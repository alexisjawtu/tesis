\chapter{Approximation}\label{auxLabel100}
\section*{Introducci\'on al cap\'itulo}
En el presente cap\'itulo probamos existencia y unicidad
para un problema discreto el\'iptico modelo y establecemos
los errores de aproximaci\'on.
Primero tratamos el caso convexo y despu\'es el no convexo. 
Para este \'ultimo presentamos nuestro proceso de mallado
en t\'erminos de prismas, tetraedros y pir\'amides, que produce
mallas adecuadamente graduadas para problemas en dominios poliedrales singulares
y escribimos y demostramos nuestro teorema de error global de
interpolaci\'on en estas mallas, el cual es probado usando los espacios
de Sobolev con pesos presentados en el Cap\'itulo~\ref{auxlabel207},
y con este probamos nuestro teorema de error de aproximaci\'on.

Ahora recordamos brevemente la comparaci\'on hecha en
la Introducci\'on entre nuestro m\'etodo y 
aquellos presentados en~\cite{MR1866274}. En ese art\'iculo
los autores proponen un m\'etodo con elementos 
prism\'aticos anis\'otropos solamente para
dominios cil\'indricos y aqu\'i proponemos un m\'etodo
que admite dominios polihedrales arbitrarios en
dimensi\'on 3 mediante una combinaci\'on
de elementos de geometr\'ias diferentes (prismas,
pir\'amides y tetraedros).

Adem\'as los autores en~\cite{MR1866274} incluyen
un resultado que usa mallas de tetraedros subdividiendo
cada prisma en tres, que requieren del uso de
tetraedros que no verifican una condici\'on 
$\pazocal{RVP}$ uniforme (cfr. Figure~\ref{fig:tetraedros})
con la consecuencia de requerir regularidad
adicional al dato $f$ del problema. En nuestro m\'etodo,
la construcci\'on de una malla
h\'ibrida evita el uso de los mencionados tetraedros
y por esto recuperamos el orden \'optimo de convergencia
del error de aproximaci\'on con dato $f$ en $L^2$.

\section*{Introduction to the chapter}
In the present chapter we prove
discrete well posedness of a model elliptic problem
and then we state the approximation errors.
First we deal with the convex case and then with the  non--convex
one. For the latter, we present our general meshing procedure
in terms of prisms, tetrahedra and pyramids, and which yields
suitable graded meshes for singular polihedral domains, 
and state first our main
global interpolation result in those meshes, which is proved using the 
weighted Sobolev spaces presented in Chapter~\ref{auxlabel207},
and whith this theorem we prove our approximation error theorem.

Now we recall briefly the comparison, that was made in the
Introduction of this thesis, between our method and the
ones presented in~\cite{MR1866274}. There the authors propose
a method with anisotropic prismatic elements only for cartesian
product type (cylindrical) domains in $\mathbb{R}^3$ and 
here we propose
a method that allows for arbitrary polihedral domains $\mathbb{R}^3$ by
means of a combination of elements of three different geometries: prisms, pyramids
and tetrahedra.

Furthermore the authors in~\cite{MR1866274} include a result 
using tetrahedral meshes
by splitting each prism into three tetrahedra, which require
the use of tetrahedra
that don't fulfill a uniform $\pazocal{RVP}$ property 
(cfr. Figure~\ref{fig:tetraedros})
and the price paid is that they require more regularity 
to the datum $f$ of 
the problem.
In our method, the construction of a hybrid mesh avoids 
the use of the mentioned
tetrahedra and therefore we recover the optimal order of convergence for 
the approximation error with datum $f$ in $L^2$.

\section{Discrete Well Posedness of a Model Elliptic Problem} % (fold)
\label{sec:well_posedness}
If we consider the following kernel
\begin{IEEEeqnarray*}{rCl}
  \Kb_{\Th} & = & \{\bv_h\in V_h: b(\bv_h,q)=0\,\,
                        \mbox{for all } q\in Q_h\} \\[4pt]
               & = & V_{\textit{h}}\cap\ker\dv
\end{IEEEeqnarray*}
then Lemma~\ref{lemma_for_coercivity} implies the following Proposition.
\begin{proposition}
  The form $a_h$ in~(\ref{discreteGlobal_a})  is coercive over $\Kb_{\Th}$
  and the coercivity constant
  depends only on the shape regularity of the pyramids of the mesh.
\end{proposition}
\begin{proposition} \label{cont} For every $E\in\Th$ the local discrete bilinear form $a_h^E$
in~(\ref{discreteLocal_a})
is continuous en $L^2(E)$. That is,  for all $\bu$,$\bv\in V_h(E)$
\[
  a_h^E(\bu,\bv) \leqslant C \|\bu\|_{L^2(E)}\|\bv\|_{L^2(E)}\mbox{,}
\]
where $C$ equals $1$ when $E$ is a right prism or tetrahedron, and
depends only on the aspect ratio of $E$ in the case of pyramids.
\end{proposition}
\begin{proof}
When $E$ is a prism or tetrahedron, $a_h^E(\bu,\bv)=a^E(\bu,\bv)$ for $\bu$ 
and $\bv$ in $V_h(E)$, and the result is immediate. 
If $E$ is a pyramid, as $a_h^E$ is symmetric, the coercivity arising from~\eqref{L2}
implies that $a_h^E$ defines an inner product. 
Hence we have
\begin{IEEEeqnarray*}{rClCr}
  \left|a_h^E(\bu,\bv)\right| & \leqslant &      a_h^E(\bu,\bu)^\frac12 a_h^E(\bv,\bv)^\frac12 & \leqslant &\\ 
                              & \leqslant & C_E\,a^E(\bu,\bu)^\frac12 a^E(\bv,\bv)^\frac12     & \leqslant & C_E\|\bu\|_{L^2(E)}\|\bv\|_{L^2(E)}.
\end{IEEEeqnarray*}
\end{proof}
\begin{lemma} \label{lemma_inf_sup_bh} There exists a constant
$\beta^*>0$ depending only on $\Omega$  and the maximum aspect
ratio of the pyramids of $\Th$ 
such that for all $q^*\in Q_h$ there exists $\bw_h^*\in V_h$ such
that
\[
\dv\bw_h^*=q^* \mbox{\quad and\quad} \beta^*\|\bw_h^*\|_{Q}\leqslant\|q^*\|_{Q}.
\]
\end{lemma}
\begin{proof} Start considering the infinite dimensional version
of the statement. There is, in fact, a constant $\beta^*$ depending only on
$\Omega$ such that
for every $q^*\in Q_h \subseteq Q$ there exists 
$\bw^*\in H^1_0(\Omega)^3$ with $\dv\bw^*=q^*$ such that
\begin{IEEEeqnarray}{rCl} \label{bound_w}
  \beta^*\|\bw^*\|_{H^1(\Omega)} & \leqslant &
  \|q^*\|_{Q}.
\end{IEEEeqnarray}
Now, for each $q^*\in Q_h$ take $\bw_h^*$ such that,
in every $E\in\Th$, $\bw_h^*|_E := I(\bw^*|_E)$, as defined in
Corollary~\ref{interpolant}. As a consequence of Proposition~\ref{vem_equal_fem}
and because we are considering the exact same degrees of freedom, this
interpolation operator coincides, in the lowest order case, with the
$H(\Div)$--conforming operators in Definitions~\ref{sub:definition_of_the_h_div_element_on_prisms}
and~\ref{defi_face_element_tetra}. 

For prismatic elements we use the estimate in Remark~\ref{auxlabel4} which, together
with the anisotropic rescalings  used in Theorem~\ref{aux_label46}, applies 
to any right prism, and Theorem $3.1$ in page $149$ of~\cite{aadl} for tetrahedral elements.
In the case of a pyramidal element we draw upon estimate~(\ref{estab2}). With
all these together and~(\ref{bound_w}) for all cases we get
\[
  \|\bw_{\textit{h}}^*\|_{Q} =
  \|I\bw^*\|_{Q}\leqslant
  C(1+\textit{h})\|\bw^*\|_{H^1(\Omega)}
  \leqslant\frac{C(1+\textit{h})}{\beta^*}\|q^*\|_{Q}.
\]
Besides, since $q^*\in Q_h$, by~(\ref{p0_projection}) we have
\[
  \dv I\bw^*=P_0\,\dv\bw^*=P_0\,q^*=q^*.
\]
\end{proof}
Now we can prove the discrete inf-sup condition for the $b_{\textit h}$ form 
in~(\ref{aux_label44}).
\begin{theorem}\label{inf_sup_b_h}
Consider the bilinear form $b_{\textit{h}}$ in~(\ref{aux_label44}).  
There exists $\beta > 0$ such that for all $q^*\in Q_h$ 
\begin{IEEEeqnarray}{rCl}\label{discrete_inf_sup_b} 
  \sup_{0\ne\bv\in V_h} \frac{b_h(\bv,q^*)}{\|\bv\|_{V_h}} &\geqslant& \beta\|q^*\|_{Q_h}.
\end{IEEEeqnarray}
\end{theorem}
\begin{proof} By Lemma~\ref{lemma_inf_sup_bh} for $q^*\in Q_h$
there exists $\bw_h^*\in V_h$ such that $\dv\bw_h^*=q^*$ and
$\beta^*\|\bw_h^*\|_{L^2(\Omega)}\leqslant\|q^*\|_{Q_h}$ (the constant $\beta^*$
is independent of $q^*$). Then
\begin{IEEEeqnarray*}{rcCcl}
  \|\bw_h^*\|_{V_h}^2 \, & \, = \, & \, \|\bw_h^*\|_{L^2(\Omega)}^2 + \|q^*\|_{Q_h}^2 
    \, & \,\leqslant\, & \, \left(\frac1{(\beta^*)^2}+1\right) \|q^*\|_{Q_h}^2
\end{IEEEeqnarray*}
and
\begin{IEEEeqnarray*}{rcCcl}
\sup_{0\ne\bv\in V_h} \frac{b_h(\bv,q^*)}{\|\bv\|_{V_h}}
      \,&\,\geqslant\,&\,
\frac{b_h(\bw_h^*,q^*)}{\|\bw_h^*\|_{V_h}}
      \,&\,\geqslant\,&\,
\frac1{\sqrt{\dfrac1{(\beta^*)^2}+1}}\,\|q^*\|_{Q_h}.
\end{IEEEeqnarray*}
\end{proof}
\begin{theorem} Problem~(\ref{mixedDiscrete}) has a unique solution.
\end{theorem}
\begin{proof}
  Theorem 5.2 of~\cite{ricardoMixed} states that 
  Problem~(\ref{mixedDiscrete}) has a unique solution
  provided~(\ref{discrete_inf_sup_b})  holds and 
  that there exists some $\alpha>0$ such that for all $\bu\in\Kb_{\Th}$
\begin{IEEEeqnarray*}{rCl}\label{discrete_inf_sup_a} 
  \sup_{0\ne\bv\in \Kb_{\Th}}
  \frac{a_{\textit{h}}(\bu,\bv)}{\|\bv\|_{V_h}} &\geqslant& \alpha\|\bu\|_{V_h}
\end{IEEEeqnarray*}
  but this is implied by the coercivity of $a_{\textit{h}}$
since
\begin{IEEEeqnarray*}{rCcCl}
  \sup_{0\ne\bv\in \Kb_{\Th}}
  \frac{a_{\textit{h}}(\bu,\bv)}{\|\bv\|_{V_h}}
  & \geqslant &
  \frac{a_{\textit{h}}(\bu,\bu)}{\|\bu\|_{V_h}}
  & \geqslant & c\|\bu\|_{V_h}.
\end{IEEEeqnarray*}
\end{proof}
We introduce the following global interpolation operator.
\begin{defi}\label{aux_label52}
  Given a positive
  parameter $\textit{h}\downarrow 0$ and an admissible mesh $\Th$ 
  made up of prisms, pyramids and tetrahedra,
  let $\rZerou$  be the \emph{global interpolation operator}
  \begin{IEEEeqnarray}{rCl}\label{global_interpolator}
    \rZerou & : & W^{1,1}(\Omega) \to V_{\textit{h}}
  \end{IEEEeqnarray}
  such that, for each $\{{E}:E\in\Th\}$,
  \begin{equation*}
    (\rZerou)|_{E} = 
      \left\{
      \begin{array}{rll}
        \br_E\,[\bu|_E] & \mbox{\,as in Definition~\ref{defi_face_element}} & \mbox{if $E$ is a prism}\\[5pt]
                           I\bu    & \mbox{\,as in Corollary~\ref{interpolant}} & \mbox{if $E$ is a pyramid}\\[5pt]
        \br_E\,[\bu|_E] & \mbox{\,as in Definition~\ref{defi_face_element_tetra}} & \mbox{if $E$ is a tetrahedron,}
      \end{array}
      \right.
  \end{equation*}
  in all cases with the lowest interpolation degree $k=0$ and
  with consistent normal components for interelementary faces.
\end{defi}
\begin{theorem}\label{aux_label47} The solution $(\bu_{\textit{h}},p_{\textit{h}})$
of~(\ref{mixedDiscrete}) satisfies,
for every approximation $\bw$ of $\bu$ in
$W(\Th)$,
\begin{IEEEeqnarray}{rCl}
  \label{aux_label48}
\|\bu-\bu_h\|_{L^2(\Omega)^3} &\leqslant&
  C\{\|\bu-\br_0\bu\|_{L^2(\Omega)^3} + \|\bu-\bw\|_{L^2(\Omega)^3}\} \\[5pt]
  \label{aux_label49}
\|P_{0,{\tau_{\textit{h}}}}p-p_h\|_{L^2(\Omega )} &\leqslant&
  C\{\|\bu-\bu_h\|_{L^2(\Omega)^3} + \|\bu-\bw\|_{L^2(\Omega)^3}\}
\end{IEEEeqnarray} 
\end{theorem}
\begin{proof} Cfr. proof of Theorem 5.1 in~\cite{bfm}.
\end{proof}
\begin{defi}\label{aux_label51}
  let $P_{0,E_{\ell}}$ be the projection onto the constants over $E_{\ell}$;
  let a $\bw_{\bu}$ such that, for every $E_\ell$
  \begin{equation*}
    (\bw_{\bu})|_{E_\ell} = 
      \left\{
      \begin{array}{rll}
        \br_{E_\ell}\,[\bu|_{E_\ell}] & \mbox{if $E_\ell$ is a prism or a tetrahedron}\\[5pt]
                           P_{0,E_{\ell}}\bu    & \mbox{if $E_{\ell}$ is a pyramid}\\[5pt]
      \end{array}
      \right.
  \end{equation*}
\end{defi}
\begin{remark}
The one in Definition~\ref{aux_label51} is is the approximation $\bw$ of $\bu$
piecewise in $W(E)$ that we will use on the right hand side of~(\ref{aux_label48})  and~(\ref{aux_label49}).
Observe that $\bw_{\bu_r+\bu_s} = \bw_{\bu_r} + \bw_{\bu_s}.$
\end{remark}
% section well_posedness (end)
\section{Approximation Error for the Convex Case} % (fold)
\label{sec:convex Case}
For the case in which $\Omega$ is convex and $f\in L^2(\Omega)$ 
we obtain from Theorem~\ref{aux_label47} the following Corollary.
\begin{corollary}\label{auxlabel418}
\begin{eqnarray*}
\|\bu-\bu_h\|_{L^2(\Omega)}&\leqslant& C\textit{h}|p|_{H^2(\Omega)}\\ 
\|p-p_h\|_{L^2(\Omega)}&\leqslant& C\textit{h}\|p\|_{H^2(\Omega)}
\end{eqnarray*}
where the constant $C$ depends only on the aspect ratios of tetrahedra 
and pyramids and the maximum angle of the triangular faces
of the right prisms on the mesh, provided that the tetrahedra fulfill a
uniform maximum angle condition. 
\end{corollary}
\begin{proof}
For a convex $\Omega$ and $f\in L^2(\Omega)$ the solution $p$ of the
problem belongs to $H^2(\Omega)$ and so $\bu\in H^1(\Omega)^3$. In this case, using the 
interpolation error estimates we proved in Theorem~\ref{aux_label46} for general
prisms
and Proposition~\ref{propErrorInterpolacionPiramidesTetraedros} for 
pyramids and the analogue for tetrahedra from~\cite{aadl},
we have, for $\br_0\bu$ as in Definition~\ref{aux_label52}, that 
\begin{IEEEeqnarray*}{rCl}
  \|\bu-\br_0\bu\|_{L^2(\Omega)^3} &\leqslant & C\textit{h}|p|_{H^2(\Omega)}.
\end{IEEEeqnarray*}
Next
again Theorem~\ref{aux_label46},
Proposition~\ref{propErrorInterpolacionPiramidesTetraedros} for pyramids and 
the analogue for tetrahedra from~\cite{aadl}, and Proposition~\ref{propupi}
for pyramids
yield, for $\bw_{\bu}$ as in Definition~\ref{aux_label51},
\begin{IEEEeqnarray*}{rCl}
  \|\bu-\bw_{\bu}\|_{L^2(\Omega)^3} & \leqslant & C\textit{h}|p|_{H^2(\Omega)}.
\end{IEEEeqnarray*}
So joining the last two inequalities,  from~\eqref{aux_label48} in Theorem~\ref{aux_label47},
\begin{IEEEeqnarray*}{rCl}
  \|\bu-\bu_h\|_{L^2(\Omega)^3} & \leqslant &
  C\{\|\bu-\bw_{\bu}\|_{L^2(\Omega)^3} + \|\bu-\br_0\bu\|_{L^2(\Omega)^3}\}\\[5pt]
  &\leqslant & C\textit{h}|p|_{H^2(\Omega)}\\[5pt]
  &\leqslant & C\textit{h}\,\|f\|_{L^2(\Omega)}
\end{IEEEeqnarray*}
as stated in~\cite{alw}	in expression after $(3.42)$ on page $19$.
The bound for the scalar variable error
in the convex case follows using~\eqref{aux_label49} and the estimate
$\|p-P_{0,\tau_{\textit{h}}}\,p\|\leqslant C\,h_{E}\,|p|_{1,E}$ for the orthogonal
projection.
\end{proof}
Arbitrarily narrow right prisms can be used in the mesh without 
affecting this estimate. This fact can be further exploited when the
domain $\Omega$ is not convex or $f$ is not in $L^2(\Omega)$. Besides, we could
also allow arbitrary narrow pyramids in the mesh (as long as they are not \textit{flat},
that is, at least one side of the basis must be smaller than the height) and
continue from Theorems~\ref{aux_label53} and~\ref{aux_label54}, but in the meshes we 
designed, which appear in what follows, pyramids happened to be regular. 
The same can be said about the $\pazocal{MAC}$ condition mentioned
in Corollary~\ref{auxlabel418}. Again, this is not a restriction, since
our method required only the use of shape--regular tetrahedra. This will be made clear
in Subsection~\ref{meshes}.

The non--convex case including anisotropic prismatic elements is what follows. It is in
this non--convex case where we will need our estimates of anisotropic type,
in contrast with the previous convex case, where we didn't need them.

\section{Approximation Error for the Non--Convex Case}
\label{sec:non_convex_case}
Our main approximation error Theorem will be the following statement:
There exists a family of anisotropic graded meshes
$\{\Th\}_{{\textit{h}}\downarrow 0}\,$
made up of
prisms, tetrahedra and pyramids 
for which 
\begin{IEEEeqnarray*}{rCl}
  \|\bu-\bu_{\textit{h}}\|_{0,\Omega}&\leqslant &C {\textit{h}} \|f\|_{0,\Omega}\\[5pt]
  \|p-p_{\textit{h}}\|_{0,\Omega}&\leqslant &C \textit{h} \|f\|_{0,\Omega}
\end{IEEEeqnarray*}
with $\textit{h}$ being smaller than  $C N_{\textit{h}}^{\nicefrac{-1}{3}}$, where
$N_{\textit{h}}$ is the  number of elements in $\Th$.

The rest of the chapter contains its proof, which is summarized at the end of it.
To avoid what would result in a proof causing weariness we decided to partition
the proof in subsections, paragraphs and items. The first part of the proof
will be the exhibition of the meshes and after that we will prove the estimates.
%=====================================================
%{\color{BrickRed}\paragraph{TODO} % (fold)
%\label{par:todo}
%ver qu'e pasa con lo de distance to the vertices, distance to de 
%singular vertices.
%
%ver si se acomoda definiendo $\lambda_e, \lambda_v$ como $+\infty$ en caso de
%ser arista o v'ertice regulares, y as'i poder usar las cotas con las 
%distancias al \'unico v\'ertice o arista de cada macroelemento, etc...
%}
%=====================================================
\subsection{Meshes and Meshing Algorithm in Dimension $3$}\label{meshes}
%%=========== GRADING ==============================================
Our algorithm starts taking a first partition of $\Omega$
into macro--elements, which may be
prisms or tetrahedra, according to the main regularity result, 
Theorem~\ref{thm_regularity}. Read Section~\ref{sec:regularity} again and recall Figures~\ref{macro_prism_reg}  and~\ref{macro_tetra_reg}.
Here we recall it.

Let $\Omega=\cup_{\ell=1}^N \Lambda_\ell$ be 
a decomposition in
macro--elements having, each one of them,
at most a singular edge $\lambda_{\be}^{(\ell)}$ and a singular vertex
$\lambda_{\bv}^{(\ell)}$. 

Please recall also the notation for the distance  $R(\bx)$ to the singular vertex 
and the angular distance $\theta(\bx)$ to the singular edge in Definition~\ref{auxlabel300}.

We will show that, in each prismatic or tetrahedral macro--element, our meshes
fulfill the following grading. Given a macro--element $\Lambda_\ell$ with
just a singular edge $\be$, for
any element $E\subseteq\Lambda_\ell$ it holds
\begin{equation}\label{label_grading3}
\begin{IEEEeqnarraybox*}{lCl}
  h_{E,1}, h_{E,2} &\sim&
    \begin{cases}
      \textit{h}^{\frac{1}{\mu}}  & \text{ if\, }d(E,\be)=0\\
      \textit{h}\,d(E,\be)^{1-\mu}  & \text{ if\, }0<d(E,\be)\lesssim1\\
      \textit{h}          & \text{ if\, }d(E,\be)\sim1
    \end{cases}\\[5pt]
  h_{E,3}   &\sim&\textit{h}
\end{IEEEeqnarraybox*}
\end{equation}
Given a macro--element $\Lambda_\ell$ with
just a singular vertex $\bv$, for
any element $E\subseteq\Lambda_\ell$ it holds
\begin{equation}\label{label_grading2}
\begin{IEEEeqnarraybox*}{lCl}
  h_{E,1}, h_{E,2}, h_{E,3}   &\sim& 
    \begin{cases}
      \textit{h}^{\frac{1}{\nu}}  & \text{ if\, }d(E,v)=0\\
      \textit{h}\,d(E,\bv)^{1-\nu}  & \text{ if\, }0<d(E,v)\lesssim1\\
      \textit{h}          & \text{ if\, }d(E,v)\sim1.
    \end{cases}
\end{IEEEeqnarraybox*}
\end{equation}
Given a macro--element $\Lambda_\ell$ with
a singular edge $\be$ and a singular vertex $\bv$ at an
endpoint of $\be$, for
any element $E\subseteq\Lambda_\ell$ it holds
\begin{equation}\label{label_grading}
\begin{IEEEeqnarraybox*}{lCl}
  h_{E,1}, h_{E,2} &\sim&
    \begin{cases}
      \textit{h}^{\frac{1}{\mu}}  & \text{ if\, }d(E,\be)=0\\
      \textit{h}\,d(E,\be)^{1-\mu}  & \text{ if\, }0<d(E,\be)\lesssim1\\
      \textit{h}          & \text{ if\, }d(E,\be)\sim1
    \end{cases}\\[5pt]
  h_{E,3}   &\sim& 
    \begin{cases}
      \textit{h}^{\frac{1}{\nu}}  & \text{ if\, }d(E,v)=0\\
      \textit{h}\,d(E,\bv)^{1-\nu}  & \text{ if\, }0<d(E,v)\lesssim1\\
      \textit{h}          & \text{ if\, }d(E,v)\sim1
    \end{cases}
\end{IEEEeqnarraybox*}
\end{equation}
with $\mu\leqslant 1-\delta$ for $\delta>1-\lambda_{\be}$ and 
$\nu\leqslant 1-\beta$ for $\beta>\frac12-\lambda_{\bv}$ for all cases, 
respectively.
%%==================================================================
\subsubsection{Macro--element with singular edge and singular vertex}\label{caso4}
Let $\Lambda$ be a tetrahedral macro--element with vertices $P_0, P_1, P_2$ and $P_3$.
We suppose that $P_0$ is the singular vertex and that the
singular edge is $P_0P_1$. The mesh $\pazocal T_h$ on $\Lambda$ will 
contain tetrahedra, prisms and pyramids. 
In Tables~\ref{prisms_barycentric_a}--\ref{tetrahedral_barycentric}
we explicit the elements in terms
of the barycentric coordinates of the vertices of each element corresponding to 
the ordered vertices $P_0, P_1, P_2$ and $P_3$.
\bigskip

\prismsBaryCoordA                        %% tables of coordinates

\prismsBaryCoordB

\pyramidsBaryCoord

\tetrahedraBaryCoord
In Figure~\ref{aux_label70} there are a couple of examples.
\begin{figure}
  \centering
  \subfloat[$n=0$]
  {
    \tetrahedralmacroelementA
  }\hspace{1cm}
  \subfloat[$n=1$]
  {
    \tetrahedralmacroelementB
  }\hspace{1cm}
  \subfloat[$n=2$]
  {
    \tetrahedralmacroelementC
  }\\
  \subfloat[$n=3$]
  {
    \tetrahedralmacroelementD
  }\hspace{1cm}
  \subfloat[$n=4$]
  {
    \tetrahedralmacroelementE
  }\hspace{1cm}
  \subfloat[$n=5$]
  {
    \tetrahedralmacroelementF
  }
  \caption{Tetrahedral macro--elements with singularities of both types.}
  \label{aux_label70}
\end{figure}
\newpage
\subsubsection{Macro--element with a singular edge}
This case corresponds to the prismatic macro--element as well as the
tetrahedral macro--element of Subsubsection~\ref{caso4} but only with
a singular edge. Therefore, we only need to present the meshing for the former case.
A mesh in the prismatic macro--elements is constructed as the cartesian product 
between a graded mesh
in a triangle and a quasi--uniform mesh along the singular edge. % Table~\ref{prisms_product} shows the points
For $0\leqslant i\leqslant n$ and $0\leqslant j\leqslant n-1$ let $p_{i,j}$
be the point with barycentric coordinates, with respect to $P_0,P_1,P_2$,
equal to      
\begin{eqnarray*}
&&\lambda_0(i,j)=1-\lambda_1(i,j)-\lambda_2(i,j),\\[5pt]
&&\lambda_1(i,j)=\frac in\left(\frac{i+j}n\right)^{\nicefrac{1}{\mu}-1},\quad
  \lambda_2(i,j)=\frac jn\left(\frac{i+j}n\right)^{\nicefrac{1}{\mu}-1},\quad
\end{eqnarray*}
Now let $\Tb$ the family of triangles with vertices
\begin{IEEEeqnarray*}{lllCLL}
p_{i,j} & p_{i+1,j} & p_{i,j+1} & \quad & 0\leqslant i  < n\mbox{,\quad}&0\leqslant j < n - i\\
p_{i,j} & p_{i+1,j-1} & p_{i+1,j} & \quad & 0\leqslant i  < n-1\mbox{,\quad}&1\leqslant j < n - i.
\end{IEEEeqnarray*}
The elements in $\Lambda_\ell$ will be
$\tau\times (P_{0,3} + \frac kn h_{\Lambda_\ell,3}, P_{0,3} + \frac{k+1}{n} h_{\Lambda_\ell,3})$
for each $\tau\in\Tb$ and each $0\leqslant k<n$. See Figure~\ref{prismatic_macroelements}.
%borrar ---->>>  \prismsProductCoord

\def\col{black}
\def\singCol{red}
\begin{figure}[!h]\centering
  \subfloat
  {
    \begin{tikzpicture}[scale=2]
      \prismaticMacroelement{8}{2}{.63}{4}{\col}{\singCol}
    \end{tikzpicture}\hspace{1cm}
    \begin{tikzpicture}[scale=2]
      \prismaticMacroelement{8}{3}{.63}{4}{\col}{\singCol}
      \draw[red] (0,0,0) -- (0,-1.5,0);
    \end{tikzpicture}\hspace{1cm}
    \begin{tikzpicture}[scale=2]
      \prismaticMacroelement{8}{4}{.63}{4}{\col}{\singCol}
      \draw[red] (0,0,0) -- (0,-1.5,0);
    \end{tikzpicture}
  }
  \caption{Elements of the family of
    meshes restricted to a prismatic macro--element.$\mu = .63$}
  \label{prismatic_macroelements}
\end{figure}

\subsubsection{Macro--element with a singular vertex}

We consider again a tetrahedral macro--element $T$ with vertices $P_0, P_1, P_2$
and $P_3$, assuming that it has a singular vertex at $P_0$ and no singular edge.
We construct a triangulation of $T$ made of tetrahedra describing 
the barycentric coordinates of the vertices of each tetrahedron with respect
to $P_0, P_1, P_2$ and $P_3$. %This case corresponds to Case 1 of \cite{AN}.

Let $p_{i,j,k}$ be the points with barycentric coordinates
\begin{eqnarray*}
&&\lambda_0=1-\lambda_1-\lambda_2-\lambda_3,\\[5pt]
&&\lambda_1=\tfrac in\left(\tfrac{i+j+k}n\right)^{\nicefrac{1}{\mu}-1},\quad
  \lambda_2=\tfrac jn\left(\tfrac{i+j+k}n\right)^{\nicefrac{1}{\mu}-1},\quad
  \lambda_3=\tfrac kn\left(\tfrac{i+j+k}n\right)^{\nicefrac{1}{\mu}-1}
\end{eqnarray*}
for $0\leqslant i\leqslant n$, $0\leqslant j\leqslant n-i$, $0\leqslant k\leqslant n-i-j$.
Then, the tetrahedra are the $n^3$ ones with the following vertices
\begin{IEEEeqnarray*}{lcl}
p_{i,j,k}, p_{i+1,j,k}, p_{i,j+1,k}, p_{i,j,k+1}, &\quad& 0\leqslant i+j+k\leqslant n-1\mbox{,}\\
p_{i+1,j,k}, p_{i,j+1,k}, p_{i,j,k+1}, p_{i+1,j,k+1}, &\quad& 0\leqslant i+j+k\leqslant n-2\mbox{,}\\
p_{i,j+1,k}, p_{i,j,k+1}, p_{i+1,j,k+1}, p_{i,j+1,k+1}, &\quad& 0\leqslant i+j+k\leqslant n-2\mbox{,}\\
p_{i+1,j,k}, p_{i,j+1,k}, p_{i+1,j+1,k}, p_{i+1,j,k+1}, &\quad& 0\leqslant i+j+k\leqslant n-2\mbox{,}\\
p_{i,j+1,k}, p_{i+1,j+1,k}, p_{i+1,j,k+1}, p_{i,j+1,k+1}, &\quad& 0\leqslant i+j+k\leqslant n-2\mbox{,}\\
p_{i+1,j+1,k}, p_{i+1,j,k+1}, p_{i,j+1,k+1}, p_{i+1,j+1,k+1}, &\quad& 0\leqslant i+j+k\leqslant n-3.
\end{IEEEeqnarray*}


\begin{proposition} Pyramids and tetrahedra in the meshes in Subsection~\ref{meshes}  are isotropic.
\end{proposition}
\begin{proof} Consider a pyramid with vertices $p_0, \ldots, p_4$ in a 
macro--element of vertices $P_0,P_1,P_2$ and $P_3$ as in Subsection~\ref{caso4}.
With $|p_i-p_j|$ or $|P_i-P_j|$ we mean euclidean distance.
Note that the basis of the pyramid is the paralelogram $p_0p_1p_3p_2$ with
\begin{eqnarray}
\label{once}
&&p_1-p_0=p_3-p_2=\tfrac1n\left(\tfrac{n-l-1}n\right)^{\nicefrac{1}{\mu}-1}(P_3-P_2)\\
\label{doce}
&&p_2-p_0=p_3-p_1=\left[\left(\tfrac{n-l}n\right)^{\nicefrac{1}{\mu}}-\left(\tfrac{n-l-1}n\right)^{\nicefrac{1}{\mu}}\right](P_0-P_1).
\end{eqnarray}
So there hold
\begin{IEEEeqnarray*}{rCl}
  \tfrac{\nicefrac{1}{\mu}}{n}\left(\tfrac{n-l-1}n\right)^{\nicefrac{1}{\mu}-1}|P_0-P_1|
  & \leqslant & |p_2-p_0|=|p_3-p_1| \\[4pt]
  & \leqslant &
  \tfrac{\nicefrac{1}{\mu}}{n}\left(\tfrac{n-l}n\right)^{\nicefrac{1}{\mu}-1}|P_0-P_1|\mbox{,}
\end{IEEEeqnarray*}
and 
\[
\mu\left(\tfrac12\right)^{\nicefrac{1}{\mu}-1}\leqslant\mu\left(\tfrac{n-l-1}{n-l}\right)^{\nicefrac{1}{\mu}-1}\leqslant\tfrac{|p_1-p_0|}{|p_2-p_0|}\leqslant \mu.
\]
Then the parallelogram $p_0p_1p_3p_2$ is shape-regular since the angle between $P_0-P_1$ and $P_3-P_2$ depends only on the macro--element, and so it is away from $0$ and $\pi$. 

Now we prove that there exists constants $c_0$ and $c_1$ depending only on
 $\nicefrac{1}{\mu}$ and the vertices of the macro--element such that
\begin{IEEEeqnarray}{rCcCl}
\label{diez}
c_0&\leqslant&\tfrac{|p_4-p_2|}{|p_2-p_0|}&\leqslant& c_1
\end{IEEEeqnarray}
and  
\begin{IEEEeqnarray}{rCcCl}
\label{trece}
c_0&\leqslant&\tfrac{|p_4-p_3|}{|p_2-p_0|}&\leqslant& c_1.
\end{IEEEeqnarray}
After simple computations we obtain
\begin{eqnarray*}
p_4-p_2 &=& \left[\left(\tfrac{n-l}n\right)^{\nicefrac{1}{\mu}} - \left(\tfrac{n-l-1}n\right)^{\nicefrac{1}{\mu}}\right](P_3-P_0) \\ &&\qquad + \tfrac in \left[\left(\tfrac{n-l}n\right)^{\nicefrac{1}{\mu}-1} - \left(\tfrac{n-l-1}n\right)^{\nicefrac{1}{\mu}-1}\right](P_2-P_3)\\ &=& \left[\left(\tfrac{n-l}n\right)^{\nicefrac{1}{\mu}} - \left(\tfrac{n-l-1}n\right)^{\nicefrac{1}{\mu}}\right]\bigg\{P_3-P_0 +\\ &&\qquad + \tfrac{\tfrac in \left[\left(\tfrac{n-l}n\right)^{\nicefrac{1}{\mu}-1} - \left(\tfrac{n-l-1}n\right)^{\nicefrac{1}{\mu}-1}\right]}{\left[\left(\tfrac{n-l}n\right)^{\nicefrac{1}{\mu}} - \left(\tfrac{n-l-1}n\right)^{\nicefrac{1}{\mu}}\right]} (P_2-P_3)\bigg\}\\ &\sim& \left[\left(\tfrac{n-l}n\right)^{\nicefrac{1}{\mu}} - \left(\tfrac{n-l-1}n\right)^{\nicefrac{1}{\mu}}\right](P_3-P_0)\\ &\sim& p_3-p_1
\end{eqnarray*}
where we used the following relation 
\[
  \tfrac{\tfrac in \left[\left(\tfrac{n-l}n\right)^{\nicefrac{1}{\mu}-1} - \left(\tfrac{n-l-1}n\right)^{\nicefrac{1}{\mu}-1}\right]}{\left[\left(\tfrac{n-l}n\right)^{\nicefrac{1}{\mu}} - \left(\tfrac{n-l-1}n\right)^{\nicefrac{1}{\mu}}\right]}\leqslant \tfrac{\nicefrac{1}{\mu}-1}{\nicefrac{1}{\mu}},
\]
and also that the angle between $P_3-P_0$ and $P_2-P_3$ is fixed
(and depends only on the macro--element) and equation~\eqref{doce}. 
This proves~\eqref{diez}. Inequalities in~\eqref{trece} follow analogously. 
To finish the proof of the isotropy of the pyramids we have to observe that the 
basis $p_0p_1p_3p_2$ is contained in a plane ${\Pi}_1$ that is parallel to the one generated 
by the directions $P_1-P_0$ and $P_3-P_2$, and the face $p_2p_3p_4$ is in a 
plane parallel to the plane ${\Pi}_2$ in which lies the triangle $P_0P_2P_3$ 
(this is a face of the macro--element), 
and the angle between  planes ${\Pi}_1$ and ${\Pi}_2$ depends only on the macro--element.      
\end{proof}
%===========================================================
%Another way, less formal but clearer, to look
%at the  mesh points in a tetrahedral macro--element
%with both kinds of singularities is the following.
%
%Fijamos:
%\begin{IEEEeqnarray*}{rCl}
%  T &=& \left[\textbf{p}_0,\textbf{p}_1,\textbf{p}_2,\textbf{p}_3\right]\\[7pt]
%  \textbf{p}_0 &=& 0\\[7pt]
%  \left[\textbf{p}_0,\textbf{p}_3\right] &=& 
%  \text{la arista singular}\\[7pt]
%  \textbf{p}_3 & = & \text{v\'ertice singular}\\[7pt]
%  \left[\textbf{p}_0, \textbf{p}_1, 
%  \textbf{p}_2\right]&\subseteq&\{z=0\}.
%\end{IEEEeqnarray*}
  %
%$\lambda_0,\lambda_1$ y $\lambda_2$: {\color{RedOrange}coordenadas  baric\'entricas} con respecto a
%$\textbf{p}_0,\textbf{p}_1$ y $\textbf{p}_2$.\\[10pt]
%$\textbf{p}_{i,j}$ es tal que\\[5pt]
%\begin{IEEEeqnarray*}{rCl}
%  \lambda_0(\textbf{p}_{i,j})  &=& 1-\lambda_1(\textbf{p}_{i,j}) - \lambda_2(\textbf{p}_{i,j}) \\[5pt]
%  \lambda_1(\textbf{p}_{i,j})  &=& \frac{i}{n}\left(\frac{i+j}{n}\right)^{\frac{1}{\mu}-1}\\[5pt]
%  \lambda_2(\textbf{p}_{i,j})  &=& \frac{j}{n}\left(\frac{i+j}{n}\right)^{\frac{1}{\mu}-1}.\\[15pt]
%  &&0\leqslant i\leqslant n\\[5pt]
%  &&0\leqslant j\leqslant n-i
%\end{IEEEeqnarray*}
  %
%Los puntos son la uni\'on de\\[5pt]
%\begin{IEEEeqnarray*}{lCl}
%  \left\{ \textbf{p}_{i,j}\,:\,0\leqslant i\leqslant n,\quad
%    0\leqslant j\leqslant n-i \right\}&&\\[7pt]
%  \left\{ \textbf{p}_{i,j}\,:\,0\leqslant i\leqslant n-1,\quad
%  0\leqslant j\leqslant n-1-i \right\} & \quad+\quad & 
%  [{\scriptstyle 1-\left(\frac{n-1}{n}\right)^{1/\mu}}]\,\textbf{e}_3\\[7pt]
%  \IEEEeqnarraymulticol{3}{c}{\vdots}\\[7pt]
%  \left\{\textbf{p}_{i,j}\,:\,0\leqslant i\leqslant n-k,\quad
%  0\leqslant j\leqslant n-k-i \right\} & \quad+\quad &
%  [{\scriptstyle 1-\left(\frac{n-k}{n}\right)^{1/\mu}}]\,\textbf{e}_3\\[7pt]
%  \IEEEeqnarraymulticol{3}{c}{\vdots}\\[7pt]
%  \left\{ \textbf{p}_{i,j}\,:\,0\leqslant i\leqslant 1,\quad
%  0\leqslant j\leqslant 1-i \right\} & \quad+\quad &
%  [{\scriptstyle 1-\left(\frac{1}{n}\right)^{1/\mu}}]\,\textbf{e}_3\\[10pt]
%  \IEEEeqnarraymulticol{3}{l}{\left\{\textbf{e}_3\right\}}\text{,}
%\end{IEEEeqnarray*}
%i.e.:
%\begin{IEEEeqnarray*}{rCl}
%  \mathcal{P} & = & \bigcup_{k=0}^n\;\left\{ \textbf{p}_{i,j}\,:\,0\leqslant i\leqslant n - k,\quad
%  0\leqslant j\leqslant n-k-i \right\}\\[8pt]
%  &&\quad+\,[{\scriptstyle1-\left(\frac{n-k}{n}\right)^{1/\mu}}]\,\textbf{e}_3.
%\end{IEEEeqnarray*}
%===========================================================
\begin{remark}
  Our meshes sastify the property that for each $\ell$ and each $\textit{h}$
  there exists an index set $I_{\textit{h},\ell}$ such that
  \begin{IEEEeqnarray*}{rCl}
    \Lambda_\ell = \cup_{i\in I_{\textit{h},\ell}} E_{\textit{h},i}.
  \end{IEEEeqnarray*}
  In this case we are denoting 
  $\Th = \{E_{\textit{h},i}\,:\,1\leqslant i\leqslant N_{\textit{h}}\}$.
\end{remark}
This means the meshes resolve the macro--elements in a conforming way and
refinement is done whithin the macro--elements of the 
initial mesh.
\begin{remark} 
We remark that if a mesh satisfies conditions~(\ref{label_grading}
for $\mu=\mu_0$ and $\nu=\nu_0$, then it satisfies the same for 
$\mu>\mu_0$ and $\nu>\nu_0$. Thus, the mesh can be constructed for 
$\mu_0=\nu_0=1-\gamma$ with $1>\gamma>\max\{1-\lambda_e,\frac12-\lambda_v\}$, 
and it still verifies the former conditions. 
The possibility to use $\mu_0=\nu_0$ for the construction of the mesh 
allows us to validate the assumptions at the beginning of Section~\ref{auxlabel290}.  

Therefore, we assume that the mesh $\Th$ on a macroelement $\Lambda_\ell$
with a singular edge 
$\be$ and a singular vertex $\bv$ 
satisfies conditions~\eqref{label_grading} 
for $\mu,\nu\geqslant 1-\gamma$. And in particular, those conditions hold
for $\mu=1-\delta$ for some $\delta>1-\lambda_{\be}$ and for $\nu=1-\beta$ 
for some $\beta>\frac12-\lambda_{\bv}$. 

For the case of a tetrahedral macroelement with just a singular edge, which is 
assumed to have a face $f$ perpendicular to the singular edge, 
we assume that it is meshed as in the case in
which the opposite vertex to the face $f$ is singular, satisfying 
conditions~\eqref{label_grading} with $\mu=\nu=1-\delta$.
This assumption does not affect the asymptotic relation between 
$\textit{h}$ and the number of elements.


We remark that if a mesh satisfies~(\ref{label_grading})
for $\mu=\mu_0$ and $\nu=\nu_0$, then it satisfies the
same for $\mu>\mu_0$ and $\nu>\nu_0$. Thus, the mesh can be overrefined by
taking $\mu=\nu=\min\{1-\delta,1-\beta\}$, and it still
verifies~(\ref{label_grading}) for
$\min\{1-\delta,1-\beta\}\leqslant\mu\leqslant 1-\delta$ and
$\min\{1-\delta,1-\beta\}\leqslant\nu\leqslant 1-\beta$. The possibility to use $\mu=\nu$
for the construction of the mesh, allows to validate condition stating
that tetrahedra and pyramids in $\Th$ are not necessarily anisotropic.
\end{remark}
\begin{remark}
If the diameter $\delta(\Lambda)$ of a macro--element $\Lambda$ were greater than $1$, 
then
\begin{IEEEeqnarray*}{rCl}
  R_K(\bx) & \leqslant & \delta(\Lambda)\min\{ R_K(\bx), 1\}
\end{IEEEeqnarray*}
so we will assume $R_K(\bx)\leqslant 1$.
\end{remark}
\begin{remark}
  $\mu < 1 \Rightarrow \mu \leqslant \nu$.
\end{remark}
\section{Main Global Interpolation Error Theorem}
In the present section we perform the anisotropic approximation error estimates
over the meshes we constructed in the previous section.

To estimate the error, we start with inequality~(\ref{aux_label48}) of
Theorem~\ref{aux_label47} and work with each term in the right--hand side
in the non--convex case.

As in the meshing algorithm we proposed the pyramids turned out to be isotropic,
we may use Proposition~\ref{propErrorInterpolacionPiramidesTetraedros} as the
local interpolation estimate which.
Nevertheless, starting with the anisotropic Finite Element stability estimates
we proved in Theorem~\ref{aux_label53}  and in Theorem~\ref{aux_label54}   
we could arrive at inequalities of the form~(\ref{aux_label39}) and~(\ref{aux_label55})
for the interpolation operators determined by the 
finite elements in Definitions~\ref{aux_label50} and~\ref{aux_label71}
and the estimates in terms of weighted norms would follow exactly as the ones
we will perform in short for $h_E|u|_{1,E}$.                        %%%  At the end of this proof we include weighted

The following is the main interpolation error theorem.
\begin{theorem}\label{interpolation_theorem} Let $\Omega$ and $f$ be the data and $(\bu,p)$ be the 
solution
of Problem~\ref{weakMixedContinuous}. Let  $\br_0$ be the operator~(\ref{global_interpolator}), then
  \begin{IEEEeqnarray}{rCl}
  \label{auxlabel75}
    \|\bu-\rZerou\|_{0,\Omega} &\leqslant& C \textit{h}\|f\|_{0,\Omega}\\[5pt]
  \label{auxlabel76}
    \|p-P_{\scriptscriptstyle \Th}p\|_{0,\Omega} &\leqslant& 
    C \textit{h}\|\nabla p\|_{0,\Omega}.
  \end{IEEEeqnarray}
\end{theorem}

\begin{remark}
  The operator~(\ref{global_interpolator}) is well defined because, for any
  polyhedral element considered, for any of its faces, if a field $\bv$ of the discrete
  global space has all
  the
  degrees of freedom with respect to that face equal to zero, then the restriction
  of $\bv$ to that face vanishes identically.
\end{remark}

\section{Proof of Theorem~\ref{interpolation_theorem}}
For the scalar variable the estimate follows from Lemma~\ref{aux_label40}.
For the vectorial variable start splitting the field $\bu$ as in~(\ref{splitting}). By 
Lemma~\ref{well_defined_dofs}, the field $\rZerou = \br_0\bu_s + \br_0\bu_r$ 
is well defined, so we will work 
with the regular part $\bu_r - \br_0\bu_r$ and the singular part $\bu_s - \br_0\bu_s$
separately. The bound for the error of the interpolation of the singular part will be 
performed in each type of macro--element.

First we'd like to remark that, for a given macro--element in 
$\pazocal{T}_{\textit{h}_0}$ the case when there is no weight with respect to 
the distance $R(\bx)$
to the vertex is also available by simply putting 
$R\equiv 1$ in the computations we show. The case when there is no weight with 
respect to the angular distance
$\theta(\bx)$ is also available by simply putting 
$\theta\equiv 1.$ Additionally, if $\be$ is not singular, the condition for 
$\delta$ in~(\ref{label_grading}) 
turns $\delta > -\infty$ and if $\bv$ is not singular, the condition for $\beta$
turns $\beta > -\infty.$

Now it is the time to make a remark of crucial importance. If we take the
solution $(\bu, p)$ of Problem~\ref{mixedContinuous},
the singular part of the vectorial variable, $\bu_s$, has a well defined $H(div, \Omega)$--conforming interpolate.
This is implied by the next result, since the normal traces on faces
of the elements in $W^{1,1}(\Lambda_\ell)$ are well defined.
\begin{lemma}\label{well_defined_dofs}
If $\beta,\delta\in[0,1)$ and $\beta + \delta\leqslant 1$ then 
\[
  V^{1,2}_{\beta,\delta}(\Lambda_\ell) \subseteq W^{1,1}(\Lambda_\ell)
\]
for all macro--element $\Lambda_{\ell}$.
\end{lemma}
\begin{proof}
Every function ${\bv}\in V^{1,2}_{\beta,\delta}(\Lambda_\ell)$ has a finite
$L^1(\Lambda_\ell)$ norm. On the other hand,
\[
  R^{-\beta}\theta^{-\delta}\leqslant\left(\max_{\bx
  \in\Lambda_\ell}R(\bx)^\delta\right)
  r^{-\beta-\delta}
  \leqslant C r^{-\beta-\delta}\in L^2(\Lambda_\ell)
\]
which implies $\partial^\alpha{\bv}\in L^1(\Lambda_\ell)$ for all
$\bv\in V^{1,2}_{\beta,\delta}(\Lambda_\ell)$.
\end{proof}
%%======================================================================================= 
% Recall that $\|\bw\|^2_{\mathbb{V}} = \|\bw\|^2_{0,\Omega} + \|\dv\,\bw\|^2_{0,\Omega}$ 
% and that, 
% First, for the divergence term,
% Primero usar el lemma de taylor promediado aplicado al caso sin derivar. Despu'es:
% If we apply this after the commutative diagram property~(\ref{div_commutativity})
% \begin{IEEEeqnarray*}{rCl}
% \|\dv(\bu - \rZerou)\|^2_{0,\Omega} & = &
%   \|\dv\,\bu - \dv\,\rZerou\|^2_{0,\Omega}\\
%     & = &\|\dv\,\bu - \boldsymbol{P}_{\pazocal{T}_h}\dv\,\bu\|^2_{0,\Omega}\\
%   &\leqslant&\textit{h}^2\,\|\dv\,\bu\|^2_{0,\Omega}\\
%   &=&\textit{h}^2\,\|f\|^2_{0,\Omega}.|
% \end{IEEEeqnarray*}
%%======================================================================================= 
\subsection{Bound for the Regular Part in~(\ref{auxlabel75})} % (fold)
\label{sub:bound_for_the_regular_part}

Take any macro--element $\Lambda_\ell$. Given an element $E\subset\Lambda_\ell$, 
let $\boldsymbol{h}=(h_{1},h_{2},h_{3})'$. By Theorem~\ref{aux_label46}
\begin{IEEEeqnarray*}{rCl}
  \|\bu_r - \br_0\bu_r\|_{L^2(\Omega)}^2 & = &
  \sum_{E\in\Th}
  \|\bu_r - \br_0\bu_r\|_{L^2(E)}^2\\
\IEEEeqnarraymulticol{3}{l}{\qquad
  \leqslant\,C\,\sum_{E\in\Th}\left( \sum_{|{\balpha}| = 1} 
  \boldsymbol{h}^{\balpha} \|\partial^{\balpha}\bu_r\|_{L^2(E)} + 
  h_E\|\dv\bu_r\|_{L^2(E)}
  \right)^2
}
  \\[5pt]
\IEEEeqnarraymulticol{3}{l}{\qquad
  \leqslant\,\sum_{E\in\Th}\left\{\sum_{|{\balpha}| = 1}
  \boldsymbol{h}^{2{\balpha}} + h_E^{2} \right\}
  \left\{\sum_{|{\balpha}| = 1}\|\partial^{\balpha}\bu_r\|_{L^2(E)}^2 + 
  \|\dv\bu_r\|_{L^2(E)}^2\right\}.
}
\end{IEEEeqnarray*}
%================= TODO
%\noindent{\color{Orange}\#\#\#\#\#\#\# aca hay que poner la interpolacion en piramides
%aparte tambien.} 
%============================
Now following the grading~(\ref{label_grading}) it holds
$\boldsymbol{h}^{\balpha}\leqslant C\,\textit{h}$ and $h_E \leqslant C\,h_3 \sim \textit{h}$ 
for all the elements of the
  mesh, so the last expression is bounded above by
\begin{IEEEeqnarray*}{lCl}
  C^2\sum_{E\in\Th}4\,\textit{h}^{2}
  \left\{\sum_{|{\balpha}| = 1}\|\partial^{\balpha}\bu_r\|_{L^2(E)}^2 + 
  \|\text{div}\bu_r\|_{L^2(E)}^2\right\}
  &=&\qquad\\[5pt]
\IEEEeqnarraymulticol{3}{R}{
  =\,C\,\textit{h}^{2}
  \left\{\sum_{|{\balpha}| = 1}\|\partial^{\balpha}\bu_r\|_{L^2(\Omega)}^2 + 
  \|\text{div}\bu_r\|_{L^2(\Omega)}^2\right\}.
}
\end{IEEEeqnarray*}
Then, finally
\begin{IEEEeqnarray*}{rCl}
  \|\bu_r - \br_0\bu_r\|_{L^2(\Omega)}&\leqslant&
  C\,\textit{h}\,|\bu_r|_{1,\Omega}\\
  \mbox{by~(\ref{aux_label11})\qquad}&\leqslant&C\,\textit{h}\,\|f\|_{0,\Omega}.
\end{IEEEeqnarray*}
% subsection bound_for_the_regular_part (end)
\subsection{Bound for the Singular Part in~(\ref{auxlabel75}) in a 
prismatic macro--element with
a singular Edge} % (fold)
\label{sub:bound_singular_part_prismatic_macroelement}

Let $\Lambda_\ell$ be a prismatic element of $\pazocal{T}_{\textit{h}_0}$.
Let $\be$ be the singular edge of $\Lambda_\ell$ and let 
$\bxi = (\xi_1,\xi_2,\xi_3)'$ be it's local coordinates, all as in the hypotheses
of Theorem~\ref{thm_regularity}. In this case we need to
distinguish between the elements $E$ with $d(E,\be) > 0$ and those with
$d(E,\be) = 0$.
\begin{IEEEeqnarray*}{rCl}
  \| \bu_s - \br_0\bu _s\|_{L^2(\Lambda_\ell)}^2 &\leqslant&
  \sum_{d(E,\be) = 0} \left(\| \bu_s\|_{L^2(E)} + 
  \|\br_0\bu_s\|_{L^2(E)}\right)^2\\
  & &\:+\sum_{d(E,\be) > 0} \left( \sum_{|{\balpha}| = 1} 
  \boldsymbol{h}^{\balpha} \|\partial^{\balpha}\bu_s\|_{L^2(E)} + 
  h_E\|\text{div}\bu_s\|_{L^2(E)}
  \right)^2.\\
  \yesnumber\label{aux_label}&&
\end{IEEEeqnarray*}
The objective of the following paragraphs and items is to bound each one of the
terms involved in the right hand side of~(\ref{aux_label})
by a constant times
$\textit{h}$ times the norm of $f$, to sum everything up afterwards.
\paragraph{Elements with $d(E,\be) > 0$ in~(\ref{aux_label}).} % (fold)
\label{par:elements_with_d_pos}
\begin{enumerate}
\item Bound for $\|\partial_{\xi_1}\bu_s\|_{L^2(E)}$.                    %%, $\alpha = (1,0,1).$ % (fold)
\label{subp:bound_for_100} Pick any weight $w(\bx)\geqslant h_1$.
\begin{IEEEeqnarray*}{rCl}
  h_1\|\partial_{\xi_1}\bu_s\|_{L^2(E)}& = & 
    \sum_{i=1,2} \|h_1\partial_{\xi_1}u_{s,i}\|_{L^2(E)} +
      h_1\|\partial_{\xi_1}u_{s,3}\|_{L^2(E)}\\
  & \leqslant & \sum_{i=1,2} \|w(\bx)\partial_{\xi_1}u_{s,i}\|_{L^2(E)} +
      h_1\|u_{s,3}\|_{V_0^{1,2}(E)}.
\end{IEEEeqnarray*}
Now it turns out that if we could choose $w(\bx) = \textit{h}\,r(\bx)^\delta \geqslant h_1$
for some $\delta$, then
we would have
\begin{IEEEeqnarray*}{rCl}
  h_1\|\partial_{\xi_1}\bu_s\|_{L^2(E)}
  &\leqslant&\textit{h} \sum_{i=1,2} \|u_{s,i}\|_{V_\delta^{1,2}(E)} +
      \textit{h}\|u_{s,3}\|_{V_0^{1,2}(E)}\\
  \yesnumber\label{aux_label1}
  &\leqslant&C\,\textit{h}\|\bu_s\|_{\pazocal{V}_{\beta,\delta}(E)}.
\end{IEEEeqnarray*}
Let's look for $\delta$. We have $r(\bx)\geqslant d(E,\be)$ so, according to the
grading~(\ref{label_grading}) , in the 
case $0< d(E,\be) <1$, it holds
\begin{IEEEeqnarray*}{rClCl}
  \textit{h}\,r(\bx)^{1-\mu}&\geqslant&\textit{h}\,d(E,\be)^{1-\mu}&\sim& h_1.
\end{IEEEeqnarray*}
In the 
case $d(E,\be) \sim 1$
\begin{IEEEeqnarray*}{rClCl}
  \textit{h}\,r(\bx)^{1-\mu}&\gtrsim&\textit{h}&\sim& h_1.
\end{IEEEeqnarray*}
In both cases we have to take $\delta \sim 1-\mu > 1 - \frac{\pi}{\omega_\ell}$
to get the estimate~(\ref{aux_label1}).
The bound for the term $\|\partial_{\xi_22}\bu_s\|$ is done similarly.
\item Bound for $\|\partial_{\xi_3}\bu_s\|_{L^2(E)}$. % (fold)
\label{subp:bound_for_001}
If $i = 1$,$2$ or $3$, $\partial_{\xi_3}u_i$ equals $\partial_{\xi_i}u_3$, so
\begin{IEEEeqnarray*}{rCl}
  h_3^2\|\partial_{\xi_3}\bu_s\|^2_{L^2(E)}& = & 
     h_3^2\sum_{i=1,2,3}\|\partial_{\xi_3}u_{s,i}\|^2_{L^2(E)}\\
    &\lesssim&\textit{h}^2\,\| u_3 \|^2_{V_0^{1,2}(E)}.
\end{IEEEeqnarray*}
\item {Bound for the divergence.} By the grading in the present macro--element and
the triangle inequality we have
\label{subp:bound_for_the_div}
\begin{IEEEeqnarray*}{rCl}
  h_E\,\| \dv\bu_s \|_{L^2(E)} & \lesssim &
  {h_3}\,\|\dv\bu_s\|_{L^2(E)}\sim
  \textit{h}\,\|\dv\bu_s\|_{L^2(E)}\\[7pt]
  \yesnumber\label{aux_label64}&\leqslant&\textit{h}\,\left\{\|\dv\bu\|_{L^2(E)}+
  \|\dv\bu_r\|_{L^2(E)}\right\}.
\end{IEEEeqnarray*}
\end{enumerate}
\paragraph{Elements with $d(E,\be) = 0$ in~(\ref{aux_label}).}
Take $\delta \sim 1-\mu$ in this paragraph again.
\begin{enumerate}
  \item Bound for $\| \bu_s\|_{L^2(E)}$. % (fold)
\begin{IEEEeqnarray*}{rCl}
  \|\bu_s\|^2_{L^2(E)}& = & \sum_{i=1,2}\|{u_{s,i}}\|^2_{L^2(E)} + 
    \|{u_{s,3}}\|^2_{L^2(E)}\\
\IEEEeqnarraymulticol{3}{r}{
\begin{IEEEeqnarraybox*}{rCl}
\qquad&=&\sum_{i=1,2}\|r(\bx)^{1-\delta} r(\bx)^{\delta-1} {u_{s,i}}\|^2_{L^2(E)}
    + \|r\,r^{-1} {u_{s,3}}\|^2_{L^2(E)}\\
  &\leqslant& 
  \max r(\bx)^{2(1-\delta)}\sum_{i=1,2}\|{u_{s,i}}\|^2_{V_\delta^{1,2}(E)}
  + \max_{\bx\in E} r(\bx)^{2}\|{u_{s,3}}\|^2_{V_0^{1,2}(E)}\\
  &\lesssim& h_1^{2(1-\delta)}\sum_{i=1,2}\|{u_{s,i}}\|^2_{V_\delta^{1,2}(E)}
  + h_1\|{u_{s,3}}\|^2_{V_0^{1,2}(E)}\\
  &\sim&  (\textit{h}^{\nicefrac{2}{\mu}})^\mu\sum_{i=1,2}\|{u_{s,i}}\|^2_{V_\delta^{1,2}(E)}
  + (\textit{h}^{\nicefrac{1}{\mu}})^2\|{u_{s,3}}\|^2_{V_0^{1,2}(E)}\\
  &\leqslant& C\,\textit{h}^2\|\bu_{s}\|^2_{\pazocal{V}_\delta^{1,2}(E)}
\end{IEEEeqnarraybox*}
}    
\end{IEEEeqnarray*}
(because $\mu \leqslant 1$).
\item \label{aux_label61} {Bound for $\| \br_E \bu_s \|_{L^2(E)^3}$.} % (fold)
First recall that if $\phi$ is a scalar polynomial defined on a physical element $E$, then
\begin{IEEEeqnarray}{rCl}\label{normaL2L1}
  \| \phi \|_{L^{2}(E)} & \leqslant & C\,|E|^{-1/2}\,\| \phi \|_{L^{1}(E)}.
\end{IEEEeqnarray}
Now we estimate the $L^1$ norms, starting with the stability estimate in 
the rescaled element $\tilde{E}$ of Figure~(\ref{rescaled_prism}).
As in our mesh we are considering right prisms and local coordinates
in $\Lambda_\ell$ such that $\xi_3$ is the direction containing the singular
edge $\be_\ell$, there is a matrix $M_E$ of the form 
\begin{equation}\label{matrix_A}
  M_E=
    \left(\begin{array}{ccc}a_{11}&a_{12}&0\\a_{21}&a_{22}&0\\0&0&1\end{array}\right)
\end{equation}
for which an affine transform $F_E$ with matrix $M_E$ maps $\tilde{E}$ onto $E$.
The infinity
norm of $\left(\begin{array}{cc}a_{11}&a_{12}\\a_{21}&a_{22}\end{array}\right)$
is bounded by a quantity $c_E$ depending only on the maximum angle of the projection
of $E$ onto the $(\xi_1,\xi_2)$ plane and the infinity norm of its inverse
is bounded by one.
So changing variables and pulling
back to $\tilde E$ we get 
\begin{IEEEeqnarray*}{rCcCl}
  \| (\br_E\bu_s)_1 \|_{L^{1}(E)}  & = & \int_E |(\br_E\bu_s)_1|\,d\bx
        & = & \int_{\tilde{E}} |(M_E\,\tilde{\br}_{\tilde E}\tilde{\bu}_s)_1|\,d\tilde{\bx}\\
   \IEEEeqnarraymulticol{5}{R}{\leqslant  c_E\,\Big\{
    \sum_{i=1,2} \|\tilde{u}_{s,i}\|_{L^1(\tilde{E})}
      + \sum_{i=1}^3 h_i \left(\|\partial_{i} \tilde{u}_{s,1} \|_{L^{1}(\tilde{E})} +
           \| \partial_{i} \tilde{u}_{s,2} \|_{L^{1}(\tilde{E})}\right)}\\
    && & & +\,2 h_1\|\dv(\tilde{u}_{s,1},\tilde{u}_{s,2},0)\|_{L^{1}(\tilde{E})}\Big\}.\\[4pt]
\yesnumber\label{aux_label3}&&&&
\end{IEEEeqnarray*}
Now the work comes to rewrite and bound each term in~(\ref{aux_label3}).
Recalling the properties~(\ref{transfDiv}) and~(\ref{derivadaPiola}) about the derivatives
with coordinate changes,
\begin{IEEEeqnarray*}{rCl}
  \|\tilde{u}_{s,1}\|_{L^1(\tilde{E})} & = &
  \frac{1}{|\det(M_E)|} \int_{E} \left|\tilde{u}_{s,1}(F_E^{-1}(\bx))\right|\,d\bx\\[4pt]
    & =   &         \int_{E} |(M_E^{-1})_{\text{row}_1}\bu_s(\bx)|\,d\bx\\[4pt]
    & \leqslant & \|M_E^{-1}\|_\infty   \sum_{i=1,2} \|{u}_{s,i}\|_{L^1({E})}\\[4pt]
    & \leqslant & \sum_{i=1,2} \|{u}_{s,i}\|_{L^1({E})}
\end{IEEEeqnarray*}
and by the same reasons
\begin{IEEEeqnarray*}{rCl}
  \|\tilde{u}_{s,2}\|_{L^1(\tilde{E})} & \leqslant & \sum_{i=1,2} \|{u}_{s,i}\|_{L^1({E})}.
\end{IEEEeqnarray*}
For the first derivatives, let $k,l = 1$ or $2$. 
\begin{IEEEeqnarray*}{rCl}
  \|\partial_{\tilde{x}_l} \tilde{u}_{s,k}\|_{L^1(\tilde{E})} 
  & = & \frac{1}{|\det(M_E)|} \int_{E} \left|\det(M_E)\,(M_E^{-1})_{\text{row}_k}\,
  D\bu_s(\bx)\,(M_E)_{\text{col}_l}\right|\,d\bx\\[4pt]
  %& \leqslant & \|M_E^{-1}_{\text{row}_k}\|_\infty\|A_{\text{col}_l}\|_\infty\sum_{i,j = 1,2} \|\partial_{\xi_j}u_{s,i}\|_{L^{1}(E)}\\
  & \leqslant & \sum_{i=1,2}\|\partial_{\xi_l}u_{s,i}\|_{L^{1}(E)}
\end{IEEEeqnarray*}
and, similarly, if $k=1,2$
\begin{IEEEeqnarray*}{rCl}
  \|\partial_{\tilde{x}_3} \tilde{u}_{s,k}\|_{L^1(\tilde{E})} & \leqslant &
  \|M_E^{-1}\|_\infty\sum_{i = 1,2} \|\partial_{\xi_3}u_{s,i}\|_{L^{1}(E)}.
\end{IEEEeqnarray*}
To cope with the divergence term, because of the blocks of the matrix $M_E$
we have
\begin{IEEEeqnarray*}{rCl}
  (u_1(F_E\tilde\bx), u_2(F_E\tilde\bx), 0)' & = &
\frac{1}{\det M_E}\,M_E\, (\tilde{u}_1(\tilde{\bx}), \tilde{u}_2(\tilde{\bx}), 0)'\mbox{,}\\
{\dv}({u}_1,{u}_2,0)'&=&\frac{1}{\det M_E} \dv(\tilde{u}_1,\tilde{u}_2,0)'.
\end{IEEEeqnarray*}
Then, changing variables,
\begin{IEEEeqnarray*}{rCl}
  \|\dv(\tilde{u}_{s,1},\tilde{u}_{s,2},0)'\|_{L^{1}(\tilde{E})} &=&
  \|\dv(u_{s,1},u_{s,2},0)'\|_{L^{1}(E)}.
\end{IEEEeqnarray*}
Finally joining everything,
\begin{IEEEeqnarray*}{rCl}
  \| (\br_E \bu_s)_1 \|_{L^{1}(E)} & \leqslant & C\,\bigg\{ 
  \sum_{i=1,2} 
\Big\{
  \|u_{s,i}\|_{L^1(E)}\\[5pt]
  & &+\;h_i\sum_{j=1,2}\|\partial_{\xi_i}u_{s,j}\|_{L^{1}(E)^3} +
  h_3 \|\partial_{\xi_3}u_{s,i}\|_{L^{1}(E)}\Big\}\\[5pt]
  \yesnumber\label{aux_label15}
  & & +\;h_1 \|\dv(u_{s,1}, u_{s,2}, 0)'\|_{L^{1}(E)}\bigg\}.
\end{IEEEeqnarray*}
Now we enumerate the bound for each term on the right of~(\ref{aux_label15}).
\begin{enumerate}
    \item
  For $\|u_{s,i}\|$ we have
  \begin{IEEEeqnarray*}{rCl}
    \|u_{s,i}\|_{L^1(E)} & = & 
    \| r(\bx)^{1-\delta} r(\bx)^{\delta-1} u_{s,i}\|_{L^1(E)}\\
    &\leqslant& \|r^{1-\delta}\|_{L^2(E)} \|u_{s,i}\|_{V_\delta^{1,2}(E)}\\
    &\leqslant& C\,h_1^{1-\delta}\,|E|^{\nicefrac12}\,\|u_{s,i}\|_{V_\delta^{1,2}(E)}\\
    &\leqslant& C\,\textit{h}\,|E|^{\nicefrac12}\,\|u_{s,i}\|_{V_\delta^{1,2}(E)}.
  \end{IEEEeqnarray*}
    \item
  With respect to the derivatives orthogonal to the singular edge, take $i$,$j = 1$,$2$ and,
  by H\"older's inequality,
  \begin{IEEEeqnarray}{rCl}\label{aux_label16}
    h_1\|\partial_{\xi_j}u_{s,i}\|_{L^1(E)} &\leqslant&
    h_1\,\|r^{-\delta}\|_{L^2(E)}\,\|u_{s,i}\|_{V_\delta^{1,2}(E)}.
  \end{IEEEeqnarray}
  Integrating the radial weight $r^{-\delta}$ we get
  \begin{IEEEeqnarray*}{rCl}
    \|r^{-\delta}\|_{L^2(E)} & \leqslant & C\, h_1^{-\delta}\,h_1\,h_3^{\nicefrac12}\\
    & \sim & h_1^{-\delta}(h_1h_2h_3)^{\nicefrac12}\\
    &\leqslant& C\, h_1^{-\delta} |E|^{\nicefrac12},
  \end{IEEEeqnarray*}
  (where we used $h_1 \sim h_2$ and the minimun angle condition to bound the
  area of a triangular face from below with $h_1h_2$), so in~(\ref{aux_label16})
  we have
  \begin{IEEEeqnarray*}{rCl}
    h_1\|\partial_{\xi_j}u_{s,i}\|_{L^1(E)} & \leqslant & C\,h_1^{1-\delta} |E|^{\nicefrac12}
    \|u_{s,i}\|_{V_\delta^{1,2}(E)}\\
    &\sim& (\textit{h}^{1/\mu})^{\mu} |E|^{\nicefrac12}
    \|u_{s,i}\|_{V_\delta^{1,2}(E)}\\
    \label{derivOrtog} \yesnumber &=&|E|^{\nicefrac12}\,\textit{h}\,\|u_{s,i}\|_{V_\delta^{1,2}(E)}.
  \end{IEEEeqnarray*}
    \item
  For the derivatives along the singular edge we use $\bu = \nabla p$ to
  commute the indices of the derivatives and H\"older's inequality. We have
  \begin{IEEEeqnarray*}{rCl}
    h_3\,\| \partial_{\xi_3}u_{s,i} \|_{L^1(E)} &=& h_3\,\| \partial_{\xi_i}u_{s,3} \|_{L^1(E)}\\
    &\leqslant& h_3\,|E|^{\nicefrac12}\,\| \partial_{\xi_i}u_{s,3} \|_{L^2(E)}\\
    \yesnumber\label{alongSingular}&\leqslant& C\,|E|^{\nicefrac12}\,\textit{h}\,\| u_{s,3} \|_{V_0^{1,2}(E)}.
  \end{IEEEeqnarray*}
    \item
  For the divergence we can recall the grading condition $h_1\sim h_2$ and 
  observe that
  \begin{IEEEeqnarray*}{rCl}
    h_1 \|\text{div}(u_{s,1}, u_{s,2}, 0)\|_{L^{1}(E)} & \leqslant &
    h_1 \sum_{j=1,2} \|\partial_{\xi_j} u_{s,j}\|_{L^{1}(E)}
  \end{IEEEeqnarray*}
  and reuse the estimates~(\ref{derivOrtog}). 
\end{enumerate}
The estimate for 
$\| (\br_0\bu_s)_2 \|_{L^{2}(E)}$ is the same. The estimate for 
component $(\br_0\bu_s)_3$ is as follows. By the commutativity of the
local interpolator and the coordinate change in Corollary~\ref{aux_label17} and
the estimate~(\ref{aux_label18}) in the proof Theorem~\ref{thmStabilityKtildeRT} we have
\begin{IEEEeqnarray*}{rCl}
  \| (\br_E\bu_s)_3 \|_{L^{1}(E)}
  & =     &
  \| (\br_{\tilde E}\tilde{\bu}_s)_3 \|_{L^{1}(\tilde{E})}\\
  & \leqslant & C\,\big{\{}\,\|\tilde{u}_{s,3}\|_{L^1(\tilde{E})} +
    \sum_{j=1}^3 h_j\,\|\partial_{\tilde{x}_j}\tilde{u}_{s,3}\|_{L^1(\tilde{E})}\,\big{\}}
\end{IEEEeqnarray*}
which by~(\ref{matrix_A}) and the considerations made there is bounded by
\begin{IEEEeqnarray*}{rCl}
  C\,(\,\|{u}_{s,3}\|_{L^1({E})} +
    \sum_{j=1}^3 h_j\|\partial_{\xi_j}{u}_{s,3}\|_{L^1({E})}).
\end{IEEEeqnarray*}
Now
\begin{IEEEeqnarray*}{rCl}
  \|u_{s,3}\|_{L^1(E)} &\leqslant& \|r\|_{L^2(E)} \,\|r^{-1}\,u_{s,3}\|_{L^2(E)}\\
              &\leqslant&C\,h_1\,|E|^{\nicefrac12}\,\|u_{s,3}\|_{V_0^{1,2}(E)}\\
              &\leqslant&C\,\textit{h}\,|E|^{\nicefrac12}\,\|u_{s,3}\|_{V_0^{1,2}(E)}. 
\end{IEEEeqnarray*}
We have also                                   %% TODO {\color{red} (al final no va $(h_1 + h_2)$)}
\begin{IEEEeqnarray*}{rCl}
  (h_1 + h_2)\sum_{i=1,2} \|\partial_{{x}_i}{u}_{s,3}\|_{L^1({E})} &\leqslant&
  C\,|E|^{\nicefrac12}\,\textit{h}\,\|u_{s,3}\|_{V_0^{1,2}(E)}
\end{IEEEeqnarray*}
and by~(\ref{alongSingular})
\begin{IEEEeqnarray*}{rCl}
  h_3\,\|\partial_{\xi_3}u_{s,3}\|_{L^1(E)} &\leqslant& C\,|E|^{\nicefrac12}\,
  \textit{h}\,\|u_{s,3}\|_{V_{0}^{1,2}(E)}.
\end{IEEEeqnarray*}

At last, having the bound for each of the three terms in 
$\|\br_E \bu_s\|_{L^2(E)^3}$, (\ref{normaL2L1}) yields    %% TODO {\color{BrickRed}(redactar mejor)}
\begin{IEEEeqnarray*}{rCl}
  \|\br_E\bu_s\|_{L^{2}(E)^3} &\leqslant& C\,\textit{h}\,\|\bu_s\|_{\pazocal{V}_{\delta}(E)}.
\end{IEEEeqnarray*}
\end{enumerate}
% paragraph elements_with_d0 (end)
% subsection bound_singular_part_prismatic_macroelement (end)
\subsection{Bound for the Singular Part  of~(\ref{auxlabel75})
in a Tetrahedral Macro--Element with Singular Edge and Vertex}\label{auxlabel205} % (fold)
Again we organize separating cases in terms of distance to the singular edge 
and then, for this type of macro--element, the terms with $d(E,\be) = 0$
will be separated in terms of distance to the singular vertex.

Table~\ref{element_classification} shows what has to be done for each type of
element. The rest of the needed estimates are contained in the proof of
the former case.
\begin{table}[!h]
\centering
\caption{Singular Part.}
\label{element_classification}
  \begin{IEEEeqnarraybox}
  [\IEEEeqnarraystrutmode
   \IEEEeqnarraystrutsizeadd{0pt}{0pt}]{v/c/v/c/v/c/v/c/v}
    \IEEEeqnarrayrulerow\\
    \IEEEeqnarrayseprow[5pt]\\
    &\hfill\raisebox{22pt}[0pt][0pt]{$d(E,\be)>0$}\hfill
                & & \referencePrismTikz{.9} 
              & & \referencePyramidTikz{0.9}
                & & \referenceTetrahedronTikz{.94}&\\
    \IEEEeqnarrayrulerow\\
    \IEEEeqnarrayseprow[5pt]\\
    &\hfill\raisebox{30pt}[0pt][0pt]{$d(E,\be)=0$}\hfill& &
      \begin{IEEEeqnarraybox}{c}
      \referencePrismTikz{.9}\\{\scriptstyle d(E,\bv) > 0}
      \end{IEEEeqnarraybox}
    &&&&
      \begin{IEEEeqnarraybox}{c}
        \referenceTetrahedronTikz{.94}\\{\scriptstyle d(E,\bv) = 0}
      \end{IEEEeqnarraybox}
    &\\
    \IEEEeqnarrayseprow[3pt]\\
    \IEEEeqnarrayrulerow
  \end{IEEEeqnarraybox}
\end{table}

Take a tetrahedral macro--element $\Lambda_\ell$ in $\pazocal{T}_{\textit{h}_0}$ as
in the title of the subsection. Let us start with
\begin{IEEEeqnarray}{rCl}\nonumber
  \|\bu_s - \br_0\bu_s\|^2_{\scriptscriptstyle L^2(\Lambda_\ell)^3}
  &=&\sum_{d(E,\be) = 0}
  \|\bu_s - \br_0\bu_s\|^2_{\scriptscriptstyle L^2(E)^3}+
  \sum_{d(E,\be) > 0}
  \|\bu_s - \br_0\bu_s\|^2_{\scriptscriptstyle L^2(E)^3}.\\
\label{auxlabel350}
\end{IEEEeqnarray}
\paragraph{Terms in~\eqref{auxlabel350} such that $d(E,\be)>0$. }
%% (y,z,x)
\begin{figure}[!h]
  \centering
\begin{tikzpicture}[rotate=0,scale=1,shape border uses incircle,shape border rotate=-30]
  \node [name=t,shape=trapezium,draw,minimum width=2cm,
      trapezium left angle=120] at (0,0,0) {};
  
  \coordinate (punta) at (0,0,3);

  \coordinate (tl) at (t.top left corner);
  \coordinate (tr) at (t.top right corner);
  \coordinate (bl) at (t.bottom left corner);
  \coordinate (br) at (t.bottom right corner);

  \draw (tl) -- (punta);
  \draw (bl) -- (punta);
  \draw (tr) -- (punta);
  \draw (br) -- (punta);
  
  \coordinate (k1) at ($0.70*(tl) + 0.30*(punta)$);
  \coordinate (k2) at ($0.5*(tl) + 0.5*(tr)$);
  \coordinate (k3) at ($0.4*(tl) + 0.6*(bl)$);

  \node [left=1mm] (eta1) at (k1) {$\boldsymbol{\eta_1}$};
  \node [right=1mm] (eta2) at (k2) {$\boldsymbol{\eta_2}$};
  \node [below=0.5mm,right] (eta3) at (k3) {$\boldsymbol{\eta_3}$};

  \draw [->,thick] (tl) -- (k1);
  \draw [->,thick] (tl) -- (k2);
  \draw [->,thick] (tl) -- (k3);
\end{tikzpicture}
\caption{Macro--element directions in a pyramid}
\label{auxlabel8}
\end{figure}
\begin{enumerate}
  \item 
Case $E$ is a pyramid.  
Remember the local ordering for the directions
of the edges of sizes $h_1$, $h_2$ and $h_3$ for pyramids in our setting
(cfr. Figure~\ref{auxlabel8}).

As pyramids in $\Th$ don't touch singularities
and are regular, and $1-\mu\geqslant 1-\nu$, we have
\begin{IEEEeqnarray*}{rCcCl}
  \|\bu_s - \br_0\bu_s\|^2_{\sss L^2(E)} & \leqslant & h_E^2|\bu_s|_{1,E}^2
  &\leqslant&
  \textit{h}^2d(E,\be)^{2(1-\mu)}\sum_{i=1}^3 |u_{s,i}|^2_{1,E}\\
\IEEEeqnarraymulticol{5}{R}{
\begin{IEEEeqnarraybox*}{rCl}
  \qquad&\leqslant&  
    \textit{h}^2\sum_{i=1}^3\sum_{j=1}^3 \|r^{1-\mu}\gancho_{\eta_j}u_{s,i}\|_{\sss L^2(E)}^2\\
       &\leqslant&
    \textit{h}^2\left(
     \sum_{i=1}^2 \sum_{j=1}^3\|R^{1-\mu}\theta^{1-\mu}\gancho_{\eta_j}u_{s,i}\|_{\sss L^2(E)}^2
     + \sum_{j=1}^3\|R^{1-\mu}\gancho_{\eta_j}u_{s,3}\|_{\sss L^2(E)}^2\right)\\[7pt]
       &\leqslant&
    \textit{h}^2\left(
     \sum_{i=1}^2 \sum_{j=1}^3\|R^{1-\nu}\theta^{1-\mu}\gancho_{\eta_j}u_{s,i}\|_{\sss L^2(E)}^2
    + \sum_{j=1}^3\|R^{1-\nu}\gancho_{\eta_j}u_{s,3}\|_{\sss L^2(E)}^2\right)\\[7pt]
       &\leqslant&
    \textit{h}^2\left(
    \sum_{i=1}^2 \sum_{j=1}^3\|R^{\beta}\theta^{\delta}\gancho_{\eta_j}u_{s,i}\|_{\sss L^2(E)}^2+
     \sum_{j=1}^3\|R^{\beta}\gancho_{\eta_j}u_{s,3}\|_{\sss L^2(E)}^2\right)\\[7pt]
  \yesnumber\label{aux_label63} 
       &\leqslant&
    \textit{h}^2
    \|\bu_{s}\|^2_{\pazocal{V}_{\beta,\delta}(E)}.
\end{IEEEeqnarraybox*}
}
\end{IEEEeqnarray*}
\item Case $E$ is a prism (with $d(E,\be)>0$) or a tetrahedron. In the case of prisms we use 
the result of Theorem~\ref{aux_label46} and in the case of tetrahedra
we use Theorem 6.2 of~\cite{aadl}  to get
\begin{IEEEeqnarray}{rCl}\label{auxlabel9}
  \|\bu_s - \br_0\bu_s\|^2_{\sss L^2(K)}&\leqslant& C\left\{
    \sum_{i=1}^3 h_i^2\|\gancho_{\xi_i}\bu_s\|^2+
    h_K^2\|\dv\bu_s\|^2
  \right\}.
\end{IEEEeqnarray}
Now the treatment is a slight variation of the previous item including a few small
tricks. In fact,
\begin{IEEEeqnarray*}{rCcCl}
  h_1^2\|\gancho_{\xi_1}\bu_s\|^2_{L^2(E)^3}&=&
  h_1^2\sum_{i=1}^3 \|\gancho_{\xi_1}u_{s,i}\|^2_{L^2(E)}&\leqslant&
    \textit{h}^2\sum_{i=1}^3 \|r^{1-\mu}\gancho_{\xi_1}u_{s,i}\|^2_{L^2(E)}\\
\IEEEeqnarraymulticol{5}{r}{
\begin{IEEEeqnarraybox}{rCl}
\qquad&\leqslant&
    \textit{h}^2\left(
     \sum_{i=1}^2 \|R^{1-\mu}\theta^{1-\mu}\gancho_{\xi_1}u_{s,i}\|^2_{L^2(E)}+
     \|R^{1-\mu}\gancho_{\xi_1}u_{s,3}\|^2_{L^2(E)}\right)\\[7pt]
  &\leqslant&
    \textit{h}^2\left(
     \sum_{i=1}^2 \|R^{1-\nu}\theta^{1-\mu}\gancho_{\xi_1}u_{s,i}\|^2_{L^2(E)}+
     \|R^{1-\nu}\gancho_{\xi_1}u_{s,3}\|^2_{L^2(E)}\right)\\[7pt]
  &\leqslant&
    \textit{h}^2\left(
    \sum_{i=1}^2 \|R^{\beta}\theta^{\delta}\gancho_{\xi_1}u_{s,i}\|^2_{L^2(E)}+
     \|R^{\beta}\gancho_{\xi_1}u_{s,3}\|^2_{L^2(E)}\right)\\[7pt]
  &\leqslant&
    \textit{h}^2 \|\bu_{s}\|^2_{\pazocal{V}_{\beta,\delta}(E)}.
\end{IEEEeqnarraybox}
}
\end{IEEEeqnarray*}
Now, since $h_2\sim h_1$, with the same argument,
\begin{IEEEeqnarray*}{rCl}
  h_2\|\gancho_{\xi_2}\bu_s\|_{L^2(E)^3}&=&
  h_2\sum_{i=1}^3 \|\gancho_{\xi_2}u_{s,i}\|\,\leqslant\,
    \textit{h}
    \|\bu_{s}\|_{\pazocal{V}_{\beta,\delta}(E)}.
\end{IEEEeqnarray*}
For the derivative with respect to $\xi_3$ in~\eqref{auxlabel9},
\begin{IEEEeqnarray*}{rCl}
  h_3^2\|\gancho_{\xi_3}\bu_s\|^2_{L^2(E)^3}&=&
  h_3^2\sum_{i=1}^3 \|\gancho_{\xi_3}u_{s,i}\|^2_{L^2(E)}\\
\IEEEeqnarraymulticol{3}{R}{
\begin{IEEEeqnarraybox*}{rCl}
\qquad&=&h_3^2\left(\sum_{i=1}^2\|\gancho_{\xi_i}u_{s,3}\|^2_{L^2(E)}+
    \|\gancho_{\xi_3}u_{s,3}\|^2_{L^2(E)}\right)\\[7pt]
  &\leqslant&\textit{h}^2\left(\sum_{i=1}^2\|d(E,v)^{1-\nu}\gancho_{\xi_i}u_{s,3}\|^2_{L^2(E)}
      +\|d(E,v)^{1-\nu}\gancho_{\xi_3}u_{s,3}\|^2_{L^2(E)}\right)\\[7pt]
  &\leqslant&
  \textit{h}^2\left(\sum_{i=1}^2\|R^{\beta}\gancho_{\xi_i}u_{s,3}\|^2_{L^2(E)}+
    \|R^{\beta}\gancho_{\xi_3}u_{s,3}\|^2_{L^2(E)}\right)\\[7pt]
  &=&
  \textit{h}^2\sum_{i=1}^3\|R^{\beta}\gancho_{\xi_i}u_{s,3}\|^2_{L^2(E)}\\[7pt]
  &\leqslant&
  3\textit{h}^2\|u_{s,3}\|^2_{\scriptscriptstyle V^{1,2}_{\beta,0}(E)}.
\end{IEEEeqnarraybox*} 
}
\end{IEEEeqnarray*}
The divergence term in~\eqref{auxlabel9} goes like in~(\ref{aux_label64}).
\end{enumerate}
\paragraph{Terms in~\eqref{auxlabel350} such that $d(E,\be_\ell)=0$.}
In this case we have anisotropic prisms and tetrahedra from an isotropic 
subfamily, one in each tetrahedral macro--element of the present type.

From~\eqref{auxlabel350}, after triangle inequality, we write
\begin{IEEEeqnarray}{rCl} %
  \IEEEeqnarraymulticol{3}{L}{\nonumber
    \sum_{d(E,\be) = 0}
    \|\bu_s - \br_0\bu_s\|^2_{\scriptscriptstyle L^2(E)^3}
    \,\lesssim\,}
  \\
  \IEEEeqnarraymulticol{3}{R}{
  \label{distancia_cero_arista}
  \,\lesssim\,\|\br_0\bu_s\|^2_{\scriptscriptstyle L^2(T_\ell)^3}
  \,+\sum_{
  \substack{{\small{\text{prisms\,}}P}\\
          d(P,\be) = 0}} \|\br_0\bu_s\|^2_{\scriptscriptstyle L^2(P)^3}
  \,+\sum_{d(E,\be) = 0} \|\bu_s\|^2_{\scriptscriptstyle L^2(E)^3}}.
\end{IEEEeqnarray}
Here we put $T_\ell$ for the mentioned tetrahedron and
$E$ for generic elements in $\Lambda_\ell$.
Now let us bound each term.%% on the right of~(\ref{distancia_cero_arista}).
\begin{enumerate}
  \item 
  For the last term on the right of~(\ref{distancia_cero_arista}), let
$E$ be any of both the tetrahedron with the
singular
 vertex or a prism. If $E$ is a prism, recall in this case we have
 $d(E,\be) = 0$ and $d(E,\bv) > 0$. So
\begin{IEEEeqnarray*}{rCl}
   \|\bu_s\|_{L^2(E)^3} &=&
    \sum_{i=1}^2 \|R^\nu\theta^\mu R^{-\nu}\theta^{-\mu} u_{s,i}\|_{0,E}
    + \|R^\nu\theta^{-1}R^{-\nu}\theta u_{s,3}\|_{0,E}\\[7pt]
  &\leqslant&\|R^\nu\theta^\mu\|_{L^\infty(E)}
  \sum_{i=1}^2 \|R^{-\nu}\theta^{-\mu} u_{s,i}\|_{0,E}\\[4pt]
\IEEEeqnarraymulticol{3}{r}{
    +\,\|R^\nu\theta\|_{\infty,E}
    \|R^{-\nu}\theta^{-1} u_{s,3}\|_{0,E}.
}
\end{IEEEeqnarray*}
Since $\theta < 1$ and $\mu \leqslant \nu < 1$, then
\begin{IEEEeqnarray}{rClClCr}
  \label{cota_pesos}
  R(\bx)^\nu\theta(\bx)&\leqslant&
  R(\bx)^\nu\theta(\bx)^\mu&\leqslant&R(\bx)^\mu\theta(\bx)^\mu&=&
  r(\bx)^\mu.
\end{IEEEeqnarray}
Using this we get
\begin{IEEEeqnarray*}{rCl}
  \|\bu_s\|_{0,E}&\leqslant&\|r^\mu\|_{\infty,E}
  \left(
    \sum_{i=1}^2 \|R^{-\nu}\theta^{-\mu}u_{s,i}\|_{0,E}
    + \|R^{-\nu}\theta^{-1}u_{s,3}\|_{0,E}
  \right)\\[7pt]
  &\leqslant&\max\{h_1,h_2\}^\mu
  \left(
    \sum_{i=1}^2 \|R^{-\nu}\theta^{-\mu}u_{s,i}\|_{0,E}
    + \|R^{-\nu}\theta^{-1}u_{s,3}\|_{0,E}
  \right)\\[7pt]
  &\lesssim&\textit{h}
    \|\bu_{s}\|_{\pazocal{V}_{\beta,\delta}(E)}
\end{IEEEeqnarray*}
provided we take $\beta\sim 1-\nu$ and $\delta\sim 1-\mu$.
\item For the first term in~\eqref{distancia_cero_arista},
although, as we said, the tetrahedra touching the vertex remain regular, 
we estimate with 
the weighted norms anisotropically for the sake of completeness and generality.
Recall $T_\ell$ also has an edge
contained in the singular edge of $\Lambda_\ell$. %${\color{blue}\|\br_0\bu_s\|^2_{\scriptscriptstyle L^2(Tetra)^3}}$.
We work for the first component of the local interpolate in the $L^1(T_\ell)$ norm
using~(\ref{normaL2L1}).
\begin{IEEEeqnarray*}{rCl}
  \|(\br_0\bu_s)_1\|_{\scriptscriptstyle L^2(T_\ell)}
    & \leqslant & C|T_\ell|^{-1/2} \|(\br_0\bu_s)_1\|_{\scriptscriptstyle L^1(T_\ell)}.
\end{IEEEeqnarray*}
Remember that by Lemma~\ref{well_defined_dofs} we know $\bu_s$ is in 
$W^{1,1}(\Lambda_\ell)^3$. We start pulling $\br_0\bu_s$ back to a rescaled
reference tetrahedron $\tilde{T}$ 
with the mapping 
$F_{T_\ell}(\tilde{\bx}) = M_{T_\ell}\tilde{\bx} + \bx_{T_\ell}$
(cfr. Figure~\ref{rescaled_tetra}). 
\rescaledTetraTikz

Using Lemma 5.22 in page 123 of~\cite{monk},
which gives the analog of our statement in~(\ref{div_interp_commutes}) for 
tetrahedra, and then using Proposition 3.4 of~\cite{aadl}, we get
\begin{IEEEeqnarray*}{rCl}
  \|(\br_0\bu_s)_1\|_{L^1(T_\ell)} &=& \int_{\tilde{T}}|(M_{T_\ell}\tilde{\br}_0\tilde{\bu}_s)_1|\,d\tilde{\bx}\\[7pt]
    &\leqslant& C\,\bigg\{ \sum_{1\leqslant i\leqslant 3} \|\tilde{u}_{s,i}\|_{L^1(\tilde{T})} + 
      \sum_{1\leqslant j\leqslant 3} h_j\|\partial_j\tilde{u}_{s,i}\|_{L^1(\tilde{T})}\\
      \IEEEeqnarraymulticol{3}{r}{ + \,
             h_i\|\dvg \tilde{\bu}_s\|_{L^1(\tilde{T})}\bigg\}} \\[7pt] %%% & = & \ldots \text{paso al tetraedro f\'isico} \ldots \\[7pt] % VER BORRADOR INTERPOLACION PIRAMIDE
    & \leqslant & C\,
    \bigg\{\|\bu_{s}\|_{L^1(T_\ell)^3} + 
    \sum_{1\leqslant j \leqslant 3} h_j\|\partial_{\xi_j}\bu_{s}\|_{L^1(T_\ell)^3}\\
    \IEEEeqnarraymulticol{3}{r}{
    + (h_1+h_2+h_3) \|\dvg \bu_s\|_{L^1(T_\ell)}\bigg\}}\\[3pt]
&&\yesnumber\label{aux_label60}
\end{IEEEeqnarray*}
where $C$ depends only on the maximum angle of $T_\ell$. 
Now we estimate each term on the right hand side of~(\ref{aux_label60}).
\begin{enumerate}
  \item[(2a)] By H\"older's inequality
\begin{IEEEeqnarray*}{rCl}
  \|\bu_{s}\|_{L^1(T_\ell)^3} &\leqslant&
\sum_{i=1,2}\|R^{\nu}\theta^{\mu}\|_{L^2(T_\ell)}\|R^{-\nu}\theta^{-\mu}u_{s,i}\|_{L^2(T_\ell)}\\ %%&\leqslant&\|R^\nu\theta^{\mu}\|_\infty|T_\ell|^{\nicefrac12}\sum_{i=1,2}\|\ldots u_{s,i}\|_{L^2(T\ell)} + \|R^\nu\theta\|_\infty|T_\ell|^{\nicefrac12}\|\ldots u_{s,3}\|_{L^2(T\ell)}\\[7pt]
 & &\qquad+\,\|R^{\nu}\theta\|_{L^2(T_\ell)}\|R^{-\nu}\theta^{-1}u_{s,3}\|_{L^2(T_\ell)}
\end{IEEEeqnarray*}
and by~(\ref{cota_pesos}) and the grading criterion~(\ref{label_grading}) we have                                        %&\leqslant&(\textit{h}^{1/\mu})^\mu|T_\ell|^{\nicefrac12}\sum_{i=1,2}\|R^{-\nu}\theta^{-\mu}u_{s,i}\|_{L^{2}(T_\ell)} +\|R^{-\nu}\theta^{-1} u_{s,3}\|_{L^{2}(T_\ell)}\\[7pt]
\begin{IEEEeqnarray*}{rCl} 
\|\bu_{s}\|_{L^1(T_\ell)^3} &\leqslant&(\textit{h}^{1/\mu})^\mu|T_\ell|^{\nicefrac12}
\|\bu_s\|_{V_{\beta,\delta}^{1,2}(T_\ell)^2\times V_{\beta,0}^{1,2}(T_\ell)}\\
&=&\textit{h}\,|T_\ell|^{\nicefrac12}\|\bu_s\|_{\pazocal{V}_{\beta,\delta}(T_\ell)}.
\end{IEEEeqnarray*}
  \item[(2b)]
Take $1\leqslant j,i \leqslant 3$. Then %%% TODO (HACER APARTE PARA $i=3$; creo que es un poco diferente, como pas\'o antes):
\begin{IEEEeqnarray*}{rCl}
  h_j\|\partial_{\xi_j} u_{s,i}\|_{L^1(T_\ell)} & \leqslant &
    h_j\|R^{\nu-1}\theta^{\mu-1}\|_{0,T_\ell}\|R^{1-\nu}\theta^{1-\mu}\partial_{\xi_j} u_{s,i}\|_{0,T_\ell}.
\end{IEEEeqnarray*}
As it holds $0\leqslant1-\nu\leqslant1-\mu<1$, we get 
\[
  R(\bx)^{1-\nu}\geqslant R(\bx)^{1-\mu}
\]
and
\[
  R(\bx)^{\nu-1}\theta(\bx)^{\mu-1}\leqslant
  R(\bx)^{\mu-1}\theta(\bx)^{\mu-1} = r(\bx)^{\mu-1}.
\]
From this we will use that
\[
  \|R^{\nu-1}\theta^{\mu-1}\|_{L^2(T_\ell)} \leqslant \|r^{\mu-1}\|_{L^2(T_\ell)}.
\]
Now, calculating the right hand side integrating in a cylindrical section,
\[
  \|r^{\mu-1}\|_{L^2(T_\ell)} = \sqrt{\tfrac{\pi}{4\mu}}\sqrt{h_3}\,h_1^{\mu}\sim \sqrt{h_3}\,\textit{h}.
\]
Then
\begin{IEEEeqnarray*}{rCl}
  h_j\|r^{\mu-1}\|_{L^2} & \lesssim & \sqrt{h_1h_2h_3}\,\textit{h}\\[7pt]
    & \sim & \sqrt{|T_\ell|}\,\textit{h}
\end{IEEEeqnarray*}
and then
\begin{IEEEeqnarray}{rCl}
\nonumber
  h_j\|\partial_{\xi_j} u_{s,i}\|_{L^1(T_\ell)} & \lesssim &
    \textit{h}\,\sqrt{|T_\ell|}\,\|R^{1-\nu}\theta^{1-\mu}\partial_{\xi_j} u_{s,i}\|_{L^2(T_\ell)}\\
\label{cuentita_integral}
    &\leqslant& \textit{h}\,\sqrt{|T_\ell|}\,\|u_{s,i}\|_{\scriptscriptstyle V^{1,2}_{\beta,\delta}(T_\ell)}.
\end{IEEEeqnarray}
\item[(2c)]\label{aux_label65} For the divergence term in~(\ref{aux_label60}):
\begin{IEEEeqnarray*}{rClCr}
  (h_1+h_2+h_3) \|\dvg \bu_s\|_{L^1(T_\ell)} & \leqslant &
    (2\textit{h}^{1/\mu} + \textit{h}^{1/\nu}) \|\dvg \bu_s\|_{L^1(T_\ell)} \\[7pt]
\IEEEeqnarraymulticol{3}{r}{
  \begin{IEEEeqnarraybox*}{rCl}
    & \leqslant & 3\,\textit{h}\,\sqrt{|T_\ell|}\,\|\dvg \bu_s\|_{L^2(T_\ell)} 
    \\[7pt]
    & \leqslant & 3\,\textit{h}\,\sqrt{|T_\ell|}\,\left\{\|\dv\bu\|_{L^2(T_\ell)}
     + \|\dv\bu_r\|_{L^2(T_\ell)}\right\}.
  \end{IEEEeqnarraybox*}
}\\&&
\yesnumber\label{aux_label66}
\end{IEEEeqnarray*}
\end{enumerate}
\item 
Now prisms, the middle term on the right hand of~(\ref{distancia_cero_arista}). Remember
the difference with the tetrahedron is that $d(P,\bv_\ell) \gtrsim h_1$.
We start pulling back from $P$ to the rescaled prism in Figure~\ref{rescaled_prism}
called here $\tilde P$.   
By the stability estimate in Theorem~\ref{thmStabilityKtildeRT}, 
\begin{IEEEeqnarray*}{rCl}
  \|\br_0\bu_s\|_{\scriptscriptstyle L^1(P)^3} & \leqslant & 
    \|M_P\|_\infty \|\tilde{\br}_0\tilde{\bu}_s\|_{L^1({\tilde{P}})^3} \\ [7pt]
\IEEEeqnarraymulticol{3}{r}{
\begin{IEEEeqnarraybox*}{rCl}
 \qquad&\lesssim& \left\| \tilde{\bu}_s \right\|_{L^1(\tilde{P})^3}
    + \sum_{j=1}^3 h_j \left\| \partial_{\tilde{x}_j}\tilde{\bu}_s \right\|_{L^1(\tilde{P})^3}
    + h_{\tilde{P}}\left\|\dv(\tilde{u}_{s,1}, \tilde{u}_{s,2}, 0) \,\right\|_{L^1(\tilde{P})}\\[7pt]
  &\lesssim& \left\| \bu_s \right\|_{L^1(P)^3}
    + \sum_{j=1}^3 h_j \left\| \partial_{\xi_j}\bu_s \right\|_{L^1(P)^3}
    + h_{P}\left\|\dv(u_{s,1}, u_{s,2}, 0) \,\right\|_{L^1(P)}\mbox{,}    
\end{IEEEeqnarraybox*}
}\\[4pt]
&&\yesnumber\label{auxlabel10}
\end{IEEEeqnarray*}
by the same argument as in~(\ref{aux_label15}) for 
elements touching the singular edge in the prismatic macro--element. Now
to estimate each term, again for $i=1$ or $2$ we have
\begin{IEEEeqnarray*}{rCcCl}
  \|u_{s,i}\|_{L^1(P)} & \leqslant & 
    \|R^{\nu}\theta^{\mu}\|_{0,P} \|R^{-\nu}\theta^{-\mu}u_{s,i}\|_{0,P}
      & \leqslant & \textit{h}\,|P|^{\nicefrac12} \|R^{-\nu}\theta^{-\mu}u_{s,i}\|_{0,P}
\end{IEEEeqnarray*}
and
\begin{IEEEeqnarray*}{rCcCl}
  \|u_{s,3}\|_{L^1(P)} & \leqslant & \|R^{\nu}\theta\|_{0,P} \|R^{-\nu}\theta^{-1}u_{s,3}\|_{0,P}
  & \leqslant & \textit{h}\,|P|^{\nicefrac12} \|R^{-\nu}\theta^{-1}u_{s,3}\|_{0,P}.
\end{IEEEeqnarray*}
Next fix $j = 1$ or $2$.         %\noindent DErivatives: $j = 1$ or $2$.\\
If $i=1$ or $2$, with the same argument as in~(\ref{cuentita_integral}), we state that
\begin{IEEEeqnarray}{rCl}
  h_j\|\partial_{\xi_j} u_{s,i}\|_{L^1(P)} & \lesssim &
    \textit{h}\,\sqrt{|P|}\,\|R^{1-\nu}\theta^{1-\mu}\partial_{\xi_j} u_{s,i}\|_{0,P}.
\end{IEEEeqnarray}
For the derivatives of $u_{s,3}$ it holds that 
\begin{IEEEeqnarray*}{rCl}
  h_j\|\partial_{\xi_j}u_{s,3}\|_{L^1(P)} &\leqslant&
    h_j \|R^{\nu-1}\|_{0,P}\|R^{1-\nu}\partial_{\xi_j}u_{s,3}\|_{0,P}\\[7pt]
  &\leqslant& h_j \|r^{\mu-1}\|_{0,P}\|R^{1-\nu}\partial_{\xi_j}u_{s,3}\|_{0,P}\\[7pt]
  &\lesssim& \textit{h}\,\sqrt{|P|}\,\|R^{1-\nu}\partial_{\xi_j}u_{s,3}\|_{0,P}.
\end{IEEEeqnarray*}
Now, for the  derivative along the direction of the singular edge 
$\partial_{\xi_3}(\cdot)$, remember in this case it is 
$h_3\sim\textit{h}\,d(P,\bv)^{1-\nu}$. If we take first and second components,
\begin{IEEEeqnarray*}{rCl}
  h_3\|\partial_{\xi_3}u_{s,i}\|_{L^1(P)}& = & h_3\|\partial_{\xi_i}u_{s,3}\|_{L^1(P)}\\[7pt]
  & \leqslant & C\textit{h}\|R^{1-\nu}\partial_{\xi_i}u_{s,3}\|_{L^1(P)}\\[7pt]
  & \leqslant & C\textit{h}|P|^{\nicefrac12}\|R^{1-\nu}\partial_{\xi_i}u_{s,3}\|_{L^2(P)}.
\end{IEEEeqnarray*}
And if we take the third component,
\begin{IEEEeqnarray*}{rCl}
  h_3\|\partial_{\xi_3}u_{s,3}\|_{L^1(P)}& \sim & \textit{h}\|d(P,\bv)^{1-\nu}
    \partial_{\xi_3}u_{s,3}\|_{L^1(P)}\\[7pt]
  & \leqslant & C\textit{h}\|R^{1-\nu}\partial_{\xi_3}u_{s,3}\|_{L^1(P)}\\[7pt]
  & \leqslant & C\textit{h}|P|^{\nicefrac12}\|R^{1-\nu}\partial_{\xi_3}u_{s,3}\|_{L^2(P)}.
\end{IEEEeqnarray*}
For the divergence term in~\eqref{auxlabel10},
\begin{IEEEeqnarray*}{rCl}
  h_P\|\dv(u_{s,1},u_{s,1},0)\|_{L^1(P)}&\leqslant&h_P\|\dv\bu_s\|_{L^1(P)}
    +h_P\|\partial_{\xi_3}u_{s,3}\|_{L^1(P)}\\[7pt]
    &\lesssim&h_P\|\dv\bu_s\|_{L^1(P)} + h_3\|\partial_{\xi_3}u_{s,3}\|_{L^1(P)}\mbox{;}
\end{IEEEeqnarray*}
the estimate for $h_P\|\dv\bu_s\|_{L^1(P)}$ follows as in~(\ref{aux_label66}), 
and the estimate for $h_3\|\partial_{\xi_3}u_{s,3}\|_{L^1(P)}$ should be
\begin{IEEEeqnarray*}{rCl}
  h_3\|\partial_{\xi_3}u_{s,3}\|_{L^1(P)} & \lesssim &
  \textit{h} |P|^{\nicefrac12}\|R^{1-\nu}\partial_{\xi_3}u_{s,3}\|_{L^2(P)} \\
  &\lesssim& \textit{h} |P|^{\nicefrac12}\|u_{s,3}\|_{V_{\beta,0}(P)}.
\end{IEEEeqnarray*}
\end{enumerate}
Collecting all the estimates in items $2$ and $3$ above, using~(\ref{normaL2L1})
we bound the first and second term in~(\ref{distancia_cero_arista}), 
belonging to the case with zero distance
to the edge, in the $L^2$ norm.

\subsection{Bound for the Singular Part in a Tetrahedral Macro--Element
with Singular Vertex and no Singular Edge}
This macroelement is made up with a graded subfamily of isotropic tetrahedra.
The grading fulfills the relations~\eqref{label_grading2} and the estimates
follow repeating (in fact, simpler) arguments used in Subsection~\ref{auxlabel205}.  
%% {\color{blue}\# continue here; anotar para seguir 
%% en el remark del caso de macroel. que me marcó ariel que faltar'ia 
%% (ver mi cuaderno)}

\section{Main Approximation Error Theorem}
\label{auxlabel400}
\begin{theorem}\label{auxlabel11}
Let $\Omega$ be a polihedral domain in $\mathbb{R}^3$, let $f\in L^2(\Omega)$ and 
let $(\bu,p)$ and 
$(\bu_{\textit{h}},p_{\textit{h}})$ 
be the solutions of Problems~\ref{weakMixedContinuous}
and~\ref{mixedDiscrete} respectively. There exists a family of anisotropic meshes
$\{\Th\}_{{\textit{h}}\to 0}\,$
made up of
prisms, tetrahedra and pyramids,  graded according to~\eqref{label_grading3}--\eqref{label_grading} 
for which 
\begin{IEEEeqnarray*}{rCl}
  \|\bu-\bu_{\textit{h}}\|_{0,\Omega}&\leqslant &C {\textit{h}} \|f\|_{0,\Omega}\\[5pt]
  \|p-p_{\textit{h}}\|_{0,\Omega}&\leqslant &C \textit{h} \|f\|_{0,\Omega}
\end{IEEEeqnarray*}
with $\textit{h}$ being smaller than  $C N_{\textit{h}}^{\nicefrac{-1}{3}}$, where
$N_{\textit{h}}$ is the  number of elements in $\Th$.
\end{theorem}
\begin{proof}
Recall from inequality~(\ref{aux_label48})  for every $\bw\in W(\Th)$ it holds
\begin{IEEEeqnarray*}{rCl}
  \|\bu-\bu_h\|_{L^2(\Omega)^3} &\leqslant& C\{\|\bu-\br_0\bu\|_{L^2(\Omega)^3} + \|\bu-\bw\|_{L^2(\Omega)^3}\}.
\end{IEEEeqnarray*}
For each $\bw$ we decompose the norm with the macro--elements as follows
\begin{IEEEeqnarray*}{rCl}
  \|\bu-\bw\|^2_{0,\Omega} &=& \sum_{\ell=1}^{N_{\textit{h}_0}} \|\bu-\bw\|^2_{0,\Lambda_\ell}\\
\IEEEeqnarraymulticol{3}{r}{
    =\,\sum_{\ell=1}^{N_{\textit{h}_0}} \left\{
\sum_{\substack{{\scriptscriptstyle{\text{prisms\,}}E_\ell}\\E_\ell\subseteq\Lambda_\ell}}
      \|\bu-\bw\|^2_{\scriptscriptstyle 0,E_\ell} +
\sum_{\substack{{\scriptscriptstyle{\text{tetrahedra\,}}E_\ell}\\E_\ell\subseteq\Lambda_\ell}}
      \|\bu-\bw\|^2_{\scriptscriptstyle 0,E_\ell} +
\sum_{\substack{{\scriptscriptstyle{\text{pyramids\,}}E_\ell}\\E_\ell\subseteq\Lambda_\ell}}
      \|\bu-\bw\|^2_{\scriptscriptstyle 0,E_\ell}
    \right\} 
}
\end{IEEEeqnarray*}
%%=========================================================================
%For each macro--element $\Lambda_\ell$, for each pyramid $E_\ell\in\Lambda_\ell$,
%\begin{IEEEeqnarray*}{rCl}
%  \|\bu_s - \pi\bu_s\|^2_{\sss L^2(E)}&\leqslant&
%  h_1^2\|\gancho_{\xi_1}\bu_s\|^2+
%  h_2^2\|\gancho_{\eta_2}\bu_s\|^2+
%  h_3^2\|\gancho_{\xi_3}\bu_s\|^2+
%  h_K^2\|\dvg\bu_s\|^2\\[7pt]
%\end{IEEEeqnarray*}
%\begin{IEEEeqnarray*}{rCl}
%  h_1\|\gancho_{\xi_1}\bu_s\|&=&
%  h_1\sum_{i=1}^3 \|\gancho_{\xi_1}\bu_{s,i}\|\\
%  &\leqslant&
%    \textit{h}\sum_{i=1}^3 \|r^{1-\mu}\gancho_{\xi_1}\bu_{s,i}\|\\
%  &\leqslant&
%    \textit{h}\left(
%     \sum_{i=1}^2 \|R^{1-\mu}\theta^{1-\mu}\gancho_{\xi_1}\bu_{s,i}\|+
%     \|R^{1-\mu}\gancho_{\xi_1}\bu_{s,3}\|\right)\\[7pt]
%  &\leqslant&
%    \textit{h}\left(
%     \sum_{i=1}^2 \|R^{1-\nu}\theta^{1-\mu}\gancho_{\xi_1}\textbf{u}_{s,i}\|+
%     \|R^{1-\nu}\gancho_{\xi_1}\textbf{u}_{s,3}\|\right)\\[7pt]
%  &\leqslant&
%    \textit{h}\left(
%    \sum_{i=1}^2 \|R^{\beta}\theta^{\delta}\gancho_{\xi_1}\textbf{u}_{s,i}\|+
%     \|R^{\beta}\gancho_{\xi_1}\textbf{u}_{s,3}\|\right)\\[7pt]
%  &\leqslant&
%    \textit{h}
%    \left(\sum_{i=1}^2 \|\textbf{u}_{s,i}\|_{\scriptscriptstyle \beta,\delta}+
%         \|\textbf{u}_{s,3}\|_{\scriptscriptstyle \beta,0}\right).
%\end{IEEEeqnarray*}
%%=========================================================================
and then take $\bw_{\bu}$ as in Definition~\ref{aux_label51}.
For the sum term over the pyr\-amids (remember pyr\-amids do not touch singularities)
we have, restricting $\bw_{\bu}$, by the inequality we proved
in~(\ref{aux_label63})  for pyramids,
\begin{IEEEeqnarray*}{rCl}
  \|\bu-\bw_{\bu}\|^2_{0,E_\ell} & = & \|\bu-P_{0,E_{\ell}}\bu\|^2_{0,E_\ell}\\[4pt]
                           & \leqslant & C \left\{h_E|\bu_r|_{H^1(E_\ell)^3} 
                                + h_E|\bu_s|_{H^1(E_\ell)^3}\right\}\\[4pt]
                           & \leqslant & C\textit{h}\left\{ |\bu_r|_{H^1(E_\ell)^3} 
                              + \|\bu_s\|_{\pazocal{V}_{\beta,\delta}(E)}\right\}.
\end{IEEEeqnarray*}
With this and all the others for prisms and tetrahedra within the proof of the
interpolation Theorem~\ref{interpolation_theorem} we have
\begin{IEEEeqnarray*}{rCl}
  \|\bu-\bw_{\bu}\|^2_{0,\Omega}
    &=& \sum_{\ell=1}^{N_{\textit{h}_0}} \Bigg\{
\sum_{\substack{{\scriptscriptstyle{\text{prisms\,}}E_\ell}\\E_\ell\subseteq\Lambda_\ell}}
      \|\bu-\br_0\bu\|^2_{\scriptscriptstyle 0,E_\ell} +
\sum_{\substack{{\scriptscriptstyle{\text{tetrahedra\,}}E_\ell}\\E_\ell\subseteq\Lambda_\ell}}
      \|\bu-\br_0\bu\|^2_{\scriptscriptstyle 0,E_\ell}\\[7pt] 
\IEEEeqnarraymulticol{3}{R}{
      +\,
\sum_{\substack{{\scriptscriptstyle{\text{pyramids\,}}E_\ell}\\E_\ell\subseteq\Lambda_\ell}}
      \|\bu-P_{0,E_{\ell}}\bu\|^2_{\scriptscriptstyle 0,E_\ell}
    \Bigg\}
}\\[5pt]
\yesnumber\label{aux_label62}
    &\leqslant& C \sum_{\ell=1}^{N_{\textit{h}_0}}
      \textit{h}^2 \|f\|^2_{0,\Omega} \, \leqslant \,  C\,\textit{h}^2 \|f\|^2_{0,\Omega}.
\end{IEEEeqnarray*}
Note $C$ depends on the cardinal of the initial mesh.
Now by Theorem~\ref{interpolation_theorem} and~(\ref{aux_label62}) we have, for 
the vectorial variable, 
$\|\bu-\bu_h\|_{L^2(\Omega)^3} \leqslant C\textit{h}\|f\|_{L^2(\Omega)}$.
For the scalar variable estimate we may do
\begin{IEEEeqnarray*}{rCl}
  \|p-p_{\textit h}\|_{0,\Omega} &\leqslant& \|p - P_0 p\|_{0,\Omega} + \|P_0 p - p_{\textit h}\|_{0,\Omega}
\end{IEEEeqnarray*}
and then apply Theorem~\ref{interpolation_theorem} to the first term and 
Theorem~\ref{aux_label47} to the second and reuse the error 
estimate for the vectorial variable.
\end{proof}

