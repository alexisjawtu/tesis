\section{Discrete Well Posedness of a Model Elliptic Problem} % (fold)
\label{sec:well_posedness}
If we consider the following kernel
\begin{IEEEeqnarray*}{rCl}
  \Kb_{\Th} & = & \{\bv_h\in V_h: b(\bv_h,q)=0\,\,
                        \mbox{for all } q\in Q_h\} \\[4pt]
               & = & V_{\textit{h}}\cap\ker\dv
\end{IEEEeqnarray*}
then Lemma~\ref{lemma_for_coercivity} implies the following Proposition.
\begin{proposition}
  The form $a_h$ in~(\ref{discreteGlobal_a})  is coercive over $\Kb_{\Th}$
  and the coercivity constant
  depends only on the aspect ratio of the pyramids of the mesh.
\end{proposition}
\begin{proposition} \label{cont} For every $E\in\Th$ the local discrete bilinear form $a_h^E$
in~(\ref{discreteLocal_a})
is continuous en $L^2(E)$. That is,  for all $\bu$,$\bv\in V_h(E)$
\[
  a_h^E(\bu,\bv) \leqslant C \|\bu\|_{L^2(E)}\|\bv\|_{L^2(E)}\mbox{,}
\]
where $C$ equals $1$ when $E$ is a right prism or tetrahedron, and
depends only on the aspect ratio of $E$ in the case of pyramids.
\end{proposition}
\begin{proof}
When $E$ is a prism or tetrahedron, $a_h^E(\bu,\bv)=a^E(\bu,\bv)$ for $\bu$ 
and $\bv$ in $V_h(E)$, and the result is immediate. 
If $E$ is a pyramid, as $a_h^E$ is symmetric, the coercivity arising from~\eqref{L2}
implies that $a_h^E$ defines an inner product. 
Hence by Lemma~\ref{lemma_for_coercivity} we have
\begin{IEEEeqnarray*}{rClCr}
  \left|a_h^E(\bu,\bv)\right| & \leqslant &      a_h^E(\bu,\bu)^\frac12 a_h^E(\bv,\bv)^\frac12 & \leqslant &\\ 
                              & \leqslant & C_E\,a^E(\bu,\bu)^\frac12 a^E(\bv,\bv)^\frac12     & \leqslant & C_E\|\bu\|_{L^2(E)}\|\bv\|_{L^2(E)}.
\end{IEEEeqnarray*}
\end{proof}
\begin{lemma} \label{lemma_inf_sup_bh} There exists a constant
$\beta^*>0$ depending only on $\Omega$  and the maximum aspect
ratio of the pyramids of $\Th$ 
such that for all $q^*\in Q_h$ there exists $\bw_h^*\in V_h$ such
that
\[
\dv\bw_h^*=q^* \mbox{\quad and\quad} \beta^*\|\bw_h^*\|_{Q}\leqslant\|q^*\|_{Q}.
\]
\end{lemma}
\begin{proof} Start considering the infinite dimensional version
of the statement. There is, in fact, a constant $\beta^*$ depending only on
$\Omega$ such that
for every $q^*\in Q_h \subseteq Q$ there exists 
$\bw^*\in H^1_0(\Omega)^3$ with $\dv\bw^*=q^*$ such that
\begin{IEEEeqnarray}{rCl} \label{bound_w}
  \beta^*\|\bw^*\|_{H^1(\Omega)} & \leqslant &
  \|q^*\|_{Q}.
\end{IEEEeqnarray}
We refer the reader to~\cite{ricardoMixed}.
Now, for each $q^*\in Q_h$ take $\bw_h^*$ such that,
in every $E\in\Th$, $\bw_h^*|_E := I(\bw^*|_E)$, as defined in
Corollary~\ref{interpolant}. As a consequence of Proposition~\ref{vem_equal_fem}
and because we are considering the exact same degrees of freedom, this
interpolation operator coincides, in the lowest order case, with the
$H(\Div)$--conforming operators in Definitions~\ref{sub:definition_of_the_h_div_element_on_prisms}
and~\ref{defi_face_element_tetra}. 

For prismatic elements we use the estimate in Remark~\ref{auxlabel4} which, together
with the anisotropic rescalings  used in Theorem~\ref{aux_label46}, applies 
to any right prism, and Theorem $3.1$ in page $149$ of~\cite{aadl} for tetrahedral elements.
In the case of a pyramidal element we draw upon estimate~(\ref{estab2}). With
all these together and~(\ref{bound_w}) for all cases we get
\[
  \|\bw_{\textit{h}}^*\|_{Q} =
  \|I\bw^*\|_{Q}\leqslant
  C(1+\textit{h})\|\bw^*\|_{H^1(\Omega)}
  \leqslant\frac{C(1+\textit{h})}{\beta^*}\|q^*\|_{Q}.
\]
Besides, since $q^*\in Q_h$, by~(\ref{p0_projection}) we have
\[
  \dv I\bw^*=P_0\,\dv\bw^*=P_0\,q^*=q^*.
\]
\end{proof}
Now we can prove the discrete inf-sup condition for the $b_{\textit h}$ form 
in~(\ref{aux_label44}).
\begin{theorem}\label{inf_sup_b_h}
Consider the bilinear form $b_{\textit{h}}$ in~(\ref{aux_label44}).  
There exists $\beta > 0$ such that for all $q^*\in Q_h$ 
\begin{IEEEeqnarray}{rCl}\label{discrete_inf_sup_b} 
  \sup_{0\ne\bv\in V_h} \frac{b_h(\bv,q^*)}{\|\bv\|_{V_h}} &\geqslant& \beta\|q^*\|_{Q_h}.
\end{IEEEeqnarray}
\end{theorem}
\begin{proof} By Lemma~\ref{lemma_inf_sup_bh} for $q^*\in Q_h$
there exists $\bw_h^*\in V_h$ such that $\dv\bw_h^*=q^*$ and
$\beta^*\|\bw_h^*\|_{L^2(\Omega)}\leqslant\|q^*\|_{Q_h}$ (the constant $\beta^*$
is independent of $q^*$). Then
\begin{IEEEeqnarray*}{rcCcl}
  \|\bw_h^*\|_{V_h}^2 \, & \, = \, & \, \|\bw_h^*\|_{L^2(\Omega)}^2 + \|q^*\|_{Q_h}^2 
    \, & \,\leqslant\, & \, \left(\frac1{(\beta^*)^2}+1\right) \|q^*\|_{Q_h}^2
\end{IEEEeqnarray*}
and
\begin{IEEEeqnarray*}{rcCcl}
\sup_{0\ne\bv\in V_h} \frac{b_h(\bv,q^*)}{\|\bv\|_{V_h}}
      \,&\,\geqslant\,&\,
\frac{b_h(\bw_h^*,q^*)}{\|\bw_h^*\|_{V_h}}
      \,&\,\geqslant\,&\,
\frac1{\sqrt{\dfrac1{(\beta^*)^2}+1}}\,\|q^*\|_{Q_h}
\end{IEEEeqnarray*}
as we wanted.
\end{proof}
\begin{theorem} Problem~(\ref{mixedDiscrete}) has a unique solution.
\end{theorem}
\begin{proof}
  Theorem 5.2 of~\cite{ricardoMixed} states that 
  Problem~(\ref{mixedDiscrete}) has a unique solution
  provided~(\ref{discrete_inf_sup_b})  holds and 
  that there exists some $\alpha>0$ such that for all $\bu\in\Kb_{\Th}$
\begin{IEEEeqnarray*}{rCl}\label{discrete_inf_sup_a} 
  \sup_{0\ne\bv\in \Kb_{\Th}}
  \frac{a_{\textit{h}}(\bu,\bv)}{\|\bv\|_{V_h}} &\geqslant& \alpha\|\bu\|_{V_h}
\end{IEEEeqnarray*}
  but this is implied by the coercivity of $a_{\textit{h}}$
since
\begin{IEEEeqnarray*}{rCcCl}
  \sup_{0\ne\bv\in \Kb_{\Th}}
  \frac{a_{\textit{h}}(\bu,\bv)}{\|\bv\|_{V_h}}
  & \geqslant &
  \frac{a_{\textit{h}}(\bu,\bu)}{\|\bu\|_{V_h}}
  & \geqslant & c\|\bu\|_{V_h}.
\end{IEEEeqnarray*}
\end{proof}
We introduce the following global interpolation operator.
\begin{defi}\label{aux_label52}
  Given a positive
  parameter $\textit{h}$ which tends to zero and is strictly decreasing
  and a conforming mesh $\Th$ 
  made up of prisms, pyramids and tetrahedra,
  let $\rZerou$  be the \emph{global interpolation operator}
  \begin{IEEEeqnarray}{rCl}\label{global_interpolator}
    \rZerou & : & W^{1,1}(\Omega) \to V_{\textit{h}}
  \end{IEEEeqnarray}
  such that, for each $\{{E}:E\in\Th\}$,
  \begin{equation*}
    (\rZerou)|_{E} = 
      \left\{
      \begin{array}{rll}
        \br_E\,[\bu|_E] & \mbox{\,as in Definition~\ref{defi_face_element}} & \mbox{if $E$ is a prism}\\[5pt]
                           I\bu    & \mbox{\,as in Corollary~\ref{interpolant}} & \mbox{if $E$ is a pyramid}\\[5pt]
        \br_E\,[\bu|_E] & \mbox{\,as in Definition~\ref{defi_face_element_tetra}} & \mbox{if $E$ is a tetrahedron,}
      \end{array}
      \right.
  \end{equation*}
  in all cases with the lowest interpolation order $k=0$. % and with consistent normal components for interelementary faces.
\end{defi}
With regard to the pyarmids we remark here that we could perform an 
analysis similar to the one based on virtual elements using the finite elements 
on pyramids 
treated in Theorems~\ref{aux_label53} and~\ref{aux_label54}. 
\begin{theorem}\label{aux_label47} The solution $(\bu_{\textit{h}},p_{\textit{h}})$
of~(\ref{mixedDiscrete}) satisfies,
for every approximation $\bw$ of $\bu$ in
$W(\Th)$,
\begin{IEEEeqnarray}{rCl}
  \label{aux_label48}
\|\bu-\bu_h\|_{L^2(\Omega)^3} &\leqslant&
  C\{\|\bu-\br_0\bu\|_{L^2(\Omega)^3} + \|\bu-\bw\|_{L^2(\Omega)^3}\} \\[5pt]
  \label{aux_label49}
\|P_{0}\,p-p_h\|_{L^2(\Omega )} &\leqslant&
  C\{\|\bu-\bu_h\|_{L^2(\Omega)^3} + \|\bu-\bw\|_{L^2(\Omega)^3}\}
\end{IEEEeqnarray} 
\end{theorem}
\begin{proof} Cfr. proof of Theorem 5.1 in~\cite{bfm}.
\end{proof}
\begin{defi}\label{aux_label51}
  Let $P_{0,E_{\ell}}$ be the projection onto the constants over $E_{\ell}$ and
  let $\bw_{\bu}$ be defined piecewise as 
  \begin{equation*}
    (\bw_{\bu})|_{E_\ell} = 
      \left\{
      \begin{array}{rll}
        \br_{E_\ell}\,[\bu|_{E_\ell}] & \mbox{if $E_\ell$ is a prism or a tetrahedron}\\[5pt]
                           P_{0,E_{\ell}}\bu    & \mbox{if $E_{\ell}$ is a pyramid}\\[5pt]
      \end{array}
      \right.
  \end{equation*}
  for every $E_\ell$
\end{defi}
\begin{remark}
The one in Definition~\ref{aux_label51} is is the approximation $\bw$ of $\bu$
piecewise in $W(E)$ that we will use on the right hand side of~(\ref{aux_label48})  and~(\ref{aux_label49}).
Observe that $\bw_{\bu_r+\bu_s} = \bw_{\bu_r} + \bw_{\bu_s}.$
\end{remark}
% section well_posedness (end)