\section{non--convex case}
\label{sec:non_convex_case}

\begin{theorem}

{\color{BrickRed} poner esto en la intro como objetivo de todo el texto que sigue y
el enunciado este, entero, ponerlo al final, en donde digo lo de la $\bw_{\bu}$ \\

no, ojo que las mallas las estoy queriendo poner antes ... pensarlo}

There exists a family of anisotropic graded meshes
$\{\mathcal{T}_{\textit{h}}\}_{{\textit{h}}\downarrow 0}\,$
made up of
Prisms, tetrahedra and pyramids 
for which 
\begin{IEEEeqnarray*}{rCl}
  \|\bu-\bu_{\textit{h}}\|_{0,\Omega}&\leqslant &C {\textit{h}} \|f\|_{0,\Omega}\\[5pt]
  \|p-p_{\textit{h}}\|_{0,\Omega}&\leqslant &C \textit{h} \|f\|_{0,\Omega}
\end{IEEEeqnarray*}
whith $\textit{h}$ being smaller than  $C N^{\nicefrac{-1}{3}}$, where
$N = N_{\textit{h}}$ is the  number of elements in $\mathcal{T}_{\textit{h}}$.
\end{theorem}
%%=========== GRADING ==============================================

{\color{BrickRed}\paragraph{TODO} % (fold)
\label{par:todo}
ver qu'e pasa con lo de distance to the vertices, distance to de 
singular vertices.

ver si se acomoda definiendo $\lambda_e, \lambda_v$ como $+\infty$ en caso de
ser arista o v'ertice regulares, y as'i poder usar las cotas con las 
distancias al \'unico v\'ertice o arista de cada macroelemento, etc...
% paragraph todo (end)
}
grading

\begin{equation}\label{label_grading}
\begin{IEEEeqnarraybox*}{lCl}
  h_1, h_2 &\sim&
    \begin{cases}
      \textit{h}^{\frac{1}{\mu}}  & \text{ if\, }d(E,\be)=0\\
      \textit{h}\,d(E,\be)^{1-\mu}  & \text{ if\, }0<d(E,\be)\lesssim1\\
      \textit{h}          & \text{ if\, }d(E,\be)\sim1
    \end{cases}\\[5pt]
  h_3   &\sim& 
    \begin{cases}
      \textit{h}^{\frac{1}{\nu}}  & \text{ if\, }d(E,v)=0\\
      \textit{h}\,d(E,v)^{1-\nu}  & \text{ if\, }0<d(E,v)\lesssim1\\
      \textit{h}          & \text{ if\, }d(E,v)\sim1
    \end{cases}
\end{IEEEeqnarraybox*}
\end{equation}
with $\mu\le 1-\delta$ for some $\delta>1-\lambda_{\be}$.\\
with $\nu\le 1-\beta$ for some $\beta>\frac12-\lambda_{\bv}$.
%%==================================================================
\begin{remark} \noindent{\color{BrickRed} ESto seguro que va, todavia no se
bien adonde.} 
We remark that if a mesh satisfies~(\ref{label_grading})
for $\mu=\mu_0$ and $\nu=\nu_0$, then it satisfies the
same for $\mu>\mu_0$ and $\nu>\nu_0$. Thus, the mesh can be overrefined by
taking $\mu=\nu=\min\{1-\delta,1-\beta\}$, and it still
verifies~(\ref{label_grading}) for
$\min\{1-\delta,1-\beta\}\le\mu\le 1-\delta$ and
$\min\{1-\delta,1-\beta\}\le\nu\le 1-\beta$. The possibility to use $\mu=\nu$
for the construction of the mesh, allows to validate condition stating
that Tetrahedra and pyramids in $\Th$ are not necessarily anisotropic.
\end{remark}

To avoid what would result in an illegible proof we decided to partition
the proof in subsections, paragraphs and items.\\

\textbf{Obs}: si $\delta(\Lambda)>1$ entonces
\[
  R_K(\bx) \leqslant \delta(\Lambda)\min\{ R_K(\bx), 1\}
\]
as\'i que asumimos $R_K(\bx)\leqslant 1$.\\[7pt]
\textbf{Obs}: $\mu < 1 \Rightarrow \mu \leqslant \nu$. \\[7pt]

\subsection{Meshes}\label{meshes}
\noindent{\color{BrickRed}*** ver si las mallas hacen falta antes del caso convexo;
ver si hace falta llegar hasta subsubseccion} 
\subsubsection{Macro--element with singular edge and vertex}\label{caso4}
Let $T$ be a tetrahedral macro--element with vertices $P_0, P_1, P_2$ and $P_3$. We suppose that $P_0$ is the singular vertex and that the
singular edge is $P_0P_1$. The mesh $\mathcal T_h$ on $T$ will contain tetrahedra, prisms and pyramids. 
In Tables~\ref{prisms_barycentric_a}--\ref{tetrahedral_barycentric}
we explicit the elements in terms
of the barycentric coordinates of the vertices of each element corresponding to the ordered vertices $P_0, P_1, P_2$ and $P_3$.
\bigskip

\prismsBaryCoordA

\prismsBaryCoordB

\pyramidsBaryCoord

\tetrahedraBaryCoord

\subsubsection{Macro--element with a singular edge}
This case corresponds to the prismatic macro--element.
A mesh is constructed as the cartesian product between a graded mesh
in a triangle and a quasi--uniform mesh along the singular edge. Table~\ref{prisms_product}
shows the points
For $0\leqslant i\leqslant n$ and $0\leqslant j\leqslant n-1$ let $p_{i,j}$
be the point with barycentric coordinates, with respect to $P_0,P_1,P_2$,
equal to      
\begin{eqnarray*}
&&\lambda_0(i,j)=1-\lambda_1(i,j)-\lambda_2(i,j),\\[5pt]
&&\lambda_1(i,j)=\frac in\left(\frac{i+j}n\right)^{\gamma-1},\quad
  \lambda_2(i,j)=\frac jn\left(\frac{i+j}n\right)^{\gamma-1},\quad
\end{eqnarray*}
Now let $\Tb$ the family of triangles with vertices  {\color{BrickRed}( controlar )}
\begin{IEEEeqnarray*}{lllCLL}
p_{i,j} & p_{i+1,j} & p_{i,j+1} & \quad & 0\leqslant i  < n\mbox{,\quad}&0\leqslant j < n - i\\
p_{i,j} & p_{i+1,j-1} & p_{i+1,j} & \quad & 0\leqslant i  < n-1\mbox{,\quad}&1\leqslant j < n - i.
\end{IEEEeqnarray*}
The elements in $\Lambda_\ell$ will be
$\tau\times (P_{0,3} + \frac kn h_{\Lambda_\ell,3}, P_{0,3} + \frac{k+1}{n} h_{\Lambda_\ell,3})$
for each $\tau\in\Tb$ and each $0\leqslant k<n$. See Figure~\ref{prismatic_macroelements}.
%borrar ---->>>  \prismsProductCoord

\def\col{black}
\def\height{0}
\def\twoPi{360}
\begin{figure}[!h]\centering
  \subfloat
  {
    \begin{tikzpicture}[scale=2]
      \prismaticMacroelement{8}{2}{.63}{4}{\col}
    \end{tikzpicture}\hspace{1cm}
    \begin{tikzpicture}[scale=2]
      \prismaticMacroelement{8}{3}{.63}{4}{\col}
      \draw[red] (0,0,0) -- (0,-1.5,0);
    \end{tikzpicture}\hspace{1cm}
    \begin{tikzpicture}[scale=2]
      \prismaticMacroelement{8}{4}{.63}{4}{\col}
      \draw[red] (0,0,0) -- (0,-1.5,0);
    \end{tikzpicture}
  }
  \caption{Elements of the family of
    meshes restricted to a prismatic macro--element.$\mu = .63$}
  \label{prismatic_macroelements}
\end{figure}

\subsubsection{Macro--element with a singular vertex}

We consider again a tetrahedral macro--element $T$ with vertices $P_0, P_1, P_2$
and $P_3$, assuming that it has a singular vertex at $P_0$ and no singular edge.
We construct a triangulation of $T$ made of tetrahedra describing 
the barycentric coordinates of the vertices of each tetrahedron with respect
to $P_0, P_1, P_2$ and $P_3$. %This case corresponds to Case 1 of \cite{AN}.

Let $p_{i,j,k}$ be the points with barycentric coordinates
\begin{eqnarray*}
&&\lambda_0=1-\lambda_1-\lambda_2-\lambda_3,\\[5pt]
&&\lambda_1=\frac in\left(\frac{i+j+k}n\right)^{\gamma-1},\quad
  \lambda_2=\frac jn\left(\frac{i+j+k}n\right)^{\gamma-1},\quad
  \lambda_3=\frac kn\left(\frac{i+j+k}n\right)^{\gamma-1},
\\[5pt] &&0\le i\le n, 0\le j\le n-i, 0\le k\le n-i-j.
\end{eqnarray*}
Then, the tetrahedra are the ones with vertices
\begin{eqnarray*}
&& p_{i,j,k}, p_{i+1,j,k}, p_{i,j+1,k}, p_{i,j,k+1}, \qquad 0\le i+j+k\le n-1\\
&& p_{i+1,j,k}, p_{i,j+1,k}, p_{i,j,k+1}, p_{i+1,j,k+1}, \qquad 0\le i+j+k\le n-2\\
&& p_{i,j+1,k}, p_{i,j,k+1}, p_{i+1,j,k+1}, p_{i,j+1,k+1}, \qquad 0\le i+j+k\le n-2\\
&& p_{i+1,j,k}, p_{i,j+1,k}, p_{i+1,j+1,k}, p_{i+1,j,k+1}, \qquad 0\le i+j+k\le n-2\\
&& p_{i,j+1,k}, p_{i+1,j+1,k}, p_{i+1,j,k+1}, p_{i,j+1,k+1}, \qquad 0\le i+j+k\le n-2\\
&& p_{i+1,j+1,k}, p_{i+1,j,k+1}, p_{i,j+1,k+1}, p_{i+1,j+1,k+1}, \qquad 0\le i+j+k\le n-3
\end{eqnarray*}


\begin{proposition} Pyramids and tetrahedra are isotropic.
\end{proposition}
\begin{proof} We deal firstly with pyramids. Consider a pyramid with vertices $p_0, \ldots, p_4$ in a macro--element of vertices $P_0,P_1,P_2$ and $P_3$ as in Subsection \ref{caso4}. Note that the basis of the pyramid is the paralelogram $p_0p_1p_3p_2$ with
\begin{eqnarray}\label{once}
&&p_1-p_0=p_3-p_2=\frac1n\left(\frac{n-l-1}n\right)^{\gamma-1}(P_3-P_2)\\\label{doce} &&p_2-p_0=p_3-p_1=\left[\left(\frac{n-l}n\right)^\gamma-\left(\frac{n-l-1}n\right)^\gamma\right](P_0-P_1).
\end{eqnarray}
So 
\[
\frac{\gamma}{n}\left(\frac{n-l-1}n\right)^{\gamma-1}|P_0-P_1|\le |p_2-p_0|=|p_3-p_1|\le \frac{\gamma}{n}\left(\frac{n-l}n\right)^{\gamma-1}|P_0-P_1|,
\]
and 
\[
\frac1\gamma\left(\frac12\right)^{\gamma-1}\le\frac1\gamma\left(\frac{n-l-1}{n-l}\right)^{\gamma-1}\le\frac{|p_1-p_0|}{|p_2-p_0|}\le \frac1\gamma.
\]
Then the parallelogram $p_0p_1p_3p_2$ is shape-regular since the angle between $P_0-P_1$ and $P_3-P_2$ depends only on the macro--element, and so it is away from $0$ and $\pi$. 

Now we prove that there exists constants $c_0$ and $c_1$ depending only on $\gamma$ and the macro--element's vertices such that
\begin{eqnarray}\label{diez}
&c_0\le\frac{|p_4-p_2|}{|p_2-p_0|}\le c_1&\\ \label{trece}
&c_0\le\frac{|p_4-p_3|}{|p_2-p_0|}\le c_1&.
\end{eqnarray}
After simple computations we obtain
\begin{eqnarray*}
p_4-p_2 &=& \left[\left(\frac{n-l}n\right)^\gamma - \left(\frac{n-l-1}n\right)^\gamma\right](P_3-P_0) \\ &&\qquad + \frac in \left[\left(\frac{n-l}n\right)^{\gamma-1} - \left(\frac{n-l-1}n\right)^{\gamma-1}\right](P_2-P_3)\\ &=& \left[\left(\frac{n-l}n\right)^\gamma - \left(\frac{n-l-1}n\right)^\gamma\right]\bigg\{P_3-P_0 +\\ &&\qquad + \frac{\frac in \left[\left(\frac{n-l}n\right)^{\gamma-1} - \left(\frac{n-l-1}n\right)^{\gamma-1}\right]}{\left[\left(\frac{n-l}n\right)^\gamma - \left(\frac{n-l-1}n\right)^\gamma\right]} (P_2-P_3)\bigg\}\\ &\sim& \left[\left(\frac{n-l}n\right)^\gamma - \left(\frac{n-l-1}n\right)^\gamma\right](P_3-P_0)\\ &\sim& p_3-p_1
\end{eqnarray*}
where we used that 
\[
\frac{\frac in \left[\left(\frac{n-l}n\right)^{\gamma-1} - \left(\frac{n-l-1}n\right)^{\gamma-1}\right]}{\left[\left(\frac{n-l}n\right)^\gamma - \left(\frac{n-l-1}n\right)^\gamma\right]}\le \frac{\gamma-1}{\gamma},
\]
that the angle between $P_3-P_0$ and $P_2-P_3$ is fixed (and depends only on the macro--element) and equation \eqref{doce}. This prove \eqref{diez}, and \eqref{trece} follows analogously. In order to prove that the pyramid is isotropic now we have to note that the basis $p_0p_1p_3p_2$ is contained in a plane parallel to the one generated by the vectors $P_1-P_0$ and $P_3-P_2$, and the face $p_2p_3p_4$ is in a plane parallel to the plane (macro--element's face) $P_0P_2P_3$, and the angle between those planes depends only on the macro--element.      


COMPLETAR PRUEBA PARA LOS TETRAEDROS
\end{proof}



Another way, less formal but clearer, to look
at the  mesh points in a tetrahedral macro--element
with both kinds of singularities is the following.

Fijamos:
\begin{IEEEeqnarray*}{rCl}
  T &=& \left[\textbf{p}_0,\textbf{p}_1,\textbf{p}_2,\textbf{p}_3\right]\\[7pt]
  \textbf{p}_0 &=& 0\\[7pt]
  \left[\textbf{p}_0,\textbf{p}_3\right] &=& 
  \text{la arista singular}\\[7pt]
  \textbf{p}_3 & = & \text{v\'ertice singular}\\[7pt]
  \left[\textbf{p}_0, \textbf{p}_1, 
  \textbf{p}_2\right]&\subseteq&\{z=0\}.
\end{IEEEeqnarray*}
  
$\lambda_0,\lambda_1$ y $\lambda_2$: {\color{RedOrange}coordenadas  baric\'entricas} con respecto a
$\textbf{p}_0,\textbf{p}_1$ y $\textbf{p}_2$.\\[10pt]
$\textbf{p}_{i,j}$ es tal que\\[5pt]
\begin{IEEEeqnarray*}{rCl}
  \lambda_0(\textbf{p}_{i,j})  &=& 1-\lambda_1(\textbf{p}_{i,j}) - \lambda_2(\textbf{p}_{i,j}) \\[5pt]
  \lambda_1(\textbf{p}_{i,j})  &=& \frac{i}{n}\left(\frac{i+j}{n}\right)^{\frac{1}{\mu}-1}\\[5pt]
  \lambda_2(\textbf{p}_{i,j})  &=& \frac{j}{n}\left(\frac{i+j}{n}\right)^{\frac{1}{\mu}-1}.\\[15pt]
  &&0\leqslant i\leqslant n\\[5pt]
  &&0\leqslant j\leqslant n-i
\end{IEEEeqnarray*}
  
Los puntos son la uni\'on de\\[5pt]
\begin{IEEEeqnarray*}{lCl}
  \left\{ \textbf{p}_{i,j}\,:\,0\leqslant i\leqslant n,\quad
    0\leqslant j\leqslant n-i \right\}&&\\[7pt]
  \left\{ \textbf{p}_{i,j}\,:\,0\leqslant i\leqslant n-1,\quad
  0\leqslant j\leqslant n-1-i \right\} & \quad+\quad & 
  [{\scriptstyle 1-\left(\frac{n-1}{n}\right)^{1/\mu}}]\,\textbf{e}_3\\[7pt]
  \IEEEeqnarraymulticol{3}{c}{\vdots}\\[7pt]
  \left\{\textbf{p}_{i,j}\,:\,0\leqslant i\leqslant n-k,\quad
  0\leqslant j\leqslant n-k-i \right\} & \quad+\quad &
  [{\scriptstyle 1-\left(\frac{n-k}{n}\right)^{1/\mu}}]\,\textbf{e}_3\\[7pt]
  \IEEEeqnarraymulticol{3}{c}{\vdots}\\[7pt]
  \left\{ \textbf{p}_{i,j}\,:\,0\leqslant i\leqslant 1,\quad
  0\leqslant j\leqslant 1-i \right\} & \quad+\quad &
  [{\scriptstyle 1-\left(\frac{1}{n}\right)^{1/\mu}}]\,\textbf{e}_3\\[10pt]
  \IEEEeqnarraymulticol{3}{l}{\left\{\textbf{e}_3\right\}}\text{,}
\end{IEEEeqnarray*}
i.e.:
\begin{IEEEeqnarray*}{rCl}
  \mathcal{P} & = & \bigcup_{k=0}^n\;\left\{ \textbf{p}_{i,j}\,:\,0\leqslant i\leqslant n - k,\quad
  0\leqslant j\leqslant n-k-i \right\}\\[8pt]
  &&\quad+\,[{\scriptstyle1-\left(\frac{n-k}{n}\right)^{1/\mu}}]\,\textbf{e}_3.
\end{IEEEeqnarray*}
\begin{remark}
  Our meshes sastify the property that for each $\ell$ and each $\textit{h}$
  there exists an index set $I_{\textit{h},\ell}$ such that
  \begin{IEEEeqnarray*}{rCl}
    \Lambda_\ell = \cup_{i\in I_{\textit{h},\ell}} E_{\textit{h},i}.
  \end{IEEEeqnarray*}
  In this case we are denoting 
  $\Th = \{E_{\textit{h},i}\,:\,1\leqslant i\leqslant N_{\textit{h}}\}$.
\end{remark}
