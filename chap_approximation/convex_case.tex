\section{convex Case} % (fold)
\label{sec:convex Case}
%% ============================================================================
% {\color{brown}
% esta version de intercalado se usa?? Dur\'an Mixed lemma 3.9 es la segunda de estas
% \begin{proof}
% \begin{IEEEeqnarray*}{lCl}
%   \|\boldsymbol{u} - \boldsymbol{u}_{\textit{h}}\|_{L^2(\Omega)^3}
%   &\leqslant& 
%   C\,\inf_{v\in\mathbb{V}_h}\|\boldsymbol{u} - \textbf{v}\|_{\mathbb{V}}\\[7pt]
%   &\leqslant& 
%   C\,\|\boldsymbol{u} - \pi\boldsymbol{u}\|_{\mathbb{V}}\\[10pt]
%   \|p-p_{\textit{h}}\|_{\mathbb{Q}}&\leqslant&
%   C\,[\|\boldsymbol{u} - \pi\boldsymbol{u}\|_{L2(\Omega)}+\|p - \pi_{\textit{h}}^{\bot} p \|_{\mathbb{Q}}].
% \end{IEEEeqnarray*}
% \end{proof}
% }
%% ============================================================================

If $\Omega$ were convex and $f\in L^2(\Omega)$ 
we obtain from Theorem~\ref{aux_label47} the following Corollary.
\begin{corollary}
\begin{eqnarray*}
\|\bu-\bu_h\|_{L^2(\Omega)}&\leqslant& C\textit{h}|p|_{H^2(\Omega)}\\ 
\|p-p_h\|_{L^2(\Omega)}&\leqslant& C\textit{h}|p|_{H^1(\Omega)}
\end{eqnarray*}
where the constant $C$ depends only on the aspect ratios of tetrahedra 
and pyramids and the triangular bases of the right prisms on the mesh. 
\end{corollary}
\begin{proof}
If $\Omega$ were convex and $f\in L^2(\Omega)$ we know that the solution $p$ of
problem $H^2(\Omega)$, and so $\bu\in H^1(\Omega)^3$. In this case, using the 
interpolation error estimates we proved in Propositions~\ref{aux_label46}
and~\ref{propErrorInterpolacionPiramidesTetraedros}, the result of 

we have 
\begin{IEEEeqnarray*}{rCl}
  \|\bu-I(\bu)\|_{L^2(\Omega)^3} &\leqslant & C\textit{h}|p|_{H^2(\Omega)}.
\end{IEEEeqnarray*}
Next
Proposition~\ref{propupi}, and a 
%%%% standard estimate for $L^2$-projection error,  --- > bramble hilbert?
yield



\end{proof}
Arbitrarily narrow right prisms can be used in the mesh without 
affecting this estimate. This fact can be further exploited when the
domain $\Omega$ is not convex or $f$ is not in $L^2(\Omega)$. This is what follows!
