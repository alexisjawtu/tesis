\section{Approximation Error for the Convex Case} % (fold)
\label{sec:convex Case}
For the case in which $\Omega$ is convex and $f\in L^2(\Omega)$ 
we obtain from Theorem~\ref{aux_label47} the following Corollary.
\begin{corollary}\label{auxlabel418}
\begin{eqnarray*}
\|\bu-\bu_h\|_{L^2(\Omega)}&\leqslant& C\textit{h}|p|_{H^2(\Omega)}\\ 
\|p-p_h\|_{L^2(\Omega)}&\leqslant& C\textit{h}\|p\|_{H^2(\Omega)}
\end{eqnarray*}
where the constant $C$ depends only on the aspect ratios of tetrahedra 
and pyramids and the maximum angle of the triangular faces
of the right prisms on the mesh, provided that the tetrahedra fulfill a
uniform maximum angle condition. 
\end{corollary}
\begin{proof}
For a convex $\Omega$ and $f\in L^2(\Omega)$ the solution $p$ of the
problem belongs to $H^2(\Omega)$ and so $\bu\in H^1(\Omega)^3$. In this case, using the 
interpolation error estimates we proved in Theorem~\ref{aux_label46} for general
prisms
and Proposition~\ref{propErrorInterpolacionPiramidesTetraedros} for 
pyramids and the analogue for tetrahedra from~\cite{aadl},
we have, for $\br_0\bu$ as in Definition~\ref{aux_label52}, that 
\begin{IEEEeqnarray*}{rCl}
  \|\bu-\br_0\bu\|_{L^2(\Omega)^3} &\leqslant & C\textit{h}|p|_{H^2(\Omega)}.
\end{IEEEeqnarray*}
Next
again Theorem~\ref{aux_label46},
Proposition~\ref{propErrorInterpolacionPiramidesTetraedros} for pyramids and 
the analogue for tetrahedra from~\cite{aadl}, and Proposition~\ref{propupi}
for pyramids
yield, for $\bw_{\bu}$ as in Definition~\ref{aux_label51},
\begin{IEEEeqnarray*}{rCl}
  \|\bu-\bw_{\bu}\|_{L^2(\Omega)^3} & \leqslant & C\textit{h}|p|_{H^2(\Omega)}.
\end{IEEEeqnarray*}
So joining the last two inequalities,  from~\eqref{aux_label48} in Theorem~\ref{aux_label47},
\begin{IEEEeqnarray*}{rCl}
  \|\bu-\bu_h\|_{L^2(\Omega)^3} & \leqslant &
  C\{\|\bu-\bw_{\bu}\|_{L^2(\Omega)^3} + \|\bu-\br_0\bu\|_{L^2(\Omega)^3}\}\\[5pt]
  &\leqslant & C\textit{h}|p|_{H^2(\Omega)}\\[5pt]
  &\leqslant & C\textit{h}\,\|f\|_{L^2(\Omega)}
\end{IEEEeqnarray*}
as stated in~\cite{alw}	in expression after $(3.42)$ on page $19$.
The bound for the scalar variable error
in the convex case follows using~\eqref{aux_label49} and the estimate
$\|p-P_{0,\tau_{\textit{h}}}\,p\|\leqslant C\,h_{E}\,|p|_{1,E}$ for the orthogonal
projection.
\end{proof}
Arbitrarily narrow right prisms can be used in the mesh without 
affecting this estimate. This fact can be further exploited when the
domain $\Omega$ is not convex or $f$ is not in $L^2(\Omega)$. Besides, we could
also allow arbitrary narrow pyramids in the mesh (as long as they are not \textit{flat},
that is, at least one side of the basis must be smaller than the height) and
continue from Theorems~\ref{aux_label53} and~\ref{aux_label54}, but in the meshes we 
designed, which appear in what follows, pyramids happened to be regular. 
The same can be said about the $\pazocal{MAC}$ condition mentioned
in Corollary~\ref{auxlabel418}. Again, this is not a restriction, since
our method required only the use of shape--regular tetrahedra. This will be made clear
in Subsection~\ref{meshes}.

The non--convex case including anisotropic prismatic elements is what follows. It is in
this non--convex case where we will need our estimates of anisotropic type,
in contrast with the previous convex case, where we didn't need them.
