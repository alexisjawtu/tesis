\chapter{Implementation}

\section*{Introducci\'on al cap\'itulo}
En este cap\'itulo mostramos algunos ejemplos obtenidos 
con programas que desarrollamos para esta tesis.

En la Secci\'on~\ref{auxlabel214} mostramos ejemplos de mallado 
obtenidos con la implementaci\'on de nuestro proceso propuesto 
en la Subsecci\'on~\ref{meshes}. 
En este caso usamos el dominio de Fichera como ejemplo can\'onico 
de dominio singular en $\mathbb{R}^3$.

Esta primera versi\'on del programa de mallado se puede ver en el siguiente 
repositorio \href{https://github.com/alexisjawtu/mesher}{github.com/alexisjawtu/mesher}.

El programa que implementa la resoluci\'on num\'erica con nuestro esquema
$FEM/VEM$ en las mallas h\'ibridas se encuentra en desarrollo y ser\'a
usado en experimentos num\'ericos en trabajos futuros.

En la Secci\'on~\ref{auxlabel215} mostramos resultados num\'ericos
cuando aplicamos nuestro m\'etodo a un dominio descompuesto en macro--elementos
de un solo tipo, el macro--elemento prism\'atico con una arista singular y sin v\'ertices
singulares.

\section*{Introduction to the chapter}
In this chapter we show some examples obtained with programs we 
developed for the present thesis.

In Section~\ref{auxlabel214} we show a meshing example made with the implementation
of the procedure we proposed in Subsection~\ref{meshes}. In this case we used
the Fichera domain as a canonical example of a singular domain in $\mathbb{R}^3$.

This first version of the meshing program can be found in the following repository
\href{https://github.com/alexisjawtu/mesher}{github.com/alexisjawtu/mesher}.

The program that implements the numerical solution with our $FEM/VEM$ scheme
over the hybrid meshes of Subsection~\ref{meshes} is being developed and will
be used in numerical experiments of future works.

In Section~\ref{auxlabel215} we show numerical results when we
apply our method to a domain made exclusively of one kind of macro--elements,
namely, the prismatic macro--element with a singular edge and no singular vertex.

\section{Examples Of The Meshing Procedure In Dimension $3$} % (fold)
\label{auxlabel214}
Figure~\ref{auxlabel206} shows the partition $\pazocal{T}_{\textit{h}_0}$ (cfr. Remark~\ref{auxlabel216}) of the Fichera Domain 
$\Omega:=(-1,1)^3\,\setminus\,(0,1)^3$ from four different azimuthal 
angles. 

\tauZero 

Then in Figure~\ref{auxlabel300} we show how the hybrid mesh 
looks like in a tetrahedral
macro--element as in Subsubsection~\ref{caso4}. 

\macroElement3

Figure~\ref{auxlabel301} shows the detail of the decomposition into macro--elements
of a cube defined as the intersection of $\Omega$ with an arbitrary octant. The 
cube consists of five tetrahedral macro--elements; a regular tetrahedron
in the center of the cube meshed as a macro--element from 
Subsubsection~\ref{auxlabel302} and four more macro--tetrahedra as in 
Subsubsection~\ref{caso4}.

\tauOneEnCube

Finally in Figure~\ref{auxlabel303} we can see the whole domain $\Omega$
meshed with our anisotropic grading procedure.

\tauOneEn
\vspace{5cm}
Additionally, we found interesting to illustrate some experiments
that confirm the conformity of the meshes. These can be seen in 
Figure~\ref{auxlabel305}.

\conform
\section{Numerical Experiments with Cartesian product anisotropic Meshes in Domains with Edge Singularities}
\label{auxlabel215}

\edgedomain $--->$ \ref{auxlabel204}
\unifSection $--->$ \ref{auxlabel212}
\gradSection $--->$ \ref{auxlabel310}
\tableErrorsUniformCylinder $--->$ \ref{table_errors_1}
\tableErrorsAnisoCylinder $--->$ \ref{table_errors_2}


%%%%%%%%%%%%%%%%%%%%%%%%%%%%%%%%%%%%%%%%%%%%%
%% TODO
%\begin{algorithm}
%	\begin{algorithmic}
%		\State vertices $\leftarrow$ load(verticesCoordinates)
%		\State elementVertices $\leftarrow$ load(elementsByVerticesIndices)
%		\State faces $\leftarrow$ load(facesCoordinates)
%		\State elementFaces $\leftarrow$ load(elementsByFacesIndices)
%
%		\ForAll{element}
%			\State v $\gets$ verticesOf(element,vertices)
%		\EndFor
%	\end{algorithmic}
%\end{algorithm}
%