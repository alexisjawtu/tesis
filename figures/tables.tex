\def\tableErrorsUniformCylinder{
\begin{table}[!ht]
    \centering
    \caption{$\omega = 3\pi/2$, $\gamma = 1$}
    \label{table_errors_1}
    \begin{IEEEeqnarraybox}[\IEEEeqnarraystrutmode\IEEEeqnarraystrutsizeadd{3pt}{1pt}]
    {v / c / v / c / v / c / v / c / v / c / v / c / v}
        \IEEEeqnarrayrulerow\\
        & n && \text{Nel} && \boldsymbol{u} && p  
            && \frac{\Delta_i \log(e_{\boldsymbol{u}})}{- \Delta_i \log(n)}
            && \frac{\Delta_i \log(e_{p})}{- \Delta_i \log(n)} & 
        \IEEEeqnarraystrutsizeadd{7pt}{7pt}\\\IEEEeqnarrayrulerow\\
        & 6  && 1512    && 0.632043 && 0.205469 &&          &&          & \\ \IEEEeqnarrayrulerow\\
        & 10 && 7000    && 0.324654 && 0.123666 && 1.304158 && 0.993902 & \\ \IEEEeqnarrayrulerow\\
        & 20 && 56000   && 0.243846 && 0.064251 && 0.412932 && 0.944658 & \\ \IEEEeqnarrayrulerow\\
        & 30 && 189000  && 0.130700 && 0.042238 && 1.538066 && 1.034557 & \\ \IEEEeqnarrayrulerow\\
        & 40 && 448000  && 0.108696 && 0.031693 && 0.640810 && 0.998409 & \\ \IEEEeqnarrayrulerow\\
        & 50 && 875000  && 0.091443 && 0.025360 && 0.774564 && 0.999010 & \\ \IEEEeqnarrayrulerow\\
        & 60 && 1512000 && 0.079951 && 0.021128 && 0.736620 && 1.001384 & \\ \IEEEeqnarrayrulerow
    \end{IEEEeqnarraybox}
\end{table}
}

\def\tableErrorsAnisoCylinder{
    \begin{table}[!ht]
    \centering
    \caption{$\omega = 3\pi/2$, $\gamma = 1.5$}
    \label{table_errors_2}
    \begin{IEEEeqnarraybox}[\IEEEeqnarraystrutmode\IEEEeqnarraystrutsizeadd{3pt}{1pt}]
    {v / c / v / c / v / c / v / c / v / c / v / c / v}
        \IEEEeqnarrayrulerow\\
        & n && \text{Nel} && \boldsymbol{u} && p  
            && \frac{\Delta_i \log(e_{\boldsymbol{u}})}{- \Delta_i \log(n)}
            && \frac{\Delta_i \log(e_{p})}{- \Delta_i \log(n)} & 
        \IEEEeqnarraystrutsizeadd{7pt}{7pt}\\\IEEEeqnarrayrulerow\\
        & 6  && 1512     && 0.612728  && 0.206872  &&           &&                    & \\ \IEEEeqnarrayrulerow\\
        & 10 && 7000     && 0.292898  && 0.124441  && 1.444907  && 0.994997  & \\ \IEEEeqnarrayrulerow\\
        & 20 && 56000    && 0.218083  && 0.064388  && 0.425517  && 0.950588  & \\ \IEEEeqnarrayrulerow\\
        & 30 && 189000   && 0.094069  && 0.042391  && 2.073776  && 1.030897  & \\ \IEEEeqnarrayrulerow\\
        & 40 && 448000   && 0.074568  && 0.031808  && 0.807546  && 0.998343  & \\ \IEEEeqnarrayrulerow\\
        & 50 && 875000   && 0.057800  && 0.025453  && 1.141533  && 0.998845  & \\ \IEEEeqnarrayrulerow\\
        & 60 && 1512000  && 0.047035  && 0.021210  && 1.130357  && 1.000286  & \\ \IEEEeqnarrayrulerow\\
        & 70 && 2401000  && 0.040214  && 0.018179  && 1.016342  && 1.000308  & \\ \IEEEeqnarrayrulerow
    \end{IEEEeqnarraybox}
\end{table}
}

\def\facesOfPrism{
\begin{table}[!ht]
    \centering  
    \caption{Notation for the faces and positive normals of the reference prism.}
    \label{prismNotationTableFaces}
    \begin{IEEEeqnarraybox*}
      [\IEEEeqnarraystrutmode
      \IEEEeqnarraystrutsizeadd{2pt}{6pt}]{v/c/x/c/x/c/x/v/x/c/x/c/x/c/v/}
        \IEEEeqnarrayrulerow\\
        \IEEEeqnarrayseprow[5pt]\\
          & \hat f_1 && \subseteq &&  \{\hat x_1 = 0 \}            && && \hat{\bn}_1 && = && (-1,0,0)' & \\
        \IEEEeqnarrayrulerow\\
        \IEEEeqnarrayseprow[5pt]\\
          & \hat f_2 && \subseteq &&  \{\hat x_2 = 0 \}            && && \hat{\bn}_2 && = && (0,-1,0)' &\\
        \IEEEeqnarrayrulerow\\
        \IEEEeqnarrayseprow[5pt]\\
          & \hat f_3 && \subseteq &&  \{\hat x_3 = 0 \} && && \hat{\bn}_3 && = && (0,0,-1)' &\\
        \IEEEeqnarrayrulerow\\
        \IEEEeqnarrayseprow[5pt]\\
          & \hat f_4 && \subseteq &&  \{\hat x_3 = 1 \} && && \hat{\bn}_4 && = && (0,0,1)' &\\
        \IEEEeqnarrayrulerow\\
        \IEEEeqnarrayseprow[5pt]\\
          & \hat f_5 && \subseteq &&  \{\hat x_1+\hat x_2 = 1\} && && \hat{\bn}_5 && = && 2^{-\nicefrac{1}{2}}(1,1,0)' &\\
        \IEEEeqnarrayrulerow
    \end{IEEEeqnarraybox*}
\end{table}
}

\def\edgesOfPrism{
\begin{table}[!ht]
    \centering  
    \caption{Notation for the edges and positive tangents of the reference prism.}
    \label{prismNotationTableEdges}
    \begin{IEEEeqnarraybox*}
      [\IEEEeqnarraystrutmode
      \IEEEeqnarraystrutsizeadd{2pt}{6pt}]{v/c/x/c/x/c/x/v/x/c/x/c/x/c/v/}
        \IEEEeqnarrayrulerow\\
        \IEEEeqnarrayseprow[5pt]\\
   & \hat \be_1 && = && \{(\hat x_1,0,0)^t\,:\,0\leqslant\hat x_1\leqslant 1\} && && \hat \btau_1 && = && (1,0,0)' & \\
        \IEEEeqnarrayrulerow\\
        \IEEEeqnarrayseprow[5pt]\\
   & \hat \be_2 && = && \{(0,\hat x_2,0)^t\,:\,0\leqslant\hat x_2\leqslant 1\} && && \hat \btau_2 && = && (0,1,0)' & \\
        \IEEEeqnarrayrulerow\\
        \IEEEeqnarrayseprow[5pt]\\
   & \hat \be_3 && = && \{(0,0,\hat x_3)^t\,:\,0\leqslant\hat x_3\leqslant 1\} && && \hat \btau_3 && = && (0,0,1)' & \\
        \IEEEeqnarrayrulerow\\
        \IEEEeqnarrayseprow[5pt]\\
   & \hat \be_4 && = && \{(\hat x_1,0,1)^t\,:\,0\leqslant\hat x_1\leqslant 1\} && && \hat \btau_4 && = && (0,0,1)' & \\
        \IEEEeqnarrayrulerow\\
        \IEEEeqnarrayseprow[5pt]\\
   & \hat \be_5 && = && \{(0,\hat x_2,1)^t\,:\,0\leqslant\hat x_2\leqslant 1\} && && \hat \btau_5 && = && (0,0,1)' & \\
        \IEEEeqnarrayrulerow\\
        \IEEEeqnarrayseprow[5pt]\\
   & \hat \be_6 && = && \{(1,0,\hat x_3)^t\,:\,0\leqslant\hat x_3\leqslant 1\} && && \hat \btau_6 && = && (0,0,1)' & \\
        \IEEEeqnarrayrulerow\\
        \IEEEeqnarrayseprow[5pt]\\
   & \hat \be_7 && = && \{(0,1,\hat x_3)^t\,:\,0\leqslant\hat x_3\leqslant 1\} && && \hat \btau_7 && = && (0,0,1)' & \\
        \IEEEeqnarrayrulerow\\
        \IEEEeqnarrayseprow[5pt]\\
   & \hat \be_8 && = && \{(\hat x_1,1-\hat x_1,1)^t\,:\,0\leqslant\hat x_1\leqslant 1\} && && \hat \btau_8 && = && 2^{-\nicefrac{1}{2}}(1,-1,0)' & \\
        \IEEEeqnarrayrulerow\\
        \IEEEeqnarrayseprow[5pt]\\
   & \hat \be_9 && = && \{(\hat x_1,1-\hat x_1,0)^t\,:\,0\leqslant\hat x_1\leqslant 1\} && && \hat \btau_9 && = && 2^{-\nicefrac{1}{2}}(1,-1,0)' & \\
        \IEEEeqnarrayrulerow  
    \end{IEEEeqnarraybox*}
\end{table}
}

\def\edgeShapeTable{\begin{table}[!ht]
    \centering  
    \caption{{Edge shape functions} on the reference pyramid}
    \label{shape_edge_table}
    \begin{IEEEeqnarraybox*}
    [\IEEEeqnarraystrutmode
    \IEEEeqnarraystrutsizeadd{2pt}{25pt}]{x/c/x/c/x/c/x}
        \IEEEeqnarrayseprow[5pt]\\
        &\IEEEeqnarraymulticol{6}{c}{
                {%\tiny
                {\scriptstyle\hat\bgamma_1} = 
                \left(
                    \begin{array}{c}
                        {1-z-y} \\[5pt]             
                        0 \\[5pt]
                        x-\frac{xy}{1-z}               
                    \end{array}
                \right)}\;\;
                {%\tiny
                {\scriptstyle\hat\bgamma_2} = 
                \left(
                    \begin{array}{c}
                        0 \\[5pt]
                        x \\[5pt]                
                        \frac{xy}{1-z}               
                    \end{array}
                \right)}\;\;
                {%\tiny
                {\scriptstyle\hat\bgamma_3} = 
                \left(
                    \begin{array}{c}
                        y \\[5pt]
                        0 \\[5pt]                
                        \frac{xy}{1-z}               
                    \end{array}
                \right)}\;\;
        {%\tiny
        {\scriptstyle\hat\bgamma_4} = 
        \left(
                    \begin{array}{c}
                        0 \\[8pt]                
                        {1-z-x} \\[8pt]
                        y-\frac{xy}{1-z}  
                    \end{array}
                \right)}}\\
        \IEEEeqnarrayseprow[5pt]\\
        &\IEEEeqnarraymulticol{6}{c}{
        {%\tiny
        {\scriptstyle\hat\bgamma_5} = 
                \left(
                        \begin{array}{c}
                            z-\frac{yz}{1-z} \\[8pt]               
                            z-\frac{xz}{1-z} \\[8pt]               
                            1-x-y+\frac{xy}{1-z}-\frac{xyz}{(1-z)^2}               
                        \end{array}
               \right)}\;\;
                {%\tiny
                {\scriptstyle\hat\bgamma_6} = 
                \left(
                    \begin{array}{c}
                        -z+\frac{yz}{1-z} \\[8pt]               
                        \frac{xz}{1-z} \\[8pt]               
                        x-\frac{xy}{1-z}+\frac{xyz}{(1-z)^2}               
                    \end{array}
                        \right)}}\\
        \IEEEeqnarrayseprow[5pt]\\
        &\IEEEeqnarraymulticol{6}{c}{
        {%\tiny
        {\scriptstyle\hat\bgamma_7} = 
                \left(
                        \begin{array}{c}
                            \frac{yz}{1-z} \\[8pt]               
                            -z+\frac{xz}{1-z} \\[8pt]               
                            y-\frac{xy}{1-z}+\frac{xyz}{(1-z)^2}               
                        \end{array}
               \right)}\;\;
        {%\tiny
                {\scriptstyle\hat\bgamma_8} = 
                \left(
                    \begin{array}{c}
                        -\frac{yz}{1-z} \\[8pt]               
                        -\frac{xz}{1-z} \\[8pt]               
                        \frac{xy}{1-z}-\frac{xyz}{(1-z)^2}               
                    \end{array}
                        \right)}}
    \end{IEEEeqnarraybox*}
\end{table}}

\def\faceShapeTable{
 \begin{table}[!ht]
    \centering  
    \caption{{Face shape functions} on the reference pyramid}
    \label{shape_face_table}
    \begin{IEEEeqnarraybox*}
    [\IEEEeqnarraystrutmode
    \IEEEeqnarraystrutsizeadd{2pt}{25pt}]{x/c/x/c/x/c/x}
        \IEEEeqnarrayseprow[5pt]\\
        &\IEEEeqnarraymulticol{5}{c}{
                \raisebox{-15pt}{}\;\;
                {%\tiny
                {\scriptstyle\hat\bz_1} = 
                \left(
                        \begin{array}{c}
                            -\frac{xz}{1-z} \\[8pt]             
                            y-2+\frac{y}{1-z} \\[8pt]
                            z               
                        \end{array}
                        \right)}\;\;
                {%\tiny
                {\scriptstyle\hat\bz_2} = 
                \left(
                            \begin{array}{c}
                                x-2+\frac{x}{1-z} \\[8pt]
                                -\frac{yz}{1-z} \\[8pt]                
                                z               
                            \end{array}
                        \right)}}&\\
        \IEEEeqnarrayseprow[10pt]\\
        &
        {%\tiny
        {\scriptstyle\hat\bz_3} = 
        \left(
                    \begin{array}{c}
                        x+\frac{x}{1-z} \\[8pt]
                        -\frac{yz}{1-z} \\[8pt]                
                        z               
                    \end{array}
                \right)}&&
        {%\tiny
        {\scriptstyle\hat\bz_4} = 
        \left(
                    \begin{array}{c}
                        -\frac{xz}{1-z} \\[8pt]                
                        y+\frac{y}{1-z} \\[8pt]
                        z               
                    \end{array}
                \right)}&&
        {%\tiny
        {\scriptstyle\hat\bz_5} = 
        \left(
                    \begin{array}{c}
                        x \\[8pt]               
                        y \\[8pt]
                        z-1                 
                    \end{array}
                \right)}&\\
        \IEEEeqnarrayseprow[10pt]
    \end{IEEEeqnarraybox*}
\end{table}}

\def\facesOfPyramid{
\begin{table}[!ht]
    \centering  
    \caption{Notation for the faces and positive normals of the
    reference pyramid.}
    \label{pyramidNotationTableFaces}
    \begin{IEEEeqnarraybox*}
      [\IEEEeqnarraystrutmode
      \IEEEeqnarraystrutsizeadd{2pt}{6pt}]{v/c/x/c/x/c/x/v/x/c/x/c/x/c/v/}
        \IEEEeqnarrayrulerow\\
        \IEEEeqnarrayseprow[5pt]\\
          & \hat f_1 && \subseteq &&  \{\hat x_2 = 0 \}            && && \hat{\bn}_1 && = && (0,-1,0)' & \\
        \IEEEeqnarrayrulerow\\
        \IEEEeqnarrayseprow[5pt]\\
          & \hat f_2 && \subseteq &&  \{\hat x_1 = 0 \}            && && \hat{\bn}_2 && = && (-1,0,0)' &\\
        \IEEEeqnarrayrulerow\\
        \IEEEeqnarrayseprow[5pt]\\
          & \hat f_3 && \subseteq &&  \{\hat x_1 + \hat x_3 = 1 \} && && \hat{\bn}_3 && = && 2^{-\nicefrac{1}{2}}(1,0,1)' &\\
        \IEEEeqnarrayrulerow\\
        \IEEEeqnarrayseprow[5pt]\\
          & \hat f_4 && \subseteq &&  \{\hat x_2 + \hat x_3 = 1 \} && && \hat{\bn}_4 && = && 2^{-\nicefrac{1}{2}}(0,1,1)' &\\
        \IEEEeqnarrayrulerow\\
        \IEEEeqnarrayseprow[5pt]\\
          & \hat f_5 && \subseteq &&  \{\hat x_3 = 0\}             && && \hat{\bn}_5 && = && (0,0,-1)' &\\
        \IEEEeqnarrayrulerow
    \end{IEEEeqnarraybox*}
\end{table}}

\def\edgesOfPyramid{
\begin{table}[!ht]
    \centering  
    \caption{Notation for the edges and positive tangents of the
    reference pyramid.}
    \label{pyramidNotationTableEdges}
    \begin{IEEEeqnarraybox*}
      [\IEEEeqnarraystrutmode
      \IEEEeqnarraystrutsizeadd{2pt}{6pt}]{v/c/x/c/x/c/x/v/x/c/x/c/x/c/v/}
        \IEEEeqnarrayrulerow\\
        \IEEEeqnarrayseprow[5pt]\\
   & \hat \be_1 && = && \{(\hat x_1,0,0)^t\,:\,0\leqslant\hat x_1\leqslant 1\} && && \hat \btau_1 && = && (1,0,0)' & \\
        \IEEEeqnarrayrulerow\\
        \IEEEeqnarrayseprow[5pt]\\
   & \hat \be_2 && = && \{(1,\hat x_2,0)^t\,:\,0\leqslant\hat x_2\leqslant 1\} && && \hat \btau_2 && = && (0,1,0)' & \\
        \IEEEeqnarrayrulerow\\
        \IEEEeqnarrayseprow[5pt]\\
   & \hat \be_3 && = && \{(\hat x_1,1,0)^t\,:\,0\leqslant\hat x_1\leqslant 1\} && && \hat \btau_3 && = && (-1,0,0)' & \\
        \IEEEeqnarrayrulerow\\
        \IEEEeqnarrayseprow[5pt]\\
   & \hat \be_4 && = && \{(0,\hat x_2,0)^t\,:\,0\leqslant\hat x_2\leqslant 1\} && && \hat \btau_4 && = && (0,1,0)' & \\
        \IEEEeqnarrayrulerow\\
        \IEEEeqnarrayseprow[5pt]\\
   & \hat \be_5 && = && \{(0,0,\hat x_3)^t\,:\,0\leqslant\hat x_3\leqslant 1\} && && \hat \btau_5 && = && (0,0,1)' & \\
        \IEEEeqnarrayrulerow\\
        \IEEEeqnarrayseprow[5pt]\\
   & \hat \be_6 && = && \{(1-\hat x_3,0,\hat x_3)^t\,:\,0\leqslant\hat x_3\leqslant 1\} && && \hat \btau_6 && = && 2^{-\nicefrac{1}{2}}(-1,0,1)' & \\
        \IEEEeqnarrayrulerow\\
        \IEEEeqnarrayseprow[5pt]\\
   & \hat \be_7 && = && \{(0,1-\hat x_3,\hat x_3)^t\,:\,0\leqslant\hat x_3\leqslant 1\} && && \hat \btau_7 && = && 2^{-\nicefrac{1}{2}}(0,-1,1)' & \\
        \IEEEeqnarrayrulerow\\
        \IEEEeqnarrayseprow[5pt]\\
   & \hat \be_8 && = && \{(1-\hat x_3,1-\hat x_3,\hat x_3)^t\,:\,0\leqslant\hat x_3\leqslant 1\} && && \hat \btau_8 && = && 3^{-\nicefrac{1}{2}}(-1,-1,1) & \\
        \IEEEeqnarrayrulerow
    \end{IEEEeqnarraybox*}
\end{table}}