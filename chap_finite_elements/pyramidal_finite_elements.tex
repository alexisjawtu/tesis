\section{Pyramidal Finite Elements}
The Finite Elements defined here are the ones found in~\cite{gh99}. There the authors
perform a
construction of discrete differential
forms by solving a local interpolation problem on the reference pyramid. The
finite element in an arbitrary mesh pyramid is obtained by pushing forward
the vector proxies of those discrete forms.
\begin{table}[!h]
    \centering  
    \caption{Notation for the faces and positive normals of $\partial\hat{E}$.}
    \label{pyramidNotationTableFaces}
    \begin{IEEEeqnarraybox*}
      [\IEEEeqnarraystrutmode
      \IEEEeqnarraystrutsizeadd{2pt}{6pt}]{v/c/x/c/x/c/x/v/x/c/x/c/x/c/v/}
        \IEEEeqnarrayrulerow\\
        \IEEEeqnarrayseprow[5pt]\\
          & \hat f_1 && \subseteq &&  \{\hat x_2 = 0 \}            && && \hat{\bn}_1 && = && (0,-1,0)' & \\
        \IEEEeqnarrayrulerow\\
        \IEEEeqnarrayseprow[5pt]\\
          & \hat f_2 && \subseteq &&  \{\hat x_1 = 0 \}            && && \hat{\bn}_2 && = && (-1,0,0)' &\\
        \IEEEeqnarrayrulerow\\
        \IEEEeqnarrayseprow[5pt]\\
          & \hat f_3 && \subseteq &&  \{\hat x_1 + \hat x_3 = 1 \} && && \hat{\bn}_3 && = && 2^{-\nicefrac{1}{2}}(1,0,1)' &\\
        \IEEEeqnarrayrulerow\\
        \IEEEeqnarrayseprow[5pt]\\
          & \hat f_4 && \subseteq &&  \{\hat x_2 + \hat x_3 = 1 \} && && \hat{\bn}_4 && = && 2^{-\nicefrac{1}{2}}(0,1,1)' &\\
        \IEEEeqnarrayrulerow\\
        \IEEEeqnarrayseprow[5pt]\\
          & \hat f_5 && \subseteq &&  \{\hat x_3 = 0\}             && && \hat{\bn}_5 && = && (0,0,-1)' &\\
        \IEEEeqnarrayrulerow
    \end{IEEEeqnarraybox*}
\end{table}
\begin{table}[!h]
    \centering  
    \caption{Notation for the edges and positive tangents of $\partial\hat{E}$.}
    \label{pyramidNotationTableEdges}
    \begin{IEEEeqnarraybox*}
      [\IEEEeqnarraystrutmode
      \IEEEeqnarraystrutsizeadd{2pt}{6pt}]{v/c/x/c/x/c/x/v/x/c/x/c/x/c/v/}
        \IEEEeqnarrayrulerow\\
        \IEEEeqnarrayseprow[5pt]\\
   & \hat \be_1 && = && \{(\hat x_1,0,0)^t\,:\,0\leqslant\hat x_1\leqslant 1\} && && \hat \btau_1 && = && (1,0,0)' & \\
        \IEEEeqnarrayrulerow\\
        \IEEEeqnarrayseprow[5pt]\\
   & \hat \be_2 && = && \{(1,\hat x_2,0)^t\,:\,0\leqslant\hat x_2\leqslant 1\} && && \hat \btau_2 && = && (0,1,0)' & \\
        \IEEEeqnarrayrulerow\\
        \IEEEeqnarrayseprow[5pt]\\
   & \hat \be_3 && = && \{(\hat x_1,1,0)^t\,:\,0\leqslant\hat x_1\leqslant 1\} && && \hat \btau_3 && = && (-1,0,0)' & \\
        \IEEEeqnarrayrulerow\\
        \IEEEeqnarrayseprow[5pt]\\
   & \hat \be_4 && = && \{(0,\hat x_2,0)^t\,:\,0\leqslant\hat x_2\leqslant 1\} && && \hat \btau_4 && = && (0,1,0)' & \\
        \IEEEeqnarrayrulerow\\
        \IEEEeqnarrayseprow[5pt]\\
   & \hat \be_5 && = && \{(0,0,\hat x_3)^t\,:\,0\leqslant\hat x_3\leqslant 1\} && && \hat \btau_5 && = && (0,0,1)' & \\
        \IEEEeqnarrayrulerow\\
        \IEEEeqnarrayseprow[5pt]\\
   & \hat \be_6 && = && \{(1-\hat x_3,0,\hat x_3)^t\,:\,0\leqslant\hat x_3\leqslant 1\} && && \hat \btau_6 && = && 2^{-\nicefrac{1}{2}}(-1,0,1)' & \\
        \IEEEeqnarrayrulerow\\
        \IEEEeqnarrayseprow[5pt]\\
   & \hat \be_7 && = && \{(0,1-\hat x_3,\hat x_3)^t\,:\,0\leqslant\hat x_3\leqslant 1\} && && \hat \btau_7 && = && 2^{-\nicefrac{1}{2}}(0,-1,1)' & \\
        \IEEEeqnarrayrulerow\\
        \IEEEeqnarrayseprow[5pt]\\
   & \hat \be_8 && = && \{(1-\hat x_3,1-\hat x_3,\hat x_3)^t\,:\,0\leqslant\hat x_3\leqslant 1\} && && \hat \btau_8 && = && 3^{-\nicefrac{1}{2}}(-1,-1,1) & \\
        \IEEEeqnarrayrulerow
    \end{IEEEeqnarraybox*}
\end{table}
\begin{figure}[!h]
\centering
  \unitTangentsPyramid
  \caption{Directions of the positive unit tangents (cfr. Table~\ref{pyramidNotationTableEdges}).}
  \label{reference_pyramid}
\end{figure}

\subsection{Edge} % (fold)
\label{sub:edge}
\begin{table}[!h]
    \centering  
    \caption{{Edge Shape Functions} on $\hat{E}$}
    \label{shape_edge_table}
    \begin{IEEEeqnarraybox*}
    [\IEEEeqnarraystrutmode
    \IEEEeqnarraystrutsizeadd{2pt}{25pt}]{x/c/x/c/x/c/x}
        \IEEEeqnarrayseprow[5pt]\\
        &\IEEEeqnarraymulticol{6}{c}{
                {%\tiny
                {\scriptstyle\bgamma_1} = 
                \left(
                            \begin{array}{c}
                                {1-z-y} \\[5pt]             
                                0 \\[5pt]
                                x-\frac{xy}{1-z}               
                            \end{array}
                \right)}\;\;
                {%\tiny
                {\scriptstyle\bgamma_2} = 
                \left(
                    \begin{array}{c}
                        0 \\[5pt]
                        x \\[5pt]                
                        \dfrac{xy}{1-z}               
                    \end{array}
                \right)}\;\;
                {%\tiny
                {\scriptstyle\bgamma_3} = 
                \left(
                    \begin{array}{c}
                        y \\[5pt]
                        0 \\[5pt]                
                        \dfrac{xy}{1-z}               
                    \end{array}
                \right)}\;\;
        {%\tiny
        {\scriptstyle\bgamma_4} = 
        \left(
                    \begin{array}{c}
                        0 \\[8pt]                
                        {1-z-x} \\[8pt]
                        y-\dfrac{xy}{1-z}  
                    \end{array}
                \right)}}\\
        \IEEEeqnarrayseprow[5pt]\\
        &\IEEEeqnarraymulticol{6}{c}{
        {%\tiny
        {\scriptstyle\bgamma_5} = 
                \left(
                        \begin{array}{c}
                            z-\dfrac{yz}{1-z} \\[8pt]               
                            z-\dfrac{xz}{1-z} \\[8pt]               
                            1-x-y+\dfrac{xy}{1-z}-\dfrac{xyz}{(1-z)^2}               
                        \end{array}
               \right)}\;\;
                {%\tiny
                {\scriptstyle\bgamma_6} = 
                \left(
                    \begin{array}{c}
                        -z+\dfrac{yz}{1-z} \\[8pt]               
                        \dfrac{xz}{1-z} \\[8pt]               
                        x-\dfrac{xy}{1-z}+\dfrac{xyz}{(1-z)^2}               
                    \end{array}
                        \right)}}\\
        \IEEEeqnarrayseprow[5pt]\\
        &\IEEEeqnarraymulticol{6}{c}{
        {%\tiny
        {\scriptstyle\bgamma_7} = 
                \left(
                        \begin{array}{c}
                            \dfrac{yz}{1-z} \\[8pt]               
                            -z+\dfrac{xz}{1-z} \\[8pt]               
                            y-\dfrac{xy}{1-z}+\dfrac{xyz}{(1-z)^2}               
                        \end{array}
               \right)}\;\;
        {%\tiny
                {\scriptstyle\bgamma_8} = 
                \left(
                    \begin{array}{c}
                        -\dfrac{yz}{1-z} \\[8pt]               
                        -\dfrac{xz}{1-z} \\[8pt]               
                        \dfrac{xy}{1-z}-\dfrac{xyz}{(1-z)^2}               
                    \end{array}
                        \right)}}
    \end{IEEEeqnarraybox*}
\end{table}
\begin{defi}\label{aux_label50}
  The following items define a least order $\bcurl$--conforming finite element
  on the reference Pyramid.
  \begin{enumerate}
    \item $\hat E$ is the reference Pyramid in Definition~\ref{defi_of_ref_pyr}. 
    \item The rational space $P_{\hat E}$ is the span of
    $\{\hat{\bgamma}_1,\,\ldots,\,\hat{\bgamma}_8\}$ with $\hat{\bgamma}_i$
    as in Table~\ref{shape_edge_table}.
    \item The degrees of freedom are the line integrals
      \begin{IEEEeqnarray*}{c}
        \int_{\hat\be_j}\hat\bu\cdot\,d\hat\balpha
      \end{IEEEeqnarray*}
      for every edge $\hat\be_j$ of $\hat E$, $1\leqslant j\leqslant 8$.
  \end{enumerate}
\end{defi}
\begin{remark}
  These rational funtions satisfy $\int_{e_j}\bgamma_i\cdot\btau = \delta_{ij}$.   
\end{remark}
\begin{lemma}
  The finite element in Definition~(\ref{aux_label50})  is $H(\bcurl)$--conforming and 
  unisolvent.
\end{lemma}
% subsection edge (end)
\subsection{Face} % (fold)
\label{sub:face}
\begin{table}[!h]
    \centering  
    \caption{\emph{Face Shape Functions} on $\hat{E}$}
    \label{shape_face_table}
    \begin{IEEEeqnarraybox*}
    [\IEEEeqnarraystrutmode
    \IEEEeqnarraystrutsizeadd{2pt}{25pt}]{x/c/x/c/x/c/x}
        \IEEEeqnarrayseprow[5pt]\\
        &\IEEEeqnarraymulticol{5}{c}{
                \raisebox{-15pt}{}\;\;
                {%\tiny
                {\scriptstyle\bzeta_1} = 
                \left(
                            \begin{array}{c}
                                -\frac{xz}{1-z} \\[8pt]             
                                y-2+\frac{y}{1-z} \\[8pt]
                                z               
                            \end{array}
                        \right)}\;\;
                {%\tiny
                {\scriptstyle\zeta_2} = 
                \left(
                            \begin{array}{c}
                                x-2+\dfrac{x}{1-z} \\[8pt]
                                -\dfrac{yz}{1-z} \\[8pt]                
                                z               
                            \end{array}
                        \right)}}&\\
        \IEEEeqnarrayseprow[10pt]\\
        &
        {%\tiny
        {\scriptstyle\zeta_3} = 
        \left(
                    \begin{array}{c}
                        x+\dfrac{x}{1-z} \\[8pt]
                        -\dfrac{yz}{1-z} \\[8pt]                
                        z               
                    \end{array}
                \right)}&&
        {%\tiny
        {\scriptstyle\zeta_4} = 
        \left(
                    \begin{array}{c}
                        -\dfrac{xz}{1-z} \\[8pt]                
                        y+\dfrac{y}{1-z} \\[8pt]
                        z               
                    \end{array}
                \right)}&&
        {%\tiny
        {\scriptstyle\zeta_5} = 
        \left(
                    \begin{array}{c}
                        x \\[8pt]               
                        y \\[8pt]
                        z-1                 
                    \end{array}
                \right)}&\\
        \IEEEeqnarrayseprow[10pt]
    \end{IEEEeqnarraybox*}
\end{table}
\begin{defi}\label{defi_h_div_conforme_pyramid} Pyramidal $H\dv$--conforming 
Finite Element.
\begin{enumerate}
    \item $\hat{E}$ is the reference Pyramid (Figure~\ref{reference_pyramid}).
    \item El espacio $Q_{\hat{E}}$ is $\mbox{span}\{\hat{\bz}_1,\,\ldots,\,\hat{\bz}_5\}$ with $\hat{\bz}_i$
    as in Table~\ref{shape_face_table}
    \item The degrees of freedom are:
\begin{IEEEeqnarray}{lcl}
    \label{dofsdivpyramid} \int\limits_{\hat{f}} \bv\cdot\boldsymbol{\nu}\,d\hat{\gamma}
        && 
\end{IEEEeqnarray}
\end{enumerate}
\end{defi}
\begin{remark}
  These rational funtions satisfy $\iint_{f_j}\bz_i\cdot\bn = \delta_{ij}$   
\end{remark}
\begin{lemma}
  The finite element~(\ref{aux_label50})  is $H(div)$--conforming and 
  unisolvent.
\end{lemma}
% subsection face (end)