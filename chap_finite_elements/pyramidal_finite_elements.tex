\section{Pyramidal Finite Elements}
The Finite Elements defined here are the ones found in~\cite{gh99}. There the authors
perform a
construction of discrete differential
forms by solving a local interpolation problem on the reference pyramid. The
finite element in an arbitrary mesh pyramid is obtained by pushing forward
the vector proxies of those discrete forms.
\begin{table}[!h]
    \centering  
    \caption{Notation for the faces and positive normals of the
    reference pyramid.}
    \label{pyramidNotationTableFaces}
    \begin{IEEEeqnarraybox*}
      [\IEEEeqnarraystrutmode
      \IEEEeqnarraystrutsizeadd{2pt}{6pt}]{v/c/x/c/x/c/x/v/x/c/x/c/x/c/v/}
        \IEEEeqnarrayrulerow\\
        \IEEEeqnarrayseprow[5pt]\\
          & \hat f_1 && \subseteq &&  \{\hat x_2 = 0 \}            && && \hat{\bn}_1 && = && (0,-1,0)' & \\
        \IEEEeqnarrayrulerow\\
        \IEEEeqnarrayseprow[5pt]\\
          & \hat f_2 && \subseteq &&  \{\hat x_1 = 0 \}            && && \hat{\bn}_2 && = && (-1,0,0)' &\\
        \IEEEeqnarrayrulerow\\
        \IEEEeqnarrayseprow[5pt]\\
          & \hat f_3 && \subseteq &&  \{\hat x_1 + \hat x_3 = 1 \} && && \hat{\bn}_3 && = && 2^{-\nicefrac{1}{2}}(1,0,1)' &\\
        \IEEEeqnarrayrulerow\\
        \IEEEeqnarrayseprow[5pt]\\
          & \hat f_4 && \subseteq &&  \{\hat x_2 + \hat x_3 = 1 \} && && \hat{\bn}_4 && = && 2^{-\nicefrac{1}{2}}(0,1,1)' &\\
        \IEEEeqnarrayrulerow\\
        \IEEEeqnarrayseprow[5pt]\\
          & \hat f_5 && \subseteq &&  \{\hat x_3 = 0\}             && && \hat{\bn}_5 && = && (0,0,-1)' &\\
        \IEEEeqnarrayrulerow
    \end{IEEEeqnarraybox*}
\end{table}
\begin{table}[!h]
    \centering  
    \caption{Notation for the edges and positive tangents of the
    reference pyramid.}
    \label{pyramidNotationTableEdges}
    \begin{IEEEeqnarraybox*}
      [\IEEEeqnarraystrutmode
      \IEEEeqnarraystrutsizeadd{2pt}{6pt}]{v/c/x/c/x/c/x/v/x/c/x/c/x/c/v/}
        \IEEEeqnarrayrulerow\\
        \IEEEeqnarrayseprow[5pt]\\
   & \hat \be_1 && = && \{(\hat x_1,0,0)^t\,:\,0\leqslant\hat x_1\leqslant 1\} && && \hat \btau_1 && = && (1,0,0)' & \\
        \IEEEeqnarrayrulerow\\
        \IEEEeqnarrayseprow[5pt]\\
   & \hat \be_2 && = && \{(1,\hat x_2,0)^t\,:\,0\leqslant\hat x_2\leqslant 1\} && && \hat \btau_2 && = && (0,1,0)' & \\
        \IEEEeqnarrayrulerow\\
        \IEEEeqnarrayseprow[5pt]\\
   & \hat \be_3 && = && \{(\hat x_1,1,0)^t\,:\,0\leqslant\hat x_1\leqslant 1\} && && \hat \btau_3 && = && (-1,0,0)' & \\
        \IEEEeqnarrayrulerow\\
        \IEEEeqnarrayseprow[5pt]\\
   & \hat \be_4 && = && \{(0,\hat x_2,0)^t\,:\,0\leqslant\hat x_2\leqslant 1\} && && \hat \btau_4 && = && (0,1,0)' & \\
        \IEEEeqnarrayrulerow\\
        \IEEEeqnarrayseprow[5pt]\\
   & \hat \be_5 && = && \{(0,0,\hat x_3)^t\,:\,0\leqslant\hat x_3\leqslant 1\} && && \hat \btau_5 && = && (0,0,1)' & \\
        \IEEEeqnarrayrulerow\\
        \IEEEeqnarrayseprow[5pt]\\
   & \hat \be_6 && = && \{(1-\hat x_3,0,\hat x_3)^t\,:\,0\leqslant\hat x_3\leqslant 1\} && && \hat \btau_6 && = && 2^{-\nicefrac{1}{2}}(-1,0,1)' & \\
        \IEEEeqnarrayrulerow\\
        \IEEEeqnarrayseprow[5pt]\\
   & \hat \be_7 && = && \{(0,1-\hat x_3,\hat x_3)^t\,:\,0\leqslant\hat x_3\leqslant 1\} && && \hat \btau_7 && = && 2^{-\nicefrac{1}{2}}(0,-1,1)' & \\
        \IEEEeqnarrayrulerow\\
        \IEEEeqnarrayseprow[5pt]\\
   & \hat \be_8 && = && \{(1-\hat x_3,1-\hat x_3,\hat x_3)^t\,:\,0\leqslant\hat x_3\leqslant 1\} && && \hat \btau_8 && = && 3^{-\nicefrac{1}{2}}(-1,-1,1) & \\
        \IEEEeqnarrayrulerow
    \end{IEEEeqnarraybox*}
\end{table}
\begin{figure}[!h]
\centering
  \unitTangentsPyramid
  \caption{Directions of the positive unit tangents (cfr. Table~\ref{pyramidNotationTableEdges}).}
  \label{reference_pyramid}
\end{figure}

\subsection{$H(\bcurl)$--Conforming Element on Pyramids} % (fold)
\label{sub:edge}
\begin{defi}\label{aux_label50}
  The following items define a least order $\bcurl$--conforming finite element
  on the reference Pyramid.
  \begin{enumerate}
    \item $\hat E$ is the reference Pyramid in Definition~\ref{defi_of_ref_pyr}. 
    \item The rational space $P_{\hat E}$ is the span of
    $\{\hat{\bgamma}_1,\,\ldots,\,\hat{\bgamma}_8\}$ with $\hat{\bgamma}_i$
    as in Table~\ref{shape_edge_table}.
    \item The degrees of freedom are the line integrals
      \begin{IEEEeqnarray*}{c}
        \int_{\hat\be_j}\hat\bu\cdot\,d\hat\balpha
      \end{IEEEeqnarray*}
      for every edge $\hat\be_j$ of $\hat E$, $1\leqslant j\leqslant 8$.
  \end{enumerate}
\end{defi}
\edgeShapeTable
A direct computation yields the following Lemma.
\begin{lemma}
  For $1\leqslant i,j\leqslant 8$,
  $\int_{\hat\be_j}\hat\bgamma_i\cdot d\hat\balpha = \delta_{ij}$ which
  implies immediately that the finite element in Definition~\ref{aux_label50}
  is unisolvent in $\hat E$. %$H(\bcurl)$--conforming and 
\end{lemma}
% subsection edge (end)
\subsection{$H(\Div)$--Conforming Element on Pyramids} % (fold)
\label{sub:face}
\begin{defi}\label{aux_label71}
The following items define a least order $\Div$--conforming finite element
  on the reference Pyramid.
\begin{enumerate}
  \item $\hat{E}$ is the reference pyramid of Figure~\ref{reference_pyramid}.
  \item The space $P_{\hat{E}}$ is the span of 
  $\{\hat{\bz}_1,\,\ldots,\,\hat{\bz}_5\}$ with $\hat{\bz}_i$
    as in Table~\ref{shape_face_table}.
  \item The degrees of freedom are the surface integrals
  \begin{IEEEeqnarray*}{c}
    \label{dofsdivpyramid} \iint_{\hat{f}_j} \hat\bv\cdot\hat\bn\,d\hat S
  \end{IEEEeqnarray*}
  for every face $\hat{f}_j$ of $\hat E$, $1\leqslant j\leqslant 5$.
\end{enumerate}
\end{defi}
\faceShapeTable
A direct computation yields the following Lemma.
\begin{lemma}
  For $1\leqslant i,j\leqslant 5$,
  $\iint_{f_j}\hat\bz_i\cdot\hat\bn\,d\hat S = \delta_{ij}$ which
  implies immediately that the finite element in Definition~\ref{aux_label71}
  is unisolvent in $\hat E$. %$H(\bcurl)$--conforming and 
\end{lemma}
% subsection face (end)