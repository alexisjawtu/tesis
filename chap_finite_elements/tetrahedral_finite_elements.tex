\section{Tetrahedral Finite Elements}\label{sec:tetrahedralFEs}
\subsection{$H(\Div)$--Conforming Element on Tetrahedra} % (fold)
\label{sub:definition_of_the_h_div_element_on_tetrahedra}
For $k > 0$, let
\begin{IEEEeqnarray}{rCl}\label{tetrahedralSpace}
    P_{\hat{E}} & = & (P_{k-1})^3 + P_{k-1}\,\bx\\
    &=& (P_{k-1})^3 \otimes \tilde{P}_{k-1}\,\bx
\end{IEEEeqnarray}
where all the polynomial spaces are taken over $\hat{E}$.
\begin{lemma}
  The dimension of $P_{\hat{E}}$ is $\nicefrac{1}{2}(k+3)(k+1)k$.
\end{lemma}
\begin{lemma}\label{lema_div} $\dv P_{\hat E} = P_{k-1}$.
\noindent{\color{BrickRed}\#\#\#\#\#\#\# esto ya no lo tenemos en
virtuales :)}
\end{lemma}
\begin{defi}[Divergence conforming element] The element is defined by
\label{defi_face_element_tetra}
\begin{itemize}
    \item $\hat{E}$ is the reference tetrahedron of Definition~\ref{def_of_ref_elems}. 
    \item $P_{\hat{E}}$ is the polynomial space in~(\ref{tetrahedralSpace}).
    \item The degrees of freedom $\Sigma_{\hat{E}}$ are
    \begin{IEEEeqnarray*}{lll}
      \iint_{\hat f}\hat{q}\hat\bu\cdot\hat\bn\,d\hat{S}
      \quad  &\mbox{for all $\hat q\in P_{k-1}(\hat f)$}&\mbox{for all face $\hat f$ of $\hat E$}\\
      \int_{\hat E} \hat\bu\cdot\hat{\bq}\,d\hat\bx
      \quad  &\mbox{for all $\hat\bq\in (P_{k-2}(\hat E))^3$}&
    \end{IEEEeqnarray*} 
\end{itemize}
\end{defi}
% subsection definition_of_the_h_div_element_on_tetrahedra

% subsection defEdgeElement (end)
% \begin{proposition} On the Finite Element~(\ref{defi_h_div_conforme}). 
% $\dim(P_{\hat{E}}) = k^2\,(k+2) + k\,(k+1)^2/2$
% which is, at the same time, equal to the number of its independent degrees of freedom.
% \end{proposition}
% \begin{proof}
%     From~(\ref{tensor_prod_dim}) and~(\ref{dk}) it is really straightforward to do
%     \begin{IEEEeqnarray*}{rCl}
%         \dim (P_{k-1}^2(x,y) \oplus \tilde{P}_{k-1}(x,y)) \otimes P_{k-1}(z) + \dim P_{k-1}(x,y)\otimes P_k(z) & = &\\[5pt]
%         \IEEEeqnarraymulticol{3}{r}{=\,(k(k+1)+k)k+\dfrac{k(k+1)}{2}(k+1)} 
%     \end{IEEEeqnarray*}
%     and the same quantity is obtained by summing up the dimensions of all the
%     \emph{test} spaces on the right of~(\ref{momentos1hdiv})--(\ref{momentos4hdiv}).
% \end{proof}