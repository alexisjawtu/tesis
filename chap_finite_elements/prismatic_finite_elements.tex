\section{Prismatic Finite Elements}
%>this intends to be an extension of the treatment
%in~\cite{giraultRaviart} to prismatic elements?
\subsection{$H(\bcurl)$--Conforming Element on Prisms} % (fold)
\label{sub:defEdgeElement}
First we introduce two more polynomial spaces on the reference prism of
Figure~\ref{reference_prism}.
$\hat{T}$ denotes the triangle in Definition~\ref{defi_of_ref_prism} and 
$\hat I = [0,1]$.
\begin{defi} For an integer $k\geqslant 1$, let $R_k(\hat{T})$ denote the space of polynomials, defined over the
triangle $\hat{T}$, given by
\begin{IEEEeqnarray}{rCl}
    \label{defRk}
    R_k(\hat{T}) & := & P_{k-1}(\hat{T})^2 \oplus S_k(\hat{T})
\end{IEEEeqnarray}
where
\begin{IEEEeqnarray}{rClCrCl}
    \label{defSk}
    S_k(\hat{T}) & := & \{ \bp\in \tilde{P}_k^2 \,:\;\bp\cdot\hat\bx =
    0\}\mbox{,}\quad\hat\bx & = & (\hat x_1, \hat x_2)'.
\end{IEEEeqnarray}
\end{defi}
\facesOfPrism
\edgesOfPrism
\begin{figure}[!h]
  \centering
  \subfloat
  {
    \label{unitTanPrism}
    \unitTangentsPrism
  }
  \caption{Directions of positive unit tangents (cfr. Table~\ref{prismNotationTableEdges}).}
\end{figure}
\begin{defi}\label{edgeelement} Given a natural $k$, the $H(\bcurl)$--Conforming 
Finite Element of degree $k$ is defined by the following triple.
\begin{enumerate}
  \item $\hat{E}$ is the reference prism in Definition~\ref{defi_of_ref_prism}.
  \item The polynomial space $P_{\hat{E}}$ is
        \begin{IEEEeqnarray}{rCl} \label{spaceFEprismHcurl}
            P_{\hat{E}} & = & R_k(\hat{T}) \otimes P_k(\hat{I}) \times 
            P_k(\hat{T}) \otimes P_{k-1}(\hat{I}).
         \end{IEEEeqnarray} 
  \item The degrees of freedom are (cfr. Tables~\ref{prismNotationTableFaces} 
and~\ref{prismNotationTableEdges}):
  {\color{Orange} ver si en~\ref{momentos2hcurl} y los otros de caras
    hay que cambiar orden de componentes o signos por lo que acomod'e para definir el elemento
    af'in; ver~(\ref{momentos2hcurlPhys})}.
\begin{IEEEeqnarray}{ll}
    \label{momentos1hcurl}  
    \hat\varphi_{\hat{\be},\hat{q}}\,(\hat\bu) = 
    \int_{\hat\be} \hat q\,\hat\bu\cdot d\hat\balpha\mbox{,}  
      & \hat q\in P_{k-1}(\hat\be)\mbox{, for each edge $\hat\be$;}\\[8pt]%with unit tangent } \boldsymbol{\tau} \mbox{;}}
    \nonumber\hat\varphi_{\hat f,\hat\bq}\,(\hat\bu) =  
    \int_{\hat f} \hat\bu \times \hat\bn \cdot \hat\bq\,
    d\hat S\mbox{, }\quad&\hat\bq = (\hat q_1,\hat q_2,0) \in P_{k-2}^2 \times \{ 0 \},\\[4pt] 
    \label{momentos2hcurl} 
      &\mbox{ for each face $f=\hat f_3$ or$\hat f_4$;}\\[8pt]
    \nonumber\hat\varphi_{\hat f_1,\hat\bq}\,(\hat\bu) =  
    \int_{\hat f_1} \hat\bu \times \hat\bn_1 \cdot \hat\bq\,
    d\hat S\mbox{, }\quad&
      \hat\bq = (0,\hat q_3,\hat q_2)\mbox{, }\hat q_3\in Q_{k-2,k-1}\mbox{,}\\[4pt]
    \label{momentos3hcurl}
      &\hat q_2 \in Q_{k-1,k-2}\mbox{;}\\[8pt]   %\mbox{, for the face } \hat f_1
    \nonumber\hat\varphi_{\hat f_2,\hat\bq}\,(\hat\bu) =  
    \int_{\hat f_2} \hat\bu \times \hat\bn_2 \cdot \hat\bq\,
    d\hat S\mbox{, }\quad& 
      \hat\bq = (\hat q_3,0,\hat q_1)\mbox{, }\hat q_3 \in Q_{k-2,k-1}\mbox{,}\\[4pt]
    \label{momentos4hcurl}
      &\hat q_1 \in Q_{k-1,k-2}\mbox{;}\\[8pt]   %\mbox{, for the face } \hat f_2
    \nonumber\hat\varphi_{\hat f_5,\hat\bq}\,(\hat\bu) =  
    \int_{\hat f_5} \hat\bu \times \hat\bn_5 \cdot \hat\bq\,
    d\hat S\mbox{, }\quad&
      \hat\bq = (0,\hat q_3,\hat q_1)\mbox{, }\hat q_3 \in Q_{k-2,k-1}\mbox{,}\\[4pt]
    \label{momentos5hcurl}
      &\hat q_1 \in Q_{k-1,k-2}\mbox{;}\\[8pt]   %\mbox{, for the face } \hat f_5
    \nonumber\hat\varphi_{\hat\br}\,(\hat\bu) = 
    \int_{\hat{E}} \hat\bu \cdot \hat\br \, d\hat\bx\mbox{, }&\hat r_1\mbox{, } 
    \hat r_2 \in P_{k-2,k-2}\mbox{, }\\[4pt]
    \label{momentos6hcurl}
      &\hat r_3 \in P_{k-3,k-1}.
\end{IEEEeqnarray}
\end{enumerate}
\end{defi}

%%==============================================================================
%% \noindent{\color{blue}\#\#\#\#\#\#\# poner mas sinteticos los dofs de superficie
%% aca arriba y aca abajo aclararlos para mostrar como se computan}
%% In order to clarify how to compute the degrees of
%% freedom~(\ref{momentos2hcurl})--(\ref{momentos5hcurl}) for an implementation
%% we write their test spaces more explicitly here.
%% \begin{IEEEeqnarray}{ll}
%%     (\ref{momentos2hcurl}) \int\limits_{f} \textbf{u} \times \boldsymbol{\nu} \cdot \hat\bq\,
%%     d\gamma\mbox{, } &\bq = (q_1,q_2,0) \in P_{k-2}^2 \times \{ 0 \},\\ 
%%     \IEEEeqnarraymulticol{2}{l}{\nonumber\mbox{ for each horizontal face $f$ with normal } \boldsymbol{\nu} = (0,0,\pm1) \mbox{;}}\\[8pt]
%%     (\ref{momentos3hcurl}) \int\limits_{f} \textbf{u} \times \boldsymbol{\nu} \cdot \bq\,
%%     d\gamma\mbox{, } &\bq = (0,q_3,q_2) \in \{ 0 \} \times Q_{k-2,k-1} \times 
%%     Q_{k-1,k-2}\mbox{, } \\
%%     \IEEEeqnarraymulticol{2}{l}{\nonumber\mbox{ for the face } f \subseteq \{ x=0 \} \mbox{ with normal }\boldsymbol{\nu} = (-1,0,0) \mbox{;}}\\[8pt]
%%     (\ref{momentos4hcurl}) \int\limits_{f} \textbf{u} \times \hat\bn \cdot \bq\,
%%     d\gamma\mbox{, } & \bq = (q_3,0,q_1) \in Q_{k-2,k-1} \times \{ 0 \} \times
%%     Q_{k-1,k-2},\\
%%     \IEEEeqnarraymulticol{2}{l}{\nonumber\mbox{ for the face } f \subseteq \{ y=0 \} \mbox{ with normal }\boldsymbol{n} = (0,-1,0) \mbox{;}}\\[8pt]
%%     (\ref{momentos5hcurl}) \int\limits_{f} \textbf{u} \times \boldsymbol{n} \cdot \bq\,
%%     d\gamma\mbox{, } & \bq = (0,q_3,q_1) \in \{ 0 \} \times Q_{k-2,k-1} \times
%%     Q_{k-1,k-2}\mbox{, }\\
%%     \IEEEeqnarraymulticol{2}{l}{\nonumber\mbox{ for the face }f \subseteq \{x+y=1\} \mbox{ with normal }\boldsymbol{n} = (1,1,0) \mbox{;}}
%% \end{IEEEeqnarray}
%% \noindent{\color{blue}\#\#\#\#\#\#\# }
%%==============================================================================
In the following Remark we make an explicitation of the elements
of $P_{\hat E}$.
\begin{remark} \label{aux_label6}
Take $\hat{\textbf{s}}=(\hat s_1,\hat s_2)\in {S}_k$ as defined in~(\ref{defSk}).
Set $\hat s_1 = \sum_{i+j=k} a_{i, j}\,\hat x_1^i\hat x_2^j$, 
$\hat s_2 = \sum_{i+j=k} b_{i, j}\,\hat x_1^i\hat x_2^j$.
By definition is 
\begin{IEEEeqnarray*}{rCl}
    0&=&\hat x_1\hat s_1 + \hat x_2\hat s_2\\[4pt]
     &=&a_{k,0}\hat x_1^{k+1} + b_{0,k}\hat x_2^{k+1}+\sum_{i+j=k}(a_{i-1,j+1} + b_{i, j})\hat x_1^i\hat x_2^{j+1}
\end{IEEEeqnarray*}
so $a_{k,0} = b_{0,k} = 0$ and for all pair $(i,j)$ with $i+j=k$ and
$i\geqslant 1$ it holds the relation $a_{i-1, j+1} = -b_{i, j}$.
Then
\begin{IEEEeqnarray*}{rCcCl}
    \hat s_1 & = & \sum_{i+j = k,\,j\geqslant 1} a_{i, j}\,\hat x_1^i\hat x_2^j
        & = & \hat x_2\sum_{i+j = k,\,j\geqslant 1} a_{i, j}\,\hat x_1^i\hat x_2^{j-1} \\[5pt]
    \hat s_2 & = & -\sum_{i+j = k,\,j \geqslant 1} a_{i, j}\,\hat x_1^{i+1}\hat x_2^{j-1}
        & = & -\hat x_1\sum_{i+j = k,\,j \geqslant 1} a_{i, j}\,\hat x_1^{i}\hat x_2^{j-1}.
\end{IEEEeqnarray*}
So any $\boldsymbol{p} \in P_{\hat E}$ may be written as
\begin{IEEEeqnarray*}{rCl}
  \boldsymbol{p} & =   & (p_1, p_2, p_3)'\\
  \yesnumber\label{elemento_P_k} & =   & (\xi_1 + x_2\,h, \xi_2 - x_1\,h, \xi_3), \\[6pt]
  \xi_1, \xi_2   & \in & P_{k-1}(x_1,x_2) \otimes P_k(x_3),\\
         \xi_3   & \in & P_{k}(x_1,x_2) \otimes P_{k-1}(x_3),\\
             h   & \in & \tilde{P}_{k-1}(x_1,x_2) \otimes P_k(x_3).\\
\end{IEEEeqnarray*}
\end{remark}
An illustrative example.
\begin{example}[edge elements of degree 1]
\begin{IEEEeqnarray*}{rCl}
\hat{\textbf{u}}\,\xyz &=& 
\left(
    \begin{array}{c}
        a_1 + a_3\hat{y} + a_4\hat{z} + a_6\hat{y}\hat{z} \\[8pt]
        a_2 - a_3\hat{x} + a_5\hat{z} - a_6\hat{x}\hat{z} \\[8pt]
        a_7 + a_8\hat{x} + a_9\hat{y}
    \end{array}
\right)\mbox{,}\\[10pt]
(\hat{\textbf{u}}\cdot\hat{\boldsymbol{\tau}})|_{\hat{\be}}
    &\in&\mathbb{P}_0(\hat{\be}).
\end{IEEEeqnarray*}
\end{example}
\begin{defi} the interpolator is well defined and bounded... ver
como dice monk o pensar y poner espacios.

Given $p>2$
$\boldsymbol{w}_{\hat{E}}\,:\,W^{1,p}(\hat{E})\to P_{\hat E}$
  \begin{IEEEeqnarray}{lClc}
    \label{aux_label45}\varphi_{\hat\be,\hat p}\,(\hat{\bu} - \wku) & = & 0 
      &\quad\mbox{for $\hat\be$ and $\hat p$ as in ~(\ref{momentos1hcurl}).}  \\
    \varphi_{f,\bq}\,(\hat{\bu} - \wku) & = & 0 &\quad\mbox{for $f$ and $\bq$ as in}  \\
    \varphi_{\boldsymbol{r}}\,(\hat{\bu} - \wku) & = & 0 &\quad\mbox{for $\br$ as in  }.
  \end{IEEEeqnarray}
\end{defi}
Because of the presence of degrees of freedom~(\ref{aux_label45})
no se puede definir
el interpolador para un campo arbitrario
en $H(\bcurl, \hat E)$, así que por eso to\-ma\-mos co\-mo hi\-pó\-te\-sis
que sea $p>2$ (ver~\cite{monk}, página 134,
~\cite{adams}, Theorem 5.4).
\begin{remark}[Dimesion of the space~(\ref{spaceFEprismHcurl}).]
$\dim R_k(\hat{T}) \otimes P_k(\hat{I})$.
\begin{IEEEeqnarray*}{rCl}
    \dim\left(R_k(\hat{T}) \otimes P_k(\hat{I})\right) 
    & = & \dim\left(R_k(\hat{T})\right) \dim\left(P_k(\hat{I})\right) \\
    & = & \dim\left(R_k(\hat{T})\right) (k+1).
\end{IEEEeqnarray*}
\begin{IEEEeqnarray*}{rCl}
    \dim\left(R_k(\hat{T})\right) 
    & = & \dim\left(P_{k-1}(\hat{T})^2 \oplus S_k(\hat{T}) \right)\\
    & = & 2\dim\left(P_{k-1}(\hat{T})\right) + \dim\left(S_k(\hat{T}) \right)\\
    & = & k(k+1) + \dim\left(S_k(\hat{T}) \right).
\end{IEEEeqnarray*}
Para la dimensi\'on de $S_k(\hat{T})$:
\begin{IEEEeqnarray*}{rCl}
S_k(\hat{T}) = \{ \textbf{p} \in \widetilde{P}_k^2 \, : \, \textbf{p}\cdot\textbf{x} = 0 \}.
\end{IEEEeqnarray*}
Consid\'erese
\begin{IEEEeqnarray*}{lll}
    \phi\,:\,\widetilde{P}_k^2 & \longrightarrow & \widetilde{P}_{k+1}\\
    \phi(\textbf{p})    & := & \textbf{p}\cdot\textbf{x}\\
                        & := & x_1p_1 + x_2p_2.
\end{IEEEeqnarray*}
Resulta
\begin{IEEEeqnarray*}{rCl}
    S_k(\hat{T})        & = & \ker(\phi)\\
    \dim(S_k(\hat{T}))  & = & \dim(\widetilde{P}_k^2) - \dim(\img(\phi)).
\end{IEEEeqnarray*}
cualquier $p \in \widetilde{P}_{k+1}$ es
\begin{IEEEeqnarray*}{rCl}
    x_1(a_{k+1,0} x_1^k + a_{k,1} x_1^{k-1}x_2 + \ldots + a_{1,k} x_2^k) + x_2(a_{0,k+1} x_2^k)
        & = & x_1p_1 + x_2p_2
\end{IEEEeqnarray*}
en donde precisamente $p_1$ y $p_2$ pertenecen a $\widetilde{P}_k$, es decir que $\phi$ es
sobreyectiva. Volviendo:
\begin{IEEEeqnarray*}{rCl}
    \dim(S_k(\hat{T}))  & = & \dim(\widetilde{P}_k^2) - \dim(\widetilde{P}_{k+1})\\
                            & = & 2 \dim(\widetilde{P}_k) - \dim(\widetilde{P}_{k+1})\\
                            & = & 2 (k+1) - (k+2)\\
                            & = & k,
\end{IEEEeqnarray*}
entonces
\begin{IEEEeqnarray*}{rCl}
    \dim\left(R_k(\hat{T})\right)   & = & k(k+1) + k\\
                                        & = & k(k+2)
\end{IEEEeqnarray*}
y finalmente
\begin{IEEEeqnarray*}{rCl}
    \dim\left(R_k(\hat{T}) \otimes P_k(\hat{I})\right) 
        & = & k(k+1)(k+2).
\end{IEEEeqnarray*}
($\dim\left(P_K\right) = 3\frac{k(k+1)(k+2)}{2}$).
\end{remark}
The following result can be found in~\cite[page 75]{nedelec2}
\begin{lemma}
  The Finite Element in Definition~\ref{edgeelement} is unisolvent in $E$.
\end{lemma}
Later we will establish the unisolvence of a general finite element of
this family on an arbitrary physical prism.
\begin{remark} The interpolation operator
can be written 
{\color{Orange}\#\#\#\#\# poner una basecita de 
polinomios para cada dof para después usarla
en la suma.}
\begin{IEEEeqnarray}{rCl}\label{edge_interp_explicit}  
  \wku & = & 
  \sum_{\be,p} \varphi_{\be,p}(\hat{\bu})\,\hat{\bv}_{\be,p} +
  \sum_{f,\bq} \varphi_{f,\bq}(\hat{\bu})\,\hat{\bv}_{f,\bq} +
  \sum_{\boldsymbol{r}}   \varphi_{\boldsymbol{r}}  (\hat{\bu})\,\hat{\bv}_{\boldsymbol{r}}
\end{IEEEeqnarray}
\end{remark}

\subsection{$H(\Div)$--Conforming Element on Prisms} % (fold)
\label{sub:definition_of_the_h_div_element_on_prisms}
With the notation introduced in Subsection~\ref{sub:polynomials} we consider
the polynomial space
\begin{IEEEeqnarray*}{rCl}
    \yesnumber\label{dk}
    D_k & = & P_{k-1}^2(x,y) \oplus \tilde{P}_{k-1}(x,y) \bx,\\
    \bx & = & (x,y).
\end{IEEEeqnarray*}
\begin{defi}\label{defi_h_div_conforme} The $RT$ element
\begin{enumerate}
  \item $\hat{E}$ is the reference prism (Figure~\ref{reference_prism}).
  \item The polynomial space $P_{\hat{E}}$ is
    \begin{IEEEeqnarray*}{rCl}
      P_{\hat{E}} & = & \{ \bv = (v_1,v_2,v_3):\,(v_1,v_2)\in D_k\otimes P_{k-1}(z),\\ 
      \yesnumber\label{prismaticSpace}&   & v_3\in P_{k-1}(x,y)\otimes P_k(z) \}.
    \end{IEEEeqnarray*} 
  \item The degrees of freedom are of two types, surface and volumen integrals:
\begin{IEEEeqnarray}{cCccl}
    \label{momentos1hdiv} 
    \rho_{\hat f,q}(\bv) & = & \int_{\hat f} (\hat\bv\cdot\hat{\boldsymbol{\nu}})\hat{q}\,d\hat{S} 
        &\quad & \mbox{for } \hat{q} \in P_{k-1}(\hat{f})\mbox{,}\\
    \nonumber&&&\quad&\mbox{if $\hat f = \hat f_3$ or $\hat f_4$;}\\[5pt]
    \label{momentos2hdiv}
    \rho_{\hat f,q}(\bv) & = & \int_{\hat f} (\hat\bv\cdot\hat{\boldsymbol{\nu}})\hat{q}\,d\hat{S} 
        &\quad & \mbox{for } \hat{q} \in Q_{k-1, k-1}(\hat f)\mbox{,}\\
    \nonumber&&&\quad&\mbox{ if $\hat f = \hat f_1$, $\hat f_2$ or $\hat f_5$;}\\[5pt]
    \nonumber
    \rho_{\hat \br}(\bv) & = & \int\limits_{\hat{E}} (v_1\,r_1 + v_2\,r_2)\,d\bx 
        &\quad& \mbox{for }r_1\mbox{, }r_2\in P_{k-2}(x,y) \otimes P_{k-1}(z);\\
    \label{momentos3hdiv}&&&&\\
    \label{momentos4hdiv}
    \rho_{\hat \br}(\bv) & = & \int\limits_{\hat{E}} v_3\,r_3\,d\bx 
        &\quad& \mbox{for }r_3\in P_{k-1}(\hat{f}_3) \otimes P_{k-2}(\hat{x}_3). 
\end{IEEEeqnarray}
\end{enumerate}
%%========
% {\color{red}TODO: tal vez en los dofs (\ref{momentos2hdiv}) haya que separar los espacios test como hice
% en pyramids}
%%========
%%=======================================================
%% original version of dofs
%\begin{IEEEeqnarray}{lcl}
%   \label{momentos1hdiv} \int\limits_{f} (\bv\cdot\boldsymbol{\nu})q\,d\gamma 
%       && \mbox{for } q \in P_{k-1}(x,y)\mbox{,}\\
%   \nonumber&& \mbox{ if $f\subseteq\{\hat{z}=0\}$ or $f\subseteq\{\hat{z}=1\}$; }\\
%   \label{momentos2hdiv} \int\limits_{f} (\bv\cdot\boldsymbol{\nu})q\,d\gamma 
%       && \mbox{for } q \in Q_{k-1, k-1, k-1}\mbox{,}\\
%   \nonumber&& \mbox{ if $f\subseteq\{\hat{x}=0\}$ or $f\subseteq\{\hat{y}=0\}$
%    or $f\subseteq\{\hat{x} + \hat{y} = 1\}$; } \\
%   \label{momentos3hdiv} \int\limits_{\hat{E}} (v_1q_1 + v_2q_2)\,d\textbf{x} 
%       &\quad& {q_1\mbox{, } q_2 \in P_{k-2}(x,y) \otimes P_{k-1}(z);}\\
%   \label{momentos4hdiv} \int\limits_{\hat{E}} v_3q_3\,d\textbf{x} 
%       &\quad& { q_3\in P_{k-1}(x,y) \otimes P_{k-2}(z).} 
%\end{IEEEeqnarray}
%%=======================================================
\end{defi}
The following result can be found in~\cite[page 66]{nedelec2}.
\begin{lemma} The Finite Element in Definition~\ref{defi_h_div_conforme} is
  unisolvent in $\hat E$.
\end{lemma}
\begin{defi}\label{defi_face_element} Let $P_{\hat E}$ be as in~(\ref{prismaticSpace}).
The interpolator $\boldsymbol{r}_{\hat{E}}\,:\,W^{1,1}(\hat{E})\to P_{\hat E}$
is defined as the operator such that, 
for each $\hat\bu\in W^{1,1}(\hat E)$, $\rku$ is
defined as the unique element in $P_{\hat E}$ satisfying
  \begin{IEEEeqnarray}{lClc}
    \rho_{f,\bq}\,(\hat{\bu} - \rku) & = & 0 &
    \quad\mbox{for $\rho_{\hat f,q}$ as in~(\ref{momentos1hdiv})
      and~(\ref{momentos2hdiv})}\\
    \rho_{\br}\,(\hat{\bu} - \rku) & = & 0 &
    \quad\mbox{for $\rho_{\br}$ as in~(\ref{momentos3hdiv})
      and~(\ref{momentos4hdiv})}.
  \end{IEEEeqnarray}
\end{defi}
Because of the degrees of freedom~(\ref{momentos1hdiv}) and~(\ref{momentos2hdiv})
we had to restrict the condition $\hat\bu\in H(div, \hat E)$
to the one of $\hat\bu\in W^{1,1}(\hat E)$ in order to define the 
interpolation operator. The proof of the following Lemma follows
from Trace Theorems in Sobolev spaces.
\begin{lemma}
  The operator of Definition~\ref{defi_face_element} is well defined and
  bounded.
\end{lemma}
\begin{proposition} On the Finite Element~(\ref{defi_h_div_conforme}). 
$\dim(P_{\hat{E}}) = k^2\,(k+2) + k\,(k+1)^2/2$
which is, at the same time, equal to the number of its independent degrees of freedom.
\end{proposition}
\begin{proof}
    From~(\ref{tensor_prod_dim}) and~(\ref{dk}) it is really straightforward to do
    \begin{IEEEeqnarray*}{rCl}
        \dim (P_{k-1}^2(x,y) \oplus \tilde{P}_{k-1}(x,y)) \otimes P_{k-1}(z) + \dim P_{k-1}(x,y)\otimes P_k(z) & = &\\[5pt]
        \IEEEeqnarraymulticol{3}{r}{=\,(k(k+1)+k)k+\dfrac{k(k+1)}{2}(k+1)} 
    \end{IEEEeqnarray*}
    and the same quantity is obtained by summing up the dimensions of all the
    polynomial spaces on the right--hand sides of~(\ref{momentos1hdiv})--(\ref{momentos4hdiv}).
\end{proof}

\begin{remark} The interpolation operator
can be written
{\color{Orange}\#\#\#\#\#\# poner bases para después usarlas
en la suma.}\\[5pt]
we take a basis...
\begin{IEEEeqnarray*}{rCl}
  \int\limits_{\hat f}(\bv_{f,\hat{p}}\cdot\boldsymbol{\nu})\hat{q}\,d\gamma  & = & \delta_{\hat{p},\hat{q}}
  \,\,etc\,\,etc
\end{IEEEeqnarray*}
\begin{IEEEeqnarray}{rCl}\label{face_interp_explicit}  
  \hat{\boldsymbol{r}}_k\hat{\bu} & = & \sum_{f,\bq} \rho_{f,\bq}(\hat{\bu})\,\hat{\bv}_{f,\bq} +
                                        \sum_{r}   \rho_{r}  (\hat{\bu})\,\hat{\bv}_{r}
\end{IEEEeqnarray}
\end{remark}
