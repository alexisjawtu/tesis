\usepackage[utf8x]{inputenc}
%\usepackage{babel-spanish}
%\usepackage[spanish]{babel}
\usepackage{cite}
\usepackage{amsmath}
\usepackage{amssymb}
\usepackage{amsfonts}
\usepackage{amsthm}
\usepackage{mathtools}
\usepackage{calrsfs}
%\usepackage{helvet}
\usepackage{nicefrac}
\usepackage{palatino}
%\usepackage{charter}
\usepackage[usenames,dvipsnames]{xcolor}
\usepackage{tikz}
\usepackage[retainorgcmds]{IEEEtrantools}
\usepackage{subfig}

\usetikzlibrary{calc,shapes.geometric,arrows}

% numbering before the word Theorem
\swapnumbers
\setcounter{secnumdepth}{3}
%\numberwithin{equation}{section}

%%%%%%%%%%%%%%%%%%%%%%%%%%%%%%%%
%\usepackage{syntonly}
%\syntaxonly
%%%%%%%%%%%%%%%%%%%%%%%%%%%%%%%%

\DeclareMathAlphabet{\pazocal}{OMS}{zplm}{m}{n}


\def\ok{{\color{green}\quad ok}}
\def\xyz{(\hat x_1, \hat x_2, \hat x_3)}
\def\be{\boldsymbol{e}}
\def\bx{\boldsymbol{x}}
\def\by{\boldsymbol{y}}
\def\bq{\boldsymbol{q}}
\def\bp{\boldsymbol{p}}
\def\bv{\boldsymbol{v}}
\def\bw{\boldsymbol{w}}
\def\bu{\boldsymbol{u}}
\def\br{\boldsymbol{r}}
\def\bn{\boldsymbol{n}}
\def\btau{\boldsymbol{\tau}}
\def\bz{\boldsymbol{\zeta}}
\def\bzeta{\boldsymbol{\zeta}}
\def\bgamma{\boldsymbol{\gamma}}
\def\s{\scriptstyle}
\def\diam{\text{diam}}
\def\dv{\mbox{div\,}}
\def\curl{\textbf{curl\,}}
\def\wku{\bw_{\hat{E}}\hat{\bu}}
\def\wkutilde{\bw_{\tilde{E}}\tilde{\bu}}
\def\rku{\br_{\hat{E}}\hat{\bu}}
\def\rkutilde{\br_{\tilde{E}}\tilde{\bu}}

\newcommand{\Qb}{\pazocal{Q}}
{\newcommand{\vertiii}[1]{{\left\vert\kern-0.25ex\left\vert\kern-0.25ex\left\vert #1 
    \right\vert\kern-0.25ex\right\vert\kern-0.25ex\right\vert}}
\newcommand{\forma}[2]{\int\limits_\Omega {\bf #1}\cdot{\bf #2}\,d\boldsymbol{x}}
\newcommand{\formb}[2]{\int\limits_\Omega #2\,\dv {\bf #1}\,d\boldsymbol{x}}
\newcommand{\hmtn}[2]{\textrm{H}^#1(\textrm{T})^#2}
\newcommand{\RTk}{\textrm{RT}_k}
\newcommand{\RTksomb}{\widehat{\textrm{RT}_k}}
\newcommand{\xSombrero}[1]{\widehat{\textbf{#1}}}
\newcommand{\raviart}[1]{\mathcal{RT}_k(#1)}
\newcommand{\Vh}{\mathbb{V}_h}
\newcommand{\Qh}{\mathbb{Q}_h}
\newcommand{\Th}{\mathcal{T}_{\textit{h}}}
\newcommand{\supp}[1]{\textrm{supp}(#1)}
\newcommand{\p}[2]{\mathcal{P}_{#1}(x,y) \otimes \mathcal{P}_{#2}(z)}
\newcommand{\pp}[2]{\mathcal{P}_{#1}(\hat f_1)\otimes\mathcal P_{#2}(\hat z)}
\newcommand{\wpcurl}[1]{W^p(\textbf{curl}, #1)}
\newcommand{\diag}[3]{
	\left(
	\begin{array}{ccc}
		#1 	& 0 	& 0\\
		0	& #2 	& 0\\
		0	& 0		& #3
	\end{array}
	\right)}
\newcommand{\dvg}{\text{div}\,}
\newcommand{\lf}[1]{\lambda_{f_{#1}}}
\newcommand{\lhatf}[1]{\lambda_{\hat{f}_{#1}}}

\newtheorem{theorem}{Theorem}[section]
\newtheorem{proposition}[theorem]{Proposition}
\newtheorem{corollary}[theorem]{Corollary}
\newtheorem{lemma}[theorem]{Lemma}
\newtheorem{obs}[theorem]{Observaci\'on}
\newtheorem{defi}[theorem]{Definition}
\newtheorem{notation}[theorem]{Notation}
\newtheorem{problem}[theorem]{Problem}
\newtheorem{remark}[theorem]{Remark}
\newtheorem{example}[theorem]{Example}

\DeclarePairedDelimiter{\abs}{\lvert}{\rvert}
\DeclarePairedDelimiter{\Abs}{\left|}{\right|}
\DeclarePairedDelimiter{\norm}{\lVert}{\rVert}
\DeclarePairedDelimiter{\Norm}{\left\|}{\right\|} %TODO este delimitador tiene un inconveniente

\DeclareMathOperator{\img}{Im}
\DeclareMathOperator{\Div}{\textrm{div}}
