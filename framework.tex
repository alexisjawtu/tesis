\usepackage[utf8x]{inputenc}
%\usepackage{babel-spanish}
%\usepackage[spanish]{babel}
\usepackage{amsmath}
\usepackage{amssymb}
\usepackage{amsfonts}
\usepackage{amsthm}
\usepackage{mathtools}
\usepackage{helvet}
%\usepackage{palatino}
%\usepackage{charter}
\usepackage[usenames,dvipsnames]{xcolor}
\usepackage{tikz}
\usepackage[retainorgcmds]{IEEEtrantools}
\usepackage{subfig}

\usetikzlibrary{calc,shapes.geometric}

%\numberwithin{equation}{section}

%%%%%%%%%%%%%%%%%%%%%%%%%%%%%%%%
%\usepackage{syntonly}
%\syntaxonly
%%%%%%%%%%%%%%%%%%%%%%%%%%%%%%%%

%\def\w{\textbf{w}}
%\def\v{\textbf{v}}
%\def\n{\textbf{n}}

\def\diam{\text{diam}}
\newcommand\bv{{\bf v}}
\newcommand\bu{{\bf u}}
\newcommand\bn{{\bf n}}
\newcommand\bw{{\bf w}}
\newcommand\dv{\mbox{div\,}}


\newcommand{\forma}[2]{\int\limits_\Omega {\bf #1}\cdot{\bf #2}\,d\boldsymbol{x}}
\newcommand{\formb}[2]{\int\limits_\Omega #2\,\dv {\bf #1}\,d\boldsymbol{x}}
\newcommand{\hmtn}[2]{\textrm{H}^#1(\textrm{T})^#2}
\newcommand{\RTk}{\textrm{RT}_k}
\newcommand{\RTksomb}{\widehat{\textrm{RT}_k}}
\newcommand{\xSombrero}[1]{\widehat{\textbf{#1}}}
\newcommand{\raviart}[1]{\mathcal{RT}_k(#1)}
\newcommand{\Vh}{\mathbb{V}_h}
\newcommand{\Qh}{\mathbb{Q}_h}
\newcommand{\Th}{\mathcal{T}_h}
\newcommand{\supp}[1]{\textrm{supp}(#1)}
\newcommand{\p}[2]{\mathcal{P}_{#1}(x,y) \otimes \mathcal{P}_{#2}(z)}
\newcommand{\wpcurl}[1]{W^p(\emph{\textbf{curl}}, #1)}
\newcommand{\diag}[3]{
	\left(
	\begin{array}{ccc}
		#1 	& 0 	& 0\\
		0	& #2 	& 0\\
		0	& 0		& #3
	\end{array}
	\right)}
\newcommand{\dvg}{\text{div}\,}
\newcommand{\lf}[1]{\lambda_{f_{#1}}}
\newcommand{\lhatf}[1]{\lambda_{\hat{f}_{#1}}}

\newtheorem{theorem}{Theorem}[section]
%\newtheorem{theorem}{theorem}[section]
\newtheorem{corolario}{Colorario}[section]
\newtheorem{lema}{Lema}[section]
\newtheorem{lemma}{Lemma}[section]
\newtheorem{obs}{Observaci\'on}[section]
\newtheorem{defi}{Definition}[section]
\newtheorem{problema}{Problema}[section]
\newtheorem{remark}{Remark}[section]
\newtheorem{ejemplo}{Ejemplo}[section]

\DeclarePairedDelimiter{\abs}{\lvert}{\rvert}
\DeclarePairedDelimiter{\Abs}{\left|}{\right|}
\DeclarePairedDelimiter{\norm}{\lVert}{\rVert}
\DeclarePairedDelimiter{\Norm}{\left\|}{\right\|} %TODO este delimitador tiene un inconveniente

\DeclareMathOperator{\img}{Im}
\DeclareMathOperator{\Div}{\textrm{div}}
