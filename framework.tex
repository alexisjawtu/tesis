\usepackage[utf8x]{inputenc}
%\usepackage{babel-spanish}
%\usepackage[spanish]{babel}
%\usepackage{graphicx}
%\usepackage{txfonts}
\usepackage{cite}
\usepackage{amsmath}
\usepackage{algorithm,algorithmicx}
\usepackage[noend]{algpseudocode}
\usepackage{amssymb}
\usepackage{amsfonts}
\usepackage{amsthm}
\usepackage{mathtools}
\usepackage{calrsfs}
%\usepackage{helvet}
\usepackage{nicefrac}
\usepackage{palatino}
%\usepackage{charter}
\usepackage[usenames,dvipsnames]{xcolor}
\usepackage{tikz}
\usepackage[retainorgcmds]{IEEEtrantools}
\usepackage{subfig}
\usepackage[format=hang,margin=10pt,font=small,labelfont=bf,labelsep=endash]{caption}
\usepackage{tocbibind}
\usepackage[pdftex,colorlinks=true]{hyperref}

\def\rescaledTetraTikz
{
\begin{figure}[!ht]
  \centering
    \begin{tikzpicture}[scale=.4]
      \draw[->] (0, 0, 0) -- node[label={above:$_{_{h_2}}$}] { } (3,0,0);
      \draw[->] (0, 0, 0) -- node[label={left:$_{_{h_3}}$}] { } (0, 3, 0);
      \draw (0, 0, 1) -- (2, 0, 0); 
%      \draw (0, 3.5, 1) -- (2, 3.5, 0); 
      \draw (2, 0, 0) -- (0, 3, 0);
      \draw (0, 0, 1) -- (0, 3, 0); 
      \draw (0,0,2) -- node[label={left:$_{_{h_1}}$}] { }  (0,0,0) ;
      %\draw (0,3.5,0) -- (2,3.5,0);
    \end{tikzpicture}
  \caption{Rescaled Tetrahedron}
  \label{rescaled_tetra}    
\end{figure}
}

\def\referenceTriangleTikz
{
\begin{tikzpicture}[scale=2]
  \draw (0,0,0) -- (1.2,0,0);
	\draw (0,0,0) -- (0,1.2,0); 
  \draw (1,0,0) -- (0,1,0);
  \draw[white] (1.2,0,0) -- (1.5,0,0);
  \draw[white] (0,0,0) -- (0,0,.7);
\end{tikzpicture}
}

\newcommand{\referencePrismTikz}[1]
{
\begin{tikzpicture}[scale={#1}]
  \draw (0, 0, 0) -- (0, 0, 2); 
  \draw (0, 0, 0) -- (2, 0, 0);
  \draw (0, 0, 0) -- (0, 2, 0);
	\draw (0, 0, 2) -- (2, 0, 0);
	\draw (0, 2, 2) -- (2, 2, 0);
	\draw (2, 0, 0) -- (2, 2, 0);
	\draw (0, 0, 2) -- (0, 2, 2);
	\draw (0, 2, 2) -- (0,2,0);
	\draw (0, 2, 0) -- (2,2,0);
\end{tikzpicture}
}

\newcommand{\referencePyramidTikz}[1]
{
\begin{tikzpicture}[scale={#1}]
  \coordinate (o) at (0,0,0);
  \coordinate (e1) at (0,0,1);
  \coordinate (e2) at (1,0,0);
  \coordinate (e3) at (0,1,0);

  \draw (o) -- (e1) -- ($(e1)+(e2)$) -- (e2) -- (e3) -- 
      (o)-- (e2);
  \draw (e3) -- (e1);
  \draw (e3) -- ($(e1)+(e2)$);
  \draw[white] (e1) -- ($1.4*(e1)$);
  \draw[white] (e2) -- ($1.2*(e2)$);
\end{tikzpicture}
}

\newcommand{\referenceTetrahedronTikz}[1]
{
\begin{tikzpicture}[scale={#1}]
  \coordinate (o) at (0,0,0);
  \coordinate (e1) at (0,0,1);
  \coordinate (e2) at (1,0,0);
  \coordinate (e3) at (0,1,0);

  \draw (o) -- (e1) -- (e2) -- (o) -- (e3) -- (e2);
  \draw (e3) -- (e1);

\end{tikzpicture}
}

\def\macroRegularity{  
\begin{figure}
  \centering
  \caption{Macroelements.}
  \subfloat[Tetrahedral Macro\-ele\-ment.]
  {
    \label{macro_tetra_reg}
    \begin{tikzpicture}[scale=1.5]
      \draw[white] (-.8,0,0) -- (1.3,0,0);
      \draw (0,0,0) -- node[midway] {\tiny{\color{black}$\xi_1$}} (0,1,0);
      \draw (0,0,0) -- node[midway] {\tiny{\color{black}$\xi_2$}} (1,0,0);
      \draw (0,1,0) -- (1,0,0);
      \draw[orange] (0,0,0) -- node[midway] {\tiny{\color{black}$\xi_3$}} (0,0,1);
      \draw[white] (0,0,1) -- (0,0,1.5);
      \draw (0,1,0) -- (0,0,1);
      \draw (1,0,0) -- (0,0,1);
      \fill[red] (0,0,1) circle (1pt);
    \end{tikzpicture}
  }\hspace{1.5cm}
  \subfloat[Prismatic Macro\-ele\-ment.]
  {
    \label{macro_prism_reg}
    \begin{tikzpicture}[scale=1.5]
      \fill[black] (0,0,0) circle (1pt);
      \draw (0,0,0) -- node[midway] {\tiny{\color{black}$\xi_1$}} (0,1,0);
      \draw (0,0,0) -- node[midway] {\tiny{\color{black}$\xi_2$}} (1,0,0);
      \draw (0,1,0) -- (1,0,0) -- (1,0,1) -- (0,0,1);
      \draw (0,1,0) -- (0,1,1) -- (0,0,1);
      \draw (1,0,1) -- (0,1,1);
      \draw[orange] (0,0,0) -- node[midway] {\tiny{\color{black}$\xi_3$}} (0,0,1);
      \draw[white] (0,0,1) -- (0,0,1.5);
      \draw[white] (1,0,0) -- (1.3,0,0);
    \end{tikzpicture}
  }
\end{figure}
}

\def\unitTangentsPyramid
{
\begin{tikzpicture}[scale=2,post/.style={->, shorten >=1pt, >=stealth', semithick}]
  \coordinate (o) at (0,0,0);
  \coordinate (e1) at (0,0,1);
  \coordinate (e2) at (1,0,0);
  \coordinate (e3) at (0,1,0);
  \coordinate (e4) at ($(e1)+(e2)$);

  \coordinate (a1start) at ($.2*(e1)$);
  \coordinate (a1end)   at ($.5*(e1)$);
  \coordinate (a2start)   at ($.8*(e1)+.2*(e4)$);
  \coordinate (a2end)   at ($.5*(e1)+.5*(e4)$);
  \coordinate (a3start)   at ($.8*(e2)+.2*(e4)$);
  \coordinate (a3end)   at ($.5*(e2)+.5*(e4)$);
  \coordinate (a4start) at ($.17*(e2)$);
  \coordinate (a4end)   at ($.4*(e2)$);
  \coordinate (a5start) at ($.2*(e3)$);
  \coordinate (a5end)   at ($.5*(e3)$);
  \coordinate (a6start) at ($.2*(e3)+.8*(e1)$);
  \coordinate (a6end)   at ($.5*(e3)+.5*(e1)$);
  \coordinate (a7start) at ($.2*(e3)+.8*(e2)$);
  \coordinate (a7end)   at ($.5*(e3)+.5*(e2)$);
  \coordinate (a8start) at ($.2*(e3)+.8*(e4)$);
  \coordinate (a8end)   at ($.5*(e3)+.5*(e4)$);

  \draw (o) -- (e1) -- ($(e1)+(e2)$) -- (e2) -- (e3) -- 
      (o)-- (e2);
  \draw (e3) -- (e1);
  \draw (e3) -- (e4);
  \foreach \i in {1,...,8}{
    \draw [post,very thick] (a\i start) -- (a\i end);
  }

\end{tikzpicture}
}

\def\unitTangentsPrism
{
\begin{tikzpicture}[scale=2,post/.style={->, shorten >=1pt, >=stealth', semithick}]
  \coordinate (o) at (0,0,0);
  \coordinate (e1) at (0,0,1);
  \coordinate (e2) at (1,0,0);
  \coordinate (e3) at (0,1,0);
  \coordinate (e4) at ($(e1)+(e3)$);
  \coordinate (e5) at ($(e2)+(e3)$);

  \coordinate (a1start) at ($.2*(e1)$);
  \coordinate (a1end)   at ($.5*(e1)$);
  \coordinate (a2start)   at ($.2*(e2)$);
  \coordinate (a2end)   at ($.5*(e2)$);

  \coordinate (a3start)   at ($.2*(e3)$);
  \coordinate (a3end)   at ($.5*(e3)$);

  \coordinate (a4start) at ($.83*(e3)+.17*(e4)$);
  \coordinate (a4end)   at ($.5*(e3)+.5*(e4)$);
  
  \coordinate (a5start) at ($.8*(e3)+.2*(e5)$);
  \coordinate (a5end)   at ($.5*(e3)+.5*(e5)$);
  
  \coordinate (a6start) at ($.2*(e4)+.8*(e1)$);
  \coordinate (a6end)   at ($.5*(e4)+.5*(e1)$);
  \coordinate (a7start) at ($.2*(e5)+.8*(e2)$);
  \coordinate (a7end)   at ($.5*(e5)+.5*(e2)$);
  \coordinate (a8start) at ($.2*(e4)+.8*(e5)$);
  \coordinate (a8end)   at ($.5*(e4)+.5*(e5)$);
  \coordinate (a9start) at ($.2*(e1)+.8*(e2)$);
  \coordinate (a9end)   at ($.5*(e1)+.5*(e2)$);

  \draw (o) -- (e1) -- (e2) -- (o) -- (e3) -- (e4) -- (e5) -- (e3);
  \draw (e4) -- (e1);
  \draw (e5) -- (e2);

  \foreach \i in {1,...,9}{
    \draw [post, very thick] (a\i start) -- (a\i end);
  }
\end{tikzpicture}
}

\newcommand{\prismaticMacroelement}[5]{
\def\m{#1} % piso y tapa como m-'agonos
\def\n{#2} % (n + 1) puntos en cada arista del tri'angulo
\def\mu{#3} % graduaci'on
\def\init{#4} % hacia d'onde queremos que apunte la boca

\foreach \l in {0,...,\n} {
  \pgfmathparse{-\l*2/\n} \let\L\pgfmathresult
  %\pgfmathparse{\m + \init - 2} \let\last\pgfmathresult
  %%% TODO perspectiva: hacer unos defs con los canonicos y el entero 1, 2, 3 de entrada
  %\pgfmathparse{\n + 1} \let\N\pgfmathresult
  \begin{scope}[shift={(0,\L,0)}]   %% --->>> altura
    \coordinate (point1) at ({cos((.15+\init-2)*\twoPi/\m)},0,{sin((.15+\init-2)*\twoPi/\m)});  %% -->> perspectiva
    \coordinate (point2) at ({cos((.15+\init-1)*\twoPi/\m)},0,{sin((.15+\init-1)*\twoPi/\m)});
    \draw[color=#5] (point1) -- (point2);

    \pgfmathparse{\n-1} \let\to\pgfmathresult
    \foreach \i in {0,...,\to} {  
      \pgfmathparse{pow(\i/\n,1/\mu)} \let\rad\pgfmathresult
      \draw[color=#5] ($\rad*(point1)$) -- ($\rad*(point2)$);
      
      \pgfmathparse{\n-\i-1} \let\ni\pgfmathresult
      \foreach \j in {0,...,\ni} {
        \pgfmathparse{(\i/\n)*pow((\i+\j)/\n,1/\mu-1)} \let\alfa\pgfmathresult
        \pgfmathparse{(\j/\n)*pow((\i+\j)/\n,1/\mu-1)} \let\beta\pgfmathresult

        \pgfmathparse{(\i/\n)*pow((\i+(\j+1))/\n,1/\mu-1)}    \let\alfaSuc\pgfmathresult
        \pgfmathparse{((\j+1)/\n)*pow((\i+(\j+1))/\n,1/\mu-1)}  \let\betaSuc\pgfmathresult
        
        \draw[color=#5] ($\alfa*(point1) + \beta*(point2)$) -- 
            ($\alfaSuc*(point1) + \betaSuc*(point2)$);
        \draw[color=#5] ($\alfa*(point2) + \beta*(point1)$) -- 
            ($\alfaSuc*(point2) + \betaSuc*(point1)$);
      };
    }
  \end{scope}
  }
  \coordinate (point1) at ({cos((.15+\init-2)*\twoPi/\m)},0,{sin((.15+\init-2)*\twoPi/\m)});
  \coordinate (point2) at ({cos((.15+\init-1)*\twoPi/\m)},0,{sin((.15+\init-1)*\twoPi/\m)}); 
  \pgfmathparse{\n} \let\to\pgfmathresult
  \foreach \i in {0,...,\to} {  
    \pgfmathparse{\n-\i} \let\ni\pgfmathresult
    \foreach \j in {0,...,\ni} {
      \pgfmathparse{(\i/\n)*pow((\i+\j)/\n,1/\mu-1)} \let\alfa\pgfmathresult
      \pgfmathparse{(\j/\n)*pow((\i+\j)/\n,1/\mu-1)} \let\beta\pgfmathresult
      \pgfmathparse{(\i/\n)*pow((\i+(\j+1))/\n,1/\mu-1)}    \let\alfaSuc\pgfmathresult
      \pgfmathparse{((\j+1)/\n)*pow((\i+(\j+1))/\n,1/\mu-1)}  \let\betaSuc\pgfmathresult        
      \draw[color=#5] ($\alfa*(point1) + \beta*(point2)$) -- 
        ($\alfa*(point1)+ \beta*(point2) + (0,-2,0)$); 
   };
  };
  \draw[red] (0,0,0) -- (0,-2,0);
}

\def\prismsBaryCoordA{
\begin{table}
  \centering  
  \caption{{Barycentric coordinates of prismatic mesh points} in $\Lambda_\ell$
   \newline $0\leqslant l\leqslant n-2$, $i+j\leqslant n-l-2$}
  \label{prisms_barycentric_a}
  \begin{IEEEeqnarraybox*}
    [\IEEEeqnarraystrutmode
    \IEEEeqnarraystrutsizeadd{2pt}{6pt}]{v/c/v/c/v/c/v/c/v/c/v/}
      \IEEEeqnarrayrulerow\\
      \IEEEeqnarrayseprow[5pt]\\
         & p_0 && {\textstyle 1-\left(\frac{n-l}n\right)^{\nicefrac{1}{\mu}}      } 
               && {\textstyle \left(\frac{n-l}n\right)^{\nicefrac{1}{\mu}}-\left(\frac{i+j}n\right)^{\nicefrac{1}{\mu}}   }
               && {\textstyle \frac in \left(\frac{i+j}n\right)^{\nicefrac{1}{\mu}-1}   }
               && {\textstyle \frac jn \left(\frac{i+j}n\right)^{\nicefrac{1}{\mu}-1} }
         & \\
      \IEEEeqnarrayrulerow\\
      \IEEEeqnarrayseprow[5pt]\\
         & p_1 && {\textstyle 1-\left(\frac{n-l}n\right)^{\nicefrac{1}{\mu}}} 
               && {\textstyle \left(\frac{n-l}n\right)^{\nicefrac{1}{\mu}}-\left(\frac{i+j+1}n\right)^{\nicefrac{1}{\mu}} }
               && {\textstyle \frac{i+1}n \left(\frac{i+1+j}n\right)^{\nicefrac{1}{\mu}-1} }
               && {\textstyle \frac jn \left(\frac{i+1+j}n\right)^{\nicefrac{1}{\mu}-1} }
         & \\
      \IEEEeqnarrayrulerow\\
      \IEEEeqnarrayseprow[5pt]\\
         & p_2 && {\textstyle 1-\left(\frac{n-l}n\right)^{\nicefrac{1}{\mu}} }
               && {\textstyle \left(\frac{n-l}n\right)^{\nicefrac{1}{\mu}}-\left(\frac{i+j+1}n\right)^{\nicefrac{1}{\mu}} }
               && {\textstyle \frac in \left(\frac{i+j+1}n\right)^{\nicefrac{1}{\mu}-1} }
               && {\textstyle \frac{j+1}n \left(\frac{i+1+j}n\right)^{\nicefrac{1}{\mu}-1}}
         & \\
      \IEEEeqnarrayrulerow\\
      \IEEEeqnarrayseprow[5pt]\\
         & p_3 && {\textstyle 1-\left(\frac{n-l-1}n\right)^{\nicefrac{1}{\mu}} }
               && {\textstyle \left(\frac{n-l-1}n\right)^{\nicefrac{1}{\mu}}-\left(\frac{i+j}n\right)^{\nicefrac{1}{\mu}} }
               && {\textstyle \frac in \left(\frac{i+j}n\right)^{\nicefrac{1}{\mu}-1} }
               && {\textstyle \frac jn \left(\frac{i+j}n\right)^{\nicefrac{1}{\mu}-1}}
         & \\
      \IEEEeqnarrayrulerow\\
      \IEEEeqnarrayseprow[5pt]\\
         & p_4 && {\textstyle 1-\left(\frac{n-l-1}n\right)^{\nicefrac{1}{\mu}} }
               && {\textstyle \left(\frac{n-l-1}n\right)^{\nicefrac{1}{\mu}}-\left(\frac{i+1+j}n\right)^{\nicefrac{1}{\mu}} }
               && {\textstyle \frac{i+1}n \left(\frac{i+1+j}n\right)^{\nicefrac{1}{\mu}-1} }
               && {\textstyle \frac jn \left(\frac{i+1+j}n\right)^{\nicefrac{1}{\mu}-1}}
         & \\
      \IEEEeqnarrayrulerow\\
      \IEEEeqnarrayseprow[5pt]\\
         & p_5 && {\textstyle 1-\left(\frac{n-l-1}n\right)^{\nicefrac{1}{\mu}} }
               && {\textstyle \left(\frac{n-l-1}n\right)^{\nicefrac{1}{\mu}}-\left(\frac{i+j+1}n\right)^{\nicefrac{1}{\mu}} }
               && {\textstyle \frac in \left(\frac{i+j+1}n\right)^{\nicefrac{1}{\mu}-1} }
               && {\textstyle \frac{j+1}n \left(\frac{i+1+j}n\right)^{\nicefrac{1}{\mu}-1}}
         & \\
      \IEEEeqnarrayrulerow
  \end{IEEEeqnarraybox*}
\end{table}}

\def\prismsBaryCoordB{
\begin{table}
  \centering  
  \caption{{Barycentric coordinates of prismatic mesh points} in $\Lambda_\ell$
  \newline $0\leqslant l\leqslant n-2,\qquad i\ge1, \qquad \mbox{and}\qquad i+j\leqslant n-l-2$}
  \label{prisms_barycentric_b}
  \begin{IEEEeqnarraybox*}
    [\IEEEeqnarraystrutmode
    \IEEEeqnarraystrutsizeadd{2pt}{6pt}]{v/c/v/c/v/c/v/c/v/c/v/}
      \IEEEeqnarrayrulerow\\
      \IEEEeqnarrayseprow[5pt]\\
         & p_0  && {\textstyle 1-\left(\frac{n-l}n\right)^{\nicefrac{1}{\mu}}      } 
               && {\textstyle \left(\frac{n-l}n\right)^{\nicefrac{1}{\mu}}-\left(\frac{i+j}n\right)^{\nicefrac{1}{\mu}}   }
               && {\textstyle \frac in \left(\frac{i+j}n\right)^{\nicefrac{1}{\mu}-1}   }
               && {\textstyle \frac jn \left(\frac{i+j}n\right)^{\nicefrac{1}{\mu}-1} }
         & \\
      \IEEEeqnarrayrulerow\\
      \IEEEeqnarrayseprow[5pt]\\
         & p_1  && {\textstyle 1-\left(\frac{n-l}n\right)^{\nicefrac{1}{\mu}} }  
               && {\textstyle \left(\frac{n-l}n\right)^{\nicefrac{1}{\mu}}-\left(\frac{i+j+1}n\right)^{\nicefrac{1}{\mu}} }
               && {\textstyle \frac in \left(\frac{i+j+1}n\right)^{\nicefrac{1}{\mu}-1} }
               && {\textstyle   \frac{j+1}n \left(\frac{i+j+1}n\right)^{\nicefrac{1}{\mu}-1}}
         & \\
      \IEEEeqnarrayrulerow\\
      \IEEEeqnarrayseprow[5pt]\\
         & p_2 && {\textstyle 1-\left(\frac{n-l}n\right)^{\nicefrac{1}{\mu}}  }
               && {\textstyle \left(\frac{n-l}n\right)^{\nicefrac{1}{\mu}}-\left(\frac{i+j}n\right)^{\nicefrac{1}{\mu}} }
               && {\textstyle \frac{i-1}n \left(\frac{i+j}n\right)^{\nicefrac{1}{\mu}-1} }
               && {\textstyle \frac{j+1}n \left(\frac{i+j}n\right)^{\nicefrac{1}{\mu}-1}}
         & \\
      \IEEEeqnarrayrulerow\\
      \IEEEeqnarrayseprow[5pt]\\
         & p_3  && {\textstyle 1-\left(\frac{n-l-1}n\right)^{\nicefrac{1}{\mu}} }
               && {\textstyle \left(\frac{n-l-1}n\right)^{\nicefrac{1}{\mu}}-\left(\frac{i+j}n\right)^{\nicefrac{1}{\mu}} }
               && {\textstyle \frac in \left(\frac{i+j}n\right)^{\nicefrac{1}{\mu}-1} }
               && {\textstyle   \frac jn \left(\frac{i+j}n\right)^{\nicefrac{1}{\mu}-1}}
         & \\
      \IEEEeqnarrayrulerow\\
      \IEEEeqnarrayseprow[5pt]\\
         & p_4 && {\textstyle 1-\left(\frac{n-l-1}n\right)^{\nicefrac{1}{\mu}} }
               && {\textstyle \left(\frac{n-l-1}n\right)^{\nicefrac{1}{\mu}}-\left(\frac{i+j+1}n\right)^{\nicefrac{1}{\mu}} }
               && {\textstyle \frac in \left(\frac{i+j+1}n\right)^{\nicefrac{1}{\mu}-1} }
               && {\textstyle \frac{j+1}n \left(\frac{i+j+1}n\right)^{\nicefrac{1}{\mu}-1}}
         & \\
      \IEEEeqnarrayrulerow\\
      \IEEEeqnarrayseprow[5pt]\\
         & p_5  && {\textstyle 1-\left(\frac{n-l-1}n\right)^{\nicefrac{1}{\mu}} }
               && {\textstyle \left(\frac{n-l-1}n\right)^{\nicefrac{1}{\mu}}-\left(\frac{i+j}n\right)^{\nicefrac{1}{\mu}} }
               && {\textstyle \frac{i-1}n \left(\frac{i+j}n\right)^{\nicefrac{1}{\mu}-1} }
               && {\textstyle \frac{j+1}n \left(\frac{i+j}n\right)^{\nicefrac{1}{\mu}-1}}
         & \\
      \IEEEeqnarrayrulerow
  \end{IEEEeqnarraybox*}
\end{table}}

\def\pyramidsBaryCoord{
\begin{table}
  \centering  
  \caption{Barycentric coordinates of pyramidal mesh points in $\Lambda_\ell$.
      \newline
    $0\leqslant l\leqslant n-2 \qquad\mbox{and}\qquad 1\leqslant i\leqslant n-l-1.$
      \newline}
  \label{pyramidal_barycentric}
  \begin{IEEEeqnarraybox*}
    [\IEEEeqnarraystrutmode
    \IEEEeqnarraystrutsizeadd{2pt}{6pt}]{v/c/v/c/v/c/v/c/v/c/v/}
      \IEEEeqnarrayrulerow\\
      \IEEEeqnarrayseprow[5pt]\\
         & p_0 && {\textstyle 1-\left(\frac{n-l}n\right)^{\nicefrac{1}{\mu}}  }
               && {\textstyle \left(\frac{n-l}n\right)^{\nicefrac{1}{\mu}}-\left(\frac{n-l-1}n\right)^{\nicefrac{1}{\mu}}  }
               && {\textstyle \frac in \left(\frac{n-l-1}n\right)^{\nicefrac{1}{\mu}-1}  }
               && {\textstyle \frac{n-l-i-1}n \left(\frac{n-l-1}n\right)^{\nicefrac{1}{\mu}-1}}
         & \\
      \IEEEeqnarrayrulerow\\
      \IEEEeqnarrayseprow[5pt]\\
         & p_1 && {\textstyle 1-\left(\frac{n-l}n\right)^{\nicefrac{1}{\mu}} }
               && {\textstyle \left(\frac{n-l}n\right)^{\nicefrac{1}{\mu}}-\left(\frac{n-l-1}n\right)^{\nicefrac{1}{\mu}} }
               && {\textstyle \frac{i-1}n \left(\frac{n-l-1}n\right)^{\nicefrac{1}{\mu}-1} }
               && {\textstyle \frac{n-l-i}n \left(\frac{n-l-1}n\right)^{\nicefrac{1}{\mu}-1}}
         & \\
      \IEEEeqnarrayrulerow\\
      \IEEEeqnarrayseprow[5pt]\\
         & p_2 && {\textstyle 1-\left(\frac{n-l-1}n\right)^{\nicefrac{1}{\mu}}  }
               && {\textstyle   0  }
               && {\textstyle   \frac in \left(\frac{n-l-1}n\right)^{\nicefrac{1}{\mu}-1}  }
               && {\textstyle   \frac{n-l-i-1}n \left(\frac{n-l-1}n\right)^{\nicefrac{1}{\mu}-1}}
         & \\
      \IEEEeqnarrayrulerow\\
      \IEEEeqnarrayseprow[5pt]\\
         & p_3 && {\textstyle 1-\left(\frac{n-l-1}n\right)^{\nicefrac{1}{\mu}}  }
               && {\textstyle 0  }
               && {\textstyle \frac{i-1}n \left(\frac{n-l-1}n\right)^{\nicefrac{1}{\mu}-1}  }
               && {\textstyle \frac{n-l-i}n \left(\frac{n-l-1}n\right)^{\nicefrac{1}{\mu}-1}}
         & \\
      \IEEEeqnarrayrulerow\\
      \IEEEeqnarrayseprow[5pt]\\
         & p_4 && {\textstyle 1-\left(\frac{n-l}n\right)^{\nicefrac{1}{\mu}}  }
               && {\textstyle 0  }
               && {\textstyle \frac in \left(\frac{n-l}n\right)^{\nicefrac{1}{\mu}-1}  }
               && {\textstyle \frac{n-l-i}n \left(\frac{n-l}n\right)^{\nicefrac{1}{\mu}-1}}
         & \\
      \IEEEeqnarrayrulerow
  \end{IEEEeqnarraybox*}
\end{table}}

\def\tetrahedraBaryCoord{
\begin{table}
  \centering  
  \caption{Barycentric coordinates of tetrahedral mesh points in $\Lambda_\ell$.
      \newline
$0\leqslant l\leqslant n-1\qquad\mbox{and}\qquad 1\leqslant i\leqslant n-l-1$
      \newline}
  \label{tetrahedral_barycentric}
  \begin{IEEEeqnarraybox*}
    [\IEEEeqnarraystrutmode
    \IEEEeqnarraystrutsizeadd{2pt}{6pt}]{v/c/v/c/v/c/v/c/v/c/v/}
      \IEEEeqnarrayrulerow\\
      \IEEEeqnarrayseprow[5pt]\\
         & p_0 && {\textstyle 1-\left(\frac{n-l}n\right)^{\nicefrac{1}{\mu}} }
               && {\textstyle \left(\frac{n-l}n\right)^{\nicefrac{1}{\mu}}-\left(\frac{n-l-1}n\right)^{\nicefrac{1}{\mu}} }
               && {\textstyle \frac in \left(\frac{n-l-1}n\right)^{\nicefrac{1}{\mu}-1} }
               && {\textstyle \frac{n-l-i-1}n \left(\frac{n-l-1}n\right)^{\nicefrac{1}{\mu}-1}}
         & \\
      \IEEEeqnarrayrulerow\\
      \IEEEeqnarrayseprow[5pt]\\
         & p_1 && {\textstyle 1-\left(\frac{n-l-1}n\right)^{\nicefrac{1}{\mu}}  }
               && {\textstyle 0  }
               && {\textstyle \frac in \left(\frac{n-l-1}n\right)^{\nicefrac{1}{\mu}-1}  }
               && {\textstyle \frac{n-l-i-1}n \left(\frac{n-l-1}n\right)^{\nicefrac{1}{\mu}-1}}
         & \\
      \IEEEeqnarrayrulerow\\
      \IEEEeqnarrayseprow[5pt]\\
         & p_2 && {\textstyle 1-\left(\frac{n-l}n\right)^{\nicefrac{1}{\mu}}  }
               && {\textstyle 0  }
               && {\textstyle \frac in \left(\frac{n-l}n\right)^{\nicefrac{1}{\mu}-1}  }
               && {\textstyle \frac{n-l-i}n \left(\frac{n-l}n\right)^{\nicefrac{1}{\mu}-1}}
         & \\
      \IEEEeqnarrayrulerow\\
      \IEEEeqnarrayseprow[5pt]\\
         & p_3 && {\textstyle 1-\left(\frac{n-l}n\right)^{\nicefrac{1}{\mu}}  }
               && {\textstyle 0  }
               && {\textstyle \frac{i+1}n \left(\frac{n-l}n\right)^{\nicefrac{1}{\mu}-1}  }
               && {\textstyle \frac{n-l-i-1}n \left(\frac{n-l}n\right)^{\nicefrac{1}{\mu}-1}}
         & \\
      \IEEEeqnarrayrulerow
  \end{IEEEeqnarraybox*}
\end{table}}

%\def\prismsProductCoord{
%\begin{table}
%  \centering  
%  \caption{Cartesian coordinates of prismatic mesh points in $\Lambda_\ell$.
%      \newline
%$0\leqslant l\leqslant n-1\qquad\mbox{and}\qquad 1\leqslant i\leqslant n-l-1$
%      \newline}
%  \label{prisms_product}
%  \begin{IEEEeqnarraybox*}
%    [\IEEEeqnarraystrutmode
%    \IEEEeqnarraystrutsizeadd{2pt}{6pt}]{v/c/v/c/v/c/v/c/v/c/v/}
%      \IEEEeqnarrayrulerow\\
%      \IEEEeqnarrayseprow[5pt]\\
%         & p_0 && {\textstyle 1-\left(\frac{n-l}n\right)^{\nicefrac{1}{\mu}} }
%               && {\textstyle \left(\frac{n-l}n\right)^{\nicefrac{1}{\mu}}-\left(\frac{n-l-1}n\right)^{\nicefrac{1}{\mu}} }
%               && {\textstyle \frac in \left(\frac{n-l-1}n\right)^{\nicefrac{1}{\mu}-1} }
%               && {\textstyle \frac{n-l-i-1}n \left(\frac{n-l-1}n\right)^{\nicefrac{1}{\mu}-1}}
%         & \\
%      \IEEEeqnarrayrulerow\\
%      \IEEEeqnarrayseprow[5pt]\\
%         & p_1 && {\textstyle 1-\left(\frac{n-l-1}n\right)^{\nicefrac{1}{\mu}}  }
%               && {\textstyle 0  }
%               && {\textstyle \frac in \left(\frac{n-l-1}n\right)^{\nicefrac{1}{\mu}-1}  }
%               && {\textstyle \frac{n-l-i-1}n \left(\frac{n-l-1}n\right)^{\nicefrac{1}{\mu}-1}}
%         & \\
%      \IEEEeqnarrayrulerow\\
%      \IEEEeqnarrayseprow[5pt]\\
%         & p_2 && {\textstyle 1-\left(\frac{n-l}n\right)^{\nicefrac{1}{\mu}}  }
%               && {\textstyle 0  }
%               && {\textstyle \frac in \left(\frac{n-l}n\right)^{\nicefrac{1}{\mu}-1}  }
%               && {\textstyle \frac{n-l-i}n \left(\frac{n-l}n\right)^{\nicefrac{1}{\mu}-1}}
%         & \\
%      \IEEEeqnarrayrulerow\\
%      \IEEEeqnarrayseprow[5pt]\\
%         & p_3 && {\textstyle 1-\left(\frac{n-l}n\right)^{\nicefrac{1}{\mu}}  }
%               && {\textstyle 0  }
%               && {\textstyle \frac{i+1}n \left(\frac{n-l}n\right)^{\nicefrac{1}{\mu}-1}  }
%               && {\textstyle \frac{n-l-i-1}n \left(\frac{n-l}n\right)^{\nicefrac{1}{\mu}-1}}
%         & \\
%      \IEEEeqnarrayrulerow
%  \end{IEEEeqnarraybox*}
%\end{table}}

\def\facesOfPrism{
\begin{table}[!h]
    \centering  
    \caption{Notation for the faces and positive normals of the reference prism.}
    \label{prismNotationTableFaces}
    \begin{IEEEeqnarraybox*}
      [\IEEEeqnarraystrutmode
      \IEEEeqnarraystrutsizeadd{2pt}{6pt}]{v/c/x/c/x/c/x/v/x/c/x/c/x/c/v/}
        \IEEEeqnarrayrulerow\\
        \IEEEeqnarrayseprow[5pt]\\
          & \hat f_1 && \subseteq &&  \{\hat x_1 = 0 \}            && && \hat{\bn}_1 && = && (-1,0,0)' & \\
        \IEEEeqnarrayrulerow\\
        \IEEEeqnarrayseprow[5pt]\\
          & \hat f_2 && \subseteq &&  \{\hat x_2 = 0 \}            && && \hat{\bn}_2 && = && (0,-1,0)' &\\
        \IEEEeqnarrayrulerow\\
        \IEEEeqnarrayseprow[5pt]\\
          & \hat f_3 && \subseteq &&  \{\hat x_3 = 0 \} && && \hat{\bn}_3 && = && (0,0,-1)' &\\
        \IEEEeqnarrayrulerow\\
        \IEEEeqnarrayseprow[5pt]\\
          & \hat f_4 && \subseteq &&  \{\hat x_3 = 1 \} && && \hat{\bn}_4 && = && (0,0,1)' &\\
        \IEEEeqnarrayrulerow\\
        \IEEEeqnarrayseprow[5pt]\\
          & \hat f_5 && \subseteq &&  \{\hat x_1+\hat x_2 = 1\} && && \hat{\bn}_5 && = && 2^{-\nicefrac{1}{2}}(1,1,0)' &\\
        \IEEEeqnarrayrulerow
    \end{IEEEeqnarraybox*}
\end{table}
}

\def\edgesOfPrism{
\begin{table}[!h]
    \centering  
    \caption{Notation for the edges and positive tangents of the reference prism.}
    \label{prismNotationTableEdges}
    \begin{IEEEeqnarraybox*}
      [\IEEEeqnarraystrutmode
      \IEEEeqnarraystrutsizeadd{2pt}{6pt}]{v/c/x/c/x/c/x/v/x/c/x/c/x/c/v/}
        \IEEEeqnarrayrulerow\\
        \IEEEeqnarrayseprow[5pt]\\
   & \hat \be_1 && = && \{(\hat x_1,0,0)^t\,:\,0\leqslant\hat x_1\leqslant 1\} && && \hat \btau_1 && = && (1,0,0)' & \\
        \IEEEeqnarrayrulerow\\
        \IEEEeqnarrayseprow[5pt]\\
   & \hat \be_2 && = && \{(0,\hat x_2,0)^t\,:\,0\leqslant\hat x_2\leqslant 1\} && && \hat \btau_2 && = && (0,1,0)' & \\
        \IEEEeqnarrayrulerow\\
        \IEEEeqnarrayseprow[5pt]\\
   & \hat \be_3 && = && \{(0,0,\hat x_3)^t\,:\,0\leqslant\hat x_3\leqslant 1\} && && \hat \btau_3 && = && (0,0,1)' & \\
        \IEEEeqnarrayrulerow\\
        \IEEEeqnarrayseprow[5pt]\\
   & \hat \be_4 && = && \{(\hat x_1,0,1)^t\,:\,0\leqslant\hat x_1\leqslant 1\} && && \hat \btau_4 && = && (0,0,1)' & \\
        \IEEEeqnarrayrulerow\\
        \IEEEeqnarrayseprow[5pt]\\
   & \hat \be_5 && = && \{(0,\hat x_2,1)^t\,:\,0\leqslant\hat x_2\leqslant 1\} && && \hat \btau_5 && = && (0,0,1)' & \\
        \IEEEeqnarrayrulerow\\
        \IEEEeqnarrayseprow[5pt]\\
   & \hat \be_6 && = && \{(1,0,\hat x_3)^t\,:\,0\leqslant\hat x_3\leqslant 1\} && && \hat \btau_6 && = && (0,0,1)' & \\
        \IEEEeqnarrayrulerow\\
        \IEEEeqnarrayseprow[5pt]\\
   & \hat \be_7 && = && \{(0,1,\hat x_3)^t\,:\,0\leqslant\hat x_3\leqslant 1\} && && \hat \btau_7 && = && (0,0,1)' & \\
        \IEEEeqnarrayrulerow\\
        \IEEEeqnarrayseprow[5pt]\\
   & \hat \be_8 && = && \{(\hat x_1,1-\hat x_1,1)^t\,:\,0\leqslant\hat x_1\leqslant 1\} && && \hat \btau_8 && = && 2^{-\nicefrac{1}{2}}(1,-1,0)' & \\
        \IEEEeqnarrayrulerow\\
        \IEEEeqnarrayseprow[5pt]\\
   & \hat \be_9 && = && \{(\hat x_1,1-\hat x_1,0)^t\,:\,0\leqslant\hat x_1\leqslant 1\} && && \hat \btau_9 && = && 2^{-\nicefrac{1}{2}}(1,-1,0)' & \\
        \IEEEeqnarrayrulerow  
    \end{IEEEeqnarraybox*}
\end{table}
}

\def\edgeShapeTable{\begin{table}[!h]
    \centering  
    \caption{{Edge shape functions} on the reference pyramid}
    \label{shape_edge_table}
    \begin{IEEEeqnarraybox*}
    [\IEEEeqnarraystrutmode
    \IEEEeqnarraystrutsizeadd{2pt}{25pt}]{x/c/x/c/x/c/x}
        \IEEEeqnarrayseprow[5pt]\\
        &\IEEEeqnarraymulticol{6}{c}{
                {%\tiny
                {\scriptstyle\hat\bgamma_1} = 
                \left(
                    \begin{array}{c}
                        {1-z-y} \\[5pt]             
                        0 \\[5pt]
                        x-\frac{xy}{1-z}               
                    \end{array}
                \right)}\;\;
                {%\tiny
                {\scriptstyle\hat\bgamma_2} = 
                \left(
                    \begin{array}{c}
                        0 \\[5pt]
                        x \\[5pt]                
                        \frac{xy}{1-z}               
                    \end{array}
                \right)}\;\;
                {%\tiny
                {\scriptstyle\hat\bgamma_3} = 
                \left(
                    \begin{array}{c}
                        y \\[5pt]
                        0 \\[5pt]                
                        \frac{xy}{1-z}               
                    \end{array}
                \right)}\;\;
        {%\tiny
        {\scriptstyle\hat\bgamma_4} = 
        \left(
                    \begin{array}{c}
                        0 \\[8pt]                
                        {1-z-x} \\[8pt]
                        y-\frac{xy}{1-z}  
                    \end{array}
                \right)}}\\
        \IEEEeqnarrayseprow[5pt]\\
        &\IEEEeqnarraymulticol{6}{c}{
        {%\tiny
        {\scriptstyle\hat\bgamma_5} = 
                \left(
                        \begin{array}{c}
                            z-\frac{yz}{1-z} \\[8pt]               
                            z-\frac{xz}{1-z} \\[8pt]               
                            1-x-y+\frac{xy}{1-z}-\frac{xyz}{(1-z)^2}               
                        \end{array}
               \right)}\;\;
                {%\tiny
                {\scriptstyle\hat\bgamma_6} = 
                \left(
                    \begin{array}{c}
                        -z+\frac{yz}{1-z} \\[8pt]               
                        \frac{xz}{1-z} \\[8pt]               
                        x-\frac{xy}{1-z}+\frac{xyz}{(1-z)^2}               
                    \end{array}
                        \right)}}\\
        \IEEEeqnarrayseprow[5pt]\\
        &\IEEEeqnarraymulticol{6}{c}{
        {%\tiny
        {\scriptstyle\hat\bgamma_7} = 
                \left(
                        \begin{array}{c}
                            \frac{yz}{1-z} \\[8pt]               
                            -z+\frac{xz}{1-z} \\[8pt]               
                            y-\frac{xy}{1-z}+\frac{xyz}{(1-z)^2}               
                        \end{array}
               \right)}\;\;
        {%\tiny
                {\scriptstyle\hat\bgamma_8} = 
                \left(
                    \begin{array}{c}
                        -\frac{yz}{1-z} \\[8pt]               
                        -\frac{xz}{1-z} \\[8pt]               
                        \frac{xy}{1-z}-\frac{xyz}{(1-z)^2}               
                    \end{array}
                        \right)}}
    \end{IEEEeqnarraybox*}
\end{table}}

\def\faceShapeTable{
 \begin{table}[!h]
    \centering  
    \caption{{Face shape functions} on the reference pyramid}
    \label{shape_face_table}
    \begin{IEEEeqnarraybox*}
    [\IEEEeqnarraystrutmode
    \IEEEeqnarraystrutsizeadd{2pt}{25pt}]{x/c/x/c/x/c/x}
        \IEEEeqnarrayseprow[5pt]\\
        &\IEEEeqnarraymulticol{5}{c}{
                \raisebox{-15pt}{}\;\;
                {%\tiny
                {\scriptstyle\hat\bz_1} = 
                \left(
                        \begin{array}{c}
                            -\frac{xz}{1-z} \\[8pt]             
                            y-2+\frac{y}{1-z} \\[8pt]
                            z               
                        \end{array}
                        \right)}\;\;
                {%\tiny
                {\scriptstyle\hat\bz_2} = 
                \left(
                            \begin{array}{c}
                                x-2+\frac{x}{1-z} \\[8pt]
                                -\frac{yz}{1-z} \\[8pt]                
                                z               
                            \end{array}
                        \right)}}&\\
        \IEEEeqnarrayseprow[10pt]\\
        &
        {%\tiny
        {\scriptstyle\hat\bz_3} = 
        \left(
                    \begin{array}{c}
                        x+\frac{x}{1-z} \\[8pt]
                        -\frac{yz}{1-z} \\[8pt]                
                        z               
                    \end{array}
                \right)}&&
        {%\tiny
        {\scriptstyle\hat\bz_4} = 
        \left(
                    \begin{array}{c}
                        -\frac{xz}{1-z} \\[8pt]                
                        y+\frac{y}{1-z} \\[8pt]
                        z               
                    \end{array}
                \right)}&&
        {%\tiny
        {\scriptstyle\hat\bz_5} = 
        \left(
                    \begin{array}{c}
                        x \\[8pt]               
                        y \\[8pt]
                        z-1                 
                    \end{array}
                \right)}&\\
        \IEEEeqnarrayseprow[10pt]
    \end{IEEEeqnarraybox*}
\end{table}}

\def\facesOfPyramid{
\begin{table}[!h]
    \centering  
    \caption{Notation for the faces and positive normals of the
    reference pyramid.}
    \label{pyramidNotationTableFaces}
    \begin{IEEEeqnarraybox*}
      [\IEEEeqnarraystrutmode
      \IEEEeqnarraystrutsizeadd{2pt}{6pt}]{v/c/x/c/x/c/x/v/x/c/x/c/x/c/v/}
        \IEEEeqnarrayrulerow\\
        \IEEEeqnarrayseprow[5pt]\\
          & \hat f_1 && \subseteq &&  \{\hat x_2 = 0 \}            && && \hat{\bn}_1 && = && (0,-1,0)' & \\
        \IEEEeqnarrayrulerow\\
        \IEEEeqnarrayseprow[5pt]\\
          & \hat f_2 && \subseteq &&  \{\hat x_1 = 0 \}            && && \hat{\bn}_2 && = && (-1,0,0)' &\\
        \IEEEeqnarrayrulerow\\
        \IEEEeqnarrayseprow[5pt]\\
          & \hat f_3 && \subseteq &&  \{\hat x_1 + \hat x_3 = 1 \} && && \hat{\bn}_3 && = && 2^{-\nicefrac{1}{2}}(1,0,1)' &\\
        \IEEEeqnarrayrulerow\\
        \IEEEeqnarrayseprow[5pt]\\
          & \hat f_4 && \subseteq &&  \{\hat x_2 + \hat x_3 = 1 \} && && \hat{\bn}_4 && = && 2^{-\nicefrac{1}{2}}(0,1,1)' &\\
        \IEEEeqnarrayrulerow\\
        \IEEEeqnarrayseprow[5pt]\\
          & \hat f_5 && \subseteq &&  \{\hat x_3 = 0\}             && && \hat{\bn}_5 && = && (0,0,-1)' &\\
        \IEEEeqnarrayrulerow
    \end{IEEEeqnarraybox*}
\end{table}}

\def\edgesOfPyramid{
\begin{table}[!h]
    \centering  
    \caption{Notation for the edges and positive tangents of the
    reference pyramid.}
    \label{pyramidNotationTableEdges}
    \begin{IEEEeqnarraybox*}
      [\IEEEeqnarraystrutmode
      \IEEEeqnarraystrutsizeadd{2pt}{6pt}]{v/c/x/c/x/c/x/v/x/c/x/c/x/c/v/}
        \IEEEeqnarrayrulerow\\
        \IEEEeqnarrayseprow[5pt]\\
   & \hat \be_1 && = && \{(\hat x_1,0,0)^t\,:\,0\leqslant\hat x_1\leqslant 1\} && && \hat \btau_1 && = && (1,0,0)' & \\
        \IEEEeqnarrayrulerow\\
        \IEEEeqnarrayseprow[5pt]\\
   & \hat \be_2 && = && \{(1,\hat x_2,0)^t\,:\,0\leqslant\hat x_2\leqslant 1\} && && \hat \btau_2 && = && (0,1,0)' & \\
        \IEEEeqnarrayrulerow\\
        \IEEEeqnarrayseprow[5pt]\\
   & \hat \be_3 && = && \{(\hat x_1,1,0)^t\,:\,0\leqslant\hat x_1\leqslant 1\} && && \hat \btau_3 && = && (-1,0,0)' & \\
        \IEEEeqnarrayrulerow\\
        \IEEEeqnarrayseprow[5pt]\\
   & \hat \be_4 && = && \{(0,\hat x_2,0)^t\,:\,0\leqslant\hat x_2\leqslant 1\} && && \hat \btau_4 && = && (0,1,0)' & \\
        \IEEEeqnarrayrulerow\\
        \IEEEeqnarrayseprow[5pt]\\
   & \hat \be_5 && = && \{(0,0,\hat x_3)^t\,:\,0\leqslant\hat x_3\leqslant 1\} && && \hat \btau_5 && = && (0,0,1)' & \\
        \IEEEeqnarrayrulerow\\
        \IEEEeqnarrayseprow[5pt]\\
   & \hat \be_6 && = && \{(1-\hat x_3,0,\hat x_3)^t\,:\,0\leqslant\hat x_3\leqslant 1\} && && \hat \btau_6 && = && 2^{-\nicefrac{1}{2}}(-1,0,1)' & \\
        \IEEEeqnarrayrulerow\\
        \IEEEeqnarrayseprow[5pt]\\
   & \hat \be_7 && = && \{(0,1-\hat x_3,\hat x_3)^t\,:\,0\leqslant\hat x_3\leqslant 1\} && && \hat \btau_7 && = && 2^{-\nicefrac{1}{2}}(0,-1,1)' & \\
        \IEEEeqnarrayrulerow\\
        \IEEEeqnarrayseprow[5pt]\\
   & \hat \be_8 && = && \{(1-\hat x_3,1-\hat x_3,\hat x_3)^t\,:\,0\leqslant\hat x_3\leqslant 1\} && && \hat \btau_8 && = && 3^{-\nicefrac{1}{2}}(-1,-1,1) & \\
        \IEEEeqnarrayrulerow
    \end{IEEEeqnarraybox*}
\end{table}}
\def\rescaledPrismTikz
{
\begin{figure}[!h]
	\centering
		\begin{tikzpicture}[scale=.4]
		  \draw[->] (0, 0) -- (0, 0, 3);	
		  \draw[->] (0, 0) -- (3,0);
		  \draw[->] (0, 0) --  (0, 5, 0);
		  \draw (0, 0, 1) -- (2, 0, 0); 
		  \draw (0, 3.5, 1) -- (2, 3.5, 0); 
		  \draw (2, 0, 0) -- node[label={right:$_{_{h_3}}$}] { } (2, 3.5, 0);
		  \draw (0, 0, 1) -- (0, 3.5, 1); 
		  \draw (0,3.5,1) -- node[label={left:$_{_{h_1}}$}] { }  (0,3.5,0) ;
		  \draw (0,3.5,0) -- node[label={above:$_{_{h_2}}$}] { } (2,3.5,0);
		\end{tikzpicture}
	\caption{Rescaled Prism}
	\label{rescaled_prism}		
\end{figure}
}

\def\tetrahedralmacroelementA{
\begin{tikzpicture}[scale=1.8]	\draw (0.0,1.0,0.0);
	\draw (0.0,0.0,0.0) -- (0.0,1.0,0.0);
	\draw (0.0,0.0,0.0) -- (1.0,0.0,0.0);
	\draw (1.0,0.0,0.0);
	\draw (0.0,0.0,0.0);
	\draw (0.0,1.0,0.0) -- (1.0,0.0,0.0);
	\fill[red] (0.0,0.0,1.0) circle (1pt);
	\draw[orange] (0.0,0.0,0.0) -- (0.0,0.0,1.0);
	\draw (0.0,1.0,0.0) -- (0.0,0.0,1.0);
	\draw (1.0,0.0,0.0) -- (0.0,0.0,1.0);
\end{tikzpicture}}

\def\tetrahedralmacroelementB{
\begin{tikzpicture}[scale=1.8]	\draw (0.0,0.3869,0.0) -- (0.0,0.0,0.0);
	\draw (0.0,0.0,0.0) -- (0.0,0.3869,0.0) -- (0.0,1.0,0.0);
	\draw (0.0,0.0,0.0) -- (0.3869,0.0,0.0) -- (0.0,0.0,0.0);
	\draw (0.0,0.3869,0.0) -- (0.5,0.5,0.0);
	\draw (0.3869,0.0,0.0) -- (0.5,0.5,0.0);
	\draw (0.0,0.0,0.0) -- (0.3869,0.0,0.0) -- (1.0,0.0,0.0);
	\draw (1.0,0.0,0.0);
	\fill[red] (0.0,0.0,1.0) circle (1pt);
	\draw (0.0,0.3869,0.0) -- (0.3869,0.0,0.0);
	\draw (0.0,1.0,0.0) -- (0.5,0.5,0.0) -- (1.0,0.0,0.0);
	\draw (0.0,0.0,0.6130) -- (0.0,0.0,0.0);
	\draw (0.0,0.3869,0.6130);
	\draw (0.0,0.0,0.6130) -- (0.0,0.3869,0.6130);
	\draw (0.0,0.0,0.6130) -- (0.3869,0.0,0.6130);
	\draw (0.3869,0.0,0.6130);
	\draw (0.0,0.0,0.6130);
	\draw (0.0,0.3869,0.6130) -- (0.3869,0.0,0.6130);
	\draw[orange] (0.0,0.0,0.0) -- (0.0,0.0,0.6130) -- (0.0,0.0,1.0);
	\draw (0.0,0.3869,0.0) -- (0.0,0.3869,0.6130);
	\draw (0.0,1.0,0.0) -- (0.0,0.3869,0.6130) -- (0.0,0.0,1.0);
	\draw (1.0,0.0,0.0) -- (0.3869,0.0,0.6130) -- (0.0,0.0,1.0);
	\draw (0.3869,0.0,0.0) -- (0.3869,0.0,0.6130);
	\draw (0.5,0.5,0.0) -- (0.3869,0.0,0.6130);
	\draw (0.5,0.5,0.0) -- (0.0,0.3869,0.6130);
\end{tikzpicture}}

\def\tetrahedralmacroelementC{
\begin{tikzpicture}[scale=1.8]	\draw (0.0,0.2220,0.0) -- (0.0,0.0,0.0) -- (0.0,0.0,0.0);
	\draw (0.0,0.5738,0.0) -- (0.0,0.0,0.0);
	\draw (0.0,0.0,0.0) -- (0.0,0.2220,0.0) -- (0.0,0.5738,0.0) -- (0.0,1.0,0.0);
	\draw (0.0,0.0,0.0) -- (0.2220,0.0,0.0) -- (0.0,0.0,0.0) -- (0.0,0.0,0.0);
	\draw (0.0,0.2220,0.0) -- (0.2869,0.2869,0.0) -- (0.0,0.0,0.0);
	\draw (0.0,0.5738,0.0) -- (0.3333,0.6666,0.0);
	\draw (0.2220,0.0,0.0) -- (0.2869,0.2869,0.0) -- (0.3333,0.6666,0.0);
	\draw (0.0,0.0,0.0) -- (0.2220,0.0,0.0) -- (0.5738,0.0,0.0) -- (0.0,0.0,0.0);
	\draw (0.0,0.2220,0.0) -- (0.2869,0.2869,0.0) -- (0.6666,0.3333,0.0);
	\draw (0.5738,0.0,0.0) -- (0.6666,0.3333,0.0);
	\draw (0.0,0.0,0.0) -- (0.2220,0.0,0.0) -- (0.5738,0.0,0.0) -- (1.0,0.0,0.0);
	\draw (0.0,0.2220,0.0) -- (0.2220,0.0,0.0);
	\draw (0.0,0.5738,0.0) -- (0.2869,0.2869,0.0) -- (0.5738,0.0,0.0);
	\draw (0.0,1.0,0.0) -- (0.3333,0.6666,0.0) -- (0.6666,0.3333,0.0) -- (1.0,0.0,0.0);
	\draw (0.0,0.0,0.4261) -- (0.0,0.0,0.0) -- (0.0,0.0,0.0);
	\draw (0.0,0.2220,0.4261) -- (0.0,0.0,0.0);
	\draw (0.0,0.0,0.4261) -- (0.0,0.2220,0.4261) -- (0.0,0.5738,0.4261);
	\draw (0.0,0.0,0.4261) -- (0.2220,0.0,0.4261) -- (0.0,0.0,0.0);
	\draw (0.0,0.2220,0.4261) -- (0.2869,0.2869,0.4261);
	\draw (0.2220,0.0,0.4261) -- (0.2869,0.2869,0.4261);
	\draw (0.0,0.0,0.4261) -- (0.2220,0.0,0.4261) -- (0.5738,0.0,0.4261);
	\draw (0.0,0.2220,0.4261) -- (0.2220,0.0,0.4261);
	\draw (0.0,0.5738,0.4261) -- (0.2869,0.2869,0.4261) -- (0.5738,0.0,0.4261);
	\draw (0.0,0.0,0.7779) -- (0.0,0.0,0.0);
	\draw (0.0,0.0,0.7779) -- (0.0,0.2220,0.7779);
	\draw (0.0,0.0,0.7779) -- (0.2220,0.0,0.7779);
	\draw (0.0,0.2220,0.7779) -- (0.2220,0.0,0.7779);
	\draw[orange] (0.0,0.0,0.0) -- (0.0,0.0,0.4261) -- (0.0,0.0,0.7779) -- (0.0,0.0,1.0);
	\draw (0.0,0.2220,0.0) -- (0.0,0.2220,0.4261) -- (0.0,0.2220,0.7779);
	\draw (0.0,0.5738,0.0) -- (0.0,0.5738,0.4261);
	\draw (0.0,1.0,0.0) -- (0.0,0.5738,0.4261) -- (0.0,0.2220,0.7779) -- (0.0,0.0,1.0);
	\draw (1.0,0.0,0.0) -- (0.5738,0.0,0.4261) -- (0.2220,0.0,0.7779) -- (0.0,0.0,1.0);
	\draw (0.2220,0.0,0.0) -- (0.2220,0.0,0.4261) -- (0.2220,0.0,0.7779);
	\draw (0.2869,0.2869,0.0) -- (0.2869,0.2869,0.4261);
	\draw (0.3333,0.6666,0.0) -- (0.2869,0.2869,0.4261) -- (0.2220,0.0,0.7779);
	\draw (0.6666,0.3333,0.0) -- (0.2869,0.2869,0.4261) -- (0.0,0.2220,0.7779);
	\draw (0.5738,0.0,0.0) -- (0.5738,0.0,0.4261);
	\draw (0.6666,0.3333,0.0) -- (0.5738,0.0,0.4261);
	\draw (0.3333,0.6666,0.0) -- (0.0,0.5738,0.4261);
	\fill[red] (0,0,1) circle (1pt);
\end{tikzpicture}}

\def\tetrahedralmacroelementD{
\begin{tikzpicture}[scale=1.8]	\fill[red] (0,0,1) circle (1pt);
	\draw (0.0,0.1497,0.0) -- (0.0,0.0,0.0) -- (0.0,0.0,0.0) -- (0.0,0.0,0.0);
	\draw (0.0,0.3869,0.0) -- (0.0,0.0,0.0) -- (0.0,0.0,0.0);
	\draw (0.0,0.6742,0.0) -- (0.0,0.0,0.0);
	\draw (0.0,1.0,0.0);
	\draw (0.0,0.0,0.0) -- (0.0,0.1497,0.0) -- (0.0,0.3869,0.0) -- (0.0,0.6742,0.0) -- (0.0,1.0,0.0);
	\draw (0.0,0.0,0.0) -- (0.1497,0.0,0.0) -- (0.0,0.0,0.0) -- (0.0,0.0,0.0) -- (0.0,0.0,0.0);
	\draw (0.0,0.1497,0.0) -- (0.1934,0.1934,0.0) -- (0.0,0.0,0.0) -- (0.0,0.0,0.0);
	\draw (0.0,0.3869,0.0) -- (0.2247,0.4495,0.0) -- (0.0,0.0,0.0);
	\draw (0.0,0.6742,0.0) -- (0.25,0.75,0.0);
	\draw (0.1497,0.0,0.0) -- (0.1934,0.1934,0.0) -- (0.2247,0.4495,0.0) -- (0.25,0.75,0.0);
	\draw (0.0,0.0,0.0) -- (0.1497,0.0,0.0) -- (0.3869,0.0,0.0) -- (0.0,0.0,0.0) -- (0.0,0.0,0.0);
	\draw (0.0,0.1497,0.0) -- (0.1934,0.1934,0.0) -- (0.4495,0.2247,0.0) -- (0.0,0.0,0.0);
	\draw (0.0,0.3869,0.0) -- (0.2247,0.4495,0.0) -- (0.5,0.5,0.0);
	\draw (0.3869,0.0,0.0) -- (0.4495,0.2247,0.0) -- (0.5,0.5,0.0);
	\draw (0.0,0.0,0.0) -- (0.1497,0.0,0.0) -- (0.3869,0.0,0.0) -- (0.6742,0.0,0.0) -- (0.0,0.0,0.0);
	\draw (0.0,0.1497,0.0) -- (0.1934,0.1934,0.0) -- (0.4495,0.2247,0.0) -- (0.75,0.25,0.0);
	\draw (0.6742,0.0,0.0) -- (0.75,0.25,0.0);
	\draw (0.0,0.0,0.0) -- (0.1497,0.0,0.0) -- (0.3869,0.0,0.0) -- (0.6742,0.0,0.0) -- (1.0,0.0,0.0);
	\draw (1.0,0.0,0.0);
	\draw (0.0,0.0,0.0);
	\draw (0.0,0.1497,0.0) -- (0.1497,0.0,0.0);
	\draw (0.0,0.3869,0.0) -- (0.1934,0.1934,0.0) -- (0.3869,0.0,0.0);
	\draw (0.0,0.6742,0.0) -- (0.2247,0.4495,0.0) -- (0.4495,0.2247,0.0) -- (0.6742,0.0,0.0);
	\draw (0.0,1.0,0.0) -- (0.25,0.75,0.0) -- (0.5,0.5,0.0) -- (0.75,0.25,0.0) -- (1.0,0.0,0.0);
	\draw (0.0,0.0,0.3257) -- (0.0,0.0,0.0) -- (0.0,0.0,0.0) -- (0.0,0.0,0.0);
	\draw (0.0,0.1497,0.3257) -- (0.0,0.0,0.0) -- (0.0,0.0,0.0);
	\draw (0.0,0.3869,0.3257) -- (0.0,0.0,0.0);
	\draw (0.0,0.6742,0.3257);
	\draw (0.0,0.0,0.3257) -- (0.0,0.1497,0.3257) -- (0.0,0.3869,0.3257) -- (0.0,0.6742,0.3257);
	\draw (0.0,0.0,0.3257) -- (0.1497,0.0,0.3257) -- (0.0,0.0,0.0) -- (0.0,0.0,0.0);
	\draw (0.0,0.1497,0.3257) -- (0.1934,0.1934,0.3257) -- (0.0,0.0,0.0);
	\draw (0.0,0.3869,0.3257) -- (0.2247,0.4495,0.3257);
	\draw (0.1497,0.0,0.3257) -- (0.1934,0.1934,0.3257) -- (0.2247,0.4495,0.3257);
	\draw (0.0,0.0,0.3257) -- (0.1497,0.0,0.3257) -- (0.3869,0.0,0.3257) -- (0.0,0.0,0.0);
	\draw (0.0,0.1497,0.3257) -- (0.1934,0.1934,0.3257) -- (0.4495,0.2247,0.3257);
	\draw (0.3869,0.0,0.3257) -- (0.4495,0.2247,0.3257);
	\draw (0.0,0.0,0.3257) -- (0.1497,0.0,0.3257) -- (0.3869,0.0,0.3257) -- (0.6742,0.0,0.3257);
	\draw (0.6742,0.0,0.3257);
	\draw (0.0,0.0,0.3257);
	\draw (0.0,0.1497,0.3257) -- (0.1497,0.0,0.3257);
	\draw (0.0,0.3869,0.3257) -- (0.1934,0.1934,0.3257) -- (0.3869,0.0,0.3257);
	\draw (0.0,0.6742,0.3257) -- (0.2247,0.4495,0.3257) -- (0.4495,0.2247,0.3257) -- (0.6742,0.0,0.3257);
	\draw (0.0,0.0,0.6130) -- (0.0,0.0,0.0) -- (0.0,0.0,0.0);
	\draw (0.0,0.1497,0.6130) -- (0.0,0.0,0.0);
	\draw (0.0,0.3869,0.6130);
	\draw (0.0,0.0,0.6130) -- (0.0,0.1497,0.6130) -- (0.0,0.3869,0.6130);
	\draw (0.0,0.0,0.6130) -- (0.1497,0.0,0.6130) -- (0.0,0.0,0.0);
	\draw (0.0,0.1497,0.6130) -- (0.1934,0.1934,0.6130);
	\draw (0.1497,0.0,0.6130) -- (0.1934,0.1934,0.6130);
	\draw (0.0,0.0,0.6130) -- (0.1497,0.0,0.6130) -- (0.3869,0.0,0.6130);
	\draw (0.3869,0.0,0.6130);
	\draw (0.0,0.0,0.6130);
	\draw (0.0,0.1497,0.6130) -- (0.1497,0.0,0.6130);
	\draw (0.0,0.3869,0.6130) -- (0.1934,0.1934,0.6130) -- (0.3869,0.0,0.6130);
	\draw (0.0,0.0,0.8502) -- (0.0,0.0,0.0);
	\draw (0.0,0.1497,0.8502);
	\draw (0.0,0.0,0.8502) -- (0.0,0.1497,0.8502);
	\draw (0.0,0.0,0.8502) -- (0.1497,0.0,0.8502);
	\draw (0.1497,0.0,0.8502);
	\draw (0.0,0.0,0.8502);
	\draw (0.0,0.1497,0.8502) -- (0.1497,0.0,0.8502);
	\draw[orange] (0.0,0.0,0.0) -- (0.0,0.0,0.3257) -- (0.0,0.0,0.6130) -- (0.0,0.0,0.8502) -- (0.0,0.0,1.0);
	\draw (0.0,0.1497,0.0) -- (0.0,0.1497,0.3257) -- (0.0,0.1497,0.6130) -- (0.0,0.1497,0.8502);
	\draw (0.0,0.3869,0.0) -- (0.0,0.3869,0.3257) -- (0.0,0.3869,0.6130);
	\draw (0.0,0.6742,0.0) -- (0.0,0.6742,0.3257);
	\draw (0.0,1.0,0.0) -- (0.0,0.6742,0.3257) -- (0.0,0.3869,0.6130) -- (0.0,0.1497,0.8502) -- (0.0,0.0,1.0);
	\draw (1.0,0.0,0.0) -- (0.6742,0.0,0.3257) -- (0.3869,0.0,0.6130) -- (0.1497,0.0,0.8502) -- (0.0,0.0,1.0);
	\draw (0.1497,0.0,0.0) -- (0.1497,0.0,0.3257) -- (0.1497,0.0,0.6130) -- (0.1497,0.0,0.8502);
	\draw (0.1934,0.1934,0.0) -- (0.1934,0.1934,0.3257) -- (0.1934,0.1934,0.6130);
	\draw (0.2247,0.4495,0.0) -- (0.2247,0.4495,0.3257);
	\draw (0.25,0.75,0.0) -- (0.2247,0.4495,0.3257) -- (0.1934,0.1934,0.6130) -- (0.1497,0.0,0.8502);
	\draw (0.75,0.25,0.0) -- (0.4495,0.2247,0.3257) -- (0.1934,0.1934,0.6130) -- (0.0,0.1497,0.8502);
	\draw (0.3869,0.0,0.0) -- (0.3869,0.0,0.3257) -- (0.3869,0.0,0.6130);
	\draw (0.4495,0.2247,0.0) -- (0.4495,0.2247,0.3257);
	\draw (0.5,0.5,0.0) -- (0.4495,0.2247,0.3257) -- (0.3869,0.0,0.6130);
	\draw (0.5,0.5,0.0) -- (0.2247,0.4495,0.3257) -- (0.0,0.3869,0.6130);
	\draw (0.6742,0.0,0.0) -- (0.6742,0.0,0.3257);
	\draw (0.75,0.25,0.0) -- (0.6742,0.0,0.3257);
	\draw (0.25,0.75,0.0) -- (0.0,0.6742,0.3257);
\end{tikzpicture}}

\def\tetrahedralmacroelementE{
\begin{tikzpicture}[scale=1.8]	\fill[red] (0,0,1) circle (1pt);
	\draw (0.0,0.0,0.0);
	\draw (0.0,0.1102,0.0) -- (0.0,0.0,0.0) -- (0.0,0.0,0.0) -- (0.0,0.0,0.0) -- (0.0,0.0,0.0);
	\draw (0.0,0.2850,0.0) -- (0.0,0.0,0.0) -- (0.0,0.0,0.0) -- (0.0,0.0,0.0);
	\draw (0.0,0.4967,0.0) -- (0.0,0.0,0.0) -- (0.0,0.0,0.0);
	\draw (0.0,0.7366,0.0) -- (0.0,0.0,0.0);
	\draw (0.0,1.0,0.0);
	\draw (0.0,0.0,0.0) -- (0.0,0.1102,0.0) -- (0.0,0.2850,0.0) -- (0.0,0.4967,0.0) -- (0.0,0.7366,0.0) -- (0.0,1.0,0.0);
	\draw (0.0,0.0,0.0) -- (0.1102,0.0,0.0) -- (0.0,0.0,0.0) -- (0.0,0.0,0.0) -- (0.0,0.0,0.0) -- (0.0,0.0,0.0);
	\draw (0.0,0.1102,0.0) -- (0.1425,0.1425,0.0) -- (0.0,0.0,0.0) -- (0.0,0.0,0.0) -- (0.0,0.0,0.0);
	\draw (0.0,0.2850,0.0) -- (0.1655,0.3311,0.0) -- (0.0,0.0,0.0) -- (0.0,0.0,0.0);
	\draw (0.0,0.4967,0.0) -- (0.1841,0.5524,0.0) -- (0.0,0.0,0.0);
	\draw (0.0,0.7366,0.0) -- (0.2,0.8,0.0);
	\draw (0.1102,0.0,0.0) -- (0.1425,0.1425,0.0) -- (0.1655,0.3311,0.0) -- (0.1841,0.5524,0.0) -- (0.2,0.8,0.0);
	\draw (0.0,0.0,0.0) -- (0.1102,0.0,0.0) -- (0.2850,0.0,0.0) -- (0.0,0.0,0.0) -- (0.0,0.0,0.0) -- (0.0,0.0,0.0);
	\draw (0.0,0.1102,0.0) -- (0.1425,0.1425,0.0) -- (0.3311,0.1655,0.0) -- (0.0,0.0,0.0) -- (0.0,0.0,0.0);
	\draw (0.0,0.2850,0.0) -- (0.1655,0.3311,0.0) -- (0.3683,0.3683,0.0) -- (0.0,0.0,0.0);
	\draw (0.0,0.4967,0.0) -- (0.1841,0.5524,0.0) -- (0.4,0.6,0.0);
	\draw (0.2850,0.0,0.0) -- (0.3311,0.1655,0.0) -- (0.3683,0.3683,0.0) -- (0.4,0.6,0.0);
	\draw (0.0,0.0,0.0) -- (0.1102,0.0,0.0) -- (0.2850,0.0,0.0) -- (0.4967,0.0,0.0) -- (0.0,0.0,0.0) -- (0.0,0.0,0.0);
	\draw (0.0,0.1102,0.0) -- (0.1425,0.1425,0.0) -- (0.3311,0.1655,0.0) -- (0.5524,0.1841,0.0) -- (0.0,0.0,0.0);
	\draw (0.0,0.2850,0.0) -- (0.1655,0.3311,0.0) -- (0.3683,0.3683,0.0) -- (0.6,0.4,0.0);
	\draw (0.4967,0.0,0.0) -- (0.5524,0.1841,0.0) -- (0.6,0.4,0.0);
	\draw (0.0,0.0,0.0) -- (0.1102,0.0,0.0) -- (0.2850,0.0,0.0) -- (0.4967,0.0,0.0) -- (0.7366,0.0,0.0) -- (0.0,0.0,0.0);
	\draw (0.0,0.1102,0.0) -- (0.1425,0.1425,0.0) -- (0.3311,0.1655,0.0) -- (0.5524,0.1841,0.0) -- (0.8,0.2,0.0);
	\draw (0.7366,0.0,0.0) -- (0.8,0.2,0.0);
	\draw (0.0,0.0,0.0) -- (0.1102,0.0,0.0) -- (0.2850,0.0,0.0) -- (0.4967,0.0,0.0) -- (0.7366,0.0,0.0) -- (1.0,0.0,0.0);
	\draw (0.0,0.1102,0.0) -- (0.1102,0.0,0.0);
	\draw (0.0,0.2850,0.0) -- (0.1425,0.1425,0.0) -- (0.2850,0.0,0.0);
	\draw (0.0,0.4967,0.0) -- (0.1655,0.3311,0.0) -- (0.3311,0.1655,0.0) -- (0.4967,0.0,0.0);
	\draw (0.0,0.7366,0.0) -- (0.1841,0.5524,0.0) -- (0.3683,0.3683,0.0) -- (0.5524,0.1841,0.0) -- (0.7366,0.0,0.0);
	\draw (0.0,1.0,0.0) -- (0.2,0.8,0.0) -- (0.4,0.6,0.0) -- (0.6,0.4,0.0) -- (0.8,0.2,0.0) -- (1.0,0.0,0.0);
	\draw (0.0,0.0,0.2633) -- (0.0,0.0,0.0) -- (0.0,0.0,0.0) -- (0.0,0.0,0.0) -- (0.0,0.0,0.0);
	\draw (0.0,0.1102,0.2633) -- (0.0,0.0,0.0) -- (0.0,0.0,0.0) -- (0.0,0.0,0.0);
	\draw (0.0,0.2850,0.2633) -- (0.0,0.0,0.0) -- (0.0,0.0,0.0);
	\draw (0.0,0.4967,0.2633) -- (0.0,0.0,0.0);
	\draw (0.0,0.7366,0.2633);
	\draw (0.0,0.0,0.2633) -- (0.0,0.1102,0.2633) -- (0.0,0.2850,0.2633) -- (0.0,0.4967,0.2633) -- (0.0,0.7366,0.2633);
	\draw (0.0,0.0,0.2633) -- (0.1102,0.0,0.2633) -- (0.0,0.0,0.0) -- (0.0,0.0,0.0) -- (0.0,0.0,0.0);
	\draw (0.0,0.1102,0.2633) -- (0.1425,0.1425,0.2633) -- (0.0,0.0,0.0) -- (0.0,0.0,0.0);
	\draw (0.0,0.2850,0.2633) -- (0.1655,0.3311,0.2633) -- (0.0,0.0,0.0);
	\draw (0.0,0.4967,0.2633) -- (0.1841,0.5524,0.2633);
	\draw (0.1102,0.0,0.2633) -- (0.1425,0.1425,0.2633) -- (0.1655,0.3311,0.2633) -- (0.1841,0.5524,0.2633);
	\draw (0.0,0.0,0.2633) -- (0.1102,0.0,0.2633) -- (0.2850,0.0,0.2633) -- (0.0,0.0,0.0) -- (0.0,0.0,0.0);
	\draw (0.0,0.1102,0.2633) -- (0.1425,0.1425,0.2633) -- (0.3311,0.1655,0.2633) -- (0.0,0.0,0.0);
	\draw (0.0,0.2850,0.2633) -- (0.1655,0.3311,0.2633) -- (0.3683,0.3683,0.2633);
	\draw (0.2850,0.0,0.2633) -- (0.3311,0.1655,0.2633) -- (0.3683,0.3683,0.2633);
	\draw (0.0,0.0,0.2633) -- (0.1102,0.0,0.2633) -- (0.2850,0.0,0.2633) -- (0.4967,0.0,0.2633) -- (0.0,0.0,0.0);
	\draw (0.0,0.1102,0.2633) -- (0.1425,0.1425,0.2633) -- (0.3311,0.1655,0.2633) -- (0.5524,0.1841,0.2633);
	\draw (0.4967,0.0,0.2633) -- (0.5524,0.1841,0.2633);
	\draw (0.0,0.0,0.2633) -- (0.1102,0.0,0.2633) -- (0.2850,0.0,0.2633) -- (0.4967,0.0,0.2633) -- (0.7366,0.0,0.2633);
	\draw (0.7366,0.0,0.2633);
	\draw (0.0,0.0,0.2633);
	\draw (0.0,0.1102,0.2633) -- (0.1102,0.0,0.2633);
	\draw (0.0,0.2850,0.2633) -- (0.1425,0.1425,0.2633) -- (0.2850,0.0,0.2633);
	\draw (0.0,0.4967,0.2633) -- (0.1655,0.3311,0.2633) -- (0.3311,0.1655,0.2633) -- (0.4967,0.0,0.2633);
	\draw (0.0,0.7366,0.2633) -- (0.1841,0.5524,0.2633) -- (0.3683,0.3683,0.2633) -- (0.5524,0.1841,0.2633) -- (0.7366,0.0,0.2633);
	\draw (0.0,0.0,0.5032) -- (0.0,0.0,0.0) -- (0.0,0.0,0.0) -- (0.0,0.0,0.0);
	\draw (0.0,0.1102,0.5032) -- (0.0,0.0,0.0) -- (0.0,0.0,0.0);
	\draw (0.0,0.2850,0.5032) -- (0.0,0.0,0.0);
	\draw (0.0,0.4967,0.5032);
	\draw (0.0,0.0,0.5032) -- (0.0,0.1102,0.5032) -- (0.0,0.2850,0.5032) -- (0.0,0.4967,0.5032);
	\draw (0.0,0.0,0.5032) -- (0.1102,0.0,0.5032) -- (0.0,0.0,0.0) -- (0.0,0.0,0.0);
	\draw (0.0,0.1102,0.5032) -- (0.1425,0.1425,0.5032) -- (0.0,0.0,0.0);
	\draw (0.0,0.2850,0.5032) -- (0.1655,0.3311,0.5032);
	\draw (0.1102,0.0,0.5032) -- (0.1425,0.1425,0.5032) -- (0.1655,0.3311,0.5032);
	\draw (0.0,0.0,0.5032) -- (0.1102,0.0,0.5032) -- (0.2850,0.0,0.5032) -- (0.0,0.0,0.0);
	\draw (0.0,0.1102,0.5032) -- (0.1425,0.1425,0.5032) -- (0.3311,0.1655,0.5032);
	\draw (0.2850,0.0,0.5032) -- (0.3311,0.1655,0.5032);
	\draw (0.0,0.0,0.5032) -- (0.1102,0.0,0.5032) -- (0.2850,0.0,0.5032) -- (0.4967,0.0,0.5032);
	\draw (0.4967,0.0,0.5032);
	\draw (0.0,0.0,0.5032);
	\draw (0.0,0.1102,0.5032) -- (0.1102,0.0,0.5032);
	\draw (0.0,0.2850,0.5032) -- (0.1425,0.1425,0.5032) -- (0.2850,0.0,0.5032);
	\draw (0.0,0.4967,0.5032) -- (0.1655,0.3311,0.5032) -- (0.3311,0.1655,0.5032) -- (0.4967,0.0,0.5032);
	\draw (0.0,0.0,0.7149) -- (0.0,0.0,0.0) -- (0.0,0.0,0.0);
	\draw (0.0,0.1102,0.7149) -- (0.0,0.0,0.0);
	\draw (0.0,0.2850,0.7149);
	\draw (0.0,0.0,0.7149) -- (0.0,0.1102,0.7149) -- (0.0,0.2850,0.7149);
	\draw (0.0,0.0,0.7149) -- (0.1102,0.0,0.7149) -- (0.0,0.0,0.0);
	\draw (0.0,0.1102,0.7149) -- (0.1425,0.1425,0.7149);
	\draw (0.1102,0.0,0.7149) -- (0.1425,0.1425,0.7149);
	\draw (0.0,0.0,0.7149) -- (0.1102,0.0,0.7149) -- (0.2850,0.0,0.7149);
	\draw (0.2850,0.0,0.7149);
	\draw (0.0,0.0,0.7149);
	\draw (0.0,0.1102,0.7149) -- (0.1102,0.0,0.7149);
	\draw (0.0,0.2850,0.7149) -- (0.1425,0.1425,0.7149) -- (0.2850,0.0,0.7149);
	\draw (0.0,0.0,0.8897) -- (0.0,0.0,0.0);
	\draw (0.0,0.1102,0.8897);
	\draw (0.0,0.0,0.8897) -- (0.0,0.1102,0.8897);
	\draw (0.0,0.0,0.8897) -- (0.1102,0.0,0.8897);
	\draw (0.1102,0.0,0.8897);
	\draw (0.0,0.0,0.8897);
	\draw (0.0,0.1102,0.8897) -- (0.1102,0.0,0.8897);
	\draw[orange] (0.0,0.0,0.0) -- (0.0,0.0,0.2633) -- (0.0,0.0,0.5032) -- (0.0,0.0,0.7149) -- (0.0,0.0,0.8897) -- (0.0,0.0,1.0);
	\draw (0.0,0.1102,0.0) -- (0.0,0.1102,0.2633) -- (0.0,0.1102,0.5032) -- (0.0,0.1102,0.7149) -- (0.0,0.1102,0.8897);
	\draw (0.0,0.2850,0.0) -- (0.0,0.2850,0.2633) -- (0.0,0.2850,0.5032) -- (0.0,0.2850,0.7149);
	\draw (0.0,0.4967,0.0) -- (0.0,0.4967,0.2633) -- (0.0,0.4967,0.5032);
	\draw (0.0,0.7366,0.0) -- (0.0,0.7366,0.2633);
	\draw (0.0,1.0,0.0) -- (0.0,0.7366,0.2633) -- (0.0,0.4967,0.5032) -- (0.0,0.2850,0.7149) -- (0.0,0.1102,0.8897) -- (0.0,0.0,1.0);
	\draw (1.0,0.0,0.0) -- (0.7366,0.0,0.2633) -- (0.4967,0.0,0.5032) -- (0.2850,0.0,0.7149) -- (0.1102,0.0,0.8897) -- (0.0,0.0,1.0);
	\draw (0.1102,0.0,0.0) -- (0.1102,0.0,0.2633) -- (0.1102,0.0,0.5032) -- (0.1102,0.0,0.7149) -- (0.1102,0.0,0.8897);
	\draw (0.1425,0.1425,0.0) -- (0.1425,0.1425,0.2633) -- (0.1425,0.1425,0.5032) -- (0.1425,0.1425,0.7149);
	\draw (0.1655,0.3311,0.0) -- (0.1655,0.3311,0.2633) -- (0.1655,0.3311,0.5032);
	\draw (0.1841,0.5524,0.0) -- (0.1841,0.5524,0.2633);
	\draw (0.2,0.8,0.0) -- (0.1841,0.5524,0.2633) -- (0.1655,0.3311,0.5032) -- (0.1425,0.1425,0.7149) -- (0.1102,0.0,0.8897);
	\draw (0.8,0.2,0.0) -- (0.5524,0.1841,0.2633) -- (0.3311,0.1655,0.5032) -- (0.1425,0.1425,0.7149) -- (0.0,0.1102,0.8897);
	\draw (0.2850,0.0,0.0) -- (0.2850,0.0,0.2633) -- (0.2850,0.0,0.5032) -- (0.2850,0.0,0.7149);
	\draw (0.3311,0.1655,0.0) -- (0.3311,0.1655,0.2633) -- (0.3311,0.1655,0.5032);
	\draw (0.3683,0.3683,0.0) -- (0.3683,0.3683,0.2633);
	\draw (0.4,0.6,0.0) -- (0.3683,0.3683,0.2633) -- (0.3311,0.1655,0.5032) -- (0.2850,0.0,0.7149);
	\draw (0.6,0.4,0.0) -- (0.3683,0.3683,0.2633) -- (0.1655,0.3311,0.5032) -- (0.0,0.2850,0.7149);
	\draw (0.4967,0.0,0.0) -- (0.4967,0.0,0.2633) -- (0.4967,0.0,0.5032);
	\draw (0.5524,0.1841,0.0) -- (0.5524,0.1841,0.2633);
	\draw (0.6,0.4,0.0) -- (0.5524,0.1841,0.2633) -- (0.4967,0.0,0.5032);
	\draw (0.4,0.6,0.0) -- (0.1841,0.5524,0.2633) -- (0.0,0.4967,0.5032);
	\draw (0.7366,0.0,0.0) -- (0.7366,0.0,0.2633);
	\draw (0.8,0.2,0.0) -- (0.7366,0.0,0.2633);
	\draw (0.2,0.8,0.0) -- (0.0,0.7366,0.2633);
\end{tikzpicture}}

\def\tetrahedralmacroelementF{
\begin{tikzpicture}[scale=1.8]	\fill[red] (0,0,1) circle (1pt);
	\draw (0.0,0.0,0.0) -- (0.0,0.0,0.0) -- (0.0,0.0,0.0) -- (0.0,0.0,0.0) -- (0.0,0.0,0.0) -- (0.0,0.0,0.0) -- (0.0,0.0,0.0);
	\draw (0.0,0.0859,0.0) -- (0.0,0.0,0.0) -- (0.0,0.0,0.0) -- (0.0,0.0,0.0) -- (0.0,0.0,0.0) -- (0.0,0.0,0.0);
	\draw (0.0,0.2220,0.0) -- (0.0,0.0,0.0) -- (0.0,0.0,0.0) -- (0.0,0.0,0.0) -- (0.0,0.0,0.0);
	\draw (0.0,0.3869,0.0) -- (0.0,0.0,0.0) -- (0.0,0.0,0.0) -- (0.0,0.0,0.0);
	\draw (0.0,0.5738,0.0) -- (0.0,0.0,0.0) -- (0.0,0.0,0.0);
	\draw (0.0,0.7789,0.0) -- (0.0,0.0,0.0);
	\draw (0.0,1.0,0.0);
	\draw (0.0,0.0,0.0) -- (0.0,0.0859,0.0) -- (0.0,0.2220,0.0) -- (0.0,0.3869,0.0) -- (0.0,0.5738,0.0) -- (0.0,0.7789,0.0) -- (0.0,1.0,0.0);
	\draw (0.0,0.0,0.0) -- (0.0859,0.0,0.0) -- (0.0,0.0,0.0) -- (0.0,0.0,0.0) -- (0.0,0.0,0.0) -- (0.0,0.0,0.0) -- (0.0,0.0,0.0);
	\draw (0.0,0.0859,0.0) -- (0.1110,0.1110,0.0) -- (0.0,0.0,0.0) -- (0.0,0.0,0.0) -- (0.0,0.0,0.0) -- (0.0,0.0,0.0);
	\draw (0.0,0.2220,0.0) -- (0.1289,0.2579,0.0) -- (0.0,0.0,0.0) -- (0.0,0.0,0.0) -- (0.0,0.0,0.0);
	\draw (0.0,0.3869,0.0) -- (0.1434,0.4303,0.0) -- (0.0,0.0,0.0) -- (0.0,0.0,0.0);
	\draw (0.0,0.5738,0.0) -- (0.1557,0.6231,0.0) -- (0.0,0.0,0.0);
	\draw (0.0,0.7789,0.0) -- (0.1666,0.8333,0.0);
	\draw (0.0859,0.0,0.0) -- (0.1110,0.1110,0.0) -- (0.1289,0.2579,0.0) -- (0.1434,0.4303,0.0) -- (0.1557,0.6231,0.0) -- (0.1666,0.8333,0.0);
	\draw (0.0,0.0,0.0) -- (0.0859,0.0,0.0) -- (0.2220,0.0,0.0) -- (0.0,0.0,0.0) -- (0.0,0.0,0.0) -- (0.0,0.0,0.0) -- (0.0,0.0,0.0);
	\draw (0.0,0.0859,0.0) -- (0.1110,0.1110,0.0) -- (0.2579,0.1289,0.0) -- (0.0,0.0,0.0) -- (0.0,0.0,0.0) -- (0.0,0.0,0.0);
	\draw (0.0,0.2220,0.0) -- (0.1289,0.2579,0.0) -- (0.2869,0.2869,0.0) -- (0.0,0.0,0.0) -- (0.0,0.0,0.0);
	\draw (0.0,0.3869,0.0) -- (0.1434,0.4303,0.0) -- (0.3115,0.4673,0.0) -- (0.0,0.0,0.0);
	\draw (0.0,0.5738,0.0) -- (0.1557,0.6231,0.0) -- (0.3333,0.6666,0.0);
	\draw (0.2220,0.0,0.0) -- (0.2579,0.1289,0.0) -- (0.2869,0.2869,0.0) -- (0.3115,0.4673,0.0) -- (0.3333,0.6666,0.0);
	\draw (0.0,0.0,0.0) -- (0.0859,0.0,0.0) -- (0.2220,0.0,0.0) -- (0.3869,0.0,0.0) -- (0.0,0.0,0.0) -- (0.0,0.0,0.0) -- (0.0,0.0,0.0);
	\draw (0.0,0.0859,0.0) -- (0.1110,0.1110,0.0) -- (0.2579,0.1289,0.0) -- (0.4303,0.1434,0.0) -- (0.0,0.0,0.0) -- (0.0,0.0,0.0);
	\draw (0.0,0.2220,0.0) -- (0.1289,0.2579,0.0) -- (0.2869,0.2869,0.0) -- (0.4673,0.3115,0.0) -- (0.0,0.0,0.0);
	\draw (0.0,0.3869,0.0) -- (0.1434,0.4303,0.0) -- (0.3115,0.4673,0.0) -- (0.5,0.5,0.0);
	\draw (0.3869,0.0,0.0) -- (0.4303,0.1434,0.0) -- (0.4673,0.3115,0.0) -- (0.5,0.5,0.0);
	\draw (0.0,0.0,0.0) -- (0.0859,0.0,0.0) -- (0.2220,0.0,0.0) -- (0.3869,0.0,0.0) -- (0.5738,0.0,0.0) -- (0.0,0.0,0.0) -- (0.0,0.0,0.0);
	\draw (0.0,0.0859,0.0) -- (0.1110,0.1110,0.0) -- (0.2579,0.1289,0.0) -- (0.4303,0.1434,0.0) -- (0.6231,0.1557,0.0) -- (0.0,0.0,0.0);
	\draw (0.0,0.2220,0.0) -- (0.1289,0.2579,0.0) -- (0.2869,0.2869,0.0) -- (0.4673,0.3115,0.0) -- (0.6666,0.3333,0.0);
	\draw (0.5738,0.0,0.0) -- (0.6231,0.1557,0.0) -- (0.6666,0.3333,0.0);
	\draw (0.0,0.0,0.0) -- (0.0859,0.0,0.0) -- (0.2220,0.0,0.0) -- (0.3869,0.0,0.0) -- (0.5738,0.0,0.0) -- (0.7789,0.0,0.0) -- (0.0,0.0,0.0);
	\draw (0.0,0.0859,0.0) -- (0.1110,0.1110,0.0) -- (0.2579,0.1289,0.0) -- (0.4303,0.1434,0.0) -- (0.6231,0.1557,0.0) -- (0.8333,0.1666,0.0);
	\draw (0.7789,0.0,0.0) -- (0.8333,0.1666,0.0);
	\draw (0.0,0.0,0.0) -- (0.0859,0.0,0.0) -- (0.2220,0.0,0.0) -- (0.3869,0.0,0.0) -- (0.5738,0.0,0.0) -- (0.7789,0.0,0.0) -- (1.0,0.0,0.0);
	\draw (1.0,0.0,0.0);
	\draw (0.0,0.0,0.0);
	\draw (0.0,0.0859,0.0) -- (0.0859,0.0,0.0);
	\draw (0.0,0.2220,0.0) -- (0.1110,0.1110,0.0) -- (0.2220,0.0,0.0);
	\draw (0.0,0.3869,0.0) -- (0.1289,0.2579,0.0) -- (0.2579,0.1289,0.0) -- (0.3869,0.0,0.0);
	\draw (0.0,0.5738,0.0) -- (0.1434,0.4303,0.0) -- (0.2869,0.2869,0.0) -- (0.4303,0.1434,0.0) -- (0.5738,0.0,0.0);
	\draw (0.0,0.7789,0.0) -- (0.1557,0.6231,0.0) -- (0.3115,0.4673,0.0) -- (0.4673,0.3115,0.0) -- (0.6231,0.1557,0.0) -- (0.7789,0.0,0.0);
	\draw (0.0,1.0,0.0) -- (0.1666,0.8333,0.0) -- (0.3333,0.6666,0.0) -- (0.5,0.5,0.0) -- (0.6666,0.3333,0.0) -- (0.8333,0.1666,0.0) -- (1.0,0.0,0.0);
	\draw (0.0,0.0,0.2210) -- (0.0,0.0,0.0) -- (0.0,0.0,0.0) -- (0.0,0.0,0.0) -- (0.0,0.0,0.0) -- (0.0,0.0,0.0);
	\draw (0.0,0.0859,0.2210) -- (0.0,0.0,0.0) -- (0.0,0.0,0.0) -- (0.0,0.0,0.0) -- (0.0,0.0,0.0);
	\draw (0.0,0.2220,0.2210) -- (0.0,0.0,0.0) -- (0.0,0.0,0.0) -- (0.0,0.0,0.0);
	\draw (0.0,0.3869,0.2210) -- (0.0,0.0,0.0) -- (0.0,0.0,0.0);
	\draw (0.0,0.5738,0.2210) -- (0.0,0.0,0.0);
	\draw (0.0,0.7789,0.2210);
	\draw (0.0,0.0,0.2210) -- (0.0,0.0859,0.2210) -- (0.0,0.2220,0.2210) -- (0.0,0.3869,0.2210) -- (0.0,0.5738,0.2210) -- (0.0,0.7789,0.2210);
	\draw (0.0,0.0,0.2210) -- (0.0859,0.0,0.2210) -- (0.0,0.0,0.0) -- (0.0,0.0,0.0) -- (0.0,0.0,0.0) -- (0.0,0.0,0.0);
	\draw (0.0,0.0859,0.2210) -- (0.1110,0.1110,0.2210) -- (0.0,0.0,0.0) -- (0.0,0.0,0.0) -- (0.0,0.0,0.0);
	\draw (0.0,0.2220,0.2210) -- (0.1289,0.2579,0.2210) -- (0.0,0.0,0.0) -- (0.0,0.0,0.0);
	\draw (0.0,0.3869,0.2210) -- (0.1434,0.4303,0.2210) -- (0.0,0.0,0.0);
	\draw (0.0,0.5738,0.2210) -- (0.1557,0.6231,0.2210);
	\draw (0.0859,0.0,0.2210) -- (0.1110,0.1110,0.2210) -- (0.1289,0.2579,0.2210) -- (0.1434,0.4303,0.2210) -- (0.1557,0.6231,0.2210);
	\draw (0.0,0.0,0.2210) -- (0.0859,0.0,0.2210) -- (0.2220,0.0,0.2210) -- (0.0,0.0,0.0) -- (0.0,0.0,0.0) -- (0.0,0.0,0.0);
	\draw (0.0,0.0859,0.2210) -- (0.1110,0.1110,0.2210) -- (0.2579,0.1289,0.2210) -- (0.0,0.0,0.0) -- (0.0,0.0,0.0);
	\draw (0.0,0.2220,0.2210) -- (0.1289,0.2579,0.2210) -- (0.2869,0.2869,0.2210) -- (0.0,0.0,0.0);
	\draw (0.0,0.3869,0.2210) -- (0.1434,0.4303,0.2210) -- (0.3115,0.4673,0.2210);
	\draw (0.2220,0.0,0.2210) -- (0.2579,0.1289,0.2210) -- (0.2869,0.2869,0.2210) -- (0.3115,0.4673,0.2210);
	\draw (0.0,0.0,0.2210) -- (0.0859,0.0,0.2210) -- (0.2220,0.0,0.2210) -- (0.3869,0.0,0.2210) -- (0.0,0.0,0.0) -- (0.0,0.0,0.0);
	\draw (0.0,0.0859,0.2210) -- (0.1110,0.1110,0.2210) -- (0.2579,0.1289,0.2210) -- (0.4303,0.1434,0.2210) -- (0.0,0.0,0.0);
	\draw (0.0,0.2220,0.2210) -- (0.1289,0.2579,0.2210) -- (0.2869,0.2869,0.2210) -- (0.4673,0.3115,0.2210);
	\draw (0.3869,0.0,0.2210) -- (0.4303,0.1434,0.2210) -- (0.4673,0.3115,0.2210);
	\draw (0.0,0.0,0.2210) -- (0.0859,0.0,0.2210) -- (0.2220,0.0,0.2210) -- (0.3869,0.0,0.2210) -- (0.5738,0.0,0.2210) -- (0.0,0.0,0.0);
	\draw (0.0,0.0859,0.2210) -- (0.1110,0.1110,0.2210) -- (0.2579,0.1289,0.2210) -- (0.4303,0.1434,0.2210) -- (0.6231,0.1557,0.2210);
	\draw (0.5738,0.0,0.2210) -- (0.6231,0.1557,0.2210);
	\draw (0.0,0.0,0.2210) -- (0.0859,0.0,0.2210) -- (0.2220,0.0,0.2210) -- (0.3869,0.0,0.2210) -- (0.5738,0.0,0.2210) -- (0.7789,0.0,0.2210);
	\draw (0.7789,0.0,0.2210);
	\draw (0.0,0.0,0.2210);
	\draw (0.0,0.0859,0.2210) -- (0.0859,0.0,0.2210);
	\draw (0.0,0.2220,0.2210) -- (0.1110,0.1110,0.2210) -- (0.2220,0.0,0.2210);
	\draw (0.0,0.3869,0.2210) -- (0.1289,0.2579,0.2210) -- (0.2579,0.1289,0.2210) -- (0.3869,0.0,0.2210);
	\draw (0.0,0.5738,0.2210) -- (0.1434,0.4303,0.2210) -- (0.2869,0.2869,0.2210) -- (0.4303,0.1434,0.2210) -- (0.5738,0.0,0.2210);
	\draw (0.0,0.7789,0.2210) -- (0.1557,0.6231,0.2210) -- (0.3115,0.4673,0.2210) -- (0.4673,0.3115,0.2210) -- (0.6231,0.1557,0.2210) -- (0.7789,0.0,0.2210);
	\draw (0.0,0.0,0.4261) -- (0.0,0.0,0.0) -- (0.0,0.0,0.0) -- (0.0,0.0,0.0) -- (0.0,0.0,0.0);
	\draw (0.0,0.0859,0.4261) -- (0.0,0.0,0.0) -- (0.0,0.0,0.0) -- (0.0,0.0,0.0);
	\draw (0.0,0.2220,0.4261) -- (0.0,0.0,0.0) -- (0.0,0.0,0.0);
	\draw (0.0,0.3869,0.4261) -- (0.0,0.0,0.0);
	\draw (0.0,0.5738,0.4261);
	\draw (0.0,0.0,0.4261) -- (0.0,0.0859,0.4261) -- (0.0,0.2220,0.4261) -- (0.0,0.3869,0.4261) -- (0.0,0.5738,0.4261);
	\draw (0.0,0.0,0.4261) -- (0.0859,0.0,0.4261) -- (0.0,0.0,0.0) -- (0.0,0.0,0.0) -- (0.0,0.0,0.0);
	\draw (0.0,0.0859,0.4261) -- (0.1110,0.1110,0.4261) -- (0.0,0.0,0.0) -- (0.0,0.0,0.0);
	\draw (0.0,0.2220,0.4261) -- (0.1289,0.2579,0.4261) -- (0.0,0.0,0.0);
	\draw (0.0,0.3869,0.4261) -- (0.1434,0.4303,0.4261);
	\draw (0.0859,0.0,0.4261) -- (0.1110,0.1110,0.4261) -- (0.1289,0.2579,0.4261) -- (0.1434,0.4303,0.4261);
	\draw (0.0,0.0,0.4261) -- (0.0859,0.0,0.4261) -- (0.2220,0.0,0.4261) -- (0.0,0.0,0.0) -- (0.0,0.0,0.0);
	\draw (0.0,0.0859,0.4261) -- (0.1110,0.1110,0.4261) -- (0.2579,0.1289,0.4261) -- (0.0,0.0,0.0);
	\draw (0.0,0.2220,0.4261) -- (0.1289,0.2579,0.4261) -- (0.2869,0.2869,0.4261);
	\draw (0.2220,0.0,0.4261) -- (0.2579,0.1289,0.4261) -- (0.2869,0.2869,0.4261);
	\draw (0.0,0.0,0.4261) -- (0.0859,0.0,0.4261) -- (0.2220,0.0,0.4261) -- (0.3869,0.0,0.4261) -- (0.0,0.0,0.0);
	\draw (0.0,0.0859,0.4261) -- (0.1110,0.1110,0.4261) -- (0.2579,0.1289,0.4261) -- (0.4303,0.1434,0.4261);
	\draw (0.3869,0.0,0.4261) -- (0.4303,0.1434,0.4261);
	\draw (0.0,0.0,0.4261) -- (0.0859,0.0,0.4261) -- (0.2220,0.0,0.4261) -- (0.3869,0.0,0.4261) -- (0.5738,0.0,0.4261);
	\draw (0.5738,0.0,0.4261);
	\draw (0.0,0.0,0.4261);
	\draw (0.0,0.0859,0.4261) -- (0.0859,0.0,0.4261);
	\draw (0.0,0.2220,0.4261) -- (0.1110,0.1110,0.4261) -- (0.2220,0.0,0.4261);
	\draw (0.0,0.3869,0.4261) -- (0.1289,0.2579,0.4261) -- (0.2579,0.1289,0.4261) -- (0.3869,0.0,0.4261);
	\draw (0.0,0.5738,0.4261) -- (0.1434,0.4303,0.4261) -- (0.2869,0.2869,0.4261) -- (0.4303,0.1434,0.4261) -- (0.5738,0.0,0.4261);
	\draw (0.0,0.0,0.6130) -- (0.0,0.0,0.0) -- (0.0,0.0,0.0) -- (0.0,0.0,0.0);
	\draw (0.0,0.0859,0.6130) -- (0.0,0.0,0.0) -- (0.0,0.0,0.0);
	\draw (0.0,0.2220,0.6130) -- (0.0,0.0,0.0);
	\draw (0.0,0.3869,0.6130);
	\draw (0.0,0.0,0.6130) -- (0.0,0.0859,0.6130) -- (0.0,0.2220,0.6130) -- (0.0,0.3869,0.6130);
	\draw (0.0,0.0,0.6130) -- (0.0859,0.0,0.6130) -- (0.0,0.0,0.0) -- (0.0,0.0,0.0);
	\draw (0.0,0.0859,0.6130) -- (0.1110,0.1110,0.6130) -- (0.0,0.0,0.0);
	\draw (0.0,0.2220,0.6130) -- (0.1289,0.2579,0.6130);
	\draw (0.0859,0.0,0.6130) -- (0.1110,0.1110,0.6130) -- (0.1289,0.2579,0.6130);
	\draw (0.0,0.0,0.6130) -- (0.0859,0.0,0.6130) -- (0.2220,0.0,0.6130) -- (0.0,0.0,0.0);
	\draw (0.0,0.0859,0.6130) -- (0.1110,0.1110,0.6130) -- (0.2579,0.1289,0.6130);
	\draw (0.2220,0.0,0.6130) -- (0.2579,0.1289,0.6130);
	\draw (0.0,0.0,0.6130) -- (0.0859,0.0,0.6130) -- (0.2220,0.0,0.6130) -- (0.3869,0.0,0.6130);
	\draw (0.3869,0.0,0.6130);
	\draw (0.0,0.0,0.6130);
	\draw (0.0,0.0859,0.6130) -- (0.0859,0.0,0.6130);
	\draw (0.0,0.2220,0.6130) -- (0.1110,0.1110,0.6130) -- (0.2220,0.0,0.6130);
	\draw (0.0,0.3869,0.6130) -- (0.1289,0.2579,0.6130) -- (0.2579,0.1289,0.6130) -- (0.3869,0.0,0.6130);
	\draw (0.0,0.0,0.7779) -- (0.0,0.0,0.0) -- (0.0,0.0,0.0);
	\draw (0.0,0.0859,0.7779) -- (0.0,0.0,0.0);
	\draw (0.0,0.2220,0.7779);
	\draw (0.0,0.0,0.7779) -- (0.0,0.0859,0.7779) -- (0.0,0.2220,0.7779);
	\draw (0.0,0.0,0.7779) -- (0.0859,0.0,0.7779) -- (0.0,0.0,0.0);
	\draw (0.0,0.0859,0.7779) -- (0.1110,0.1110,0.7779);
	\draw (0.0859,0.0,0.7779) -- (0.1110,0.1110,0.7779);
	\draw (0.0,0.0,0.7779) -- (0.0859,0.0,0.7779) -- (0.2220,0.0,0.7779);
	\draw (0.2220,0.0,0.7779);
	\draw (0.0,0.0,0.7779);
	\draw (0.0,0.0859,0.7779) -- (0.0859,0.0,0.7779);
	\draw (0.0,0.2220,0.7779) -- (0.1110,0.1110,0.7779) -- (0.2220,0.0,0.7779);
	\draw (0.0,0.0,0.9140) -- (0.0,0.0,0.0);
	\draw (0.0,0.0859,0.9140);
	\draw (0.0,0.0,0.9140) -- (0.0,0.0859,0.9140);
	\draw (0.0,0.0,0.9140) -- (0.0859,0.0,0.9140);
	\draw (0.0859,0.0,0.9140);
	\draw (0.0,0.0,0.9140);
	\draw (0.0,0.0859,0.9140) -- (0.0859,0.0,0.9140);
	\draw[orange] (0.0,0.0,0.0) -- (0.0,0.0,0.2210) -- (0.0,0.0,0.4261) -- (0.0,0.0,0.6130) -- (0.0,0.0,0.7779) -- (0.0,0.0,0.9140) -- (0.0,0.0,1.0);
	\draw (0.0,0.0859,0.0) -- (0.0,0.0859,0.2210) -- (0.0,0.0859,0.4261) -- (0.0,0.0859,0.6130) -- (0.0,0.0859,0.7779) -- (0.0,0.0859,0.9140);
	\draw (0.0,0.2220,0.0) -- (0.0,0.2220,0.2210) -- (0.0,0.2220,0.4261) -- (0.0,0.2220,0.6130) -- (0.0,0.2220,0.7779);
	\draw (0.0,0.3869,0.0) -- (0.0,0.3869,0.2210) -- (0.0,0.3869,0.4261) -- (0.0,0.3869,0.6130);
	\draw (0.0,0.5738,0.0) -- (0.0,0.5738,0.2210) -- (0.0,0.5738,0.4261);
	\draw (0.0,0.7789,0.0) -- (0.0,0.7789,0.2210);
	\draw (0.0,1.0,0.0) -- (0.0,0.7789,0.2210) -- (0.0,0.5738,0.4261) -- (0.0,0.3869,0.6130) -- (0.0,0.2220,0.7779) -- (0.0,0.0859,0.9140) -- (0.0,0.0,1.0);
	\draw (1.0,0.0,0.0) -- (0.7789,0.0,0.2210) -- (0.5738,0.0,0.4261) -- (0.3869,0.0,0.6130) -- (0.2220,0.0,0.7779) -- (0.0859,0.0,0.9140) -- (0.0,0.0,1.0);
	\draw (0.0859,0.0,0.0) -- (0.0859,0.0,0.2210) -- (0.0859,0.0,0.4261) -- (0.0859,0.0,0.6130) -- (0.0859,0.0,0.7779) -- (0.0859,0.0,0.9140);
	\draw (0.1110,0.1110,0.0) -- (0.1110,0.1110,0.2210) -- (0.1110,0.1110,0.4261) -- (0.1110,0.1110,0.6130) -- (0.1110,0.1110,0.7779);
	\draw (0.1289,0.2579,0.0) -- (0.1289,0.2579,0.2210) -- (0.1289,0.2579,0.4261) -- (0.1289,0.2579,0.6130);
	\draw (0.1434,0.4303,0.0) -- (0.1434,0.4303,0.2210) -- (0.1434,0.4303,0.4261);
	\draw (0.1557,0.6231,0.0) -- (0.1557,0.6231,0.2210);
	\draw (0.1666,0.8333,0.0) -- (0.1557,0.6231,0.2210) -- (0.1434,0.4303,0.4261) -- (0.1289,0.2579,0.6130) -- (0.1110,0.1110,0.7779) -- (0.0859,0.0,0.9140);
	\draw (0.8333,0.1666,0.0) -- (0.6231,0.1557,0.2210) -- (0.4303,0.1434,0.4261) -- (0.2579,0.1289,0.6130) -- (0.1110,0.1110,0.7779) -- (0.0,0.0859,0.9140);
	\draw (0.2220,0.0,0.0) -- (0.2220,0.0,0.2210) -- (0.2220,0.0,0.4261) -- (0.2220,0.0,0.6130) -- (0.2220,0.0,0.7779);
	\draw (0.2579,0.1289,0.0) -- (0.2579,0.1289,0.2210) -- (0.2579,0.1289,0.4261) -- (0.2579,0.1289,0.6130);
	\draw (0.2869,0.2869,0.0) -- (0.2869,0.2869,0.2210) -- (0.2869,0.2869,0.4261);
	\draw (0.3115,0.4673,0.0) -- (0.3115,0.4673,0.2210);
	\draw (0.3333,0.6666,0.0) -- (0.3115,0.4673,0.2210) -- (0.2869,0.2869,0.4261) -- (0.2579,0.1289,0.6130) -- (0.2220,0.0,0.7779);
	\draw (0.6666,0.3333,0.0) -- (0.4673,0.3115,0.2210) -- (0.2869,0.2869,0.4261) -- (0.1289,0.2579,0.6130) -- (0.0,0.2220,0.7779);
	\draw (0.3869,0.0,0.0) -- (0.3869,0.0,0.2210) -- (0.3869,0.0,0.4261) -- (0.3869,0.0,0.6130);
	\draw (0.4303,0.1434,0.0) -- (0.4303,0.1434,0.2210) -- (0.4303,0.1434,0.4261);
	\draw (0.4673,0.3115,0.0) -- (0.4673,0.3115,0.2210);
	\draw (0.5,0.5,0.0) -- (0.4673,0.3115,0.2210) -- (0.4303,0.1434,0.4261) -- (0.3869,0.0,0.6130);
	\draw (0.5,0.5,0.0) -- (0.3115,0.4673,0.2210) -- (0.1434,0.4303,0.4261) -- (0.0,0.3869,0.6130);
	\draw (0.5738,0.0,0.0) -- (0.5738,0.0,0.2210) -- (0.5738,0.0,0.4261);
	\draw (0.6231,0.1557,0.0) -- (0.6231,0.1557,0.2210);
	\draw (0.6666,0.3333,0.0) -- (0.6231,0.1557,0.2210) -- (0.5738,0.0,0.4261);
	\draw (0.3333,0.6666,0.0) -- (0.1557,0.6231,0.2210) -- (0.0,0.5738,0.4261);
	\draw (0.7789,0.0,0.0) -- (0.7789,0.0,0.2210);
	\draw (0.8333,0.1666,0.0) -- (0.7789,0.0,0.2210);
	\draw (0.1666,0.8333,0.0) -- (0.0,0.7789,0.2210);
\end{tikzpicture}}

\usetikzlibrary{calc,shapes.geometric,arrows}

% numbering before the word Theorem
\swapnumbers
\setcounter{secnumdepth}{3}
%\numberwithin{equation}{section}

%%%%%%%%%%%%%%%%%%%%%%%%%%%%%%%%
%\usepackage{syntonly}
%\syntaxonly
%%%%%%%%%%%%%%%%%%%%%%%%%%%%%%%%

\DeclareMathAlphabet{\pazocal}{OMS}{zplm}{m}{n}

\def\twoPi{360}
\def\ok{{\color{green}\quad ok}}
\def\xyz{(\hat x_1, \hat x_2, \hat x_3)}
\def\bxi{\boldsymbol{\xi}}
\def\bphi{\boldsymbol{\phi}}
\def\be{\boldsymbol{e}}
\def\bx{\boldsymbol{x}}
\def\by{\boldsymbol{y}}
\def\bh{\boldsymbol{h}}
\def\bq{\boldsymbol{q}}
\def\bp{\boldsymbol{p}}
\def\bv{\boldsymbol{v}}
\def\bw{\boldsymbol{w}}
\def\bu{\boldsymbol{u}}
\def\br{\boldsymbol{r}}
\def\bn{\boldsymbol{n}}
\def\balpha{\boldsymbol{\alpha}}
\def\bbeta{\boldsymbol{\beta}}
\def\btau{\boldsymbol{\tau}}
\def\bz{\boldsymbol{\zeta}}
\def\bnu{\boldsymbol{\nu}}
\def\bgamma{\boldsymbol{\gamma}}
\def\s{\scriptstyle}
\def\sss{\scriptscriptstyle}
\def\diam{\text{diam}}
\def\dv{\mbox{div\,}}
\def\curl{\textbf{curl\,}}
\def\bcurl{\textbf{curl}}
\def\wku{\bw_{\hat{E}}\hat{\bu}}
\def\wkutilde{\bw_{\tilde{E}}\tilde{\bu}}
\def\rku{\br_{\hat{E}}\hat{\bu}}
\def\rZerou{\br_{0}\bu}
\def\rkutilde{\br_{\tilde{E}}\tilde{\bu}}

\newcommand{\Qb}{\pazocal{Q}}
\newcommand{\Qbb}{\pazocal{\bf Q}}
\newcommand{\Tb}{\pazocal{T}}
\newcommand{\Kb}{\pazocal{K}}
\newcommand{\Th}{\pazocal{T}_{\textit{h}}}

{\newcommand{\vertiii}[1]{{\left\vert\kern-0.25ex\left\vert\kern-0.25ex\left\vert #1 
    \right\vert\kern-0.25ex\right\vert\kern-0.25ex\right\vert}}
\newcommand{\forma}[2]{\int_\Omega {\boldsymbol{#1}}\cdot{\boldsymbol{#2}}\,d\bx}
\newcommand{\formb}[2]{\int_\Omega #2\,\dv{\boldsymbol{#1}}\,d\bx}
\newcommand{\hmtn}[2]{\textrm{H}^#1(\textrm{T})^#2}
\newcommand{\RTk}{\textrm{RT}_k}
\newcommand{\RTksomb}{\widehat{\textrm{RT}_k}}
\newcommand{\xSombrero}[1]{\widehat{\textbf{#1}}}
\newcommand{\raviart}[1]{\mathcal{RT}_k(#1)}
\newcommand{\Vh}{\mathbb{V}_h}
\newcommand{\Qh}{\mathbb{Q}_h}



\newcommand{\supp}[1]{\textrm{supp}(#1)}
\newcommand{\p}[2]{\mathcal{P}_{#1}(x,y) \otimes \mathcal{P}_{#2}(z)}
\newcommand{\pp}[2]{\mathcal{P}_{#1}(\hat f_1)\otimes\mathcal P_{#2}(\hat z)}
\newcommand{\wpcurl}[1]{W^p(\textbf{curl}, #1)}
\newcommand{\diag}[3]{
    \left(
    \begin{array}{ccc}
        #1  & 0     & 0\\
        0   & #2    & 0\\
        0   & 0     & #3
    \end{array}
    \right)}
\newcommand{\blockdiag}[5]{
	\left(
	\begin{array}{ccc}
		#1 	& #2 	& 0\\
		#3	& #4 	& 0\\
		0	& 0		& #5
	\end{array}
	\right)}
\newcommand{\dvg}{\text{div}\,}
\newcommand{\lf}[1]{\lambda_{f_{#1}}}
\newcommand{\lhatf}[1]{\lambda_{\hat{f}_{#1}}}
\newcommand{\gancho}{{\scriptstyle\partial}}

\newtheorem{theorem}{Theorem}[section]
\newtheorem{proposition}[theorem]{Proposition}
\newtheorem{corollary}[theorem]{Corollary}
\newtheorem{lemma}[theorem]{Lemma}
\newtheorem{obs}[theorem]{Observaci\'on}
\newtheorem{defi}[theorem]{Definition}
\newtheorem{notation}[theorem]{Notation}
\newtheorem{notacion}[theorem]{Notaci\'on}
\newtheorem{problem}[theorem]{Problem}
\newtheorem{remark}[theorem]{Remark}
\newtheorem{example}[theorem]{Example}

\DeclarePairedDelimiter{\abs}{\lvert}{\rvert}
\DeclarePairedDelimiter{\Abs}{\left|}{\right|}
\DeclarePairedDelimiter{\norm}{\lVert}{\rVert}
\DeclarePairedDelimiter{\Norm}{\left\|}{\right\|} %TODO este delimitador tiene un inconveniente

\DeclareMathOperator{\img}{Im}
\DeclareMathOperator{\Div}{\textrm{div}}
