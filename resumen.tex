\section*{Resumen}
En esta Tesis introducimos un M\'etodo combinado
de Elementos Finitos y Virtuales en dimensi\'on tres 
para
la aproximaci\'on mixta 
de un problema el\'iptico modelo para el operador
de Laplace en un poliedro arbitrario.
El m\'etodo es analizado por completo
cuando las mallas constan de prismas rectos de base triangular, pir\'amdes y tetraedros. Los espacios locales
discretizantes coinciden con los espacios de orden m\'inimo
de Raviart-Thomas 
sobre tetraedros y prismas, y constituyen una extensi\'on
de estos a elementos piramidales. Probamos que el 
esquema discreto es bien planteado y 
probamos estimaciones \'optimas de error
sobre mallas 
que admiten elementos anis\'otropos. En particular, 
 nuestras estimaciones de 
error
de 
interpolaci\'on 
local 
para el espacio discreto son \'optimas y 
anis\'otropas en prismas rectos anis\'otropos. La motivaci\'on
para trabajar con elementos anis\'otropos es que en
distintas situaciones en aproximaciones por elmentos finitos mixtos
es necesario el uso de mallas con elementos elongados.
Este es el caso, por ejemplo, con la ecuaci\'on de Poisson en un
poliedro $\Omega$ con aristas c\'oncavas y v\'ertices entrantes,
que en forma mixta puede escribirse como
\begin{equation*}\label{mf} \left\{\begin{array}{rcl}
\boldsymbol{u}&=&-\nabla p\qquad \mbox{in }\Omega\\
\dv\boldsymbol{u}&=&f\qquad \mbox{in }\Omega\\
p&=&0\qquad\mbox{on }\partial\Omega.\end{array}\right.
\end{equation*}
En este caso la variable vectorial de la soluci\'on, $\boldsymbol{u}$,
no est\'a en $H^1(\Omega)$ en el caso general debido a las singularidades 
de aristas y v\'ertices. En particular, cerca de las aristas c\'oncavas,
$\boldsymbol{u}$ es m\'as regular en la direcci\'on a lo largo de \'estas
que transversalmente, y consecuentemente las mallas tienen que ser
adecuadamente refinadas para recuperar el orden \'optimo de convergencia
con respecto al n\'umero de grados de libertad. Tales mallas  contienen
elementos arbitrariamente alargados en la direcci\'on de las aristas sigulares.

Asimismo, proponemos un proceso de mallado
para construir una familia de mallas que nos permite
obtener estimaciones de error de aproximacion global
\'optimas cuando la soluci\'on del problema modelo
presenta singularidades de arista o v\'ertice
puesto que las mallas resultan, por construcci\'on,
adecuadamente gra\-duadas y adaptadas 
a las singularidades, como mencionamos en el p\'arrafo anterior.

Adem\'as en la presente Tesis 
obtuvimos cotas de estabilidad y de error de interpolaci\'on
local para Elementos Finitos prism\'aticos anis\'otropos de orden
arbitrario, tanto para la clase de elementos
conformes en $H(\bcurl)$ como para 
la clase de elementos conformes en $H(\mbox{div})$, que 
constituyen resultados adicionales que extienden algunos hechos te\'oricos
que probamos para el problema principal de la tesis.

Tambi\'en presentamos 
cotas de estabilidad y de error de interpolaci\'on
local para Elementos Finitos piramidales anis\'otropos 
de orden bajo, tanto para la clase de elementos
conformes en $H(\bcurl)$ como para 
la clase de elementos conformes en $H(\mbox{div})$. Este resultado
est\'a incluido para mostrar una variante a nuestro m\'etodo principal,
esto es, un m\'etodo solamente con elementos finitos. Con respecto a esta variante,
como
mostramos expl\'icitamente en la Tesis, las 
funciones de forma, generadoras de 
estos \'ultimos espacios de Elementos Finitos,
son ra\-cio\-na\-les y son singulares, aunque acotadas,
en la pir\'amide de referencia. Esta es una raz\'on por la cual 
consideramos que nuestra aproximaci\'on FE--VE combinada presenta una ventaja
que es evitar la evaluaci\'on de funciones con dichas propiedades
en implementaciones en computadoras.