Here we state an anisotropic stability bound for
the least order $\bcurl$ -- conforming operator on pyramids.
As always, we write it componentwise to make the anisotropy of the 
estimate clearer.
\begin{theorem} \label{aux_label53}
Let $\hat E$ be the reference pyramid and let $p>2$. Let $\bw_{\hat E}(\cdot)$ denote
the interpolation operator determined by the degrees of freedom in Definition~\ref{aux_label50}.
There is $C>0$ such that,
for all $\hat\bu\in W^{1,p}(\hat E)$ with first derivatives in $W^{1,1}(\hat E)$, there hold
\begin{IEEEeqnarray}{rCl}
  \nonumber\|(\wku)_1\|_{L^\infty(\hat E)}&\lesssim&
  \|\hat u_1\|_{\scriptscriptstyle W^{1,p}(\hat E)} +
  \|(\nabla\times\hat\bu)_2\|_{\scriptscriptstyle W^{1,1}(\hat E)} +  
  \|(\nabla\times\hat\bu)_3\|_{\scriptscriptstyle W^{1,1}(\hat E)}\\[4pt]
  \label{auxlabel203}
  &&\, +  \left\|\tfrac{\partial \hat u_1}{\partial\hat x_2}\right\|_{\scriptscriptstyle W^{1,1}(\hat E)} +
          \left\|\tfrac{\partial^2 \hat u_3}{\partial\hat x_2\partial\hat x_1}\right\|_{\scriptscriptstyle L^{1}(\hat E)}.\\[12pt]
  \nonumber\|(\wku)_2\|_{L^\infty(\hat E)}&\lesssim&
\|\hat u_2\|_{\scriptscriptstyle W^{1,p}(\hat E)} +
  \|(\nabla\times\hat\bu)_1\|_{\scriptscriptstyle W^{1,1}(\hat E)} +  
  \|(\nabla\times\hat\bu)_3\|_{\scriptscriptstyle W^{1,1}(\hat E)}\\[4pt]
  &&\, +  \left\|\tfrac{\partial \hat u_2}{\partial\hat x_1}\right\|_{\scriptscriptstyle W^{1,1}(\hat E)} +
          \left\|\tfrac{\partial^2 \hat u_3}{\partial\hat x_2\partial\hat x_1}\right\|_{\scriptscriptstyle L^{1}(\hat E)}.\\[12pt]
  \nonumber\|(\wku)_3\|_{L^\infty(\hat E)}&\lesssim&
  \|\hat u_3\|_{\scriptscriptstyle W^{1,p}(\hat E)} +
  \|(\nabla\times\hat\bu)_2\|_{\scriptscriptstyle W^{1,1}(\hat E)} +  
  \|(\nabla\times\hat\bu)_1\|_{\scriptscriptstyle W^{1,1}(\hat E)}\\[5pt]
  \IEEEeqnarraymulticol{3}{r}{\label{auxlabel209}
  +\left\|\tfrac{\partial \hat u_3}{\partial\hat x_1}\right\|_{\scriptscriptstyle W^{1,1}(\hat E)} +
        \left\|\tfrac{\partial \hat u_2}{\partial\hat x_1}\right\|_{\scriptscriptstyle W^{1,1}(\hat E)} +
        \left\|\tfrac{\partial \hat u_1}{\partial\hat x_2}\right\|_{\scriptscriptstyle W^{1,1}(\hat E)} +
        \left\|\tfrac{\partial^2 \hat u_3}{\partial\hat x_2\partial\hat x_1}\right\|_{\scriptscriptstyle L^{1}(\hat E)}.}
\end{IEEEeqnarray}
\end{theorem}
\begin{proof}
Take an element $\hat\bu$ of $W^{1,p}(\hat{E})$ for a $p > 2$.
Let us recall the shape funtions in Table~\ref{shape_edge_table}.
For the variables of the shape functions in upcoming computations we will write $x$,$y$ and $z$ instead of
$\hat x_i$ to get a cleaner reading. Start with $\hat\bu$ of the form $(\hat u_1,0,0)'$. After calculating we have
\begin{IEEEeqnarray*}{rCl} %%\nabla\times\hat\bu &=& (0, \tfrac{{\s\partial} \hat u_1}{{\s\partial} \hat x_3},-\tfrac{{\s\partial} \hat u_1}{{\s\partial} \hat x_2})'.\\[6pt]
	\wku	&=& [{\s\int_{\hat{\be}_1}\hat\bu\cdot d\hat\balpha_1}]\hat\bgamma_1 +
				    [{\s\int_{\hat{\be}_3}\hat\bu\cdot d\hat\balpha_3}]\hat\bgamma_3 + 
				    [{\s\int_{\hat{\be}_6}\hat\bu\cdot d\hat\balpha_6}]\hat\bgamma_6 + 
				    [{\s\int_{\hat{\be}_8}\hat\bu\cdot d\hat\balpha_8}]\hat\bgamma_8\\[5pt]
			&=:& \varphi_1(\hat\bu)\hat\bgamma_1 + 
				   \varphi_3(\hat\bu)\hat\bgamma_3 + 
				   \varphi_6(\hat\bu)\hat\bgamma_6 + 
				   \varphi_8(\hat\bu)\hat\bgamma_8.
\end{IEEEeqnarray*}
\begin{IEEEeqnarray*}{rCl}
  (\wku)_1(x,y,z) 
    &  = & \varphi_1(\hat\bu)(1-z-y)+ 
	  \varphi_3(\hat\bu)y+ 
	  \varphi_6(\hat\bu)(-z+\frac{yz}{1-z})\\[4pt]
    &&\,+\,  \varphi_8(\hat\bu)(-\frac{yz}{1-z})\\[4pt]
	& = & \varphi_1(\hat\bu) - (\varphi_1 + \varphi_6)(\hat\bu)\,z+ 
	  (\varphi_3 - \varphi_1)(\hat\bu)\,y\\[4pt]
  &&\, +\, (\varphi_6-\varphi_8)(\hat\bu)\,\frac{yz}{1-z}.
\end{IEEEeqnarray*}
Now we explore the new coefficients separately. As the tangential component of $\hat\bu$
along $\hat\be_5$ equals zero, and this is an argument we are using repeatedly in the forthcoming
computations, we may write, by Stokes' Theorem,
\begin{IEEEeqnarray*}{rCl}
  (\varphi_1+\varphi_6)(\hat\bu)
  	& = & \int_{\hat{\be}_1}\hat\bu\cdot d\hat{\balpha}_1  
        +	\int_{\hat{\be}_6}\hat\bu\cdot d\hat\balpha_6 -
  		  	\int_{\hat{\be}_5}\hat\bu\cdot d\hat\balpha_5 \\[5pt]
  	& = & \iint_{\hat{f}_1} \nabla\times\hat\bu\cdot\hat\bn\,d\hat S \\[5pt]
  	& = & -\iint_{\hat{f}_1} \tfrac{{\partial} \hat u_1}{{\partial}\hat x_3}\,d\hat S.
\end{IEEEeqnarray*}
Next,
\begin{IEEEeqnarray*}{rCl}
	(\varphi_3-\varphi_1)(\hat\bu) & = & (\varphi_3-\varphi_2-\varphi_1+\varphi_4)(\hat\bu)\\
	& = & - \int_{\partial\hat{f}_5}\hat{\bu}\cdot d\hat{\balpha}\\[4pt]
	&=& -\iint_{\hat{f}_5}\nabla\times\hat{\bu}\cdot\hat{\bn}_5\,d\hat S\\
	&=&	 \iint_{\hat{f}_5}\tfrac{{\partial} \hat{u}_1}{{\partial} \hat{x}_2}\,d\hat S.
\end{IEEEeqnarray*}
And
\begin{IEEEeqnarray*}{rCl}
  (\varphi_6-\varphi_8)(\hat\bu) & = & \int_{\hat\be_6}\hat\bu\cdot d\hat\balpha_6 -
    \int_{\hat\be_8}\hat\bu\cdot\hat\btau_6\,d\hat s-
	\int_{\hat\be_2}\hat\bu\cdot\hat\btau_6\,d\hat s\\[5pt]
	& = &-\int_{\partial\hat{f}_3}\hat\bu\cdot\hat\btau\,d\hat s  
	  =  -\iint_{\hat{f}_3}\nabla\times\hat\bu\cdot\hat\bn_3\,d\hat S
	  =   \tfrac{1}{\sqrt{2}}\iint_{\hat{f}_3}\tfrac{{\s\partial} \hat u_1}{{\s\partial} \hat x_2}\,d\hat S.
\end{IEEEeqnarray*}
So in this case in which $\hat\bu$ has null first and second components, it holds
\begin{IEEEeqnarray}{rCl}\label{first_a}
	\nonumber
  (\wku)_1 & = & \int_{\hat\be_1}\hat u_1\,d\hat\alpha_1 + 
                z\iint_{\hat f_1} \tfrac{{\s\partial}\hat u_1}{{\s\partial} \hat x_3}\,d\hat S +
                y\iint_{\hat f_5} \tfrac{{\s\partial}\hat u_1}{{\s\partial} \hat x_2}\,d\hat S\\
           &&\,+\frac{yz}{1-z}\,2^{-\nicefrac12}\iint_{\hat f_3} \tfrac{{\s\partial}\hat u_1}{{\s\partial} \hat x_2}\,d\hat S.
\end{IEEEeqnarray}
By exactly the last computation,
\begin{IEEEeqnarray}{rCl}\label{second_a}
  (\wku)_2 & = & (\varphi_6-\varphi_8)(\hat\bu)\frac{xz}{1-z}
  = \iint_{\hat{f}_3}\tfrac{{\s\partial}\hat u_1}{{\s\partial} \hat x_2}\,d\hat S\,\frac{xz}{1-z}.
\end{IEEEeqnarray}
Next,
\begin{IEEEeqnarray*}{rCl}
	(\wku)_3 & = &     \varphi_1(\hat\bu)\left(x-\frac{xy}{1-z}\right) + \varphi_3(\hat\bu)\frac{xy}{1-z}\\[6pt]
			 &   &\,+\,\varphi_6(\hat\bu)\left(x-\frac{xy}{1-z}+\frac{xyz}{(1-z)^2}\right)
		            +  \varphi_8(\hat\bu)\left(\frac{xy}{1-z}-\frac{xyz}{(1-z)^2}\right)\\[6pt]
			 & = &  (\varphi_1 + \varphi_6)(\hat\bu)\,x +
			 		(\varphi_3-\varphi_1+\varphi_8-\varphi_6)(\hat\bu)\,\frac{xy}{1-z}\\[6pt]
			 &   &\,+ (\varphi_6-\varphi_8)(\hat\bu)\,\frac{xyz}{(1-z)^2}.
\end{IEEEeqnarray*}
As $\hat\bu$ has zero tangential component along $\hat\be_5$ and $\hat\be_7$,
\begin{IEEEeqnarray*}{rCl}
  (\varphi_3-\varphi_1+\varphi_8-\varphi_6)(\hat\bu)&=&
  (\varphi_3-\varphi_7+\varphi_8)(\hat\bu)+(\varphi_5-\varphi_6-\varphi_1)(\hat\bu)\\[8pt]
  &=&-\int_{\partial\hat{f}_4}\hat\bu\cdot d\hat\balpha
   -\int_{\partial\hat{f}_1}\hat\bu\cdot d\hat\balpha\\[8pt]
  &=&-\iint_{\hat{f}_1}\nabla\times\hat\bu\cdot\hat\bn\,d\hat S
   -\iint_{\hat{f}_4}\nabla\times\hat\bu\cdot\hat\bn\,d\hat S\\[8pt]
%\yesnumber\label{pyr_edge_one}
  &=&\iint_{\hat{f}_1}(\nabla\times\hat\bu)_2\,d\hat S\\[8pt]
  & &\quad-2^{-\nicefrac{1}{2}}\iint_{\hat{f}_4}[(\nabla\times\hat\bu)_2 + (\nabla\times\hat\bu)_3]\,d\hat S.\\[8pt]
  &=&\iint_{\hat{f}_1}\tfrac{\partial\hat{u}_1}{\partial\hat{x}_3}\,d\hat S
  -2^{-\nicefrac{1}{2}}\iint_{\hat{f}_4}[\tfrac{\partial\hat{u}_1}{\partial\hat{x}_3}
   + \tfrac{\partial\hat{u}_1}{\partial\hat{x}_2}]\,d\hat S.
\end{IEEEeqnarray*}
We write down this component:
\begin{IEEEeqnarray}{rCl}\label{third_a}
	\nonumber
  (\wku)_3 & = & 
    -x\,\iint_{\hat{f}_1} \tfrac{{\s\partial} \hat u_1}{{\s\partial} \hat x_3}\,d\hat S
    +\frac{xyz}{(1-z)^2}\,2^{-\nicefrac12}\iint_{\hat{f}_3}\tfrac{{\s\partial} \hat u_1}{{\s\partial} \hat x_2}\,d\hat S\\[8pt]
    &&\,+\frac{xy}{1-z}
     \left\{\iint_{\hat{f}_1}\tfrac{\partial\hat{u}_1}{\partial\hat{x}_3}\,d\hat S
      -2^{-\nicefrac{1}{2}}\iint_{\hat{f}_4}[\tfrac{\partial\hat{u}_1}{\partial\hat{x}_3}
     +\tfrac{\partial\hat{u}_1}{\partial\hat{x}_2}]\,d\hat S\right\}
\end{IEEEeqnarray}
\noindent Now, as expected, we switch to $\hat\bu = (0,\hat u_2,0)'$. In this case we have
\begin{IEEEeqnarray*}{rCl}
  \wku     & = & \varphi_2(\hat{\bu})\,\hat\bgamma_2 +
	\varphi_4(\hat{\bu})\,\hat\bgamma_4+ \varphi_7(\hat{\bu})\,\hat\bgamma_7+\varphi_8(\hat{\bu})\,\hat\bgamma_8.\\[4pt]
  (\wku)_1 & = &(\varphi_7-\varphi_8)(\hat{\bu})\,\frac{yz}{1-z}\\[4pt]
  		   & = &(\varphi_7-\varphi_8 - \varphi_3)(\hat{\bu})\,\frac{yz}{1-z}\\[4pt]
  		   & = &\int_{\partial\hat{f}_4}\hat{\bu}\cdot d\hat\balpha\,\frac{yz}{1-z}\\[4pt]
  		   \yesnumber\label{first_b}
  		   & = &\iint_{\hat{f}_4} \nabla\times\hat\bu\cdot\hat\bn_4\,d\hat S\,\frac{yz}{1-z}
  		  \, = \,2^{-\nicefrac12} \iint_{\hat{f}_4} \tfrac{\partial\hat{u}_2}{\partial\hat{x}_1}\,d\hat S\,\frac{yz}{1-z}.
\end{IEEEeqnarray*}
For the next component,
\begin{IEEEeqnarray*}{rCl}
	(\wku)_2 & = &\varphi_4(\hat\bu) + (\varphi_2-\varphi_4)(\hat\bu)\,x -
	(\varphi_4+\varphi_7)(\hat\bu)\,z\\[4pt]
  &&\,+\, (\varphi_7-\varphi_8)(\hat\bu)\,\frac{xz}{1-z}.\\[4pt]
	(\varphi_2-\varphi_4)(\hat\bu) & = & (\varphi_2-\varphi_3-\varphi_4+\varphi_1)(\hat\bu)\\[4pt]
  &=&-\int_{\partial\hat{f}_5}\hat\bu\cdot d\hat\balpha\\[4pt]
  &=&-\iint_{\hat{f}_5}\nabla\times\hat{\bu}\cdot\hat\bn_5\,d\hat S
   =  \iint_{\hat{f}_5}\tfrac{\partial\hat{u}_2}{\partial\hat{x}_1}\,d\hat S.\\[4pt]
  (\varphi_4+\varphi_7)(\hat\bu) & = & 
  (\varphi_4+\varphi_7-\varphi_5)(\hat\bu)\\[4pt]
  &=& - \int_{\partial\hat{f}_2} \hat\bu\cdot d\hat\balpha\\[4pt]
  &=& -\iint_{\hat{f}_2}\nabla\times\hat\bu\cdot\hat\bn\,d\hat S~=~
      -\iint_{\hat{f}_2}\tfrac{{\s\partial} \hat u_2}{{\s\partial} \hat x_3}\,d\hat S\mbox{,}
\end{IEEEeqnarray*}
and we write down this second component of the interpolate
\begin{IEEEeqnarray}{rCl}
  \nonumber
  (\wku)_2& = & \int_{\hat\be_4}\hat u_2\,d\hat\alpha_4
               +x\iint_{\hat{f}_5}\tfrac{\partial\hat{u}_2}{\partial\hat{x}_1}\,d\hat S.
               +z\iint_{\hat{f}_2}\tfrac{{\s\partial} \hat u_2}{{\s\partial} \hat x_3}\,d\hat S\\
\label{second_b}
&&\,+\frac{xz}{1-z}\,2^{-\nicefrac12} \iint_{\hat{f}_4} \tfrac{\partial\hat{u}_2}{\partial\hat{x}_1}\,d\hat S.
\end{IEEEeqnarray}
And for the third one,
\begin{IEEEeqnarray*}{rCl}
	(\wku)_3&=&(\varphi_4+\varphi_7)(\hat\bu)\,y + (\varphi_2-\varphi_4-\varphi_7+\varphi_8)(\hat\bu)\,\frac{xy}{1-z}\\[4pt]
	& &\,+\,(\varphi_7-\varphi_8)(\hat\bu)\,\frac{xyz}{(1-z)^2}.\\[8pt]
	&=&(\varphi_2 - \varphi_4)(\hat\bu)\,\frac{xy}{1-z} + (\varphi_4+\varphi_7)(\hat\bu)\,y\\[8pt]
	& &-(\varphi_7-\varphi_8)(\hat\bu)\,\frac{xy}{(1-z)^2}.
\end{IEEEeqnarray*}
But  expressions for $(\varphi_2 - \varphi_4)(\cdot)$, $(\varphi_4+\varphi_7)(\cdot)$ and
$(\varphi_7-\varphi_8)(\cdot)$ were already stated above, so we have
\begin{IEEEeqnarray}{rCl}
  \nonumber
  (\wku)_3 & = & 
\frac{xy}{1-z}\,\iint_{\hat{f}_5}\tfrac{\partial\hat{u}_2}{\partial\hat{x}_1}\,d\hat S 
-y\,\iint_{\hat{f}_2}\tfrac{{\s\partial} \hat u_2}{{\s\partial} \hat x_3}\,d\hat S\\[8pt]
  \label{third_b}
  & &-2^{-\nicefrac12} \iint_{\hat{f}_4} \tfrac{\partial\hat{u}_2}{\partial\hat{x}_1}\,d\hat S\,\frac{xy}{(1-z)^2}.
\end{IEEEeqnarray}
\noindent{Finally for $\hat\bu = (0,0,\hat u_3)'$} it is
$\wku = \varphi_5(\hat\bu)\hat\bgamma_5 + 
		\varphi_6(\hat\bu)\hat\bgamma_6 + 
		\varphi_7(\hat\bu)\hat\bgamma_7 +
		\varphi_8(\hat\bu)\hat\bgamma_8$ and $\nabla\times\hat\bu = (\partial_2\hat{u}_3,
		-\partial_1\hat{u}_3,0)'.$\\[7pt]
First component of the interpolate:
\begin{IEEEeqnarray*}{rCl}
	(\wku)_1 & = &(\varphi_5-\varphi_6)(\hat\bu)z+
		(-\varphi_5+\varphi_6+\varphi_7-\varphi_8)(\hat\bu)\,\frac{yz}{1-z}.
\end{IEEEeqnarray*}
On one hand,
\begin{IEEEeqnarray*}{rCl}
	(\varphi_5-\varphi_6)(\hat\bu) & = & (\varphi_5-\varphi_6-\varphi_1)(\hat\bu) \\
	&=&-\iint_{\hat f_1}\nabla\times\hat\bu\cdot\hat\bn\,d\hat S
\end{IEEEeqnarray*}
On the other hand and analogously
\begin{IEEEeqnarray*}{rCl} 	
	(\varphi_7-\varphi_8)(\hat\bu) & = &	\iint_{\hat{f}_4}\nabla\times\hat\bu\cdot\hat\bn\,d\hat S
\end{IEEEeqnarray*}
% &=&-\iint_{\hat f_1}\partial_1 \hat{u}_3\,d\hat S.
so it follows   %% \noindent{\color{blue} cambia lo de region tipo I y eso? estoy haciendo $0\leqslant t_2 \leqslant 1; 0\leqslant t_1\leqslant 1-t_2$} 
\begin{IEEEeqnarray*}{rCl}
  (-\varphi_5+\varphi_6+\varphi_7-\varphi_8)(\hat\bu) & = & 
  \iint_{\hat{f}_1}\nabla\times\hat\bu\cdot\hat\bn\,d\hat S
  +\iint_{\hat{f}_4}\nabla\times\hat\bu\cdot\hat\bn\,d\hat S\\[4pt]
\IEEEeqnarraymulticol{3}{R}{
\begin{IEEEeqnarraybox*}{rCl}
\qquad&=&
  \int_{\mathbb{D}_{\hat f_1}}\tfrac{{\s\partial} \hat u_3}{{\s\partial} \hat x_1}
  (\Phi_{\hat f_1}(t_1,t_2))\,dt_1dt_2
  -\int_{\mathbb{D}_{\hat f_4}}\tfrac{{\s\partial} \hat u_3}{{\s\partial} \hat x_1}
  (\Phi_{\hat f_4}(t_1,t_2))\,dt_1dt_2\\[4pt]
&=&
  \int_0^1\int_0^{1-t_2} 
  \left[\tfrac{{\s\partial} \hat u_3}{{\s\partial} \hat x_1}(t_1,0,t_2)
    - \tfrac{{\s\partial} \hat u_3}{{\s\partial} \hat x_1}(t_1,1-t_2,t_2)\right]
  \,dt_1dt_2\\[4pt]
&=&
  -\int_0^1\int_0^{1-t_2}\int_0^{1-t_2} 
  \tfrac{{\s\partial^2} \hat u_3}{{\s\partial} \hat x_2{\s\partial} \hat x_1}(t_1,s,t_2)
  \,dsdt_1dt_2\\[6pt]
&=&-\int_{\hat{E}}\tfrac{{\s\partial}^2\hat u_3}{{\s\partial} \hat x_2{\s\partial} \hat x_1}
\,d\hat\bx.
\end{IEEEeqnarraybox*}
}
\end{IEEEeqnarray*}
For now we obtained
\begin{IEEEeqnarray}{rCl}\label{first_c}
	(\wku)_1 & = & -z\iint_{\hat f_1}\tfrac{{\s\partial} \hat u_3}{{\s\partial} \hat x_1}\,d\hat S
	-\frac{yz}{1-z}\int_{\hat{E}}
		\tfrac{{\s\partial}^2\hat u_3}{{\s\partial} \hat x_2{\s\partial} \hat x_1}\,d\hat\bx.
\end{IEEEeqnarray}
Regarding the second component, it is the symmetrical case, so we write
\begin{IEEEeqnarray*}{rCl}
  (\wku)_2& = &(\varphi_5-\varphi_7)(\hat\bu)z
		+(-\varphi_5+\varphi_6+\varphi_7-\varphi_8)(\hat\bu)\frac{xz}{1-z}\\
	& = &z\iint_{\hat f_2}\nabla\times\hat\bu\cdot\hat\bn\,d\hat S
	-\frac{xz}{1-z}\int_{\hat{E}}
	\tfrac{{\s\partial}^2\hat u_3}{{\s\partial} \hat x_2{\s\partial} \hat x_1}\,d\hat\bx\\
	\yesnumber\label{second_c}
	& = &-z\iint_{\hat f_2}\tfrac{{\s\partial} \hat u_3}{{\s\partial} \hat{x_2}}\,d\hat S
		-\frac{xz}{1-z}\int_{\hat{E}}
	\tfrac{{\s\partial}^2\hat u_3}{{\s\partial} \hat x_2{\s\partial} \hat x_1}\,d\hat\bx.
\end{IEEEeqnarray*}
For the third component let us denote $\xi(x,y,z) = 
  \frac{xyz}{(1-z)^2}-\frac{xz}{1-z}$. Then
\begin{IEEEeqnarray*}{rCl}
  (\wku)_3& = & \varphi_5(\hat\bu) + x\,(\varphi_6-\varphi_5)(\hat\bu)+
  y\,(\varphi_7-\varphi_5)(\hat\bu)\\[5pt]
  && \,+\,\xi(x,y,z)\,(\varphi_6-\varphi_5+\varphi_7-\varphi_8)(\hat\bu)\\[5pt]
  &=& \int_{\hat\be_5}\hat\bu\cdot d\hat\balpha_5 + 
  y\,\iint_{\hat{f}_2}\tfrac{{\s\partial} \hat u_3}{{\s\partial} {x_2}}\,d\hat S +
  x\,\iint_{\hat{f}_1}\tfrac{{\s\partial} \hat u_3}{{\s\partial} {x_1}}\,d\hat S
  \\[5pt]&&\,-\,\xi(x,y,z)\int_{\hat{E}}
    \tfrac{{\s\partial}^2\hat u_3}{{\s\partial} \hat x_2{\s\partial} \hat x_1}\,d\hat\bx.
  \yesnumber\label{third_c}
\end{IEEEeqnarray*}
%\begin{IEEEeqnarray*}{rCl}
%	(\wku)_3 & = & \int_{0}^1u_3(0,0,t)\,dt
%				+ x \int_{0}^{1}\int_{0}^{1-t}
%						\tfrac{{\s\partial} \hat u_3}{{\s\partial} \hat x_1} (s,0,t) \,d\hat s\,dt
%				+ y \int_{0}^{1}\int_{0}^{1-t}
%						\tfrac{{\s\partial} \hat u_3}{{\s\partial} \hat x_2}(0,s,t) \,d\hat s\,dt\\
%				&&\,\xi\int_{\hat{E}}
%				\tfrac{{\s\partial}^2\hat u_3}{{\s\partial} \hat x_2{\s\partial} \hat x_1}\,d\hat\bx.
%\end{IEEEeqnarray*}
All together, for a $\hat\bu=(\hat u_1,\hat u_2,\hat u_3)'$, if we combine 
what was obtained in~(\ref{first_a})--(\ref{third_c}), and apply Lemma~\ref{auxlabel350}, 
then it holds
\begin{IEEEeqnarray*}{rCl}
  (\wku)_1 & = & 
    \int_{\hat\be_1}\hat u_1\,d\hat\alpha + 
  z \iint_{\hat f_1}(\nabla\times\hat\bu)_2\,d\hat S +
  y \iint_{{\hat f_5}}\tfrac{{\s\partial} \hat u_1}{{\s\partial} \hat x_2}\,d\hat S\\[6pt]
    &&\,
+\frac{yz}{1-z} \iint_{\hat f_3} \tfrac{{\s\partial} \hat u_1}{{\s\partial} \hat x_2}\,d\hat S +
 \frac{yz}{1-z} \iint_{\hat f_4} \tfrac{{\s\partial} \hat u_2}{{\s\partial} \hat x_1}\,d\hat S\\[6pt]
    &&\,
-\frac{yz}{1-z} \int_{\hat{E}}\tfrac{{\s\partial}^2\hat u_3}{{\s\partial} \hat x_2{\s\partial} \hat x_1}\,d\hat\bx.\\[12pt]
    (\wku)_2 & = & \int_{\hat\be_4}\hat u_2\,d\hat\alpha - 
    z \iint_{\hat f_2}(\nabla\times\hat\bu)_1\,d\hat S +
    x \iint_{\hat f_5}\tfrac{{\s\partial} \hat u_2}{{\s\partial} \hat x_1}\,d\hat S\\
    &&\,+\frac{xz}{1-z} \iint_{\hat f_3}
    \tfrac{{\s\partial} \hat u_1}{{\s\partial} \hat x_2}\,d\hat S +
    \frac{xz}{1-z} \iint_{\hat f_4}
    \tfrac{{\s\partial} \hat u_2}{{\s\partial} \hat x_1}\,d\hat S\\
    &&\,+\frac{xz}{1-z} \int_{\hat{E}}
    \tfrac{{\s\partial}^2\hat u_3}{{\s\partial} \hat x_2{\s\partial} \hat x_1}\,d\hat\bx.\\[12pt]
  (\wku)_3 & = & \int_{\hat\be_5}\hat u_3\,d\hat\alpha - 
    x \iint_{\hat{f}_1} (\nabla\times\hat\bu)_2\,d\hat S +
    y \iint_{\hat{f}_2} (\nabla\times\hat\bu)_1\,d\hat S\\[8pt]
  &&\,+\frac{xy}{1-z}
\left\{
  \iint_{\hat{f}_5}\tfrac{{\s\partial} \hat u_2}{{\s\partial} \hat x_1}\,d\hat S+
  \iint_{\hat{f}_1}\tfrac{{\s\partial} \hat u_1}{{\s\partial} \hat x_3}\,d\hat S-
  2^{-\nicefrac12}\iint_{\hat{f}_4}\tfrac{{\s\partial} \hat u_1}{{\s\partial} \hat x_3}\,d\hat S
\right.\\[8pt]
  &&\,-
\left.
  2^{-\nicefrac12}\iint_{\hat{f}_4}\tfrac{{\s\partial} \hat u_1}{{\s\partial} \hat x_2}\,d\hat S
\right\}-
\frac{xy}{(1-z)^2}
2^{-\nicefrac12}\iint_{\hat{f}_4}\tfrac{{\s\partial} \hat u_2}{{\s\partial} \hat x_1}\,d\hat S\\[8pt]
\yesnumber\label{aux_label42}
&&\,+
\frac{xyz}{(1-z)^2}
2^{-\nicefrac12}\iint_{\hat{f}_3}\tfrac{{\s\partial} \hat u_1}{{\s\partial} \hat x_2}\,d\hat S+
\xi(x,y,z)\,
\int_{\hat{E}}
  \tfrac{{\s\partial}^2 u_3}{{\s\partial} \hat x_1{\s\partial} \hat x_2}\,d\hat\bx.
\end{IEEEeqnarray*}
\end{proof}
We continue with the local interpolation error estimate.
\rescaledPyramidTikz
\begin{theorem} \label{auxlabel211}
  Let $E$ be any pyramid which is
  a non degenerate affine image 
  of the reference pyramid $\hat{E}$. We fix a positively oriented local system of 
  coordinates $(\bxi_1, \bxi_2, \bxi_3)$
  with origin in a vertex $\bx_E$ of the parallelogram basis, for which $(\bxi_1, \bxi_2)$
  correspond to the two basis edges incident to $\bx_E$ and $\bxi_3$ is parallel to the 
  edge joining $\bx_E$ with the top of the pyramid. Let $h_1, h_2, h_3$ be the corresponding 
  edge lengths. With $\partial^{\balpha}$ we denote 
  $\tfrac{\partial^{|\balpha|}}{\partial_{\bxi_1}^{\alpha_1}\partial_{\bxi_2}^{\alpha_2}\partial_{\bxi_3}^{\alpha_3}}$.
  Suppose that
  $h_3 \geqslant \min \{h_1, h_2\}$ and let  $p>2$.
  For all $\bu\in W^{2,p}(E)^3$
\begin{IEEEeqnarray*}{rCl}\label{aux_label55}
  \|\bu-\bw_E \bu\|_{L^p(E)} & \lesssim &
    \sum_{|{\balpha}|=1}\bh^{\balpha} \|\partial^{\balpha} \bu\|_{L^p(E)} +\\[4pt]
  \yesnumber\label{auxlabel5}
   &&\,+\,h_E\sum_{|{\balpha}|=1}\bh^{\balpha}\|\partial^{\balpha} \curl \bu\|_{L^p(E)}\\
  & &\,+\,h_E^2 |\bu|_{2,p,E} \\[5pt]
  & &\,+\,\max \{h_{1}, h_2\} \left( \|\partial_{\tilde{x}_1}\tilde{u}_2\|_{\scriptscriptstyle L^p(\tilde E)}
   + \|\partial_{\tilde{x}_2}\tilde{u}_1\|_{\scriptscriptstyle L^p(\tilde E)}\right).
\end{IEEEeqnarray*} 
\end{theorem}
\begin{proof}
  Consider the matrix $M_{\tilde{E}}$ with coefficients $h_i\delta_{i,j}$, 
  $1\leqslant i,j\leqslant 3$ and take $\tilde{E}$ as the rescaled reference
  pyramid, that is, $\tilde{E} = M_{\tilde{E}}\hat{E}$. In 
  Figure~\ref{rescaled_pyramid} we have illustrated the scaling.
  Let us start with a stability estimate in $\tilde{E}$. Given a field $\tilde{\bu}$
  in $\tilde{E}$, pulling $\tilde{\bu}$ back to $\hat{E}$, using~\eqref{auxlabel203}
  and pushing forward to $\tilde{E}$ we get
  \begin{IEEEeqnarray*}{rCl}
    \|(\bw_{\tilde{E}}\tilde{\bu})_1\|_{\scriptscriptstyle L^\infty(\tilde{E})} &\lesssim&
      |\tilde{E}|^{-1/p}
      \big\{
        \|\tilde u_1\|_{\scriptscriptstyle L^p(\tilde E)} + 
          \sum_{i=1}^3 h_i \|\partial_{\tilde{x}_i}\tilde{u}_1\|_{\scriptscriptstyle L^p(\tilde{E})}
      \big\} \\[5pt]
    \IEEEeqnarraymulticol{3}{r}{+\,
      |\tilde{E}|^{\scriptscriptstyle -1} h_2
      \big\{
        \|(\nabla\times\tilde{\bu})_3\|_{\scriptscriptstyle L^1(\tilde{E})} + 
        \sum_{i=1}^3h_i(\|\partial_{\tilde{x}_i}(\nabla\times\tilde{\bu})_3\|_{\scriptscriptstyle L^1(\tilde{E})} +
                     \|\tfrac{\partial^2\tilde{u}_1}{\partial\tilde{x}_i\partial\tilde{x}_2}\|_{\scriptscriptstyle L^1(\tilde{E})})
      \big\}} \\[5pt]
    &&\,+\,
      |\tilde{E}|^{-1} h_3
      \big\{
        \|(\nabla\times\tilde{\bu})_2\|_{\scriptscriptstyle L^1(\tilde{E})} + 
             \sum_{i=1}^3h_i(\|\partial_{\tilde{x}_i}(\nabla\times\tilde{\bu})_2\|_{\scriptscriptstyle L^1(\tilde{E})}
      \big\} \\[5pt]
    &&\,+\,
      |\tilde{E}|^{-1} h_2h_3 \|\tfrac{\partial^2\tilde{u}_3}{\partial\tilde x_1\partial\tilde x_2}\|_{L^1(\tilde E)}.
  \end{IEEEeqnarray*}
  Estimate for component number two yields the analogue and now we write
  something similar to the third component. Note that in some cases we group terms
  using 
  \begin{IEEEeqnarray}{rCl}\label{auxlabel213}
  |\tilde E|^{-\tfrac{1}{q}}\|g\|_{\scriptscriptstyle L^q(\tilde E)} &\leqslant &
  |\tilde E|^{-\tfrac{1}{p}}\|g\|_{\scriptscriptstyle L^p(\tilde E)}\mbox{,}
  \end{IEEEeqnarray}
  whenever $q<p$, for scalar 
  functions
  in $L^p$. From~\eqref{auxlabel209}
  \begin{IEEEeqnarray*}{rCl}
    \|(\tilde\bw_{\tilde{E}}\tilde{\bu})_3\|_{L^\infty(\tilde{E})}&\lesssim&
    |\tilde{E}|^{-1/p}
    \big\{ 
      \|\tilde{u}_3\|_{\scriptscriptstyle L^p(\tilde{E})} + 
      \sum_{i=1}^3 h_i \|\tfrac{\partial\tilde{u}_3}{\partial\tilde{x}_i}\|_{\scriptscriptstyle L^p(\tilde{E})}
    \big\}\\[5pt]
    &&\,+\,|\tilde{E}|^{-1}h_1
    \big\{
      \|(\nabla\times\tilde{\bu})_2\|_{\scriptscriptstyle L^1(\tilde{E})}+
      \\[5pt]
    &&\,+
      \sum_{i=1}^3h_i
      (
        \|\partial_{\tilde{x}_i}(\nabla\times\tilde{\bu})_2\|_{\scriptscriptstyle L^1(\tilde{E})}+
        \|\tfrac{\partial^2\tilde{u}_3}{\partial\tilde{x}_i\partial\tilde{x}_1}\|_{\scriptscriptstyle L^1(\tilde{E})}
      )
    \big\}\\[5pt]
    \IEEEeqnarraymulticol{3}{r}{\,+\,|\tilde{E}|^{-1}h_2
        \big\{
        \|(\nabla\times\tilde{\bu})_1\|_{\scriptscriptstyle L^1(\tilde{E})} + 
                 \sum_{i=1}^3h_i(\|\partial_{\tilde{x}_i}(\nabla\times\tilde{\bu})_1\|_{\scriptscriptstyle L^1(\tilde{E})}
        \big\}}\\[5pt]
    &&  \,+\,|\tilde{E}|^{-1}\tfrac{h_1h_2}{h_3}
    \big\{
      \|\partial_{\tilde{x}_1}\tilde{u}_2\|_{\scriptscriptstyle L^1(\tilde{E})} + 
      \|\partial_{\tilde{x}_2}\tilde{u}_1\|_{\scriptscriptstyle L^1(\tilde{E})}\\[5pt]
    &&\,+ 
\sum_{i=1}^3 h_i
      (
        \|\tfrac{\partial^2\tilde{u}_1}{\partial\tilde{x}_i\partial\tilde{x}_2}\|_{\scriptscriptstyle L^1(\tilde{E})}+
        \|\tfrac{\partial^2\tilde{u}_2}{\partial\tilde{x}_i\partial\tilde{x}_1}\|_{\scriptscriptstyle L^1(\tilde{E})}
      )
    \big\}
    \\[5pt]
    &&\,+\,|\tilde{E}|^{-1}h_1h_2\|\tfrac{\partial^2\tilde{u}_3}{\partial\tilde{x}_1\partial\tilde{x}_2}\|_{\scriptscriptstyle L^1(\tilde{E})}.
  \end{IEEEeqnarray*}
Proceeding as in the proof of Theorem~\ref{aux_label27}
we obtain the following vectorial stability estimate in $\tilde E$:
\begin{IEEEeqnarray*}{rCl}
  \| \bw_{\tilde E}\tilde{\bu} \|_{\scriptscriptstyle L^p(\tilde E)}
  & \leqslant & \|\tilde{\bu}\|_{\scriptscriptstyle L^p(\tilde E)}
     + \sum_{i=1}^3 h_i\|\partial_{\tilde{x_i}}\tilde{\bu}\|_{\scriptscriptstyle L^p(\tilde E)} \\[5pt]  
  & &\,+\, \max \{h_{i}\} \left( \|\nabla\times\tilde{\bu}\|_{\scriptscriptstyle L^p(\tilde E)} + 
   \sum_{i=1}^3 h_i \|\partial_{\tilde{x}_i}\nabla\times\tilde{\bu}\|_{\scriptscriptstyle L^p(\tilde E)} \right) \\[5pt]
  & &\,+\, \max \{h_{i}\}^2 |\tilde{\bu}|_{2,p,\tilde{E}} \\[5pt]
  & &\,+\, \max \{h_{1}, h_2\} \big( \|\partial_{\tilde{x}_1}\tilde{u}_2\|_{\scriptscriptstyle L^p(\tilde E)}
   + \|\partial_{\tilde{x}_2}\tilde{u}_1\|_{\scriptscriptstyle L^p(\tilde E)} \big)
\end{IEEEeqnarray*}
And now we proceed as in the proof of Theorem~\ref{aux_label32}. First we 
transform from
a physical pyramidal element $E$ to $\tilde{E}$. In Section 5 of~\cite{gh99}
the approximation property of the finite element is stated and then we add the 
estimate~\eqref{aux_label30} for the rescaled pyramid $\tilde E$ in the case with 
multi--indices of order two,
to use in the corresponding terms of the averaged Taylor polynomial approximation
and the result follows.
\end{proof}













%%==========================================================================
%{\color{brown}
%    \begin{IEEEeqnarray*}{rCl}
%        (\wku)_1 & = & \int_{0}^{1}u_1(t,0,0)\,dt + 
%        z \int_0^1\int_0^{1-t_1}
%        \tfrac{{\s\partial} \hat u_1}{{\s\partial} \hat x_3}(t_1,0,t_2)\,dt_2dt_1 +
%        y \int_{{\hat f_5}}
%        \tfrac{{\s\partial} \hat u_1}{{\s\partial} \hat x_2}\,d\hat S\\
%        &&\,+\frac{yz}{1-z} \int_0^1\int_0^{1-t}
%        \tfrac{{\s\partial} \hat u_1}{{\s\partial} \hat x_2}(1-t,s,t)\,d\hat sdt +
%        \frac{yz}{1-z} \int_0^1\int_0^{1-t}
%        \tfrac{{\s\partial} \hat u_2}{{\s\partial} \hat x_1}(s,1-t,t)\,d\hat sdt\\
%        &&\,-z\int_0^1\int_0^{1-t_1}
%        \tfrac{{\s\partial} \hat u_3}{{\s\partial} \hat x_1}(t_1,0,t_2)\,dt_2dt_1 -
%        \frac{yz}{1-z} \int_{\hat{E}}
%        \tfrac{{\s\partial}^2\hat u_3}{{\s\partial} \hat x_2{\s\partial} \hat x_1}\,dV.
%    \end{IEEEeqnarray*}
%}
%%=========================================================================

%%===================================================================
%{\color{brown}
%\begin{IEEEeqnarray*}{rCl}
%    (\wku)_2 & = & \int_{0}^{1}u_2(0,t,0)\,dt + 
%    z \int_0^1\int_0^{1-t}
%    \tfrac{{\s\partial} \hat u_2}{{\s\partial} \hat x_3}(0,t,s)\,d\hat sdt +
%    x \int_0^1\int_0^{1}
%    \tfrac{{\s\partial} \hat u_2}{{\s\partial} \hat x_1}(s,t,0)\,d\hat sdt\\
%    &&\,+\frac{xz}{1-z} \int_0^1\int_0^{1-t}
%    \tfrac{{\s\partial} \hat u_1}{{\s\partial} \hat x_2}(1-t,s,t)\,d\hat sdt +
%    \frac{xz}{1-z} \int_0^1\int_0^{1-t}
%    \tfrac{{\s\partial} \hat u_2}{{\s\partial} \hat x_1}(s,1-t,t)\,d\hat sdt\\
%    &&\,-z\int_0^1\int_0^{1-t}
%    \tfrac{{\s\partial} \hat u_3}{{\s\partial} \hat x_2}(0,s,t)\,d\hat sdt +
%    \frac{xz}{1-z} \int_{\hat{P}}
%    \tfrac{{\s\partial}^2\hat u_3}{{\s\partial} \hat x_2{\s\partial} \hat x_1}\,dV.
%\end{IEEEeqnarray*}
%}
%%==================================================================

%%===========================================================================
%{\color{brown}
%\begin{IEEEeqnarray*}{rCl}
%        (\wku)_3 & = & \int_{0}^{1}u_3(0,0,t)\,dt + 
%        x \int_0^1\int_0^{1-t}
%        \tfrac{{\s\partial} \hat u_3}{{\s\partial} \hat x_1}(s,0,t)\,d\hat sdt -
%        x \int_0^1\int_0^{1-t}
%        \tfrac{{\s\partial} \hat u_1}{{\s\partial} \hat x_3}(t,0,s)\,d\hat sdt\\
%        &&\,+y \int_0^1\int_0^{1-t}
%        \tfrac{{\s\partial} \hat u_3}{{\s\partial} \hat x_2}(0,s,t)\,d\hat sdt -
%        y \int_0^1\int_0^{1-t}
%        \tfrac{{\s\partial} \hat u_2}{{\s\partial} \hat x_3}(0,t,s)\,d\hat sdt\\
%        &&\,+\frac{xy}{1-z} \int_0^1\int_t^{1}
%        \tfrac{{\s\partial} \hat u_1}{{\s\partial} \hat x_2}(t,s,0)\,d\hat sdt +
%        \frac{xy}{1-z} \int_0^1\int_0^{t}
%        \tfrac{{\s\partial} \hat u_2}{{\s\partial} \hat x_1}(t,s,1-t)\,d\hat sdt\\
%        &&\,+\frac{xyz}{(1-z)^2} \int_0^1\int_0^{1-t}
%        \tfrac{{\s\partial} \hat u_2}{{\s\partial} \hat x_1}(s,1-t,t)\,d\hat sdt
%        +\frac{xyz}{(1-z)^2} \int_0^1\int_0^{1-t}
%        \tfrac{{\s\partial} \hat u_1}{{\s\partial} \hat x_2}(1-t,s,t)\,d\hat sdt\\
%        &&\,-\frac{xy}{1-z}
%        \int_{0}^{1}
%        \int_{0}^{t}
%        \int_{0}^{1-t}
%        \tfrac{{\s\partial}^2u_1}{{\s\partial}x_2{\s\partial}x_3}(t,s,r)\,dr\,d\hat s\,dt
%        -\frac{xy}{1-z}
%        \int_{0}^{1}
%        \int_{0}^{t}
%        \int_{0}^{t}
%        \tfrac{{\s\partial}^2u_2}{{\s\partial}x_1{\s\partial}x_3}(r,s,1-t)\,dr\,d\hat s\,dt\\
%        &&\,
%        +\frac{xyz}{(1-z)^2} \int_{\hat{P}}
%        \tfrac{{\s\partial}^2 u_3}{{\s\partial} \hat x_1{\s\partial} \hat x_2}\,dV
%        -\frac{xz}{1-z} \int_{\hat{P}}
%        \tfrac{{\s\partial}^2 u_3}{{\s\partial} \hat x_1{\s\partial} \hat x_2}\,dV
%    \end{IEEEeqnarray*}
%    }
%%======================================================================================

% &=&-\int_0^1\int_0^{1-t}\tfrac{{\s\partial} \hat u_3}{{\s\partial} \hat x_1}(s,0,t)\,d\hat sdt.

%	(\pi\bu)_2 & = &\varphi_4 + (\varphi_2-\varphi_4)x -
%	(\varphi_4+\varphi_7)z + (\varphi_7-\varphi_8)\frac{xz}{1-z}.\\
%	\varphi_4 = \int_{0}^{1}u_2(0,t,0)\,dt\\
%	\varphi_2-\varphi_4 & = & \int_{0}^{1} u_2(1,t,0)-u_2(0,t,0)\,dt\\
%		&=&\int_{0}^{1}\int_{0}^{1}\tfrac{{\s\partial} \hat u_2}{{\s\partial} \hat x_1}(s,t,0)\,d\hat sdt\\
%	\varphi_4+\varphi_7 & = & \int_0^1 u_2(0,t,0)-u_2(0,t,1-t)\,dt\\
%		& = & \int_0^1\int_0^{1-t}\tfrac{{\s\partial} \hat u_2}{{\s\partial} \hat x_3}(0,t,s)\,d\hat sdt.\\
%	\varphi_7-\varphi_8&=&\int_{0}^{1} u_2(1-t,1-t,t)-u_2(0,1-t,t)\,dt\\
%		&=&\int_{0}^{1}\int_{0}^{1-t}\tfrac{{\s\partial} \hat u_2}{{\s\partial} \hat x_1}(s,1-t,t)\,d\hat sdt.
