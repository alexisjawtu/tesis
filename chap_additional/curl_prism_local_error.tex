\section{Local Interpolation Estimates for $H(\bcurl)$ Conforming Prismatic Elements}
\begin{theorem} \label{aux_label32} Let $k\in\mathbb{N}$ and $p>2$ and let $E$ 
be a prism whose triangular
faces have greatest angle less than $c_0$.
There exists $C > 0$ and three edges $\be_i$ of $E$ incident to a common vertex
$\bx_E$ such that for all $\bu\in W^{m + 1,p}(E)^3$
and $m\leqslant k-1$, %with $\bcurl \bu\in W^{m,p}(E)^3$
\begin{IEEEeqnarray*}{rCl}\label{aux_label55}
  \|\bu-\bw_E \bu\|_{L^p(E)} & \leqslant & C
  \left\{
    \sum_{|{\balpha}|=m+1}\bh^{\balpha} \|\partial^{\balpha} \bu\|_{L^p(E)} +\right.\\[4pt]
  \yesnumber\label{auxlabel5}
   &&\qquad\left. h_E\sum_{|{\balpha}|=m}\bh^{\balpha}\|\partial^{\balpha} 
    (\curl \bu)_3 \|_{L^p(E)}
  \right\}.
\end{IEEEeqnarray*} 
$C$ depends only on $c_0$.
$C$ can be chosen so that, if $M_E$ is the matrix made with
$\xi_i$ as columns, then $\|M\|_\infty\leqslant C$ and 
$\|M^{-1}\|_\infty\leqslant C$.
\end{theorem}
Notice the an\-iso\-tropic character of the inequality in~\eqref{auxlabel5}. Notice only the component
of the $\curl$ corresponding to the direction that is orthogonal to the 
triangular faces.
%[Proof of Theorem~\ref{aux_label32}]
\begin{proof}[Proof of Theorem~\ref{aux_label32}] %%% TODO: {\color{BrickRed}\#\#\#\#\#\#\#\# Ariel, por favor mirar si es correcta la manera en que lo digo.}\\\\
Since $W^{m+1,p}(E)\hookrightarrow W^{1,p}(E)$ and $p$ is greater than $2$,
the edge interpolator $\bw_E$ is well--defined via Corollary~\ref{aux_label26}.
Consider the prism $\tilde E$ as in~(\ref{tilde_prism}). By the argument in the 
proof of Theorem 2.2 in~\cite{aadl}, there is an affine map
$\tilde \bx \mapsto \bx = M_E\,\tilde\bx+\bx_E = F_E\,\tilde\bx$ from $\tilde E$ 
onto $E$, such that 
$\|M_E\|$, $\|M_E\|^{-1}\leqslant C(c_0)$. Notice that this is the only place 
where the dependence on $c_0$ is found. The matrix $M_E$ is made up of vectors 
$\xi_i$, $i = 1$, $2$, $3$ as its columns. First we take $\bq := \Qbb_{m,E}\,\bu$ and
do%where $\xi_i$ are the unitary vectors in the directions of three edges $\ell_i$  of E of lengths $h_i$ sharing the vertex $\bx_E$.
\begin{IEEEeqnarray*}{rCl}
  \|\bu-\bw_E\bu\|_{L^p(E)} & \leqslant & \|\bu-\bq\|_{L^p(E)}
    +\|\bw_E(\bu-\bq)\|_{L^p(E)}
\end{IEEEeqnarray*}
For the first term we may simply do, by Remark~\ref{aux_label28} and the
transformation~(\ref{transfHcurl}),
\begin{IEEEeqnarray}{rCl}
\nonumber
  \|\bu-\bq\|_{L^p(E)} & = & \|M_E^{-t}(\tilde{\bu}-\tilde{\bq})\circ F_E^{-1}\|_{L^p(E)} \\[5pt]
\label{aux_label37}
  &\leqslant& \|M^{-1}\||\det M_E|^{\nicefrac1p}\|\tilde{\bu}-\tilde{\bq}\|_{L^p(\tilde E)}.
\end{IEEEeqnarray}
With regard to the second term, by the commutativity property~(\ref{piTransformado})
and again the coordinate transformation,
\begin{IEEEeqnarray*}{rCl}
  \|\bw_E(\bu-\bq)\|_{L^p(E)}&\leqslant&
    |M|^{\nicefrac1p}\|M_E^{-1}\|  
      \|\tilde{\bw}_{\tilde E}(\tilde\bu-\tilde\bq)\|_{L^p(\tilde E)}.
\end{IEEEeqnarray*}
Theorem~\ref{aux_label27} implies
\begin{IEEEeqnarray*}{rCl}
    \|\bw_E(\bu-\bq)\|_{L^p(E)}
    & \leqslant & \\
\IEEEeqnarraymulticol{3}{r}{
\begin{IEEEeqnarraybox*}{rl}
  \qquad & C\|M^{-1}\||\det M_E|^{\nicefrac1p}
\left[ \| \tilde\bu-\tilde\bq \|_{L^p(\tilde{E})}
    + \sum_{i=1}^3 h_i \| \partial_{\tilde{x}_i}(\tilde\bu-\tilde\bq) \|_{L^p(\tilde{E})}\right.\\
&
    \left.
    \:+\;h\left(\left\|(\curl(\tilde\bu-\tilde\bq))_3 \right\|_{L^p(\tilde{E})}
     + \sum_{i=1}^3 h_i
     \left\| \partial_{\tilde{x}_i}(\curl(\tilde\bu-\tilde\bq))_3 \right\|_{L^p(\tilde{E})}\right)
  \right].
\end{IEEEeqnarraybox*}
}\\[4pt]
&&\yesnumber\label{aux_label34}
\end{IEEEeqnarray*}
By expressions~(\ref{aux_label30}),~(\ref{aux_label24}),~(\ref{aux_label25})
the last expression is bounded by a constant times
$\|M_E^{-1}\||\det M_E|^{\nicefrac1p}$
times the following sum
\begin{IEEEeqnarray}{rCl}
\nonumber
\sum_{i+j+k=m+1} h_1^ih_2^jh_3^k \left\| \frac{\partial^{m+1}\tilde\bu}
    {\partial\tilde x_1^i\partial\tilde x_2^j\partial\tilde x_3^k}
    \right\|_{0,\tilde E} &+&
h \sum_{j+k+l=m}  h_1^jh_2^kh_3^l
  \left\|\frac{\tilde\partial^m(\curl\tilde\bu)_3}{\partial\tilde x_1^j\partial\tilde x_2^k\partial\tilde x_3^l}
  \right\|_{0,\tilde E}
\\[7pt]
\IEEEeqnarraymulticol{3}{r}{
\nonumber
+\,h \sum_{i=1}^3 h_i\sum_{j+k+l=m-1}  h_1^jh_2^kh_3^l
        \left\|\frac{\tilde\partial^{m-1}\tilde\partial(\tilde\curl\tilde\bu)_3}
               {\partial\tilde x_1^j\partial\tilde x_2^k\partial\tilde x_3^l\partial\tilde x_j}
       \right\|_{0,\tilde E}
}\\[7pt]
\IEEEeqnarraymulticol{3}{r}{\nonumber
\lesssim
\sum_{i+j+k=m+1} h_1^ih_2^jh_3^k \left\| \frac{\partial^{m+1}\tilde\bu}
    {\partial\tilde x_1^i\partial\tilde x_2^j\partial\tilde x_3^k}
    \right\|_{0,\tilde E}
+  h \sum_{j+k+l=m}  h_1^jh_2^kh_3^l
  \left\|\frac{\tilde\partial^m(\curl\tilde\bu)_3}{\partial\tilde x_1^j\partial\tilde x_2^k\partial\tilde x_3^l}
  \right\|_{0,\tilde E}.}\\[4pt]&&
\label{aux_label33}
\end{IEEEeqnarray}
From equality~(\ref{aux_label29}), for every $\balpha$ of order
$m+1$ it holds
\begin{IEEEeqnarray}{rCl}\label{aux_label36}
  \|\tilde{\partial}^{\balpha}\tilde\bu\|_{L^p(\tilde{E})} & \leqslant & 
  \|M_E\|\,|\det M_E|^{-\nicefrac1p} \|\partial^{\balpha}\bu\|_{L^p(E)}.
\end{IEEEeqnarray} %%{\color{BrickRed}\#\#\#\#\#\#\#\# esto de la matriz realmente hace falta?.}
Lastly, adapting Lemma 3.57 in page 77 of~\cite{monk}, we observe
\begin{IEEEeqnarray*}{rCl}
  \begin{pmatrix}
    0 & -(\tilde\curl\tilde\bu)_3 & 0 \\
    (\tilde\curl\tilde\bu)_3 & 0 & 0 \\
    0 & 0 & 0 
  \end{pmatrix}M_E^{-1}
  & = & M_E^{t}
  \begin{pmatrix}
    0 & -(\curl\bu)_3 & 0 \\
    (\curl\bu)_3 & 0 & 0 \\
    0 & 0 & 0 
  \end{pmatrix}\circ F_E
\end{IEEEeqnarray*}
which implies, for every $\balpha$ of order $m$,
\begin{IEEEeqnarray}{rCl} \label{aux_label35}
  \|\tilde{\partial}^{\balpha}(\tilde{\curl}\tilde\bu)_3\|_{L^p(\tilde E)}
  & \leqslant & C |\det M_E|^{-\nicefrac1p}\,\|M_E\|^{2} 
  \|\partial^{\balpha}(\curl\bu)_3\|_{L^p(E)}.
\end{IEEEeqnarray}   %%{\color{blue}\#\#\#\#\#\#\#\# De donde sale el $(2+m)$ de (6.15) en~\cite{ariel}?.}
Now combine~(\ref{aux_label33}),~(\ref{aux_label36}) and~(\ref{aux_label35}) 
with~(\ref{aux_label34}) and~(\ref{aux_label37}) to obtain the
Theorem.
\end{proof}