As in the div--conforming case, the next step is to estimate the stability in 
an anisotropically rescaled prism. Consider again the element $\tilde{E}$ defined in~(\ref{tilde_prism}).
Given a natural number $k$, denote with ${\bw}_{\tilde{E}}$ the $k$--th order 
$\bcurl$--conforming interpolation
operator over $\tilde{E}$ defined as in Corollary~\ref{aux_label26}. 
For the rest of the Subsection, $\tilde\bu$ will be an element
with a well defined $\bcurl$--conforming interpolate.
%, namely of
%$H({\bf curl},\tilde{E})\cap H^{1/2+\delta}(\tilde{E})^3$ for 
%a positive $\delta$ with 
%${\bf curl}\,\tilde{\bu}\in L^p(\tilde{E})^3$
%for some
%$p>2$.
Write the diameter of $\tilde{E}$ as $h_{\tilde{E}}$ and as
$\tilde{x}_i,\,1\leqslant i\leqslant 3$, the coordinates along the axes
in $\mathbb{R}^3$.
\begin{lemma}\label{estabLinf} There exists a positive $C$, independent
of $h_i,\,1\leqslant i\leqslant 3$ such that for all $p > 2$ and 
$\tilde{\bu}\in\wpcurl{\tilde{E}}$
\begin{IEEEeqnarray*}{rCl}
    \left\| \wkutilde \right\|_{L^\infty(\tilde{E})^3}
    & \leqslant & C \left[ |\tilde{E}|^{-\nicefrac{1}{p}} \left( \left\| \tilde{\bu} 
    \right\|_{L^p(\tilde{E})^3} +
        \sum_{i=1}^3 h_i \left\| \partial_{\tilde{x}_i}\tilde{\bu} 
        \right\|_{L^p(\tilde{E})^3} \right)\right.\\
\IEEEeqnarraymulticol{3}{c}
{\left.\:+\; (h_1+h_2)\, |\tilde{E}|^{-1} \left( \left\|(\curl\,\tilde{\bu})_3 
    \right\|_{L^1(\tilde{E})} + 
    \sum_{i=1}^3 h_i \left\| \partial_{\tilde{x}_i}(\curl\,\tilde{\bu})_3 
    \right\|_{L^1(\tilde{E})}\right)
    \right].}
\end{IEEEeqnarray*}
%============
%{\color{BrickRed} ver las cuentas donde dice $h_1 + h_2$}
%============
% tal vez esto no haga falta
%para transformar $\hat{E} $ en $\tilde{E} $ v\'ia
%
%in this particular case ...
%\begin{IEEEeqnarray*}{rCl}
%    \hat{\pi}_i & = & h_i\tilde{\pi}_i \\
%    (\textbf{curl}\,\hat{\bu})_3 & = & h_1h_2(\textbf{curl}\,\tilde{\bu})_3.
%\end{IEEEeqnarray*}
%============
\end{lemma}
\begin{proof}
The proof of this estimate will be made componentwise
using the inequalities of 
Theorem~\ref{thm_stab_edge} and the vectorial bound will hold immediately.
Bounds for $(\wkutilde)_1$ and $(\wkutilde)_3$
will be established, as the bounding for $(\wkutilde)_2$ is the same as the first one.
Pulling $\wkutilde$ back to $\hat{E}$ we get by~(\ref{piTransformado}) 
that $(\wkutilde)_i = 
\nicefrac{1}{h_i} (\wku)_i,\,1\leqslant i\leqslant 3$. By inequality~(\ref{teorema_1}) and a suitable, though elementary,
change of variables dictated by~(\ref{change_var}) we do
\begin{IEEEeqnarray*}{rCl}
  \left\| (\wkutilde)_1 \right\|_{L^\infty(\tilde{E})} & = &
    \frac{1}{h_1} \left\| (\wku)_1 \right\|_{L^\infty(\hat{E})}\\
    & \leqslant & \frac{c(\hat{E})}{h_1} \left[\|\hat{u}_1\|_{W^{1,p}(\hat{E})} + 
        \|(\curl\,\hat{\bu})_3\|_{W^{1,1}(\hat{E})}\right] \\
    & \leqslant & c(\hat{E})
  \left[
    |\tilde{E}|^{\nicefrac{-1}{p}}
    \big\{
    \|\tilde{u}_1\|_{L^p(\tilde{E})} + \sum_{i=1}^3 h_i
    \|\tfrac{\partial\tilde{u}_1}{\partial\tilde{x}_i}\|_{L^p(\tilde{E})}
    \big\}
  \right.\\
\IEEEeqnarraymulticol{3}{r}{+
  \left.
    h_2|\tilde{E}|^{-1}
    \big\{
    \|(\curl\,\tilde{\bu})_3\|_{L^1(\tilde{E})} + 
        \sum_{i=1}^3 h_i \|\partial_{\tilde{x}_i}(\curl\,\tilde{\bu})_3\|_{L^1(\tilde{E})}
    \big\}
  \right].}
  \\&&\yesnumber\label{number1}
\end{IEEEeqnarray*}
With respect to component number three, from~(\ref{teorema_3}) we write
\begin{IEEEeqnarray}{rCl}\label{number2}
  \left\| (\wkutilde)_3 \right\|_{L^\infty(\tilde{E})}
  & \leqslant & C|\tilde{E}|^{-\nicefrac{1}{p}}
  \left(
    \|\tilde{u}_3\|_{L^p(\tilde{E})} + \sum_{i=1}^3 h_i \|\partial_{\tilde{x}_i}\tilde{u}_3\|_{L^p(\tilde{E})}
  \right).
\end{IEEEeqnarray}
\end{proof}
\noindent With the previous bound we deduce the following
anisotropic stability estimate for the rescaled prismatic element $\tilde{E}$.
\begin{theorem} \label{aux_label27}
There is a $C > 0$ independent of $h_i$ such that for all
$\tilde{\bu}\in\wpcurl{\tilde{E}}$ and $p>2$.
\begin{IEEEeqnarray*}{rCl}
    \|\wkutilde\|_{L^p(\tilde{E})}
    & \leqslant & C \left[ \left\| \tilde{\bu} \right\|_{L^p(\tilde{E})}
    + \sum_{i=1}^3 h_i \left\| \partial_{\tilde{x}_i}\tilde{\bu} \right\|_{L^p(\tilde{E})}\right.
\\\IEEEeqnarraymulticol{3}{r}{
\left.
    \:+\;(h_1+h_2)\left(\left\|(\curl\,\tilde{\bu})_3 \right\|_{L^p(\tilde{E})}
     + \sum_{i=1}^3 h_i
     \left\| \partial_{\tilde{x}_i}(\curl\,\tilde{\bu})_3 \right\|_{L^p(\tilde{E})}\right)
  \right].
}
\end{IEEEeqnarray*}
\end{theorem}
\begin{proof}
    \noindent From Lemma~\ref{estabLinf}, since $|\tilde{E}|$ is finite measured,
    the H\"older inequality tells us that, for any real $q \geqslant 1$,
    \begin{IEEEeqnarray*}{rCl}
        \|(\curl\tilde{\bu})_3\|_{L^1(\tilde{E})} &\leqslant&
         |\tilde{E}|^{1-\frac{1}{q}}\,\|(\curl\,\tilde{\bu})_3\|_{L^q(\tilde{E})}\\
        \|\partial_{\tilde{x}_i}(\curl\,\tilde{\bu})_3\|_{L^1(\tilde{E})} &\leqslant&
         |\tilde{E}|^{1-\frac{1}{q}}\,\|\partial_{\tilde{x}_i}(\curl\,\tilde{\bu})_3\|_{L^q(\tilde{E})}.
    \end{IEEEeqnarray*}
    So we get to
    \begin{IEEEeqnarray*}{rCl}
    \left\| (\tilde{\bw}_{\tilde E}\tilde{{\bu}})_1 \right\|_{L^p(\tilde{E})}
      & \leqslant & |\tilde{E}|^{\nicefrac{1}{p}}\left\| (\tilde{\bw}_{\tilde E}\tilde{{\bu}})_1 \right\|_{L^\infty(\tilde{E})}\\
      \mbox{(by~(\ref{number1}))\hspace{.6cm}}   & \leqslant & C
      \left[
        \|\tilde{u}_1\|_{L^p(\tilde{E})} + \sum_{i=1}^3 h_i \|\tfrac{\partial\tilde{u}_1}{\partial\tilde{x}_i}\|_{L^p(\tilde{E})}
        \right.\\
          & & \:\:+
        \left.
            h_2
            \left(
            \|(\curl\tilde{\bu})_3\|_{L^p(\tilde{E})} + 
                \sum_{i=1}^3 h_i \|\partial_{\tilde{x}_i}(\curl\tilde{\bu})_3\|_{L^p(\tilde{E})}
            \right)
        \right].
    \end{IEEEeqnarray*}
    Now combine this with an entirely analogous argument for component two and with~(\ref{number2}) and
    the Theorem follows.
\end{proof}