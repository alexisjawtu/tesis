\section{Prismatic Finite Elements} % (fold)
\label{sec:prismatic_finite_elements}
\subsection{Anisotropic Stability Estimates for $H(\bcurl)$--Conforming Finite
Elements on Prisms}
\label{stab_edge_prism}
\section{Stability of the Edge Element in $\hat{K}$}
In the next three Lemmas $\hat\bu$ is an element
in $H({\bf curl},\hat{E})$ and we will assume there are 
a positive $\delta$ and a $p>2$ such that 
$\hat\bu$ belongs to $H^{1/2+\delta}(\hat{E})^3$ and
${\bf curl}\,\bu$ belongs to $L^p(\hat{E})^3$.
\begin{lemma}\label{lema_PIu3_k_cualquiera} 
$(\wku)_3$ is linearly and univocally 
determined by $\hat{u}_3$.
\end{lemma}
\begin{proof} If we pay attention to the directions of the unit
tangents and normals to the edges and faces, respectively, of $\hat E$,
we realize that
the degrees of freedom which involve $\wku_3$ give rise onnlyto the 
following linear equations
\begin{IEEEeqnarray}{rCrc}
\varphi_{e_i,p}\,(\wku) & = & \varphi_{e_i,p}\,(\hat{\bu}) &\quad\mbox{for $i$ = 3, 6, 7 and }p\in\mathcal{}  \\
\varphi_{f_j,\boldsymbol{q}}\,(\wku) & = & \varphi_{f_j,\boldsymbol{q}}\,(\hat{\bu})
  &\quad\mbox{for $j$ = 1, 2, 4 and }\boldsymbol{q}\in\mathcal{}  \\
\varphi_{\boldsymbol{r}}\,(\wku) & = & \varphi_{\boldsymbol{r}}\,(\hat{\bu})
  &\quad\mbox{for }\boldsymbol{r}\in\mathcal{}.
\end{IEEEeqnarray}
These are 
$3k$+$3k(k-1)$+$k(k-1)(k-2)/2 = k(k+1)(k+2)/2$ equations,
just the dimension of $P_k(\hat{T})\otimes P_{k-1}(\hat{I})$, 
which is the space $(\wku)_3$ belongs to by definition.
%$\frac{k(k+1)(k+2)}{2}$. 
%libertad en los que no des\-a\-pa\-re\-ce $(\pi\textbf{u})_3$ son \'unicamente:
Now set all those equations to zero and see that the unique solution is $(\wku)_3 = 0$.
A little more explicitly, we have:
\noindent{\color{blue}\#\#\#\#\#\#\# } 
para cada arista $e_j$, $j = 3, 6, 7$,
\begin{IEEEeqnarray}{lClc}
  \label{aristas} \int\limits_{e_j} (\pi\textbf{u})_3 q \, ds 
  & = & 0 &\quad q\in P_{k-1}(e_j).
\end{IEEEeqnarray}
Para cada cara vertical $f=f_1$, $f_2$, $f_4$,
\begin{IEEEeqnarray}{lClc}
	\label{caras} \int\limits_{f} (\pi\textbf{u})_3 q \, 
	d\gamma & = & 0 &\quad q\in Q_{k-2,k-1}(f_{ijkl}).
\end{IEEEeqnarray}
En $\hat{K}$
\begin{IEEEeqnarray}{lClc}
	\label{enK} \int\limits_{\hat{K}} (\pi\textbf{u})_3 
	q \, d\textbf{x} & = & 0 &\quad q\in P_{k-3}(f_z) \otimes P_{k-1}(e_{xy}).
\end{IEEEeqnarray}
\noindent{\color{blue}\#\#\#\#\#\#\# }
 of freedom que estos grados de libertad se anulan todos y veamos que de esta suposici\'on
se deduce que $(\pi\textbf{u})_3 = 0$, es decir que las ecuaciones li\-nea\-les mencionadas determinan un\'ivocamente a un elemento
$(\pi\textbf{u})_3$ pertenenciente a 
$P_k(\hat{T}) \otimes P_{k-1}(\hat{I})$. 

Empezamos por ver que $(\pi\textbf{u})_3$ se anula en ciertos subconjuntos de 
$\partial \hat{K} $. La restricci\'on de $(\pi\textbf{u})_3$ a $e_j, (j=3,6,7)$
es un elemento de $P_{k-1}(e_j)$. Entonces si usamos los degrees of freedom~(\ref{aristas}) obtenemos
\begin{IEEEeqnarray}{rCl}
	\label{restriccAristas}\int\limits_{e_j} [(\pi\textbf{u})_3]^2\, ds & = & 0,
\end{IEEEeqnarray}
es decir que $(\pi\textbf{u})_3$ es id\'enticamente cero en esas aristas. Con esto la restricci\'on $(\pi\textbf{u})_3|_{f_2}$ 
que pertenece a $P_k(x)\otimes P_{k-1}(z)$ se factoriza como 
\begin{IEEEeqnarray*}{rCl}
	v(x,0,z) 	& = 	& x\,(x-1)\,w_0(x,z),\\
	w_0 		& \in 	& P_{k-2}(x)\otimes P_{k-1}(z),
\end{IEEEeqnarray*}
pero este \'ultimo es precisamente el espacio de los degrees of freedom~(\ref{caras}) con los 
cuales llegamos a
\begin{IEEEeqnarray}{lClc}
	\int\limits_{f_2} x(x-1)[w_0(x,z)]^2\,d\gamma & = & 0.
\end{IEEEeqnarray}
Entonces en $f_2$ es $w_0 \equiv 0$ y $(\pi\textbf{u})_3|_{f_2} \equiv 0$. Por simetr\'ia, el argumento para probar que
$(\pi\textbf{u})_3|_{f_1} \equiv 0$ es exactamente igual.

Si volvemos a usar las igualdades~(\ref{restriccAristas}) obtenemos que la restricc\'on de $(\pi\textbf{u})_3$ a
$f_{4} = \{ (x,y,z) \;:\; 0\leqslant x\leqslant 1,\, y = 1 - x, \, 0\leqslant z\leqslant 1 \}$ se anula si $x=1-y=0$ o bien si $x=1-y=1$;
\begin{IEEEeqnarray}{rCl}
	v(x,1-x,z) 	& = 	& x\,(1-x)\,w_1(x,z)\\
	w_1 		& \in 	& P_{k-2}(x)\otimes P_{k-1}(z).
\end{IEEEeqnarray}
Si aplicamos los degrees of freedom~(\ref{caras}) llegamos 
a que en ${f_4}$ es $w_1$ id\'enticamente cero y as\'i tambi\'en $(\pi\textbf{u})_3$.\\
Recapitulando, 
\begin{IEEEeqnarray*}{rCl}
	(\pi\textbf{u})_3|_{f_j} & \equiv & 0\quad\text{ para } j = 1,2,4.
\end{IEEEeqnarray*}
Entonces en todo $\hat{K} $ tenemos la factorizaci\'on 
\begin{IEEEeqnarray*}{rCl}
	(\pi\textbf{u})_3(x,y,z) 	& = 	& x\,y\,(1-x-y)\,w_3(x,y,z),\\
						w_3		& \in 	& P_{k-3}(x)\otimes P_{k-1}(z).
\end{IEEEeqnarray*}
Simplemente aplicando los degrees of freedom~(\ref{enK}) nos queda que $w_3 \equiv 0$, y finalmente
$(\pi \textbf{u})_3 \equiv 0$.
%	debe anularse en todo $\hat{K}^{\textrm{o}}$, y por continuidad tambi\'en en 
%$\partial\hat{K}$. Si volvemos a la expresi\'on~(\ref{piu3}) nos queda que
\end{proof}
\begin{lemma}\label{lemma_PIu2_k_in_N}
\begin{IEEEeqnarray*}{rCl}
\label{piu2_k_in_N}
	\yesnumber\pi(0, u_2(y,z), 0) & = 	& (0, \xi_2(y,z) ,0)\textrm{,}\\
\label{piu1_k_in_N}	
	\yesnumber\pi(u_1(x,z), 0, 0) & = 	& (\xi_1(x,z), 0 ,0)\textrm{,}\\
	\xi_1 				& \in 	& P_{k-1}(x) \otimes P_k(z)\textrm{,}\\
	\xi_2 				& \in 	& P_{k-1}(y) \otimes P_k(z).
\end{IEEEeqnarray*}
\end{lemma}
\begin{proof} Demostramos s\'olo la primera igualdad, porque la otra es
an\'aloga. Lo que hay que ver es que, en la expresi\'on encontrada en
~(\ref{sub:elemento_P_k}), es $h \equiv 0$, $\xi_1 \equiv 0$ y que $\xi_2$ 
no depende de $x$. Gracias al lema~\ref{lema_PIu3_k_cualquiera} ya sabemos
que $\xi_3 \equiv 0$.
As\'{\i} que veamos primero $h \equiv 0$.
Observaci\'on: si $f$ es $f_3$ o $f_4$, entonces
\begin{IEEEeqnarray}{rCl}
	(\textbf{curl}\,\pi\textbf{u})_3 |_{_{f}} & \in & P_{k-1}(x,y).	
\end{IEEEeqnarray}
Por el Lema~\ref{lema_pi_star_rot_u}, si usamos los degrees of freedom~(\ref{momentos1hdiv})
obtenemos, para todo $q \in P_{k-1}(x,y)$,
\begin{IEEEeqnarray}{rCl}
	\int\limits_{f} (\textbf{curl}\,\pi\textbf{u})_3\,q \,d\gamma & = & 
		\int\limits_{f} \textrm{rot}(\textbf{u})_3\,q \,d\gamma\\
		& = & 0.	
\end{IEEEeqnarray}
Si tomamos $q = (\textbf{curl}\,\pi\textbf{u})_3 |_{_{f}}$ tenemos
$(\textbf{curl}\,\pi\textbf{u})_3 |_{_{f}} \equiv 0$, o, de otra manera,
$z\,(z-1)$ divide a $(\textbf{curl}\,\pi\textbf{u})_3$. Escribamos
\begin{IEEEeqnarray*}{rCl}
	(\textbf{curl}\,\pi\textbf{u})_3 & = 	& z\, (z-1)\, \psi\\[6pt]
	\psi						& \in 	& P_{k-1}(x,y) \otimes P_{k-2}(z).
\end{IEEEeqnarray*}
A continuaci\'on usamos los degrees of freedom~(\ref{momentos4hdiv}) en la misma 
definici\'on de antes. Para todo $q\in P_{k-1}(x,y) \otimes P_{k-2}(z)$
\begin{IEEEeqnarray*}{rCl}
	\int\limits_{\hat{K}} z\,(z-1)\,\psi \, q\,d\textbf{x} & = & 0,
\end{IEEEeqnarray*}
as\'{\i} que tomando $q = \psi$ probamos que $\psi \equiv 0$ en $\hat{K}$,
es decir que
\begin{IEEEeqnarray}{rCl}
	\label{rot_3_es_0} (\textbf{curl}\,\pi\textbf{u})_3 &\equiv& 0.
\end{IEEEeqnarray}
Ahora veamos c\'omo es $(\textbf{curl}\,\pi\textbf{u})_3$ en t\'erminos
de la expresi\'on encontrada en~\ref{sub:elemento_P_k}.
\begin{IEEEeqnarray*}{rCl}
	(\textbf{curl}\,\pi\textbf{u})_3 & = & 
	\partial_x\,\pi(\textbf{u})_2 -	\partial_y\,\pi(\textbf{u})_1\\
	\label{expre_h} \yesnumber & = & -(2\,h + y\,\partial_y\,h + 
				x\,\partial_x\,h) +	\partial_x\,\xi_2 - \partial_y\,\xi_1,
\end{IEEEeqnarray*}
en donde, observando los grados de cada t\'ermino,
\begin{IEEEeqnarray*}{rCl}
	2\,h + y\,\partial_y\,h + x\,\partial_x\,h & \in & \tilde{P}_{k-1}(x,y)
\otimes P_k(z)\textrm{ y }\\
\partial_x\,\xi_2 - \partial_y\,\xi_1 & \in & P_{k-2}(x,y) \otimes P_k(z).
\end{IEEEeqnarray*}
De esto necesariamente sigue que 
$g := 2\,h + y\,\partial_y\,h + x\,\partial_x\,h = 0$. Ahora exploramos los 
t\'erminos de $g$. Pongamos
\[
	h(x,y,z) = \sum\limits_{i+j=k-1, l\leqslant k} \alpha_{_{i,j,l}} x^i y^j z^l.
\]
Entonces, para todo $(x,y,z)$ en $\hat{K}$
\begin{IEEEeqnarray*}{rCl}
	g(x,y,z) & = & \sum\limits_{i+j=k-1, l\leqslant k} 
(2\alpha_{_{i,j,l}} + j \alpha_{_{i,j,l}} + i \alpha_{_{i,j,l}}) x^i y^j z^l\\
	& = &(k+1)\,h(x,y,z)\\
  \yesnumber\label{h_is_zero}	& = & 0,
\end{IEEEeqnarray*}
o sea, $h \equiv 0$.
Hasta ac\'a tenemos probado que $\pi(0, u_2(y,z), 0) = 
(\xi_1(x,y,z), \xi_2(x,y,z), 0)$. Let's see that $\xi_1$ vanishes identically. Empecemos con 
los degrees of freedom sobre las aristas. Si $e$ es $e_1$ o $e_4$ entonces
\[
	\xi_1|_{e} \in P_{k-1}(x)
\]
y adem\'as, para todo $q\in P_{k-1}(x)$
\[
	\int\limits_{e} \xi_1|_{e}\,q\,ds = 0\textrm{,}
\]
con lo que llegamos a que $z\,(z-1)$ divide tambi\'en a $\xi_1|_{f_2}$. Pongamos
\begin{IEEEeqnarray}{rCl}
	\xi_1|_{f_2}(x,z) &=&z\,(z-1)\phi(x,z)\\[6pt]
	\phi &\in& P_{k-1}(x)\otimes P_{k-2}(z).
\end{IEEEeqnarray}
A continuaci\'on, si usamos los degrees of freedom en $f_2$, tenemos, para todo 
$q\in P_{k-1}(x)\otimes P_{k-2}(z)$,
\begin{IEEEeqnarray}{rCl}
	\int\limits_{f_2} z\,(z-1)\phi\,q\,d\gamma &=&0.
\end{IEEEeqnarray}
Tomando $q=\phi$ sigue que $\xi_1|_{f_2}\equiv 0$, con lo cual, para $(x,y,z)
\in \hat{K}$
\begin{IEEEeqnarray}{rCl}
	\xi_1(x,y,z) & = & y\,\rho(x,y,z)\\[6pt]
	\rho &\in&P_{k-2}(x,y)\otimes P_k(z).	
\end{IEEEeqnarray}
Ahora miramos los degrees of freedom en $f = f_3, f_4$. Vale que $\xi_1|_{f}$ pertenece 
a $P_{k-2}(x,y)$, y que para todo $q\in P_{k-2}(x,y)$
\[
	\int\limits_{f} \xi_1\,q\,d\gamma = 0\textrm{,}
\]
de donde, por tomar $q = \xi_1|_{f}$, sigue que $\xi_1|_{f} \equiv 0$. Con todo
esto tenemos que $z\,(z-1)$ divide a $\xi_1$, as\'{\i} que
\begin{IEEEeqnarray*}{rCl}
	\xi_1(x,y,x) &=&y\,z\,(z-1)\,\theta(x,y,z)\\[6pt]
	\theta &\in& P_{k-2}(x,y)\otimes P_{k-2}(z).
\end{IEEEeqnarray*}
Resta mirar los degrees of freedom de volumen aplicados a $\pi(\textbf{u})$. Para 
todo $q \in  P_{k-2}(x,y)\otimes P_{k-2}(z)$ debe ser
\begin{IEEEeqnarray*}{rCl}
\int\limits_{\hat{K}} y\,z\,(z-1)\,\theta(x,y,z)\,q\,d\textbf{x} &=&0
\textrm{,} 
\end{IEEEeqnarray*}
de donde, al tomar $q = \theta$, sigue inmediatamente que, en todo $\hat{K}$,
\begin{IEEEeqnarray}{rCl}
\label{xi1_es_0}\xi_1 & \equiv & 0.
\end{IEEEeqnarray}
Hasta ahora probamos que  
\[
	\pi(0, u_2(y,z), 0)  = 	 (0, \xi_2(x,y,z) ,0).
\]
Pero si recordamos lo que probamos en~(\ref{rot_3_es_0}) y lo combinamos con
~(\ref{xi1_es_0}), inmediatamente llegamos a 
\begin{IEEEeqnarray*}{rCCCl}
	\partial_x \pi(\textbf{u})_2 &=& \partial_x \xi_2 &\equiv& 0\textrm{,}
\end{IEEEeqnarray*}
que implica que $\xi_2$ no depende de $x$, y que $\pi(0, u_2(y,z), 0)$ tiene 
la forma que quer\'{\i}amos.
\end{proof}
\begin{lemma}\label{pi00u3} Si $\pi$ es el interpolador determinado por el elemento de la
\emph{Definici\'on}~(\ref{edgeelement}), entonces
\begin{IEEEeqnarray}{rCl}
	\pi(0,0, u_3)& = & (0,0,\xi_3)\textrm{,}\\
	\nonumber		\xi_3 & \in & \p{k}{k-1}.
\end{IEEEeqnarray}
\end{lemma}
\begin{proof} La demostraci\'on ser\'a muy parecida a la del Lema~(\ref{lemma_PIu2_k_in_N}). Recordemos
la expresi\'on que encontramos en la Subsecci\'on~\ref{sub:elemento_P_k}; tenemos
\begin{IEEEeqnarray}{rCl}
	\label{expre_pi00u3} \pi(0,0,u_3) &=& (\xi_1 + y\,h,\xi_2 - x\,h, \xi_3).	
\end{IEEEeqnarray}
Empezamos por ver que $h$ se anula. Sea $f$ cualquiera de las dos caras horizontales del 
prisma de referencia. Usamos la expresi\'on~(\ref{expre_h}) para 
$(\textbf{curl}\,\pi\textbf{u})_3 \in \p{k-1}{k}$. Gracias al
Lema~\ref{lema_pi_star_rot_u} y a las igualdades~(\ref{momentos1hdiv}) vale
\begin{IEEEeqnarray}{rCl}
	(\textbf{curl}\,\pi\textbf{u})_3 |_{f}&\equiv&0.
\end{IEEEeqnarray}
Entonces los polinomios $z$ y $z-1$ dividen a $(\textbf{curl}\,\pi\textbf{u})_3=
z\,(z-1)\,\psi$, ($\psi\in P_{k-1}(x,y)\otimes P_{k-2}(z)$). Ahora vamos a los
degrees of freedom~(\ref{momentos4hdiv}) para obtener finalmente que $\psi$ es id\'enticamente cero. Es decir
que, en todo $\hat{K}$, $(\textbf{curl}\,\pi\textbf{u})_3 \equiv 0$ y, siguiendo
el argumento en la demostraci\'on del Lema~\ref{lemma_PIu2_k_in_N} probamos que $h$ es id\'enticamente
cero. As\'{\i} que podemos reescribir la expresi\'on~(\ref{expre_pi00u3}) y poner
\begin{IEEEeqnarray}{rCl}
	\label{expre_pi00u3_} \pi(0,0,u_3) &=& (\xi_1,\xi_2, \xi_3)\\
	\nonumber\xi_1, \xi_2&\in& \p{k-1}{k}.
\end{IEEEeqnarray}
Resta ver que $\xi_1$ y $\xi_2$ se anulan. Con la misma idea, si consideramos
las caras $f_1$ con normal $\boldsymbol{\nu}=(-1,0,0) $ y $f_2$ con normal 
$\boldsymbol{\nu}=(0,-1,0)$, entonces las igualdades~(\ref{momentos1hcurl}) 
para las aristas $e_1, e_4$ junto con las igualdades~(\ref{momentos4hcurl}) por un lado,
y las igualdades~(\ref{momentos1hcurl}) para las aristas $e_2, e_5$ y~(\ref{momentos3hcurl})
por el otro, implican, respectivamente, que $y$ divide a $\xi_1$ y $x$ divide a $\xi_2$. Si 
continuamos con los degrees of freedom sobre las dos caras horizontales~(\ref{momentos2hcurl})
obtenemos tambi\'en que $z\,(z-1)$ los divide a ambos. Finalmente, si aplicamos las
igualdades~(\ref{momentos6hcurl}) probamos que $\xi_1 = \xi_2 \equiv 0$ en todo $\hat{K}$.
En conclusi\'on, $\pi(0,0, u_3)$ tiene la forma deseada.
\end{proof}
\begin{theorem}\label{thm_stab_edge}
Dados $p > 2$, $\hat{\emph{\textbf{u}}} \in \wpcurl{\hat{K}}$, si $\hat{\pi}$ es 
el operador de interpolaci\'on determinado por el elemento de la Definici\'on~(\ref{edgeelement}),
entonces
\begin{IEEEeqnarray}{rCl}
\label{teorema_1} \norm{\hat{\pi}(\hat{\emph{\textbf{u}}})_1}_{L^{\infty}(\hat{E})} & 
	\lesssim & \|\hat{u}_1\|_{W^{1,p}(\hat{E})} + 
	\|\emph{\textbf{curl}}(\hat{\emph{\textbf{u}}})_3\|_{{\color{red} W^{1,1}(\hat{E})}} \\	
\label{teorema_2} \norm{\hat{\pi}(\hat{\emph{\textbf{u}}})_2}_{L^{\infty}(\hat{E})} & 
	\lesssim & \|\hat{u}_2\|_{W^{1,p}(\hat{E})} + 
	\|\emph{\textbf{curl}}(\hat{\emph{\textbf{u}}})_3\|_{{\color{red} W^{1,1}(\hat{E})}} \\	
\label{teorema_3} \norm{\hat{\pi}(\hat{\emph{\textbf{u}}})_3}_{L^{\infty}(\hat{E})} & 
	\lesssim & \|\hat{u}_3\|_{W^{1,p}(\hat{E})}.
\end{IEEEeqnarray}
\end{theorem}
\begin{proof}
The proof will rely on the three previous Lemmas, 
the triangular inequality applied on each component of 
expression~(\ref{edge_interp_explicit}) and traces inequalities.

First we will take a smooth field $\hat{\bu}$ defined on $\hat{E}$
and, by Lemma~(\ref{lemaDensidad}), we will conclude the Theorem 
with a density argumentation.\\[5pt]


Las dos primeras se 
demuestran de forma parecida; probamos~(\ref{teorema_1}). La idea es tomar una funci\'on otra,
$\hat{\textbf{w}}$, tal que su interpolada tenga igual primera coordenada que la de 
$\hat{\bu}$,
pero que tenga degrees of freedom m\'as c\'omodos de acotar en t\'erminos exclusivamente de 
$\hat{u}_1$ y $\textbf{curl}(\hat{\bu})_3$.
Definimos, dada $\hat{\bu} \in W^{1,p}(\hat{K})^3$,
\begin{IEEEeqnarray*}{rCl}
	\hat{\textbf{w}} & = & (\hat{u}_1, \hat{u}_2 - \hat{u}_2(0,y,z), 0).
\end{IEEEeqnarray*}
Como dec\'{\i}amos, observemos primero que gracias a los lemas~(\ref{lemma_PIu2_k_in_N}) 
y~(\ref{pi00u3}) es
\begin{IEEEeqnarray*}{rCl}
	\hat{\pi}(\hat{\textbf{w}})_1 & = & (\hat{\pi}\hat{\bu})_1 - 
	\hat{\pi}(0, \hat{u}_2(0,y,z), 0)_1 -
	\hat{\pi}(0, 0, \hat{u}_3)_1\\
						& = & (\hat{\pi}\hat{\bu})_1.
\end{IEEEeqnarray*}
Ahora exploremos uno por uno los degrees of freedom que definen a $\pi(\hat{\textbf{w}})$. Los \'unicos
degrees of freedom sobre aristas que no se anulan o que no dependen expl\'{\i}citamente s\'olo de $\textrm{u}_1$
son
\begin{IEEEeqnarray*}{rCl}
	\int\limits_{e_8} \hat{\textbf{w}} \cdot \boldsymbol{\tau}\,\phi\,ds & = &
	\tfrac{1}{\sqrt{2}} \int\limits_{e_8} (w_1 - w_2)\,\phi\,ds\\
	\int\limits_{e_9} \hat{\textbf{w}} \cdot \boldsymbol{\tau}\,\phi\,ds & = &
	\tfrac{1}{\sqrt{2}} \int\limits_{e_9} (w_1 - w_2)\,\phi\,ds
\end{IEEEeqnarray*}
para $q\in \mathcal{P}_{k-1}$. Para el momento sobre $e_8$ hacemos partes en la cara $f_4
\subseteq \{ z=1 \}$. Tomemos un polinomio $\phi \in \mathcal{P}_{k-1}$ sobre $e_8$, que puede
verse como $\phi(y)$, dado que en $e_8$ es $x = 1 - y$.
\begin{IEEEeqnarray*}{rCl}
	\int\limits_{f_4} \textbf{curl}(\hat{\bu})_3\,\phi\,d\gamma & = &
	\int\limits_{f_4} \textbf{curl}(\hat{\textbf{w}})_3\,\phi\,d\gamma\\
	& = & -\int\limits_{f_4} \left(w_2\,\partial_x\phi - w_1\,\partial_y\phi\right)\,d\gamma
		+ \int\limits_{\partial f_4} \left(w_2\,\nu_x - w_1\,\nu_y\right)\,\phi\,ds\\
	& = &  \int\limits_{f_4} w_1\,\partial_y\phi\,d\gamma
		+ \int\limits_{e_8} \left(w_2 - w_1\right)\,\phi\,ds + 
			\int\limits_{e_4} w_1\,\phi\,ds.
\end{IEEEeqnarray*}
Es decir
\begin{IEEEeqnarray}{rCl}\label{momentosWaristas}
	\tfrac{1}{\sqrt{2}} \int\limits_{e_8} (w_1 - w_2)\,\phi\,ds & = &
		 \tfrac{1}{\sqrt{2}} \int\limits_{e_4} u_1\,\phi\,ds - 
		 \tfrac{1}{\sqrt{2}} \int\limits_{f_4} \textbf{curl}(\hat{\bu})_3\,\phi\,d\gamma
		 + \int\limits_{f_3} u_1\,\partial_y\phi\,d\gamma.
\end{IEEEeqnarray}
De igual manera hacemos partes en $f_3 \subseteq \{ z=0 \}$ para obtener
\begin{IEEEeqnarray}{rCl}\label{momentosWaristas2}
	\tfrac{1}{\sqrt{2}} \int\limits_{e_9} (w_1 - w_2)\,\phi\,ds & = &
		 \tfrac{1}{\sqrt{2}} \int\limits_{e_1} u_1\,\phi\,ds - 
		 \tfrac{1}{\sqrt{2}} \int\limits_{f_3} \textbf{curl}(\hat{\bu})_3\,\phi\,d\gamma
		 + \int\limits_{f_3} u_1\,\partial_y\phi\,d\gamma.
\end{IEEEeqnarray}
Ahora miramos los degrees of freedom sobre las caras. Tambi\'en aqu\'{\i} hacemos notar que s\'olo hace falta
trabajar con los degrees of freedom sobre las dos caras horizontales $f_3$ y $f_4$ y sobre la cara $f_5$.
Tomemos $\phi_1$, $\phi_2 \in \mathcal{P}_{k-2}(x,y)$, $\boldsymbol{\phi} = (\phi_1, \phi_2, 0)$.
\begin{IEEEeqnarray}{rCl}
 	\label{cotaf3}\int\limits_{f_3} \hat{\textbf{w}} \times \boldsymbol{\nu} \cdot \boldsymbol{\phi}\,d\gamma
 		& = & \int\limits_{f_3} u_1\,\phi_2\,d\gamma + \int\limits_{f_3} w_2\,\phi_1\,d\gamma.
\end{IEEEeqnarray}
Ahora sea un polinomio $\varphi_1 \in \mathcal{P}_{k-1}(x,y) $ tal que 
\begin{IEEEeqnarray*}{rCl}
	\partial_x \varphi_1 & = & \phi_1\textrm{,}\\
	\varphi_1 |_{e_9} 	 & \equiv & 0\textrm{;}
\end{IEEEeqnarray*}
tomemos por ejemplo $\varphi_1 = -\int_{x}^{1-y} \phi_1(t,y)\,dt$. Entonces
\begin{IEEEeqnarray*}{rCl}
	\int\limits_{f_3} \textbf{curl}(\hat{\bu})_3\,\varphi_1\,d\gamma & = & -\int\limits_{f_3} \left(w_2\,\phi_1 - w_1\,\partial_y\varphi_1\right)\,d\gamma
		+ \int\limits_{e_1} w_1\,\nu_y\,\varphi_1\,ds,
\end{IEEEeqnarray*}
y de esto con junto con~(\ref{cotaf3}) sigue que
\begin{IEEEeqnarray}{rCl}\label{momentosWcaras}
 	\nonumber\int\limits_{f_3} \hat{\textbf{w}} \times \boldsymbol{\nu} \cdot \boldsymbol{\phi}\,d\gamma
 		& = & \int\limits_{f_3} u_1\,\phi_2\,d\gamma - \int\limits_{f_3} \textbf{curl}(\hat{\bu})_3\,\varphi_1\,d\gamma\\
 		& 	& + \int\limits_{f_3} u_1\,\partial_y\varphi_1\,d\gamma	+ \int\limits_{e_1} u_1\,\nu_y\,\varphi_1\,ds.
\end{IEEEeqnarray}
Si repetimos lo anterior para el momento en $f_4$, obtenemos 
\begin{IEEEeqnarray}{rCl}\label{momentosWcaras2}
 	\nonumber\int\limits_{f_4} \hat{\textbf{w}} \times \boldsymbol{\nu} \cdot \boldsymbol{\phi}\,d\gamma
 		& = & \int\limits_{f_4} u_1\,\phi_2\,d\gamma - \int\limits_{f_4} \textbf{curl}(\hat{\bu})_3\,\varphi_1\,d\gamma\\
 		& 	& + \int\limits_{f_4} u_1\,\partial_y\varphi_1\,d\gamma	+ \int\limits_{e_4} u_1\,\nu_y\,\varphi_1\,ds\textrm{,}
\end{IEEEeqnarray}
y con la misma técnica para $f_5$, si dado $\textbf{q} = (0,q_3,q_1) \in \{ 0 \} \times Q_{k-2,k-1} \times Q_{k-1,k-2}$ 
tomamos $\varphi_1(x,z)=-\int\limits_{x}^{1} q_1(t,z)\,dt$, entonces obtenemos
\begin{IEEEeqnarray}{rCl}\label{momentosWcaras3}
 	\nonumber\int\limits_{f_4} \hat{\textbf{w}} \times \boldsymbol{\nu} \cdot \textbf{q}\,d\gamma
 		& = & \int\limits_{f_5} u_1\,q_1\,d\gamma + \int\limits_{f_5} \textbf{curl}(\hat{\bu})_3\,\varphi_1\,d\gamma\\
 		& 	& - \int\limits_{f_5} u_1\,\partial_y\varphi_1\,d\gamma	+ \int\limits_{e_4} u_1\,\varphi_1\,ds.
\end{IEEEeqnarray}
Finalmente, estudiamos los degrees of freedom de volumen. Tomemos
$\boldsymbol{\phi} = (\phi_1, \phi_2, \phi_3) \in (P_{k-2}(x,y) \otimes P_{k-2}(z))^{\color{red}2}
\times P_{k-3}(x,y) \otimes
P_{k-1}(z)$ (cfr.~(\ref{momentos6hcurl})) y sea $\varphi_2 = - \int\limits_{x}^{1-y} \phi_2(t,y,z)\,dt$, para el cual vale 
$\partial_x\varphi_2 = \phi_2 $ y $\varphi_2|_{f_5} \equiv 0$. Entonces, por partes,
\begin{IEEEeqnarray}{rCl}\label{momentosWvolumen}
 	\nonumber\int\limits_{\hat{K}} \hat{\textbf{w}} \cdot \boldsymbol{\phi}\,d\textbf{x}
 		& = & \int\limits_{\hat{K}} u_1\,\phi_1\,d\textbf{x} -\int\limits_{\hat{K}} \textbf{curl}(\hat{\bu})_3\,
 		\varphi_2,d\textbf{x}\\
 		& 	& + \int\limits_{\hat{K}} u_1\,\partial_y\varphi_2\,d\textbf{x}	+ 
 		\int\limits_{f_2} u_1\,\varphi_2\,d\gamma.
\end{IEEEeqnarray} %,~(\ref{momentosWcaras2}),~(\ref{momentosWcaras3})
Ahora recolectamos lo que dicen las igualdades~(\ref{momentosWaristas}),~(\ref{momentosWaristas2}),
y~(\ref{momentosWcaras})--(\ref{momentosWvolumen}), para hacer simplemente
%\begin{IEEEeqnarray*}{rCl}
%	\|\hat{\pi}(\hat{\textbf{w}})\|_{L^\infty(\hat{K})} & = 
%	 & 
%	 & \leqslant & C \left(\|\hat{u}_1\|_{W^{1,p}(\hat{K})} +
%		\|{\textbf{curl}}({\hat{\bu}})_3\|_{W^{1,1}(\hat{K})}\right)
%\end{IEEEeqnarray*}
\begin{IEEEeqnarray*}{rCl}
	\|(\hat{\pi}\hat{\bu})_1\|_{L^\infty(\hat{K})} & = 
	&\|(\hat{\pi}\hat{\textbf{w}})_1\|_{L^\infty(\hat{K})}\\
	& \leqslant & C \left\|\sum F_{i}(\hat{\textbf{w}}, \textbf{q}_i)\,\hat{\textbf{v}}_i\right\|_{L^\infty(\hat{K})}\\
	& \leqslant & C \sum \left|F_{i}(\hat{\textbf{w}}, \textbf{q}_i)\right|\,
		\left\|\hat{\textbf{v}}_i\right\|_{L^\infty(\hat{K})}\\
	& \leqslant & C \left(\|\hat{u}_1\|_{W^{1,p}(\hat{K})} +
		\|{\textbf{curl}}({\hat{\bu}})_3\|_{{\color{red} W^{1,1}(\hat{K})}}\right),
\end{IEEEeqnarray*}
que es la desigualdad a la que quer\'iamos llegar.\\[5pt]
\noindent For inequality~(\ref{teorema_3}) given $\hat{\bu} \in W^{1,p}(\hat{K})^3$, define
$\hat{\bv}  =  (0,0, \hat{u}_3).$
Thanks to Lemma~(\ref{lema_PIu3_k_cualquiera}) we have 
$\hat{\bw}_k(\hat{\bv})_3 = (\hat{\bw}_k\hat{\bu})_3 - (\hat{\bw}_k(\hat{u}_1, \hat{u}_2, 0))_3 = (\hat{\bw}_k\hat{\bu})_3.$
Apply now expression~(\ref{edge_interp_explicit}) to $\bv$ writing the degrees of
freedom more explicitly. Taking another look at 
the unit tangent vector of the edges and unit normal vectors to the faces we have to
consider only a few not null terms ({\color{red}recall the numbering of the edges and faces}).
\begin{IEEEeqnarray*}{rCl}
  (\hat{\boldsymbol{w}}_k\hat{\bv})_3 & = &
  \sum_{j=3,6,7\,p\,\in\,{\color{red}\mathcal{B}_{\hat e_j}}} \int\limits_{\hat e_j} \hat{u}_3 p\,ds         \,(\hat{\bv}_{e_j,p})_3 +
  \sum_{i=1,2,4\,q\,\in\,{\color{red}\mathcal{B}_{\hat f_i}}} \int\limits_{\hat f_i} \hat{u}_3 q\,d\gamma    \,(\hat{\bv}_{f_i,q})_3 +
  \sum_{\boldsymbol{r}\,\in\,{\color{red}\mathcal{B}_{\hat E}}}
  \int\limits_{\hat E} \hat{u}_3 r_3\,d\textbf{x}\,(\hat{\bv}_{\boldsymbol{r}})_3.
\end{IEEEeqnarray*}
And now apply the triangular inequality and the proof
of Lemma $5.38$ in the page $134$ of Monk~\cite{monk}
and Theorem $3.9$ (\emph{Trace Theorem})
in page $43$ of
the same book, to get \noindent{\color{blue}\#\#\#\#\#\#\# revisar los exponentes hoelder y todo...
porque en monk pone $H^{1/2+\delta}$} 
\begin{IEEEeqnarray*}{rCl}
  \norm{(\hat{\boldsymbol{w}}_k\hat{\bu})_3}_{L^{\infty}(\hat{E})} =
  \norm{(\hat{\boldsymbol{w}}_k\hat{\bv})_3}_{L^{\infty}(\hat{E})} & 
  \leqslant & c\,\|\hat{u}_3\|_{W^{1,p}(\hat{E})}
\end{IEEEeqnarray*}
where the quantity $c$ depends, clearly, on the choice of the bases $\mathcal{B}_{\hat e_j},
\mathcal{B}_{\hat f_i}, \mathcal{B}_{\hat E}$.
\end{proof}
\noindent the next step is to estimate the stability in 
an anisotropically rescaled prism. Given three positive numbers
$h_1$, $h_2$ and $h_3$ we denote
\begin{IEEEeqnarray*}{CCl}
    \yesnumber\label{tilde_prism}
    \tilde{E}   &   =   &   \tilde{T} \times \tilde{I}\\
    \tilde{T}   &   =   &   \{ 0 < \nicefrac{\tilde{x}_1}{h_1} + \nicefrac{\tilde{x}_2  }{h_2} < 1 \}\\
    \tilde{I}   &   =   &   \{ 0 < \nicefrac{\tilde{x}_3}{h_3} < 1 \}.
\end{IEEEeqnarray*}
\rescaledPrismTikz
Of course $\tilde{E} = F(\hat{E})$ where $F$ is the linear
$\mathbb{R}^3 \rightarrow \mathbb{R}^3$ transformation such as
\begin{IEEEeqnarray}{rClCl}
  \label{change_var}
  F\hat{\bx} & = & \diag{h_1}{h_2}{h_3} \hat{\bx} & = & \tilde{\bx}
\end{IEEEeqnarray}
Denote with $\tilde{\bw}_k$ the $k$--th order curl--conforming interpolation
operator over $\tilde{E}$ defined as in~(\ref{push-forward}) via the \emph{push--forward}
$F^*$. for the rest of the subsection $\tilde\bu$ will be an element
with a well defined curl--conforming interpolate.
%, namely of
%$H({\bf curl},\tilde{E})\cap H^{1/2+\delta}(\tilde{E})^3$ for 
%a positive $\delta$ with 
%${\bf curl}\,\tilde{\bu}\in L^p(\tilde{E})^3$
%for some
%$p>2$.
Write the diameter of $\tilde{E}$ as $\textit{h}_{\tilde{E}}$ and as
$\tilde{x}_i,\,1\leqslant i\leqslant 3$, the coordinates along the axis
in $\mathbb{R}^3$.
\begin{lemma}\label{estabLinf} There exists a positive $C$, independent
of $h_i,\,1\leqslant i\leqslant 3$ such that for all $p > 2$ and 
$\tilde{\bu}\in\wpcurl{\tilde{E}}$
\begin{IEEEeqnarray*}{rCl}
    \left\| \wkutilde \right\|_{L^\infty(\tilde{E})^3}
    & \leqslant & C \left[ |\tilde{E}|^{-\nicefrac{1}{p}} \left( \left\| \tilde{\bu} 
    \right\|_{L^p(\tilde{E})^3} +
        \sum_{i=1}^3 h_i \left\| \partial_{\tilde{x}_i}\tilde{\bu} 
        \right\|_{L^p(\tilde{E})^3} \right)\right.\\
    &   & \left.\:+\; (h_1+h_2)\, |\tilde{E}|^{-1} \left( \left\|(\curl\,\tilde{\bu})_3 
    \right\|_{L^1(\tilde{E})} + 
    \sum_{i=1}^3 h_i \left\| \partial_{\tilde{x}_i}(\curl\,\tilde{\bu})_3 
    \right\|_{L^1(\tilde{E})}\right)
    \right].
\end{IEEEeqnarray*}
{\color{BrickRed} ver las cuentas donde dice $h_1 + h_2$}
\end{lemma}
\begin{proof}
The proof of this estimate will be made componentwise
using the inequalities of 
Theorem~(\ref{thm_stab_edge}) and the vectorial bound will hold immediately.
Bounds for $(\wkutilde)_1$ and $(\wkutilde)_3$
will be established, as the bounding for $(\wkutilde)_2$ is the same as the first one.
Pulling $\wkutilde$ back to $\hat{E}$ we get by
~(\ref{piTransformado}) that $(\wkutilde)_i = 
\nicefrac{1}{h_i} (\wku)_i,\,1\leqslant i\leqslant 3$. By~(\ref{teorema_1}) and a suitable, though elementary,
change of variables dictated by~(\ref{change_var})
\begin{IEEEeqnarray*}{rCl}
  \left\| (\wkutilde)_1 \right\|_{L^\infty(\tilde{E})} & = &
    \frac{1}{h_1} \left\| (\wku)_1 \right\|_{L^\infty(\hat{E})}\\
    & \leqslant & \frac{c(\hat{E})}{h_1} \left[\|\hat{u}_1\|_{W^{1,p}(\hat{E})} + 
        \|(\curl\,\hat{\bu})_3\|_{W^{1,1}(\hat{E})}\right] \\
    & \leqslant & c(\hat{E})
  \left[
    |\tilde{E}|^{\nicefrac{-1}{p}}
    \left(
    \|\tilde{u}_1\|_{L^p(\tilde{E})} + \sum_{i=1}^3 h_i \left\|\frac{\partial\tilde{u}_1}{\partial\tilde{x}_i}
    \right\|_{L^p(\tilde{E})}
    \right)
  \right.\\
\yesnumber\label{number1}      & & \:\:+
  \left.
    h_2|\tilde{E}|^{-1}
    \left(
    \|(\curl\,\tilde{\bu})_3\|_{L^1(\tilde{E})} + 
        \sum_{i=1}^3 h_i \|\partial_{\tilde{x}_i}(\curl\,\tilde{\bu})_3\|_{L^1(\tilde{E})}
    \right)
  \right].
\end{IEEEeqnarray*}
With respect to component number three, from~(\ref{teorema_3})
\begin{IEEEeqnarray}{rCl}\label{number2}
  \left\| (\wkutilde)_3 \right\|_{L^\infty(\tilde{E})}
  & \leqslant & C|\tilde{E}|^{-\nicefrac{1}{p}}
  \left(
    \|\tilde{u}_3\|_{L^p(\tilde{E})} + \sum_{i=1}^3 h_i \|\partial_{\tilde{x}_i}\tilde{u}_3\|_{L^p(\tilde{E})}
  \right).
\end{IEEEeqnarray}
\end{proof}
\noindent With the previous bound we deduce the following
anisotropic stability estimate for the rescaled element $\tilde{E}$.
\begin{theorem} There is a $c > 0$ independent of $h_i$ such that for all
$\tilde{\bu}\in\wpcurl{\tilde{E}}$
  \begin{IEEEeqnarray*}{rCl}
    \left\| \wkutilde \right\|_{L^p(\tilde{E})}
    & \leqslant & C \left[ \left\| \tilde{\bu} \right\|_{L^p(\tilde{E})}
    + \sum_{i=1}^3 h_i \left\| \partial_{\tilde{x}_i}\tilde{\bu} \right\|_{L^p(\tilde{E})}\right.\\
    & & \left.
    \:+\;(h_1+h_2)\left(\left\|(\curl\,\tilde{\bu})_3 \right\|_{L^p(\tilde{E})}
     + \sum_{i=1}^3 h_i
     \left\| \partial_{\tilde{x}_i}(\curl\,\tilde{\bu})_3 \right\|_{L^p(\tilde{E})}\right)
  \right].
\end{IEEEeqnarray*}
\end{theorem}
\begin{proof}
    \noindent From Lemma~(\ref{estabLinf}), since $|\tilde{E}|$ is finite measured,
    the H\"older inequality tells us that, for any real $q \geqslant 1$,
    \begin{IEEEeqnarray*}{rCl}
        \|(\curl\,\tilde{\textbf{u}})_3\|_{L^1(\tilde{E})} &\leqslant&
         |\tilde{E}|^{1-\frac{1}{q}}\,\|(\curl\,\tilde{\textbf{u}})_3\|_{L^q(\tilde{E})}\\
        \|\partial_{\tilde{x}_i}(\curl\,\tilde{\textbf{u}})_3\|_{L^1(\tilde{E})} &\leqslant&
         |\tilde{E}|^{1-\frac{1}{q}}\,\|\partial_{\tilde{x}_i}(\curl\,\tilde{\textbf{u}})_3\|_{L^q(\tilde{E})}.
    \end{IEEEeqnarray*}
    So we get to
    \begin{IEEEeqnarray*}{rCl}
    \left\| (\tilde{\bw}_k\tilde{{\textbf{u}}})_1 \right\|_{L^p(\tilde{E})}
        & \leqslant & |\tilde{E}|^{\nicefrac{1}{p}}\left\| (\tilde{\bw}_k\tilde{{\textbf{u}}})_1 \right\|_{L^\infty(\tilde{E})}\\
     \mbox{(by~(\ref{number1}))\hspace{.6cm}}   & \leqslant & C
        \left[
            \|\tilde{u}_1\|_{L^p(\tilde{E})} + \sum_{i=1}^3 h_i \|\frac{\partial\tilde{u}_1}{\partial\tilde{x}_i}\|_{L^p(\tilde{E})}
        \right.\\
            & & \:\:+
        \left.
            h_2
            \left(
            \|(\curl\,\tilde{\textbf{u}})_3\|_{L^p(\tilde{E})} + 
                \sum_{i=1}^3 h_i \|\partial_{\tilde{x}_i}(\curl\,\tilde{\textbf{u}})_3\|_{L^p(\tilde{E})}
            \right)
        \right].
    \end{IEEEeqnarray*}
    Now combine this with an entirely analogous argument for component $2$ and with~(\ref{number2}) and
    the Theorem follows.
\end{proof}
\subsection{Anisotropic Stability Estimates for $H(\Div)$--Conforming Finite Elements
on Prisms} % (fold)
The exposition in this section will be the $H(\mbox{div})$--conforming analogue
of the exposition in the previous section.\\
\noindent In present subsection $\hat\bu$ will be an element
of $W^{1,1}(\hat E)$ in which elements have well defined
normal traces over the faces of $\hat{E}$, another possibility
being, as mentioned in~\cite{monk}, Lemma $5.15$, page $120$, to assume
there is
a positive $\delta$ such that $\hat\bu$ belongs to
$H^{1/2+\delta}(\hat{E})^3$.
For the whole subsection, $\hat\br_{\hat E}$ will be the $k$--th order face 
interpolation operator on the reference
Prism determined by the element of
Definition~\ref{defi_h_div_conforme}.
\label{stability_of_rt_element_in_hat_k}
\begin{lemma}\label{lemmaRT3zero}
$(\rku)_3$ is linearly and univocally determined by $\hat{u}_3$.
\end{lemma}
\begin{proof}
By the unisolvence of the finite element in Definition~\ref{defi_h_div_conforme},
$(\rku)_3$ is determined by the following linear equations.
\begin{IEEEeqnarray}{rCrl}
\label{rku_3_1}
\hat\rho_{\hat f_j,\hat q}\,(\rku) & = & \hat\rho_{\hat f_j,\hat q}\,(\hat{\bu})
  &\quad\mbox{for $j$ = 3, 4 and }\hat q\in P_{k-1}({\hat{f}_j}) \\
\label{rku_3_2}
\hat\rho_{\hat\br}\,(\rku) & = & \hat\rho_{\hat\br}\,(\hat{\bu})
  &  \quad\mbox{for }\hat\br =(0, 0, \hat r_3)\mbox{, }\hat r_3 \in P_{k-1,k-2}.
\end{IEEEeqnarray}
We have $\nicefrac{k(k+1)^2}{2}$ independent equations, which is the 
number of independent coefficients in $(\rku)_3$.
Now set $\hat u_3 = 0$, which makes the right hand side of all the equations in~(\ref{rku_3_1})
and~(\ref{rku_3_2}) equal to zero.
Take $\hat f$ to be either $\hat{f}_3$ or $\hat{f}_4$. Put 
$\hat q = (\rku)_3|_{\hat{f}}$ in~(\ref{rku_3_1}) recalling~(\ref{prismaticSpace}).
It yields
\begin{IEEEeqnarray*}{rCl}
  \iint_{\hat{f}} \{(\rku)_3\}^2\,d\hat{S} & = & 0.
\end{IEEEeqnarray*}
so there is a polinomial $\hat{q}_3\in {P}_{k-1,k-1}$ such that
$(\rku)_3 \xyz = \hat{x}_3(\hat{x}_3-1)\,\hat{q}_3\xyz$
and the Lemma will follow once we apply the degrees of freedom~(\ref{rku_3_2})
with the test polynomial $\hat\br$ set equal to $(0,0,\hat{q}_3)'$. 
\end{proof}
\begin{lemma}\label{lemma_u1_u2} If $\hat\bu\xyz = (0,0, \hat{u}_3\xyz)'$,
then $\rku\xyz = (0,0,\hat{\xi}_3\xyz)'$ for some $\hat{\xi}_3\in
P_{k-1}(\hat{f}_3)\otimes P_k(\hat{x}_3)$.
\end{lemma}
\begin{proof}
In a completely analogous way as was done in Section~\ref{sub:defEdgeElement}
we can derive the following expressions for 
$\rku = (\hat{p}_1, \hat{p}_2, \hat{p}_3)' \in  P_{\hat{E}}$: 
\begin{IEEEeqnarray*}{rCl}
  \hat{p}_1\xyz & = & \hat{q}_1\xyz + \hat{x}_1\,\hat{h}\xyz\\
  \label{exprPrt}\yesnumber
  \hat{p}_2\xyz & = & \hat{q}_2\xyz + \hat{x}_2\,\hat{h}\xyz\\
  \hat{p}_3\xyz & = & \hat{q}_3\xyz
\end{IEEEeqnarray*}
for unique $\hat{q}_1, \hat{q}_2 \in {P}_{k-1}(\hat{f}_3)
\otimes{P}_{k-1}(\hat{x}_3)$,
$\hat{q}_3 \in {P}_{k-1}(\hat{f}_3)\otimes{P}_{k}(\hat{x}_3)$,
and
$\hat{h} \in \tilde{{P}}_{k-1}(\hat{f}_3)\otimes{P}_{k-1}(\hat{x}_3)$.
Take an arbitrary $\hat{q}\in{P}_{k-1}(\hat f_1)\otimes P_{k-1}(\hat z)$.
Continuing with the technique developed for Section~\ref{stab_edge_prism}, 
the trick will be to apply Green's Theorem to the field
$(\hat{p}_1, \hat{p}_2, 0)'$ and the scalar $\hat{q}$. By the surface degrees of
freedom~(\ref{momentos2hdiv}) and the volume degrees of freedom~(\ref{momentos3hdiv}) we have
  \begin{IEEEeqnarray*}{rClCl}
    \int_{\hat{E}} \mbox{div}(\hat{p}_1, \hat{p}_2, 0)'\,\hat{q}\,d\hat{\bx}&=&
    \int_{\partial\hat{E}} (\hat{p}_1, \hat{p}_2, 0)'\cdot\hat\bn\,\hat{q}\,d\hat{S}
    - \int_{\hat{E}} (\hat{p}_1, \hat{p}_2, 0)'\cdot\nabla \hat{q}\,d\hat{\bx}&&\\[5pt]
    & = &
    \int_{\hat{f}_5} (\hat{u}_1 + \hat{u}_2)
    \hat{q}_{|_{\hat{f}_5 = 1}}\,d\hat{S}
    - \int_{\hat{f}_1} \hat{u}_1 \hat{q}_{|_{\hat{f}_1}}\,d\hat{S}&&\\[5pt]
    &&\,- \int_{\hat{f}_2} \hat{u}_2 \hat{q}_{|_{\hat{f}_2}}\,d\hat{S}
    - \int_{\hat{E}} \hat{u}_1
    \tfrac{\partial\hat{q}}{\partial{\hat{x}_1}} 
      + \hat{u}_2\tfrac{\partial\hat{q}}{\partial{\hat{x}_2}}\,d\hat{\bx} & = & 0.
  \end{IEEEeqnarray*}
  And since $\mbox{div}(\hat{p}_1, \hat{p}_2, 0)'$ also belongs to
  $P{k-1}{k-1}$, we just established it vanishes on all $\hat{E}$.\\[3pt]
  In a similar way as done in the proof of Lemma~\ref{lemma_PIu2_k_in_N}, eq.~(\ref{h_is_zero}),
  we get to prove $\hat{h} \equiv 0$ in expression~(\ref{exprPrt}). It means we may
  assume that $\hat{p}_1$ and $\hat{p}_2$ belong to 
  ${P}_{k-1}(\hat{f}_3)\otimes{P}_{k-1}(\hat{x}_3)$.
  Now it is convenient to use again the degrees of freedom on the
  faces normal to $(-1, 0, 0)'$ and $(0, -1, 0)'$.
  The conditions
  \begin{IEEEeqnarray*}{rCCCll}
    \hat\rho_{\hat f_i\,\hat q}(\rku) & = & \iint_{\hat f_i} \hat{p}_i\hat q\,d\hat S
    & = & 0\qquad\mbox{for all }\hat{q}\in{P}_{k-1}(\hat f_i)\mbox{, \,}&1\leqslant i\leqslant 2
  \end{IEEEeqnarray*}
  ensure that $\hat{x}_i$ divides $\hat{p}_i$ for both $i=1$ and $2$.
%  In other words, there are $\phi$ and $\psi$ in $\pp{k-2}{k-1}$ such that
%  \begin{IEEEeqnarray*}{rCl}
%    \hat{p}_1 & = & \hat{x}\,\phi\\
%    \hat{p}_2 & = & \hat{y}\,\psi.
%  \end{IEEEeqnarray*}
But finally, if we evaluate the degrees of freedom~(\ref{momentos3hdiv}),
we see that  $\hat{p}_1 = (\rku)_1$ and 
$\hat{p}_2 = (\rku)_2$ can be no other that
constantly null over all $\hat{E}$. 
\end{proof}
\begin{lemma}
\begin{itemize}
  \item []
  \item [(a)]\label{piu2_k_in_N} If $\hat\bu(\hat x_1,\hat x_2,\hat x_3) =
  (0, \hat{u}_2(\hat x_1,\hat x_3), 0)'$,
  then $\rku\xyz = (0, \hat\xi_2(\hat x_1, \hat x_3) ,0)'$ for some 
  $\hat\xi_2 \in P_{k-1}(\hat{x}_2) \otimes P_k(\hat{x}_3)$.
  \item [(b)]\label{piu1_k_in_N} If $\hat\bu(\hat x_1,\hat x_2,\hat x_3) = 
  (\hat{u}_1(\hat x_2,\hat x_3), 0, 0)'$
  then $\rku\xyz = (\hat\xi_1(\hat x_2,\hat x_3), 0 ,0)'$ for some
    $\hat\xi_1\in P_{k-1}(\hat{x}_1) \otimes P_k(\hat{x}_3)$.
\end{itemize}
\end{lemma}
\begin{proof} Let us prove the first one of the two claims. The second one 
  has an entirely analogous proof. By Lemma~\ref{lemmaRT3zero} we get
  $(\rku)_3 = 0$.
  The nullity of $\dv\hat{\bu}$ and the commutative
  diagram property~(\ref{div_commutativity}) give us
  $\dv\rku = 0$.
  This last fact, together with the result of the evaluation of the 
  degrees of freedom~(\ref{momentos2hdiv})
  on the face $\hat f_1 \subseteq \{x_1=0\}$
  and~(\ref{momentos3hdiv}) over $\hat E$, implies, as we have seen in
  Lemma~\ref{lemma_u1_u2}, that $(\rku)_1 = 0$.
  And if look again at 
  $\dv\rku = {\partial\rku}/{\partial\hat x_2} = 0$
  we have that $(\rku)_2$ does not depend on $\hat x_2$.
\end{proof}
\noindent It is time to state and prove the Theorem that was the purpose of this section.
\begin{theorem}\label{thm_stab_div} Given $\hat{\bu} \in W^{1,1}(\hat{E})$
\begin{IEEEeqnarray}{rCl}
\label{teoremaDiv_1} \norm{(\rku)_1}_{L^{\infty}(\hat{E})} & 
    \lesssim & \|\hat{u}_1\|_{W^{1,1}(\hat{E})} + 
    \|\emph{div}(\hat{u}_1, \hat{u}_2, 0)\|_{L^{1}(\hat{E})} \\ 
\label{teoremaDiv_2} \norm{(\rku)_2}_{L^{\infty}(\hat{E})} & 
    \lesssim & \|\hat{u}_2\|_{W^{1,1}(\hat{E})} + 
    \|\emph{div}(\hat{u}_1, \hat{u}_2, 0)\|_{L^{1}(\hat{E})} \\ 
\label{teoremaDiv_3} \norm{(\rku)_3}_{L^{\infty}(\hat{E})} & 
    \lesssim & \|\hat{u}_3\|_{W^{1,1}(\hat{E})}
\end{IEEEeqnarray}
where the constants in the inequalities depend olny on $\hat{E}$.
\end{theorem}
\begin{proof}[Proof of Theorem~\ref{thm_stab_div}] The proof is based on the
last three Lemmas. By Remark~\ref{density_wpcurl} we can state the estimate for
a smooth field $\hat{\bu}$ and then finish the proof with a density argument.\\[4pt]
Take again $\omega$ as the closure of $\hat{E}$ and 
$\hat\bu\in\mathcal{C}^\infty(\omega)^3$. Let
us prove the first inequality. Set
$\hat{\bv} = (\hat{u}_1, \hat{u}_2 - \hat{u}_2(\hat{x}_1,0,\hat{x}_3), 0)$.
Then $(\hat{\br}_k\bv)_1 = (\rku)_1$.
By evaluating the degrees of freedom to $\hat{\bv}$ we observe
that some of them vanish or depend exclusively on $\hat{u}_1$ in the way
we need them to depend on $\hat{u}_1$. As for the others,
pick first 
$\hat{q}_0 \in \mathcal{P}_{\hat{f}_5} = 
P_{k-1}(\hat{x}_1)\otimes P_{k-1}(\hat{x}_3)$
and extend it as the same polynomial  
$\hat{q}$ to $Q_{k-1,k-1,k-1}$.
\begin{IEEEeqnarray*}{rCl}
  \rho_{\hat{f}_5,\,\hat{q}_0} (\hat{\bv})
  & = & \int\limits_{\hat{f}_5} \hat{u}_1\,\hat{q}_0\,d\hat{\gamma} + 
  \sqrt{2}\iint\limits_{[0,1]^2} \hat{q}_0(\hat{x}_1,\hat{x}_3)
  \int_0^{1-\hat{x}_1}\frac{\partial \hat{v}_2}{\partial\hat{x}_2}
    (\hat{x}_1,\hat{t},\hat{x}_3)\,d\hat{t}d\hat{x}_1d\hat{x}_3\\[5pt]    
  & = & \int\limits_{\hat{f}_5} \hat{u}_1\,\hat{q}_0\,d\hat{\gamma} + 
  \sqrt{2}\int\limits_{\hat{E}} \hat{q}\frac{\partial \hat{v}_2}{\partial\hat{x}_2}\,d\hat{\bx}\\[5pt]    
  & = & \int\limits_{\hat{f}_5} \hat{u}_1\,\hat{q}_0\,d\hat{\gamma} + 
  \sqrt{2}\int\limits_{\hat{E}} \hat{q}\,\{\,\mbox{div} (\hat{u}_1,\hat{u}_2,0)' -
    \frac{\partial \hat{u}_1}{\partial\hat{x}_1}\,\}\,d\hat{\bx}.
\end{IEEEeqnarray*}
For the volume degrees of freedom~(\ref{momentos3hdiv}) take $\hat{q}_2
\in P_{k-2}(\hat{f}_3) \otimes P_{k-1}(\hat{x}_3)$. Write 
$\hat{v}_2\xyz = \int_0^{\hat{x}_2} 
\nicefrac{\partial\hat{u}_2}{\partial\hat{x}_2}(\hat{x}_1,\hat{t},\hat{x}_3)\,d\hat{t}$ and do
{\color{brown}\#\#\#\#\#\#\#\# partir este array para pasar de p'agina.}
\begin{IEEEeqnarray*}{rCl}
  \int\limits_{\hat{E}} \hat{v}_2\,\hat{q}_2 
  & = &\int\limits_0^1\int\limits_0^1\int\limits_0^{1-\hat{x}_1}
  \int\limits_0^{\hat{x}_2}
    \frac{\partial \hat{u}_2}{\partial \hat{x}_2} 
    (\hat{x}_1,\hat{t},\hat{x}_3)\,d\hat{t}\,\hat{q}_2\xyz\,d\hat{x}_2\,d\hat{x}_1\,d\hat{x}_3\\
  & = &\int\limits_0^1\int\limits_0^1\int\limits_0^{1-\hat{x}_1}\int\limits_0^{\hat{x}_2}
        \frac{\partial\hat{u}_2}{\partial\hat{x}_2}(\hat{x}_1,\hat{t},\hat{x}_3)
        \,\hat{q}_2 \xyz\,d\hat{t}\,d\hat{x}_2\,d\hat{x}_1\,d\hat{x}_3\\
  & = &\int\limits_0^1\int\limits_0^1\int\limits_0^{1-\hat{x}_1}\int\limits_{\hat{t}}^{1-\hat{x}_1}
        \frac{\partial\hat{u}_2}{\partial\hat{x}_2}(\hat{x}_1,\hat{t},\hat{x}_3)\,\hat{q}_2\xyz\,
        d\hat{x}_2\,d\hat{t}\,d\hat{x}_1\,d\hat{x}_3\\
  & = &\int\limits_0^1\int\limits_0^1\int\limits_0^{1-\hat{x}_1}
        \frac{\partial \hat{u}_2}{\partial \hat{x}_2}(\hat{x}_1,\hat{t},\hat{x}_3)
        \int\limits_{\hat{t}}^{1-\hat{x}_1}\,\hat{q}_2\xyz\,d\hat{x}_2\,d\hat{t}\,d\hat{x}_1\,d\hat{x}_3\\
  & = &\int\limits_0^1\int\limits_0^1\int\limits_0^{1-\hat{x}_1}
  \frac{\partial\hat{u}_2}{\partial\hat{x}_2}(\hat{x}_1,\hat{t},\hat{x}_3)\,
       \hat{\phi} (\hat{x}_1,\hat{t},\hat{x}_3)\,d\hat{t}\,d\hat{x}_1\,d\hat{x}_3\\
& = &\int\limits_{\hat{E}} \mbox{div} (\hat{u}_1,\hat{u}_2,0)'\,\hat{\phi}\,d\hat{\bx}
    - \int\limits_{\hat{E}}\frac{\partial\hat{u}_1}{\partial\hat{x}_1}\,\hat{\phi}\,d\hat{\bx}
\end{IEEEeqnarray*}
(for some $\hat{\phi} \in  P_{k-1}(\hat{f}_3)\otimes P_{k-1}(\hat{x}_3)$),
which is what we needed. The inequality~(\ref{teoremaDiv_2}) is proved in the same way.
For inequality~(\ref{teoremaDiv_3})
\begin{IEEEeqnarray*}{rCl}
  (\rku)_3 & = & (\hat{\br}_k(0,0,\hat{u}_3)')_3\\
  & = & \sum_{i=3,4\,\hat{\bq}\,\in\,{\color{red}\mathcal{B}_{\hat f_i}}}
  \int\limits_{\hat f_i} \hat{u}_3 \hat{q}_3\,d\hat{\gamma} \,(\hat{\bv}_{\hat{f}_i,\hat{\bq}})_3
    +\sum_{\hat{\br}\,\in\,{\color{red}\mathcal{B}_{\hat E}}}
  \int\limits_{\hat E} \hat{u}_3 \hat{r}_3\,d\hat{\bx}\,(\hat{\bv}_{\hat{\br}})_3.
\end{IEEEeqnarray*}
Then, by standard result for traces in Sobolev spaces,
\begin{IEEEeqnarray*}{rCl}
  \|(\rku)_3\|_{L^\infty(\hat{E})} 
  & \leqslant & c(\hat{E}) \left\{
   \sum_{i=3,4}
     \int\limits_{\hat f_i} |\hat{u}_3|\,d\hat{\gamma}
   + \int\limits_{\hat E} |\hat{u}_3|\,d\hat{\bx}
  \right\}\\
  &\leqslant& c(\hat{E}) (\|\hat{u}_3|_{\partial\hat{E}}\|_{L^1(\partial\hat{E})} + 
    \|\hat{u}_3\|_{L^1(\hat{E})})\\
  &\leqslant& c(\hat{E}) \|\hat{u}_3\|_{W^{1,1}(\hat{E})}
\end{IEEEeqnarray*}
\end{proof}
Theorems~\ref{thm_stab_edge} and~\ref{thm_stab_div} show that the interpolations
determined by the Finite Elements~(\ref{edgeelement}) and~(\ref{defi_h_div_conforme})
are anisotropically stable, in the sense that the image of a field $\hat\bu$ under
the linear operator depends not only continously on $\hat\bu$, but also with a
\emph{componentwise} bound, with perhaps and additional
curl or divergence term, respectively.\\

\noindent Consider again the element $\tilde{E}$ defined in~(\ref{tilde_prism}).
\begin{theorem} \label{thmStabilityKtildeRT}
There is $C > 0$, independent of $h_1$, $h_2$ and $h_3$, s.t. for all $p \geqslant 1$ and 
  $\tilde{\bu}\in W^{1,p}(\tilde{E})$
  \begin{IEEEeqnarray*}{rCl}
    \left\| \rkutilde \right\|_{L^p(\tilde{E})}
    & \leqslant & C \left( \left\| \tilde{\bu} \right\|_{L^p(\tilde{E})}
    + \sum_{i=1}^3 h_i \left\| \frac{\partial\tilde{\bu}}{\partial\tilde{x}_i} \right\|_{L^p(\tilde{E})}
    + \max\{h_1,h_2\}\left\|{\dv}(\tilde{u}_1, \tilde{u}_2, 0) \,\right\|_{L^p(\tilde{E})}\right).
  \end{IEEEeqnarray*}
\end{theorem}
\begin{remark}
{\color{red}esto va en el cap. de approx tal vez, decidir, porque esta dos veces
esto mismo.}  When it comes to estimate in terms of the data $f$ we
  will assume $h_3 \geqslant C\max\{h_1,h_2\}$ to be able to do
  \begin{IEEEeqnarray*}{rCl}
    \left\| \rkutilde \right\|_{L^p(\tilde{E})}
    & \leqslant & C \left( \left\| \tilde{\bu} \right\|_{L^p(\tilde{E})}
    + \sum_{i=1}^3 h_i \left\| \frac{\partial\tilde{\bu}}{\partial\tilde{x}_i} \right\|_{L^p(\tilde{E})}
    + \max\{h_1,h_2\}\left\|{\dv}\hat\bu\right\|_{L^p(\tilde{E})}
    + \max\{h_1,h_2\}\left\|\frac{\partial\hat u_3}{\partial\hat x_3}\right\|_{L^p(\tilde{E})}\right).\\[5pt]
    &\leqslant& C \left( \left\| \tilde{\bu} \right\|_{L^p(\tilde{E})}
    + \sum_{i=1}^3 h_i \left\| \frac{\partial\tilde{\bu}}{\partial\tilde{x}_i} \right\|_{L^p(\tilde{E})}
    + \max\{h_1,h_2\}\left\|{\dv}\hat\bu\right\|_{L^p(\tilde{E})}\right).
  \end{IEEEeqnarray*}
  this is not a restriction since we needed the prisms to be 
  elongated exactly along the direction paralell to the cuadrilateral faces.
\end{remark}
\begin{proof}
Pick $p\geqslant 1$. If we pull 
$\tilde{\bu}$ back to $\hat{E}$ we get the relation
\begin{IEEEeqnarray}{rCl}\label{pull1}
  \hat{\bu}(\hat{\bx}) & = & (det\,DF)DF^{-1}\tilde{\bu}(F\hat{\bx})
\end{IEEEeqnarray}
\begin{IEEEeqnarray}{rCl}
  D\hat{\bu}(\hat{\bx}) & = & \diag{h_2\,h_3}{h_1\,h_3}{h_1\,h_2}\cdot
  D\tilde{\bu}(F\hat{\bx})\cdot\diag{h_1}{h_2}{h_3}
\end{IEEEeqnarray}
and by~(\ref{push-forward})
\begin{IEEEeqnarray}{rCl}\label{pull2}
  (\det DF)DF^{-1}\rkutilde(F(\hat{\bx})) & = & \rku(\hat{\bx}).
\end{IEEEeqnarray}
With~(\ref{pull1})~(\ref{pull2}) and 
stability inequality~(\ref{teoremaDiv_1}) 
plus H\"older's inequality we obtain 
\begin{IEEEeqnarray*}{rCl}
  \|(\rkutilde)_1\|_{L^{\infty}(\tilde{E})} & = &
  (h_2\,h_3)^{-1}
  \|(\rku)_1\|_{L^{\infty}(\hat{E})}\\[7pt]
  &\leqslant&(h_2\,h_3)^{-1}\left(\|\hat{u}_1\|_{W^{1,1}(\hat{E})}
  +\|\dvg(\hat{u}_1,\hat{u}_2,0)'\|_{L^{1}(\hat{E})}\right)\\[7pt]
  &=& 
  (h_2\,h_3)^{-1}
  \left(
    \int\limits_{\hat{E}}\left|\hat{u}_1\right|\,d\hat{\bx}
    +\sum_{i=1}^3\int\limits_{\hat{E}}\left|\frac{\partial\hat{u}_1}{\partial\hat{x}_i}\right|\,d\hat{\bx}
    +\int\limits_{\hat{E}}\left|\frac{\partial\hat{u}_1}{\partial\hat{x}_1} + \frac{\partial\hat{u}_2}{\partial\hat{x}_2}\right|
    \,d\hat{\bx}
  \right)\\[7pt]
  &=&(\det DF)^{-1}\left(
  \int\limits_{\tilde{E}}|\tilde{u}_1|\,d\tilde{\bx}
  +\sum_{i=1}^3h_i\int\limits_{\tilde{E}}|\frac{\partial\tilde{u}_1}{\partial\tilde{x}_i}|\,d\tilde{\textbf{x}}
  +h_1\int\limits_{\tilde{E}}|\frac{\partial\tilde{u}_1}{\partial\tilde{x}_1} + \frac{\partial\tilde{u}_2}{\partial\tilde{x}_2}|
  \,d\tilde{\textbf{x}}\right)\\[7pt]
  &=&(2|\tilde{E}|)^{-1}\left(
  \|\tilde{u}_1\|_{L^1(\tilde{E})}
  +\sum_{i=1}^3h_i\|\frac{\partial\tilde{u}_1}{\partial\tilde{x}_i}\|_{L^1(\tilde{E})}
  +h_1\|\dvg(\tilde{u}_1,\tilde{u}_2,0)\|_{L^1(\tilde{E})}\right)\\[7pt]
  &\leqslant&(2|\tilde{E}|^{\nicefrac{1}{p}})^{-1}\left(
  \|\tilde{u}_1\|_{L^p(\tilde{E})}
  +\sum_{i=1}^3h_i\|\frac{\partial\tilde{u}_1}{\partial\tilde{x}_i}\|_{L^p(\tilde{E})}
  +h_1\|\dvg(\tilde{u}_1,\tilde{u}_2,0)\|_{L^p(\tilde{E})}\right).
\end{IEEEeqnarray*}
Now
\begin{IEEEeqnarray*}{rCl}
  \|(\rkutilde)_1\|_{L^{p}(\tilde{E})}
  &\leqslant&
  |\tilde{E}|^{1/p}\|(\tilde{\br}_k\tilde{\boldsymbol{u}})_1\|_{L^{\infty}(\tilde{E})}\\
  &\lesssim&\|\tilde{u}_1\|_{L^p(\tilde{E})}
  +\sum_{i=1}^3h_i\|\frac{\partial\tilde{u}_1}{\partial\tilde{x}_i}\|_{L^p(\tilde{E})}
  +h_1\|\dvg(\tilde{u}_1,\tilde{u}_2,0)\|_{L^p(\tilde{E})},
\end{IEEEeqnarray*}
and again, the symmetric inequality holds for component $2$. For component $3$,
stability inequality~(\ref{teoremaDiv_3}) gives us
\begin{IEEEeqnarray*}{rCl}
  \|(\rkutilde)_3\|_{L^{\infty}(\tilde{E})} & = &({h_1h_2})^{-1}
  \|(\rku)_3\|_{L^{\infty}(\hat{E})}\\[6pt]
  &\leqslant&{C}({h_1h_2})^{-1}\,\|\hat{u}_3\|_{W^{1,1}(\hat{E})}\\[6pt]
  &=&C\,|\tilde{E}|^{-1}\,\left[\|\tilde{u}_3\|_{L^1(\tilde{E})} +
    \sum_{i=1,2,3} h_i\,\left\|\frac{\partial\tilde{u}_3}{\partial\tilde{x}_i}\right\|_{L^1(\tilde{E})}\right]\\[6pt]
  &\leqslant& {C}\,|\tilde{E}|^{-\nicefrac{1}{p}}\,\left[\|\tilde{u}_3\|_{L^p(\tilde{E})} +
    \sum_{i=1,2,3} h_i\,\left\|\frac{\partial\tilde{u}_3}{\partial\tilde{x}_i}\right\|_{L^p(\tilde{E})}\right]
\end{IEEEeqnarray*}
so, immediately,
\begin{IEEEeqnarray}{rCl} \label{aux_label18}
  \|(\rkutilde)_3\|_{L^{p}(\tilde{E})}
  &\leqslant& C\,\left(
  \|\tilde{u}_3\|_{L^p(\tilde{E})} +
    \sum_{i=1,2,3} h_i\,\left\|\frac{\partial\tilde{u}_3}{\partial\tilde{x}_i}\right\|_{L^p(\tilde{E})}
  \right)
\end{IEEEeqnarray}
and the sum of the three estimates yields the theorem.
\end{proof}
%===============================================================================
% \begin{lemma}\label{L6} Let $P$ be a right prism. There exists a constant $C$ depending only on $\alpha_P$ such that for all $\bu$ in $W^{1,1}(P)$ we have
% \begin{multline}\label{estabL1}
% \|\bu_I\|_{L^1(P)} \leqslant C\Bigg(\|\bu\|_{L^1(P)} + \sum_{i=1}^3 h_{i,P}\|\partial_{\xi_{P,i}}\bu\|_{L^1(P)}\\ + \max\{h_{P,1},h_{P,2}\}\|\mbox{div\,}(u_1,u_2,0)\|_{L^1(P)}\Bigg).
% \end{multline}
% \end{lemma}
% \begin{proof} Using the notation introduced above for the vertices of $P$, suppose that $v_0$ is the vertex with the maximum angle of the triangle $v_0v_1v_2$. Let $\tilde P$ be a prism with vertices at $(0,0,0)$, $(h_{P,1},0,0)$, $(0,h_{P,2},0)$, $(0,0,h_{P,3})$, $(h_{P,1},0,h_{P,3})$ and $(0,h_{P,2},h_{P,3})$. Then by standard rescaling arguments using the Piola Transform we can prove from Lemma \eqref{L5} that there exists a constant $C$ such that for all $\bu\in W^{1,1}(\tilde P)$ we have
% \begin{eqnarray*}
% \|\tilde\bu_I\|_{L^1(\tilde P)}&\le& C\Bigg(\|\tilde\bu\|_{L^1(\tilde P)} + \sum_{i=1}^3h_{P,i}\|\partial_{x_i}\tilde\bu\|_{L^1(\tilde P)}\\&&\qquad + 
% \max\{h_{P,1},h_{P,2}\}\|\dv(\tilde u_1,\tilde u_2,0)\|_{L^1(\tilde P)}\Bigg).
% \end{eqnarray*}
% Let $B$ be the matrix with columns $\xi_{P,1}$, $\xi_{P,2}$ and $\xi_{P,3}$ (note $B$ has the form \eqref{matrix} and $\xi_{P,3}=(0,0,1)$). Then the map $F(\tilde{\bf x})=B\tilde{\bf x}+v_0$ sends $\tilde P$ onto $P$. Then, again by a change of variables, we obtain from the previous estimate, that for all $\bu\in W^{1,1}(P)$ it holds
% \begin{eqnarray*}
% \|\bu_I\|_{L^1(P)}&\le& C\|B\|\|B^{-1}\|\bigg(\|\bu\|_{L^1(P)} + \sum_{i=1}^3h_{P,i}\|\partial_{\xi_{P,i}}\bu\|_{L^1(P)}\\ &&  \qquad +\max\{h_{P,1},h_{P,2}\}\frac1{\|B^{-1}\|}\|\dv(u_1,u_2,0)\|_{L^1(P)}\bigg). 
% \end{eqnarray*}
% Then the proof concludes by noting that $\|B\|\leqslant C$ and $\|B^{-1}\|\sim \sin\alpha_P$.
% \end{proof}
% 
% \begin{remark} Stability estimates in $L^p$-norm, $p>1$, can by proved analogously.
% In particular, from  \eqref{estabL1}, using an inverse inequality on the left hand side, and Cauchy-Schwarz inequality on the right hand side, we obtain under assumptions of Lemma \ref{L6}
% \begin{multline}\label{estabL2}
% \|\bu_I\|_{L^2(P)} \leqslant C\Bigg(\|\bu\|_{L^2(P)} + \sum_{i=1}^3 h_{i,P}\|\partial_{\xi_{P,i}}\bu\|_{L^2(P)}\\ + \max\{h_{P,1},h_{P,2}\}\|\mbox{div\,}(u_1,u_2,0)\|_{L^2(P)}\Bigg)
% \end{multline}
% \end{remark}
%===============================================================================
\begin{corollary}\label{aux_label38}
Under assumptions of Lemma \ref{L6}, and if $h_{P,3}\ge \max\{h_{P,1},h_{P,2}\}$ we obtain 
\begin{equation}\label{estabLp}
\|\br_E \bu\|_{L^p(E)} \leqslant \left(\|\bu\|_{L^p(E)} +
\sum_{i=1}^3 h_{E,i}\|\partial_{\xi_i} \bu\|_{L^p(E)}+ h_E\|\mbox{div\,}\bu\|_{L^p(E)}\right)
\end{equation}
\end{corollary}
\begin{remark}
{\color{red}esto es del corolario aca arriba}  When it comes to estimate in terms of the data $f$ we
  will assume $h_3 \geqslant C\max\{h_1,h_2\}$ to be able to do
  \begin{IEEEeqnarray*}{rCl}
    \left\| \rkutilde \right\|_{L^p(\tilde{E})}
    & \leqslant & C \left( \left\| \tilde{\bu} \right\|_{L^p(\tilde{E})}
    + \sum_{i=1}^3 h_i \left\| \frac{\partial\tilde{\bu}}{\partial\tilde{x}_i} \right\|_{L^p(\tilde{E})}
    + \max\{h_1,h_2\}\left\|{\dv}\hat\bu\right\|_{L^p(\tilde{E})}
    + \max\{h_1,h_2\}\left\|\frac{\partial\hat u_3}{\partial\hat x_3}\right\|_{L^p(\tilde{E})}\right).\\[5pt]
    &\leqslant& C \left( \left\| \tilde{\bu} \right\|_{L^p(\tilde{E})}
    + \sum_{i=1}^3 h_i \left\| \frac{\partial\tilde{\bu}}{\partial\tilde{x}_i} \right\|_{L^p(\tilde{E})}
    + \max\{h_1,h_2\}\left\|{\dv}\hat\bu\right\|_{L^p(\tilde{E})}\right).
  \end{IEEEeqnarray*}
  this is not a restriction since we needed the prisms to be 
  elongated exactly along the direction paralell to the cuadrilateral faces.
\end{remark}

\subsection{Local Interpolation Estimates for Prismatic Elements} % (fold)
\label{sub:local_interpolation_estimates_for_prismatic_elements}
We will state scaling consequences
of inequalities~(\ref{aux_label19}).
%=========
%For a non--negative integer $m$ let $\partial^m f$ denote the sum of the absolute values of all the derivatives of order $m$ of $f$.
%=========
\begin{remark}\label{aux_label28}
Denote by $\Qbb_{m,E}\,\bw$ the vector averaged Taylor polynomial of $\bw$
\[
  \Qbb_{m,E}\,\bw = (\Qb_{m,E}\,w_1,\Qb_{m,E}\,w_2,\Qb_{m,E}\,w_3)'.
\]
If $(\,\cdot\,)\,\hat{}\,$ denotes any of transformations~(\ref{transfHcurl})
and~(\ref{transfDiv}) then
\begin{IEEEeqnarray*}{rCl}
  \Qbb_{m,\hat{E}}\hat{\bw} & = & (\Qbb_{m,E}\,\bw)\,\hat{}.
\end{IEEEeqnarray*}
\end{remark}
\begin{lemma}\label{aux_label20}
Let $\tilde E$ be the rescaled reference prism in Figure~\ref{rescaled_prism}
and denote its diameter by $h$.
Given $p\geqslant 1$, $m\geqslant 0$ and
$\tilde\bu \in W^{m+1,p}(\tilde E)$,  Then
If $m \geqslant 0$ and $p \geqslant 1$.
\begin{enumerate}
  \item 
  For any component $1\leqslant i\leqslant 3$
  \begin{IEEEeqnarray}{rCl}\label{aux_label30}
    \|\tilde  u_i - \tilde\Qb_{m,\tilde E}\,\tilde u_i \|_{L^p(\tilde E)}& \leqslant &
      C \sum_{j+k+l=m+1} h_1^jh_2^kh_3^l \left\| \frac{\tilde\partial^{m+1}\tilde u_i}
      {\partial\tilde x_1^j\partial\tilde x_2^k\partial\tilde x_3^l}
      \right\|_{L^p(\tilde E)}
  \end{IEEEeqnarray}
  \item 
  For any component $1\leqslant l\leqslant 3$
  \begin{IEEEeqnarray}{rCl}\label{aux_label31}
    \left\|\frac{\partial}{\partial \tilde x_i}(\tilde u_l-\tilde q_l)\right\|_{L^p(\tilde E)}
    &\leqslant&C\sum_{|\alpha| = m} \bh^\alpha 
    \left\|\frac{\partial}{\partial\tilde x_i} (\partial^{\alpha}\tilde u_l)
    \right\|_{L^p(\tilde E)}
  \end{IEEEeqnarray}
  %%%===========================================================================
  % \begin{IEEEeqnarray}{rCl}\label{aux_label31}
  %   \left \|\frac{\partial}{\partial \tilde x_i}(\tilde\bu-\tilde\bq)\right\|_{L^p(\tilde E)}
  %   &\leqslant&C\sum_{|\alpha| = m} \bh^\alpha 
  %   \left \|\frac{\partial}{\partial\tilde x_i} (\partial^{\alpha}\tilde\bu)
  %   \right\|_{L^p(\tilde E)}
  % \end{IEEEeqnarray} 
  %%%===========================================================================
  \item 
  If $m \geqslant 1$ and $p \geqslant 1$
  %%=============================================================================
  %\begin{IEEEeqnarray}{rCl}
  %  \label{aux_label24__}
  %  \|\curl(\tilde\bu-\tilde\bq)\|_{L^p(\tilde E)}&\leqslant&
  %  C\,h^m
  %  \|\partial^m\curl\tilde\bu\|_{L^p(\tilde E)}\\[5pt]
  %  \label{aux_label25__}
  %  \|\frac{\partial}{\partial_{\tilde x_i}}\curl(\tilde\bu-\tilde\bq)\|_{L^p(\tilde E)}&\leqslant&
  %  C\,h^{m-1}
  %  \|\partial^m\curl\tilde\bu\|_{L^p(\tilde E)}
  %\end{IEEEeqnarray}
  %%=============================================================================
  \begin{IEEEeqnarray}{rCl}
    \label{aux_label24}
    \|\curl(\tilde\bu-\tilde\bq)_i\|_{L^p(\tilde E)}&\leqslant&
    C\,\sum_{j+k+l=m}  h_1^jh_2^kh_3^l
    \left\|\frac{\tilde\partial^m(\curl\tilde\bu)_i}{\partial\tilde x_1^j\partial\tilde x_2^k\partial\tilde x_3^l}
    \right\|_{L^p(\tilde E)}
  \end{IEEEeqnarray}
  \item 
  \begin{IEEEeqnarray}{rCl}
    \label{aux_label25}
    \|\frac{\partial\curl(\tilde\bu-\tilde\bq)_i}{\partial\tilde x_j}\|_{L^p(\tilde E)}
      &\leqslant& 
        C\,\sum_{j+k+l=m-1}  h_1^jh_2^kh_3^l
          \|\frac{\tilde\partial^{m-1}\tilde\partial(\tilde\curl\tilde\bu)_i}
                 {\partial\tilde x_1^j\partial\tilde x_2^k\partial\tilde x_3^l\partial\tilde x_j}
          \|_{L^p(\tilde E)}
  \end{IEEEeqnarray}
\end{enumerate}
where C depends only on $m$, $\sigma$ (cfr. Theorem~\ref{aux_label21})
and the reference element.
\end{lemma}
{\color{BrickRed} REFINAR esta de manera anisotropa y
despues el teorema correspondiente.
As a corollary of these inequalities we have immediately that
let $\tilde \bq\in P_m(\tilde E)^3$ be the
vector averaged Taylor polynomial 
of $\tilde \bu$ as in  Remark~\ref{aux_label28}.
\begin{IEEEeqnarray}{rCl}\label{aux_label23}
  \|\tilde{\dv}(\tilde\bu-\tilde\bq)\|_{L^p(\tilde E)}&\leqslant&Ch^m\|\tilde \partial^m\tilde\dv\tilde\bu\|_{L^p(\tilde E)}
\end{IEEEeqnarray}
where $C$ depends only on $m$ and \noindent{\color{BrickRed} ????}.}
\begin{proof}[Proof of Lemma~\ref{aux_label20}]
  Lemma~\ref{aux_label40} rescaling $\hat{E}$ so that $d = 1$ and then
  pushing forward to $\tilde E$. In fact, for any $\alpha$ 
  \begin{IEEEeqnarray}{rCl}\label{aux_label41}
    h_1^{\alpha_1}h_2^{\alpha_2}h_3^{\alpha_3}\tilde\partial^\alpha\tilde u_i (\tilde\bx)
    & = & (\nicefrac{1}{h_i}\hat\partial^{\alpha}\hat u_i)(F_{\tilde E}^{-1}\tilde\bx)
  \end{IEEEeqnarray}
  so
  \begin{IEEEeqnarray}{rCl}\label{aux_label30}
  \|\tilde  u_i - \tilde\Qb_{m,\tilde E}\,\tilde u_i \|^p_{L^p(\tilde E)}
  & = & \|\frac{1}{h_i}(u_i - \Qb_{m, E}\,u_i)\circ F_{\tilde E}^{-1}\|^p_{L^p(\tilde E)} \\
  & = & \frac{|\det M_E|}{h_i^p} \|  u_i - \Qb_{m, E}\, u_i\|^p_{L^p(\hat E)} \\
  & \leqslant & \frac{C|\det M_E|}{h_i^p} |  u_i|^p_{m+1,p,\hat E} \\
  & = & C|\det M_E| \sum_{j+k+l=m+1} \left\|
   \frac{\partial^{m+1}}{\partial^jx_1\partial^kx_2\partial^lx_3} ( \frac{1}{h_i} u_i)\right\|^p_{L^p(\hat{E})} \\
  (\mbox{by~(\ref{aux_label41}) })& = & C \sum_{j+k+l=m+1} \left\|
   h_1^jh_2^kh_3^l
   \frac{\tilde\partial^{m+1}}{\partial^j\tilde x_1\partial^k\tilde x_2\partial^l\tilde x_3}
   (\tilde u_i)\right\|^p_{L^p(\tilde{E})} \\
  & = & C \sum_{j+k+l=m+1} (h_1^jh_2^kh_3^l)^p \left\| \frac{\tilde\partial^{m+1}\tilde u_i}
    {\partial\tilde x_1^j\partial\tilde x_2^k\partial\tilde x_3^l}
    \right\|^p_{L^p(\tilde E)}.
\end{IEEEeqnarray}
To prove~(\ref{aux_label24}), by a straightforward manipulation we have
\begin{IEEEeqnarray*}{rCl}
  \tilde\curl \tilde \Qbb_{m,\tilde E}\tilde\bu & = & 
  \tilde \Qbb_{m-1,\tilde E} \tilde\curl \tilde\bu 
\end{IEEEeqnarray*}
then, componentwise, by~(\ref{aux_label30}), 
\begin{IEEEeqnarray*}{rCl}
  \|\tilde\curl(\tilde\bu-\tilde \Qbb_{m,\tilde E}\tilde\bu)_i\|_{L^p(\tilde E)}& = &
    \|(\tilde\curl\tilde\bu)_i - \tilde\Qb_{m-1}(\tilde\curl\tilde\bu)_i\|_{L^p(\tilde E)}\\
  &\leqslant& C \sum_{j+k+l=m} h_1^jh_2^kh_3^l 
  \|\frac{\tilde\partial^m(\tilde\curl\tilde\bu)_3}
         {\partial\tilde x_1^j\partial\tilde x_2^k\partial\tilde x_3^l}\|_{L^p(\tilde E)}
\end{IEEEeqnarray*}
Now~(\ref{aux_label25}) is an easy consecuence because
\begin{IEEEeqnarray*}{rCl}
\|\frac{\partial\curl(\tilde\bu-\tilde\bq)_i}{\partial\tilde x_j}\|_{L^p(\tilde E)}
    & = & 
\|\frac{\partial(\tilde\curl\tilde\bu)_i}{\partial\tilde x_j} - 
  \frac{\partial\tilde\Qb_{m-1,\tilde E}(\tilde\curl\tilde\bu)_i}{\partial\tilde x_j}
\|_{L^p(\tilde E)} \\
& = & 
\|\frac{\partial(\tilde\curl\tilde\bu)_i}{\partial\tilde x_j} - 
  \tilde\Qb_{m-2,\tilde E}\frac{\partial(\tilde\curl\tilde\bu)_i}{\partial\tilde x_j}
\|_{L^p(\tilde E)} \\
    &\leqslant& 
      C\,\sum_{j+k+l=m-1}  h_1^jh_2^kh_3^l
        \|\frac{\tilde\partial^{m-1}\tilde\partial(\tilde\curl\tilde\bu)_i}
               {\partial\tilde x_1^j\partial\tilde x_2^k\partial\tilde x_3^l\partial\tilde x_j}
        \|_{L^p(\tilde E)}
\end{IEEEeqnarray*}
\end{proof}
======================================================\\
One of the main results in this thesis. 
Sean un prisma obl\'{\i}cuo irregular $E$, $k\in\mathbb{N}$, el operador de interpolaci\'on 
$\bw_E$ de grado $k$ determinado por el elemento en la Definición~(\ref{edgeelement}), y
$p>2$. Let $h_E$ denote the diameter of $E$.
such that (\noindent{\color{BrickRed}hip. prisma}).
Let $\xi_i$, $1\leqslant i \leqslant 3$ be unitary vectors with the directions
of the three edges sharing a vertex $\bx_E$ of $E$, whose lengths are
$h_i$, $1\leqslant i\leqslant 3$.\\
As in Subsection~\ref{sub:functional_spaces_trace_spaces}, for $\bh = (h_1,h_2,h_3)'$
let $\bh^\alpha = h_1^{\alpha_1}h_2^{\alpha_2}h_3^{\alpha_3}.$
and $\partial^\alpha = \frac{\partial^{|\alpha|}}{\partial\xi_1^{\alpha_1}\xi_2^{\alpha_2}\xi_3^{\alpha_3}}.$\\
======================================================
\begin{theorem} \label{aux_label32} Let $p>2$ and let $E$ be a prism whose triangular
faces have >least/greatest? angle >greater/less? than $c_0$.
there exists $C > 0$ and three edges $\ell_i$ of $E$
 such that for all $\bu\in W^{m + 1,p}(E)^3$
and $m\leqslant k-1$, %with $\bcurl \bu\in W^{m,p}(E)^3$
\begin{IEEEeqnarray*}{rCl}\label{aux_label55}
  \|\bu-\bw_E \bu\|_{L^p(E)} & \leqslant & C
  \left(
    \sum_{|\alpha|=m+1}\bh^\alpha \|\partial^\alpha \bu\|_{L^p(E)} +
    h\sum_{|\alpha|=m}\bh^\alpha\|\partial^\alpha 
    (\curl \bu)_3 \|_{L^p(E)}
  \right).
\end{IEEEeqnarray*} 
$C$ depends only on $c_0$.
$C$ can be chosen so that, if $M_E$ is the matrix made with
$\xi_i$ as columns, then $\|M\|_\infty\leqslant C$ and $\|M^{-1}\|_\infty\leqslant C$ 
and $\det M_E \geqslant C$
\end{theorem}
Notice the anisotropic character of the inequality. Notice only the component
of the $\curl$ corresponding to the direction that is orthogonal to the 
triangular faces.
%[Proof of Theorem~\ref{aux_label32}]
\begin{proof}[Proof of Theorem~\ref{aux_label32}]
{\color{blue}\#\#\#\#\#\#\#\# Ariel, por favor mirar si es correcta
la manera en que lo digo.}\\\\
Since $W^{m+1,p}(E)\hookrightarrow W^{1,p}(E)$ and $p$ is greater than $2$,
the edge interpolator $\bw_E$ is well--defined via corollary~\ref{aux_label26}.
  Consider the prism $\tilde E$ as in~(\ref{tilde_prism}). There is an affine map
  $\tilde \bx \mapsto \bx = M_E\tilde\bx+\bx_E$ from $\tilde E$ onto $E$, such that 
  $\|M\|$, $\|\|$
  The matrix $M_E$ is made up of vectors $\xi_i$, $i = 1$, $2$, $3$, as its columns,
  where $\xi_i$ are the unitary vectors in the directions of three edges $\ell_i$
   of E of lengths $h_i$ sharing the vertex $\bx_E$.
\begin{IEEEeqnarray*}{rCl}
  \|\bu-\bw_E\bu\|_{L^p(E)} & \leqslant & \|\bu-\bq\|_{L^p(E)}
    +\|\bw_E(\bu-\bq)\|_{L^p(E)}
\end{IEEEeqnarray*}
For the first term we may simply do, by Remark~\ref{aux_label28} and the
transformation~(\ref{transfHcurl}),
\begin{IEEEeqnarray}{rCl}
\nonumber
  \|\bu-\bq\|_{L^p(E)} & = & \|M_E^{-t}(\tilde{\bu}-\tilde{\bq})\circ F_E^{-1}\|_{L^p(E)} \\[5pt]
\label{aux_label37}
  &\leqslant& \|M^{-1}\||\det M_E|^{\nicefrac1p}\|\tilde{\bu}-\tilde{\bq}\|_{L^p(\tilde E)}.
\end{IEEEeqnarray}
With regard to the second term, by the commutativity property~(\ref{piTransformado})
and again the coordinate transformation,
\begin{IEEEeqnarray*}{rCl}
  \|\bw_E(\bu-\bq)\|_{L^p(E)}&\leqslant&
    |M|^{\nicefrac1p}\|M_E^{-1}\|  
      \|\tilde{\bw}_{\tilde E}(\tilde\bu-\tilde\bq)\|_{L^p(\tilde E)}
\end{IEEEeqnarray*}
Theorem~\ref{aux_label27} implies
\begin{IEEEeqnarray*}{rCl}
    \|\bw_E(\bu-\bq)\|_{L^p(E)}
    & \leqslant & C\|M^{-1}\||\det M_E|^{\nicefrac1p}
\left[ \| \tilde\bu-\tilde\bq \|_{L^p(\tilde{E})}
    + \sum_{i=1}^3 h_i \| \partial_{\tilde{x}_i}(\tilde\bu-\tilde\bq) \|_{L^p(\tilde{E})}\right.\\
\yesnumber\label{aux_label34}
    & & \left.
    \:+\;h\left(\left\|(\curl\,(\tilde\bu-\tilde\bq))_3 \right\|_{L^p(\tilde{E})}
     + \sum_{i=1}^3 h_i
     \left\| \partial_{\tilde{x}_i}(\curl\,(\tilde\bu-\tilde\bq))_3 \right\|_{L^p(\tilde{E})}\right)
  \right].
\end{IEEEeqnarray*}
by expressions~(\ref{aux_label30}),~(\ref{aux_label31}),
{\color{BrickRed}
~(\ref{aux_label24}),
~(\ref{aux_label25})  
}the last expressions is bounded by a constant times
$\|M^{-1}\||\det M_E|^{\nicefrac1p}$
times
\begin{IEEEeqnarray}{C}
\nonumber
\sum_{i+j+k=m+1} h_1^ih_2^jh_3^k \left\| \frac{\partial^{m+1}\tilde\bu}
    {\partial\tilde x_1^i\partial\tilde x_2^j\partial\tilde x_3^k}
    \right\|_{L^p(\tilde E)}
+ h \left( \sum_{j+k+l=m}  h_1^jh_2^kh_3^l
  \left\|\frac{\tilde\partial^m(\curl\tilde\bu)_3}{\partial\tilde x_1^j\partial\tilde x_2^k\partial\tilde x_3^l}
  \right\|_{L^p(\tilde E)}
\right.
  \\[5pt]
\nonumber
\left.
+ \sum_{i=1}^3 h_i\sum_{j+k+l=m-1}  h_1^jh_2^kh_3^l
        \|\frac{\tilde\partial^{m-1}\tilde\partial(\tilde\curl\tilde\bu)_3}
               {\partial\tilde x_1^j\partial\tilde x_2^k\partial\tilde x_3^l\partial\tilde x_j}
        \|_{L^p(\tilde E)} \right)
\\[5pt]
\label{aux_label33}
\lesssim
\sum_{i+j+k=m+1} h_1^ih_2^jh_3^k \left\| \frac{\partial^{m+1}\tilde\bu}
    {\partial\tilde x_1^i\partial\tilde x_2^j\partial\tilde x_3^k}
    \right\|_{L^p(\tilde E)}
+ h \left( \sum_{j+k+l=m}  h_1^jh_2^kh_3^l
  \left\|\frac{\tilde\partial^m(\curl\tilde\bu)_3}{\partial\tilde x_1^j\partial\tilde x_2^k\partial\tilde x_3^l}
  \right\|_{L^p(\tilde E)}
\right)
\end{IEEEeqnarray}
From equality~(\ref{aux_label29}), for every $\alpha$ of order
$m+1$ it holds
\begin{IEEEeqnarray}{rCl}\label{aux_label36}
  \|\tilde{\partial}^{\alpha}\tilde\bu\|_{L^p(\tilde{E})} & \leqslant & 
  \|M_E\|\,|\det M_E|^{-\nicefrac1p} \|\partial^{\alpha}\bu\|_{L^p(E)}.
\end{IEEEeqnarray}
{\color{blue}\#\#\#\#\#\#\#\# esto de la matriz realmente hace falta?.}
Lastly, adapting Lemma 3.57 in page 77 of~\cite{monk}, we observe
\begin{IEEEeqnarray*}{rCl}
  \begin{pmatrix}
    0 & -(\tilde\curl\tilde\bu)_3 & 0 \\
    (\tilde\curl\tilde\bu)_3 & 0 & 0 \\
    0 & 0 & 0 
  \end{pmatrix}M_E^{-1}
  & = & M_E^{t}
  \begin{pmatrix}
    0 & -(\curl\bu)_3 & 0 \\
    (\curl\bu)_3 & 0 & 0 \\
    0 & 0 & 0 
  \end{pmatrix}\circ F_E
\end{IEEEeqnarray*}
which implies, for every $\alpha$ of order $m$,
\begin{IEEEeqnarray}{rCl} \label{aux_label35}
  \|\tilde{\partial}^{\alpha}(\tilde{\curl}\tilde\bu)_3\|_{L^p(\tilde E)}
  & \leqslant & C |\det M_E|^{-\nicefrac1p}\,\|M_E\|^{2} 
  \|\partial^{\alpha}(\curl\bu)_3\|_{L^p(E)}
\end{IEEEeqnarray}
{\color{blue}\#\#\#\#\#\#\#\# De donde sale el $(2+m)$ de (6.15) en
~\cite{ariel}?.}
Now combine~(\ref{aux_label33}),~(\ref{aux_label36}) and~(\ref{aux_label35}) 
with~(\ref{aux_label34}) and~(\ref{aux_label37}) to obtain the
Theorem.
\end{proof}
Imitating the former proof and adding the information of~(\ref{aux_label23}),
from Corollary~\ref{aux_label38} we deduce the following Theorem.
\begin{theorem}\label{aux_label46}
 $k\in\mathbb{N}_0, p \geqslant 1:$\quad$\exists$ $C > 0$\quad$\forall$ $\bu\in W^{m + 1,p}(E)$\quad and
\quad$m\leqslant k$, 
\begin{IEEEeqnarray}{C}\label{aux_label39}
  \|\bu-\br_E \bu\|_{L^p(E)} \leqslant C \left(
  \sum_{|\alpha|=m+1}h^\alpha \|D^\alpha\bu\|_{L^p(E)} +
  h_E^{m+1}\|D^m(\text{div} \,\bu)\|_{L^p(E)} \right).
\end{IEEEeqnarray}
\end{theorem}
%
%Take a kernel $\hat{\phi}\in\mathcal{C}_0^\infty(\mathbb{R}^n)$ such that
%\begin{IEEEeqnarray*}{rCl}
  %\supp{\hat{\phi}}&\subseteq&\hat{B}\\[5pt]
  %\int\limits_{\hat{B}} \hat{\phi}\,d\textbf{x} & = & 1
%\end{IEEEeqnarray*} 
%Now define the kernel on B
%\begin{IEEEeqnarray*}{rCl}
  %\phi_B&=&\frac{|\hat{B}|}{|B|}\,\hat{\phi}\circ F^{-1}\\[5pt]
  %&=&\frac{1}{\det DF}\,\hat{\phi}\circ F^{-1}
%\end{IEEEeqnarray*}
%This way we have
%\begin{IEEEeqnarray*}{rClCr}
  %\int\limits_{B}\phi\,d\textbf{x}&=&
  %\int\limits_{\hat{B}}\hat{\phi}\,\,d\hat{\textbf{x}}&=&1\\[5pt]
%\end{IEEEeqnarray*}
%and also
%\begin{IEEEeqnarray*}{rClClCr}
  %\supp{\phi_B}&=&\supp{\hat{\phi}\circ F^{-1}}
  %&=&F(\supp{\hat{\phi}})&=&B.
%\end{IEEEeqnarray*}
%Then we build an averaged Taylor polynomial over $B$ with the kernel
%$\phi_B$ and it holds that
%where 
%%==============================================================================
% \begin{theorem}\label{thmErrorInterpolacionPrismas}
% Let $P$ be a right prism, and consider a local system of coordinates $x_1x_2x_3$
% such that the triangular basis of $P$ are parallel to the $x_1x_2$-coordinate
% plane. Denote by $\xi_{P,1}$ and $\xi_{P,2}$ the versors parallel to the edges
% of the triangular basis of $P$ adjacent to its maximum angle
% $\alpha_P$, $\xi_{P,3}=(0,0,1)$ and $h_{P,i}$ are the lengths of the edges of
% $P$ parallel to $\xi_{P,i}$. We assume that $h_{P,3}>ch_{P,1}$ and
% $h_{P,3}>ch_{P,2}$. Then, there exists a constant $C$ depending only on $c$
% and $\alpha_P$, such that for all $\bu\in H^1(P)$ we have
% \begin{equation}\label{interp}
% \|\bu-\boldsymbol{r}_E\bu\|_{L^2(E)} \leqslant C\left(\sum_{i=1}^3 h_{E,i}
% \|\partial_{\xi_{E,i}}\bu\|_{L^2(E)} + h_T\|\dv\bu\|_{L^2(E)}\right).
% \end{equation}
% \end{theorem}
%%==============================================================================
% subsection local_interpolation_estimates_for_prismatic_elements (end)
% section prismatic_finite_elements (end)