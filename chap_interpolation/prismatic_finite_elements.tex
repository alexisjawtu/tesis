\section{Prismatic Finite Elements} % (fold)
\label{sec:prismatic_finite_elements}
\subsection{An\-iso\-tropic Stability Estimates for $H(\Div)$--Conforming Finite Elements
on Prisms} % (fold)
In the present subsection $\hat\bu$ will be an element
of $W^{1,1}(\hat E)$ in which elements have well defined
normal traces over the faces of $\hat{E}$, another possibility
being, as mentioned in~\cite{monk}, Lemma $5.15$, page $120$, to assume
there is
a positive $\delta$ such that $\hat\bu$ belongs to
$H^{1/2+\delta}(\hat{E})^3$.
For the whole subsection, $\hat\br_{\hat E}$ will be the $k$--th order face 
interpolation operator on the reference
Prism determined by the element of
Definition~\ref{defi_h_div_conforme}.
\label{stability_of_rt_element_in_hat_k}
\begin{lemma}\label{lemmaRT3zero}
$(\rku)_3$ is linearly and univocally determined by $\hat{u}_3$.
\end{lemma}
\begin{proof}
By the unisolvence of the finite element in Definition~\ref{defi_h_div_conforme},
$(\rku)_3$ is determined by the following linear equations.
\begin{IEEEeqnarray}{rCrl}
\label{rku_3_1}
\hat\rho_{\hat f_j,\hat q}\,(\rku) & = & \hat\rho_{\hat f_j,\hat q}\,(\hat{\bu})
  &\quad\mbox{for $j$ = 3, 4 and }\hat q\in P_{k-1}({\hat{f}_j}) \\
\label{rku_3_2}
\hat\rho_{\hat\br}\,(\rku) & = & \hat\rho_{\hat\br}\,(\hat{\bu})
  &  \quad\mbox{for }\hat\br =(0, 0, \hat r_3)\mbox{, }\hat r_3 \in P_{k-1,k-2}.
\end{IEEEeqnarray}
We have $\nicefrac{k(k+1)^2}{2}$ independent equations, which is the 
number of independent coefficients in $(\rku)_3$.
Now set $\hat u_3 = 0$, which makes the right hand side of all the equations in~(\ref{rku_3_1})
and~(\ref{rku_3_2}) equal to zero.
Take $\hat f$ to be either $\hat{f}_3$ or $\hat{f}_4$. Put 
$\hat q = (\rku)_3|_{\hat{f}}$ in~(\ref{rku_3_1}) recalling~(\ref{prismaticSpace}).
It yields
\begin{IEEEeqnarray*}{rCl}
  \iint_{\hat{f}} \{(\rku)_3\}^2\,d\hat{S} & = & 0.
\end{IEEEeqnarray*}
so there is a polinomial $\hat{q}_3\in {P}_{k-1,k-1}$ such that
$(\rku)_3 \xyz = \hat{x}_3(\hat{x}_3-1)\,\hat{q}_3\xyz$
and the Lemma will follow once we apply the degrees of freedom~(\ref{rku_3_2})
with the test polynomial $\hat\br$ set equal to $(0,0,\hat{q}_3)'$. 
\end{proof}
\begin{lemma}\label{lemma_u1_u2} If $\hat\bu\xyz = (0,0, \hat{u}_3\xyz)'$,
then $\rku\xyz = (0,0,\hat{\xi}_3\xyz)'$ for some $\hat{\xi}_3\in
P_{k-1}(\hat{f}_3)\otimes P_k(\hat{x}_3)$.
\end{lemma}
\begin{proof}
First take an explicit expression for
$\rku = (\hat{p}_1, \hat{p}_2, \hat{p}_3)' \in  P_{\hat{E}}$ as 
\begin{IEEEeqnarray*}{rCl}
  \hat{p}_1\xyz & = & \hat{q}_1\xyz + \hat{x}_1\,\hat{h}\xyz\\
  \label{exprPrt}\yesnumber
  \hat{p}_2\xyz & = & \hat{q}_2\xyz + \hat{x}_2\,\hat{h}\xyz\\
  \hat{p}_3\xyz & = & \hat{q}_3\xyz
\end{IEEEeqnarray*}
for unique $\hat{q}_1, \hat{q}_2 \in {P}_{k-1}(\hat{f}_3)
\otimes{P}_{k-1}(\hat{x}_3)$,
$\hat{q}_3 \in {P}_{k-1}(\hat{f}_3)\otimes{P}_{k}(\hat{x}_3)$,
and
$\hat{h} \in \tilde{{P}}_{k-1}(\hat{f}_3)\otimes{P}_{k-1}(\hat{x}_3)$.
Next, take an arbitrary $\hat{q}\in{P}_{k-1}(\hat f_1)\otimes P_{k-1}(\hat z)$.
The trick will be to apply Green's Theorem to the field
$(\hat{p}_1, \hat{p}_2, 0)'$ and the scalar $\hat{q}$. By the surface degrees of
freedom~(\ref{momentos2hdiv}) and the volume degrees of freedom~(\ref{momentos3hdiv}) we have
  \begin{IEEEeqnarray*}{rClCl}
    \int_{\hat{E}} \mbox{div}(\hat{p}_1, \hat{p}_2, 0)'\,\hat{q}\,d\hat{\bx}&=&
    \int_{\partial\hat{E}} (\hat{p}_1, \hat{p}_2, 0)'\cdot\hat\bn\,\hat{q}\,d\hat{S}
    - \int_{\hat{E}} (\hat{p}_1, \hat{p}_2, 0)'\cdot\nabla \hat{q}\,d\hat{\bx}&&\\[5pt]
    & = &
    \int_{\hat{f}_5} (\hat{u}_1 + \hat{u}_2)
    \hat{q}_{|_{\hat{f}_5 = 1}}\,d\hat{S}
    - \int_{\hat{f}_1} \hat{u}_1 \hat{q}_{|_{\hat{f}_1}}\,d\hat{S}&&\\[5pt]
    &&\,- \int_{\hat{f}_2} \hat{u}_2 \hat{q}_{|_{\hat{f}_2}}\,d\hat{S}
    - \int_{\hat{E}} \hat{u}_1
    \tfrac{\partial\hat{q}}{\partial{\hat{x}_1}} 
      + \hat{u}_2\tfrac{\partial\hat{q}}{\partial{\hat{x}_2}}\,d\hat{\bx} & = & 0.
  \end{IEEEeqnarray*}
  And since $\mbox{div}(\hat{p}_1, \hat{p}_2, 0)'$ also belongs to
  $P_{k-1,k-1}$, we just established it vanishes on all $\hat{E}$.\\[3pt]
  Now the nullity of $\mbox{div}(\hat{p}_1, \hat{p}_2, 0)'$ implies at once that
  $\hat{h} \equiv 0$ in expression~(\ref{exprPrt}) (it can be deduced directly
  derivating the polynomials and observing the degrees of the terms; cfr. 
  proof of Lemma~\ref{lemma_PIu2_k_in_N} onwards). 
  This means we may
  assume that $\hat{p}_1$ and $\hat{p}_2$ belong to 
  ${P}_{k-1}(\hat{f}_3)\otimes{P}_{k-1}(\hat{x}_3)$.
  Now it is convenient to use again the degrees of freedom on the
  faces normal to $(-1, 0, 0)'$ and $(0, -1, 0)'$.
  The conditions
  \begin{IEEEeqnarray*}{rCCCll}
    \hat\rho_{\hat f_i\,\hat q}(\rku) & = & \iint_{\hat f_i} \hat{p}_i\hat q\,d\hat S
    & = & 0\qquad\mbox{for all }\hat{q}\in{P}_{k-1}(\hat f_i)\mbox{, \,}&1\leqslant i\leqslant 2
  \end{IEEEeqnarray*}
  ensure that $\hat{x}_i$ divides $\hat{p}_i$ for both $i=1$ and $2$.
%  In other words, there are $\phi$ and $\psi$ in $\pp{k-2}{k-1}$ such that
%  \begin{IEEEeqnarray*}{rCl}
%    \hat{p}_1 & = & \hat{x}\,\phi\\
%    \hat{p}_2 & = & \hat{y}\,\psi.
%  \end{IEEEeqnarray*}
But finally, if we evaluate the degrees of freedom~(\ref{momentos3hdiv}),
we see that  $\hat{p}_1 = (\rku)_1$ and 
$\hat{p}_2 = (\rku)_2$ can be no other that
constantly null over all $\hat{E}$. 
\end{proof}
\begin{lemma}
\begin{itemize}
  \item []
  %\label{piu2_k_in_N}
  \item [(a)] If $\hat\bu(\hat x_1,\hat x_2,\hat x_3) =
  (0, \hat{u}_2(\hat x_1,\hat x_3), 0)'$,
  then $\rku\xyz = (0, \hat\xi_2(\hat x_1, \hat x_3) ,0)'$ for some 
  $\hat\xi_2 \in P_{k-1}(\hat{x}_2) \otimes P_k(\hat{x}_3)$.
  \item [(b)] If $\hat\bu(\hat x_1,\hat x_2,\hat x_3) = 
  (\hat{u}_1(\hat x_2,\hat x_3), 0, 0)'$
  then $\rku\xyz = (\hat\xi_1(\hat x_2,\hat x_3), 0 ,0)'$ for some
    $\hat\xi_1\in P_{k-1}(\hat{x}_1) \otimes P_k(\hat{x}_3)$.
\end{itemize}
\end{lemma}
\begin{proof} Let us prove the first one of the two claims. The second one 
  has an analogous proof. By Lemma~\ref{lemmaRT3zero} we get
  $(\rku)_3 = 0$.
  The nullity of $\dv\hat{\bu}$ and the commutative
  diagram property~(\ref{div_commutativity}) give us
  $\dv\rku = 0$.
  This last fact, together with the result of the evaluation of the 
  degrees of freedom~(\ref{momentos2hdiv})
  on the face $\hat f_1 \subseteq \{x_1=0\}$
  and~(\ref{momentos3hdiv}) over $\hat E$, implies, as we have seen in
  Lemma~\ref{lemma_u1_u2}, that $(\rku)_1 = 0$.
  And if we look again at 
  $\dv\rku = {\partial\rku}/{\partial\hat x_2} = 0$
  we have that $(\rku)_2$ does not depend on $\hat x_2$.
\end{proof}
\noindent It is time to state and prove the Theorem that was the purpose of this section.
\begin{theorem}\label{thm_stab_div} Given $\hat{\bu} \in W^{1,1}(\hat{E})$
\begin{IEEEeqnarray}{rCl}
\label{teoremaDiv_1} \norm{(\rku)_1}_{L^{\infty}(\hat{E})} & 
    \lesssim & \|\hat{u}_1\|_{W^{1,1}(\hat{E})} + 
    \|\emph{div}(\hat{u}_1, \hat{u}_2, 0)\|_{L^{1}(\hat{E})} \\ 
\label{teoremaDiv_2} \norm{(\rku)_2}_{L^{\infty}(\hat{E})} & 
    \lesssim & \|\hat{u}_2\|_{W^{1,1}(\hat{E})} + 
    \|\emph{div}(\hat{u}_1, \hat{u}_2, 0)\|_{L^{1}(\hat{E})} \\ 
\label{teoremaDiv_3} \norm{(\rku)_3}_{L^{\infty}(\hat{E})} & 
    \lesssim & \|\hat{u}_3\|_{W^{1,1}(\hat{E})}
\end{IEEEeqnarray}
where the constants in the inequalities depend olny on $\hat{E}$.
\end{theorem}
\begin{proof}      %%[Proof of Theorem~\ref{thm_stab_div}]
The proof is based on the
last three Lemmas. By Proposition~\ref{density_wpcurl} we can state the estimate for
a smooth field $\hat{\bu}$ and then finish the proof with a density argument.\\[4pt]
Take %again $\omega$ as the closure of $\hat{E}$ and 
$\hat\bu\in\pazocal{C}^\infty(\bar{\hat{E}})^3$. For 
the first inequality. Set
$\hat{\bv} = (\hat{u}_1, \hat{u}_2 - \hat{u}_2(\hat{x}_1,0,\hat{x}_3), 0)$.
Then $({\br}_{\hat E}\hat\bv)_1 = (\rku)_1$.
By evaluating the degrees of freedom of $\hat{\bv}$ we observe
that some of them vanish or depend exclusively on $\hat{u}_1$ in the way
we need them to depend on $\hat{u}_1$. As for the others,
pick first                                                %{P}_{\hat{f}_5} = 
$\hat{q}_0 \in
P_{k-1}(\hat{x}_1)\otimes P_{k-1}(\hat{x}_3)$
and extend it as the same polynomial  
$\hat{q}$ to $Q_{k-1,k-1,k-1}$.
\begin{IEEEeqnarray*}{rCl}
  \hat\rho_{\hat{f}_5,\,\hat{q}_0} (\hat{\bv})
  & = & \iint_{\hat{f}_5} \hat{u}_1\,\hat{q}_0\,d\hat{S} + 
  \sqrt{2}\iint_{[0,1]^2} \hat{q}_0(\hat{x}_1,\hat{x}_3)
  \int_0^{1-\hat{x}_1}\tfrac{\partial \hat{v}_2}{\partial\hat{x}_2}
    (\hat{x}_1,\hat{t},\hat{x}_3)\,d\hat{t}d\hat{x}_1d\hat{x}_3\\[5pt]    
  & = & \iint_{\hat{f}_5} \hat{u}_1\,\hat{q}_0\,d\hat{S} + 
  \sqrt{2}\int_{\hat{E}} \hat{q}\tfrac{\partial \hat{v}_2}{\partial\hat{x}_2}\,d\hat{\bx}\\[5pt]    
  & = & \iint_{\hat{f}_5} \hat{u}_1\,\hat{q}_0\,d\hat{S} + 
  \sqrt{2}\int_{\hat{E}} \hat{q}\,\{\,\mbox{div} (\hat{u}_1,\hat{u}_2,0)' -
    \tfrac{\partial \hat{u}_1}{\partial\hat{x}_1}\,\}\,d\hat{\bx}.
\end{IEEEeqnarray*}
For the volume degrees of freedom~(\ref{momentos3hdiv}) take $\hat{q}_2
\in P_{k-2,k-1}$. Write 
\[
\hat{v}_2\xyz = \int_0^{\hat{x}_2} 
\tfrac{\partial\hat{u}_2}{\partial\hat{x}_2}(\hat{x}_1,\hat{t},\hat{x}_3)\,d\hat{t}
\]
and do
\begin{IEEEeqnarray*}{rCl}
  \int_{\hat{E}} \hat{v}_2\,\hat{q}_2 
  & = &\int\limits_0^1\int\limits_0^1\int\limits_0^{1-\hat{x}_1}
  \int\limits_0^{\hat{x}_2}
    \tfrac{\partial \hat{u}_2}{\partial \hat{x}_2} 
    (\hat{x}_1,\hat{t},\hat{x}_3)\,d\hat{t}\,\hat{q}_2\xyz\,d\hat{x}_2\,d\hat{x}_1\,d\hat{x}_3\\
  & = &\int\limits_0^1\int\limits_0^1\int\limits_0^{1-\hat{x}_1}\int\limits_0^{\hat{x}_2}
        \tfrac{\partial\hat{u}_2}{\partial\hat{x}_2}(\hat{x}_1,\hat{t},\hat{x}_3)
        \,\hat{q}_2 \xyz\,d\hat{t}\,d\hat{x}_2\,d\hat{x}_1\,d\hat{x}_3\\
  & = &\int\limits_0^1\int\limits_0^1\int\limits_0^{1-\hat{x}_1}\int\limits_{\hat{t}}^{1-\hat{x}_1}
        \tfrac{\partial\hat{u}_2}{\partial\hat{x}_2}(\hat{x}_1,\hat{t},\hat{x}_3)\,\hat{q}_2\xyz\,
        d\hat{x}_2\,d\hat{t}\,d\hat{x}_1\,d\hat{x}_3\\
  & = &\int\limits_0^1\int\limits_0^1\int\limits_0^{1-\hat{x}_1}
        \tfrac{\partial \hat{u}_2}{\partial \hat{x}_2}(\hat{x}_1,\hat{t},\hat{x}_3)
        \int\limits_{\hat{t}}^{1-\hat{x}_1}\,\hat{q}_2\xyz\,d\hat{x}_2\,d\hat{t}\,d\hat{x}_1\,d\hat{x}_3\\
  & = &\int\limits_0^1\int\limits_0^1\int\limits_0^{1-\hat{x}_1}
  \tfrac{\partial\hat{u}_2}{\partial\hat{x}_2}(\hat{x}_1,\hat{t},\hat{x}_3)\,
       \hat{\phi} (\hat{x}_1,\hat{t},\hat{x}_3)\,d\hat{t}\,d\hat{x}_1\,d\hat{x}_3\\
& = &\int_{\hat{E}} \mbox{div} (\hat{u}_1,\hat{u}_2,0)'\,\hat{\phi}\,d\hat{\bx}
    - \int_{\hat{E}}\tfrac{\partial\hat{u}_1}{\partial\hat{x}_1}\,\hat{\phi}\,d\hat{\bx}
\end{IEEEeqnarray*}
(for some $\hat{\phi} \in  P_{k-1}(\hat{f}_3)\otimes P_{k-1}(\hat{x}_3)$),
which is what we needed. The inequality~(\ref{teoremaDiv_2}) is proved in the same way.
For inequality~(\ref{teoremaDiv_3}) we can do
\begin{IEEEeqnarray*}{rCl}
  (\rku)_3 & = & (\hat{\br}_k(0,0,\hat{u}_3)')_3\\[4pt]
  & = & \sum_{i=3,4;\,\hat{\bq}}
  \iint_{\hat f_i} \hat{u}_3 \hat{q}_3\,d\hat{S} \,(\hat{\bv}_{\hat{f}_i,\hat{\bq}})_3
    +\sum_{\hat{\br}}
  \int_{\hat E} \hat{u}_3\,\hat{r}_3\,d\hat{\bx}\,(\hat{\bv}_{\hat{\br}})_3.
\end{IEEEeqnarray*}
Then, by standard results for traces in Sobolev spaces,
\begin{IEEEeqnarray*}{rCl}
  \|(\rku)_3\|_{L^\infty(\hat{E})} 
  & \leqslant & C(\hat{E}) \left\{
   \sum_{i=3,4}
     \int\limits_{\hat f_i} |\hat{u}_3|\,d\hat{\gamma}
   + \int\limits_{\hat E} |\hat{u}_3|\,d\hat{\bx}
  \right\}\\
  &\leqslant& C(\hat{E}) (\|\hat{u}_3|_{\partial\hat{E}}\|_{L^1(\partial\hat{E})} + 
    \|\hat{u}_3\|_{L^1(\hat{E})})\\
  &\leqslant& C(\hat{E}) \|\hat{u}_3\|_{W^{1,1}(\hat{E})}.
\end{IEEEeqnarray*}
\end{proof}
Theorem~\ref{thm_stab_div} shows that the interpolation
determined by the finite element in Definition~\ref{defi_h_div_conforme}
is anisotropically stable, in the sense that the image of a field $\hat\bu$ under
the linear operator depends not only continously on $\hat\bu$, but also with a
\emph{componentwise} bound, with perhaps and additional
divergence term. We refer the reader to Theorem~\ref{thm_stab_edge} onwards to
note the same property for the $\bcurl$--conforming case.\\

The next step is to estimate the stability in 
an anisotropically rescaled prism. 
Given three positive numbers
$h_1$, $h_2$ and $h_3$ we denote
\begin{IEEEeqnarray*}{CCl}
    \yesnumber\label{tilde_prism}
    \tilde{E}   &   =   &   \tilde{T} \times \tilde{I}
\end{IEEEeqnarray*}
where
\begin{IEEEeqnarray*}{rCl}
    \tilde{T}   &   =   &   \{ 0 < \nicefrac{\tilde{x}_1}{h_1} + \nicefrac{\tilde{x}_2  }{h_2} < 1 \}\\
    \tilde{I}   &   =   &   \{ 0 < \nicefrac{\tilde{x}_3}{h_3} < 1 \}.
\end{IEEEeqnarray*}
\rescaledPrismTikz
Of course $\tilde{E} = F(\hat{E})$ where $F$ is the linear
$\mathbb{R}^3 \rightarrow \mathbb{R}^3$ transformation such that
\begin{IEEEeqnarray}{rClCl}
  \label{change_var}
  F\hat{\bx} & = & \diag{h_1}{h_2}{h_3} \hat{\bx} & = & \tilde{\bx}.
\end{IEEEeqnarray}
\begin{theorem} \label{thmStabilityKtildeRT}
There is $C > 0$, independent of $h_1$, $h_2$ and $h_3$, s.t. for all $p \geqslant 1$ and 
  $\tilde{\bu}\in W^{1,p}(\tilde{E})$
  \begin{IEEEeqnarray*}{rCl}
    \left\| \rkutilde \right\|_{L^p(\tilde{E})}
    & \leqslant & C \big( \left\| \tilde{\bu} \right\|_{L^p(\tilde{E})}
    + \sum_{i=1}^3 h_i \| \tfrac{\partial\tilde{\bu}}{\partial\tilde{x}_i} \|_{L^p(\tilde{E})}\\
    &&\qquad+ \max\{h_1,h_2\}\|{\dv}(\tilde{u}_1, \tilde{u}_2, 0) \,\|_{L^p(\tilde{E})}\big).
  \end{IEEEeqnarray*}
\end{theorem}
\begin{remark}\label{auxlabel4}
When it comes to estimate in terms of the data $f$ of a problem we
will assume $h_3 \geqslant C\max\{h_1,h_2\}$ so that 
Theorem~\ref{thmStabilityKtildeRT} implies
  \begin{IEEEeqnarray*}{rCl}
    \left\| \rkutilde \right\|_{L^p(\tilde{E})}
    &\leqslant& C \left( \left\| \tilde{\bu} \right\|_{L^p(\tilde{E})}
    + \sum_{i=1}^3 h_i \| \tfrac{\partial\tilde{\bu}}{\partial\tilde{x}_i} \|_{L^p(\tilde{E})}
    + h_{\tilde E}\left\|{\dv}\tilde\bu\right\|_{L^p(\tilde{E})}\right).
  \end{IEEEeqnarray*}
  This is not a restriction since we needed the prisms to be 
  elongated exactly along the direction paralell to the cuadrilateral faces.
\end{remark}
\begin{proof}[Proof of Theorem~\ref{thmStabilityKtildeRT}]
Pick $p\geqslant 1$. If we pull 
$\tilde{\bu}$ back to $\hat{E}$ we get the relation
\begin{IEEEeqnarray}{rCl}\label{pull1}
  \hat{\bu}(\hat{\bx}) & = & (det\,DF)DF^{-1}\tilde{\bu}(F\hat{\bx})
\end{IEEEeqnarray}
\begin{IEEEeqnarray}{rCl}
  D\hat{\bu}(\hat{\bx}) & = & \diag{h_2\,h_3}{h_1\,h_3}{h_1\,h_2}\cdot
  D\tilde{\bu}(F\hat{\bx})\cdot\diag{h_1}{h_2}{h_3}
\end{IEEEeqnarray}
and by~(\ref{div_interp_commutes})
\begin{IEEEeqnarray}{rCl}\label{pull2}
  (\det DF)DF^{-1}\rkutilde(F(\hat{\bx})) & = & \rku(\hat{\bx}).
\end{IEEEeqnarray}
With expressions~(\ref{pull1}) and~(\ref{pull2}) and 
stability inequality~(\ref{teoremaDiv_1}) 
plus H\"older's inequality we obtain 
\begin{IEEEeqnarray*}{rCl}
  \|(\rkutilde)_1\|_{L^{\infty}(\tilde{E})} & = &
  (h_2\,h_3)^{-1}
  \|(\rku)_1\|_{L^{\infty}(\hat{E})}\\[7pt]
\IEEEeqnarraymulticol{3}{r}{
\begin{IEEEeqnarraybox*}{rcL}
\qquad&\leqslant&(h_2\,h_3)^{-1}\left(\|\hat{u}_1\|_{W^{1,1}(\hat{E})}
  +\|\dvg(\hat{u}_1,\hat{u}_2,0)'\|_{L^{1}(\hat{E})}\right)\\[7pt]
  &=& 
  (h_2\,h_3)^{-1}
  \left(
    \int_{\hat{E}}\left|\hat{u}_1\right|\,d\hat{\bx}
    +\sum_{i=1}^3\int_{\hat{E}}\left|\tfrac{\partial\hat{u}_1}{\partial\hat{x}_i}\right|\,d\hat{\bx}
    +\int_{\hat{E}}\left|\tfrac{\partial\hat{u}_1}{\partial\hat{x}_1} + \tfrac{\partial\hat{u}_2}{\partial\hat{x}_2}\right|
    \,d\hat{\bx}
  \right)\\[7pt]
  &=&(\det DF)^{-1}\left(
  \int_{\tilde{E}}|\tilde{u}_1|\,d\tilde{\bx}
  +\sum_{i=1}^3h_i\int_{\tilde{E}}|\tfrac{\partial\tilde{u}_1}{\partial\tilde{x}_i}|\,d\tilde{\bx}
  +h_1\int_{\tilde{E}}|\tfrac{\partial\tilde{u}_1}{\partial\tilde{x}_1} + \tfrac{\partial\tilde{u}_2}{\partial\tilde{x}_2}|
  \,d\tilde{\bx}\right)\\[7pt]
  &=&(2|\tilde{E}|)^{-1}\left(
  \|\tilde{u}_1\|_{L^1(\tilde{E})}
  +\sum_{i=1}^3h_i\|\tfrac{\partial\tilde{u}_1}{\partial\tilde{x}_i}\|_{L^1(\tilde{E})}
  +h_1\|\dvg(\tilde{u}_1,\tilde{u}_2,0)\|_{L^1(\tilde{E})}\right)\\[7pt]
  &\leqslant&(2|\tilde{E}|^{\nicefrac{1}{p}})^{-1}\left(
  \|\tilde{u}_1\|_{L^p(\tilde{E})}
  +\sum_{i=1}^3h_i\|\tfrac{\partial\tilde{u}_1}{\partial\tilde{x}_i}\|_{L^p(\tilde{E})}
  +h_1\|\dvg(\tilde{u}_1,\tilde{u}_2,0)\|_{L^p(\tilde{E})}\right).\\[8pt]
  \yesnumber\label{auxlabel212}&&
\end{IEEEeqnarraybox*}
}
\end{IEEEeqnarray*}
Now
\begin{IEEEeqnarray*}{rCl}
  \|(\rkutilde)_1\|_{L^{p}(\tilde{E})}
  &\leqslant&
  |\tilde{E}|^{1/p}\|(\tilde{\br}_k\tilde{\boldsymbol{u}})_1\|_{L^{\infty}(\tilde{E})}\\
  &\lesssim&\|\tilde{u}_1\|_{L^p(\tilde{E})}
  +\sum_{i=1}^3h_i\|\tfrac{\partial\tilde{u}_1}{\partial\tilde{x}_i}\|_{L^p(\tilde{E})}
  +h_1\|\dvg(\tilde{u}_1,\tilde{u}_2,0)\|_{L^p(\tilde{E})},
\end{IEEEeqnarray*}
and again, the symmetric inequality holds for component two. For component three,
stability inequality~(\ref{teoremaDiv_3}) gives us
\begin{IEEEeqnarray*}{rCl}
  \|(\rkutilde)_3\|_{L^{\infty}(\tilde{E})} & = &({h_1h_2})^{-1}
  \|(\rku)_3\|_{L^{\infty}(\hat{E})}\\[6pt]
  &\leqslant&{C}({h_1h_2})^{-1}\,\|\hat{u}_3\|_{W^{1,1}(\hat{E})}\\[6pt]
  &=&C\,|\tilde{E}|^{-1}\,\left[\|\tilde{u}_3\|_{L^1(\tilde{E})} +
    \sum_{i=1,2,3} h_i\,\|\tfrac{\partial\tilde{u}_3}{\partial\tilde{x}_i}\|_{L^1(\tilde{E})}\right]\\[6pt]
  &\leqslant& {C}\,|\tilde{E}|^{-\nicefrac{1}{p}}\,\left[\|\tilde{u}_3\|_{L^p(\tilde{E})} +
    \sum_{i=1,2,3} h_i\,\|\tfrac{\partial\tilde{u}_3}{\partial\tilde{x}_i}\|_{L^p(\tilde{E})}\right]
\end{IEEEeqnarray*}
so, immediately,
\begin{IEEEeqnarray}{rCl} \label{aux_label18}
  \|(\rkutilde)_3\|_{L^{p}(\tilde{E})}
  &\leqslant& C\,\left(
  \|\tilde{u}_3\|_{L^p(\tilde{E})} +
    \sum_{i=1,2,3} h_i\,\|\tfrac{\partial\tilde{u}_3}{\partial\tilde{x}_i}\|_{L^p(\tilde{E})}
  \right)
\end{IEEEeqnarray}
and the sum of the three estimates yields the Theorem.
\end{proof}
%===============================================================================
% \begin{lemma}\label{L6} Let $P$ be a right prism. There exists a constant $C$ depending only on $\alpha_P$ such that for all $\bu$ in $W^{1,1}(P)$ we have
% \begin{multline}\label{estabL1}
% \|\bu_I\|_{L^1(P)} \leqslant C\Bigg(\|\bu\|_{L^1(P)} + \sum_{i=1}^3 h_{i,P}\|\partial_{\xi_{P,i}}\bu\|_{L^1(P)}\\ + \max\{h_{P,1},h_{P,2}\}\|\mbox{div\,}(u_1,u_2,0)\|_{L^1(P)}\Bigg).
% \end{multline}
% \end{lemma}
% \begin{proof} Using the notation introduced above for the vertices of $P$, suppose that $v_0$ is the vertex with the maximum angle of the triangle $v_0v_1v_2$. Let $\tilde P$ be a prism with vertices at $(0,0,0)$, $(h_{P,1},0,0)$, $(0,h_{P,2},0)$, $(0,0,h_{P,3})$, $(h_{P,1},0,h_{P,3})$ and $(0,h_{P,2},h_{P,3})$. Then by standard rescaling arguments using the Piola Transform we can prove from Lemma \eqref{L5} that there exists a constant $C$ such that for all $\bu\in W^{1,1}(\tilde P)$ we have
% \begin{eqnarray*}
% \|\tilde\bu_I\|_{L^1(\tilde P)}&\le& C\Bigg(\|\tilde\bu\|_{L^1(\tilde P)} + \sum_{i=1}^3h_{P,i}\|\partial_{x_i}\tilde\bu\|_{L^1(\tilde P)}\\&&\qquad + 
% \max\{h_{P,1},h_{P,2}\}\|\dv(\tilde u_1,\tilde u_2,0)\|_{L^1(\tilde P)}\Bigg).
% \end{eqnarray*}
% Let $B$ be the matrix with columns $\xi_{P,1}$, $\xi_{P,2}$ and $\xi_{P,3}$ (note $B$ has the form \eqref{matrix} and $\xi_{P,3}=(0,0,1)$). Then the map $F(\tilde{\bf x})=B\tilde{\bf x}+v_0$ sends $\tilde P$ onto $P$. Then, again by a change of variables, we obtain from the previous estimate, that for all $\bu\in W^{1,1}(P)$ it holds
% \begin{eqnarray*}
% \|\bu_I\|_{L^1(P)}&\le& C\|B\|\|B^{-1}\|\bigg(\|\bu\|_{L^1(P)} + \sum_{i=1}^3h_{P,i}\|\partial_{\xi_{P,i}}\bu\|_{L^1(P)}\\ &&  \qquad +\max\{h_{P,1},h_{P,2}\}\frac1{\|B^{-1}\|}\|\dv(u_1,u_2,0)\|_{L^1(P)}\bigg). 
% \end{eqnarray*}
% Then the proof concludes by noting that $\|B\|\leqslant C$ and $\|B^{-1}\|\sim \sin\alpha_P$.
% \end{proof}
% 
% \begin{remark} Stability estimates in $L^p$-norm, $p>1$, can by proved analogously.
% In particular, from  \eqref{estabL1}, using an inverse inequality on the left hand side, and Cauchy-Schwarz inequality on the right hand side, we obtain under assumptions of Lemma \ref{L6}
% \begin{multline}\label{estabL2}
% \|\bu_I\|_{L^2(P)} \leqslant C\Bigg(\|\bu\|_{L^2(P)} + \sum_{i=1}^3 h_{i,P}\|\partial_{\xi_{P,i}}\bu\|_{L^2(P)}\\ + \max\{h_{P,1},h_{P,2}\}\|\mbox{div\,}(u_1,u_2,0)\|_{L^2(P)}\Bigg)
% \end{multline}
% \end{remark}
%===============================================================================
\subsection{Local Interpolation Estimates for Prismatic Elements} % (fold)
\label{sub:local_interpolation_estimates_for_prismatic_elements}
We will state scaling consequences
of inequalities~(\ref{aux_label19}). Some of them will be used here
and the rest will be used in Chapter~\ref{auxlabel202}.
%=========
%For a non--negative integer $m$ let $\partial^m f$ denote the sum of the absolute values of all the derivatives of order $m$ of $f$.
%=========
\begin{remark}\label{aux_label28} Recall the vector averaged Taylor polynomial $\Qbb_{m,E}(\cdot)$
defined in~\eqref{auxlabel210}.
If $(\,\cdot\,)\,\hat{}\,$ denotes any of transformations~(\ref{transfHcurl})
and~(\ref{transfDiv}) then it holds
\begin{IEEEeqnarray*}{rCl}
  \Qbb_{m,\hat{E}}\hat{\bw} & = & (\Qbb_{m,E}\,\bw)\,\hat{}.
\end{IEEEeqnarray*}
\end{remark}
\begin{lemma}\label{aux_label20}
Let $\tilde E$ be the rescaled reference prism in Figure~\ref{rescaled_prism}.
Given $p\geqslant 1$, $m\geqslant 0$ and
$\tilde\bu \in W^{m+1,p}(\tilde E)$,  then
for $m \geqslant 0$ and $p \geqslant 1$ the following items hold.
\begin{enumerate}
  \item 
  For any component $1\leqslant i\leqslant 3$, for any $\bbeta$ with 
  $|\bbeta|\leqslant m+1$ 
  \begin{IEEEeqnarray}{rCl}\label{aux_label30} 
    \|\partial^{\bbeta}(\tilde  u_i - \tilde\Qb_{m,\tilde E}\,\tilde u_i)\|_{L^p(\tilde E)}
    & \leqslant & C 
    \sum_{|\balpha|=m-|\bbeta|+1} \bh^{\balpha} 
      \left\|\partial^{\,\balpha + \bbeta}\tilde u_i\right\|_{L^p(\tilde E)}
  \end{IEEEeqnarray}
  \item For any component $1\leqslant i \leqslant 3$,
  if $m \geqslant 1$ and $p \geqslant 1$
  %%=============================================================================
  %\begin{IEEEeqnarray}{rCl}
  %  \label{aux_label24__}
  %  \|\curl(\tilde\bu-\tilde\bq)\|_{L^p(\tilde E)}&\leqslant&
  %  C\,h^m
  %  \|\partial^m\curl\tilde\bu\|_{L^p(\tilde E)}\\[5pt]
  %  \label{aux_label25__}
  %  \|\tfrac{\partial}{\partial_{\tilde x_i}}\curl(\tilde\bu-\tilde\bq)\|_{L^p(\tilde E)}&\leqslant&
  %  C\,h^{m-1}
  %  \|\partial^m\curl\tilde\bu\|_{L^p(\tilde E)}
  %\end{IEEEeqnarray}
  %%=============================================================================
  \begin{IEEEeqnarray}{rCl}
    \label{aux_label24}
    \|\curl(\tilde\bu- \tilde\Qbb_{m,\tilde{E}}\tilde{\bu})_i\|_{L^p(\tilde E)}&\leqslant&
    C\,\sum_{j+k+l=m}  h_1^jh_2^kh_3^l
    \left\|\tfrac{\tilde\partial^m(\curl\tilde\bu)_i}{\partial\tilde x_1^j\partial\tilde x_2^k\partial\tilde x_3^l}
    \right\|_{L^p(\tilde E)}
  \end{IEEEeqnarray}
  \item For any component $1\leqslant i \leqslant 3$ and any 
  $1\leqslant j \leqslant 3$, if $m\geqslant 1$
  \begin{IEEEeqnarray}{rCl}
    \label{aux_label25}
    \left\|\tfrac{\partial\curl(\tilde\bu-\tilde\bq)_i}{\partial\tilde x_j}
    \right\|_{L^p(\tilde E)}
      &\leqslant& 
        C\,\sum_{j+k+l=m-1}  h_1^jh_2^kh_3^l
          \left\|\tfrac{\tilde\partial^{m-1}\tilde\partial(\tilde\curl\tilde\bu)_i}
                 {\partial\tilde x_1^j\partial\tilde x_2^k\partial\tilde x_3^l\partial\tilde x_j}
          \right\|_{L^p(\tilde E)}
  \end{IEEEeqnarray}
  \item For the divergence it holds
  \begin{IEEEeqnarray}{rCl}\label{aux_label23}
   \|\tilde{\dv}(\tilde\bu-\tilde\Qbb_{m,\tilde{E}}\tilde{\bu})\|_{L^p(\tilde E)}&\leqslant&
   C\sum_{j+k+l=m} h_1^jh_2^kh_3^l
   \left\|
    \tfrac{\tilde \partial^m\tilde{\text{div}}\tilde\bu}{\partial\tilde x_1^j\partial\tilde x_2^k\partial\tilde x_3^l}
   \right\|_{L^p(\tilde E)}
  \end{IEEEeqnarray}
\end{enumerate}
where C depends only on $m$, $\sigma$ (cfr. Theorem~\ref{aux_label21})
and the reference element.
\end{lemma}
\begin{proof}[Proof of Lemma~\ref{aux_label20}]
We will use  Lemma~\ref{aux_label40} rescaling $\hat{E}$ so that $d = 1$ and then
pushing forward to $\tilde E$. In fact, by~\eqref{aux_label29} for any multi--index 
$\balpha$ we have
  \begin{IEEEeqnarray}{rCl}\label{aux_label41}
    h_1^{\alpha_1}h_2^{\alpha_2}h_3^{\alpha_3}\tilde\partial^\alpha\tilde u_i (\tilde\bx)
    & = & (\nicefrac{1}{h_i}\hat\partial^{\alpha}\hat u_i)(F_{\tilde E}^{-1}\tilde\bx)
  \end{IEEEeqnarray}
  so, to prove~\eqref{aux_label30},
  \begin{IEEEeqnarray*}{rCl}
  \|\partial^{\bbeta}(\tilde  u_i - \tilde\Qb_{m,\tilde E}\,\tilde u_i)\|^p_{L^p(\tilde E)}
  & = & \|\frac{1}{\bh^{\bbeta}}\frac{1}{h_i}(\partial^{\bbeta} u_i - \Qb_{m-|\bbeta|, E}\,\partial^{\bbeta}u_i)\circ F_{\tilde E}^{-1}\|^p_{L^p(\tilde E)} \\[4pt]
  & = & \frac{|\det M_E|}{h_i^p}\frac{1}{(\bh^{\bbeta})^p} \| \partial^{\bbeta}  u_i - \Qb_{m-|\bbeta|, E}\, \partial^{\bbeta} u_i\|^p_{L^p(\hat E)} \\[4pt]
  & \leqslant & \frac{C|\det M_E|}{h_i^p}\frac{1}{(\bh^{\bbeta})^p} |\partial^{\bbeta} u_i|^p_{m-|\bbeta|+1,p,\hat E} \\[4pt]
  & \simeq & |\det M_E| \sum_{|\balpha  |=m-|\bbeta|+1} \tfrac{1}{(\bh^{\bbeta})^p} \left\|
   \tfrac{1}{h_i} \partial^{\balpha+\bbeta} u_i\right\|^p_{L^p(\hat{E})} \\[4pt]
  (\mbox{by~(\ref{aux_label41})})\qquad& = &
  C \sum_{|\balpha  |=m-|\bbeta|+1} 
  (\bh^{\balpha})^p \left\| \tilde\partial^{\balpha+\bbeta}\tilde u_i
    \right\|^p_{L^p(\tilde E)}.
\end{IEEEeqnarray*}
To prove~(\ref{aux_label24}), by a straightforward manipulation we have
\begin{IEEEeqnarray*}{rCl}
  \tilde\curl \tilde \Qbb_{m,\tilde E}\tilde\bu & = & 
  \tilde \Qbb_{m-1,\tilde E} \tilde\curl \tilde\bu.
\end{IEEEeqnarray*}
Then, componentwise, by~(\ref{aux_label30}), 
\begin{IEEEeqnarray*}{rCl}
  \|\tilde\curl(\tilde\bu-\tilde \Qbb_{m,\tilde E}\tilde\bu)_i\|_{L^p(\tilde E)}& = &
    \|(\tilde\curl\tilde\bu)_i - \tilde\Qb_{m-1}(\tilde\curl\tilde\bu)_i\|_{L^p(\tilde E)}\\[4pt]
  &\leqslant& C \sum_{j+k+l=m} h_1^jh_2^kh_3^l 
  \left\|\frac{\tilde\partial^m(\tilde\curl\tilde\bu)_3}
         {\partial\tilde x_1^j\partial\tilde x_2^k\partial\tilde x_3^l}\right\|_{L^p(\tilde E)}.
\end{IEEEeqnarray*}
Now~(\ref{aux_label25}) is an easy consecuence because
\begin{IEEEeqnarray*}{rCl}
\left\|\frac{\partial\curl(\tilde\bu-\tilde\bq)_i}{\partial\tilde x_j}\right\|_{L^p(\tilde E)}
    & = & 
\left\|\frac{\partial(\tilde\curl\tilde\bu)_i}{\partial\tilde x_j} - 
  \frac{\partial\tilde\Qb_{m-1,\tilde E}(\tilde\curl\tilde\bu)_i}{\partial\tilde x_j}
\right\|_{L^p(\tilde E)} \\
& = & 
\left\|\frac{\partial(\tilde\curl\tilde\bu)_i}{\partial\tilde x_j} - 
  \tilde\Qb_{m-2,\tilde E}\frac{\partial(\tilde\curl\tilde\bu)_i}{\partial\tilde x_j}
\right\|_{L^p(\tilde E)} \\
    &\leqslant& 
      C\,\sum_{j+k+l=m-1}  h_1^jh_2^kh_3^l
        \left\|\frac{\tilde\partial^{m-1}\tilde\partial(\tilde\curl\tilde\bu)_i}
               {\partial\tilde x_1^j\partial\tilde x_2^k\partial\tilde x_3^l\partial\tilde x_j}
        \right\|_{L^p(\tilde E)}.
\end{IEEEeqnarray*}
Equality~\eqref{aux_label23} follows in a simpler way.
\end{proof}
We arrived at one of the main results in this Thesis which is the an\-iso\-tropic
local interpolation error estimates for div--conforming elements on prisms of any order.
As we said earlier, in Chapter~\ref{auxlabel202} we will arrive
at the $\bcurl$--conforming analogue.
\\[5pt]

%% Sean un prisma obl\'{\i}cuo irregular $E$, $k\in\mathbb{N}$, el operador de interpolaci\'on 
%% $\bw_E$ de grado $k$ determinado por el elemento en la Definición~(\ref{edgeelement}), y
%% $p>2$. 
%% such that (\noindent{\color{BrickRed}hip. prisma}).
We will adopt the following notations. For a prism $E$, 
$h$ will denote its diameter and, for $1\leqslant i \leqslant 3$ ,
$\xi_i$ will denote unitary vectors with the directions
of the three edges $\be_i$ sharing a vertex $\bx_E$ of $E$ whose lengths are
$h_i$ and $\xi_3$ will be the particular direction of
the edge which is common to two quadrilateral faces. Recall that, for 
$\bh = (h_1,h_2,h_3)'$, $\bh^{\balpha}$ means 
$h_1^{\alpha_1}h_2^{\alpha_2}h_3^{\alpha_3}$
and 
$\partial^{\balpha} = \frac{\partial^{|{\balpha}|}}
{\partial\xi_1^{\alpha_1}\xi_2^{\alpha_2}\xi_3^{\alpha_3}}.$

Following the proof of Theorem 6.2 in~\cite{aadl} and adding the information 
of~\eqref{aux_label23},
from Remark~\ref{auxlabel4} we derive the following Theorem.
\begin{theorem}\label{aux_label46}
Let $k\in\mathbb{N}_0$ and $p \geqslant 1$.
Let $E$ be a prism whose triangular
faces have greatest angle less than $c_0$.
There exists $C > 0$ and three edges $\be_i$ of $E$ incident to a common vertex
$\bx_E$ such that for all $\bu\in W^{m + 1,p}(E)^3$
and $m\leqslant k$,
\begin{IEEEeqnarray}{C}\nonumber
  \|\bu-\br_E \bu\|_{L^p(E)} \leqslant C \left\{
  \sum_{|{\balpha}|=m+1}\bh^{\balpha} \|\partial^{\balpha}\bu\|_{L^p(E)} +
  h_E\sum_{|\balpha| = m}
  	\bh^{\balpha}\|\partial^{\balpha}\text{div} \,\bu\|_{L^p(E)} \right\}.\\[4pt]
  \label{aux_label39}
\end{IEEEeqnarray}
\end{theorem}

%
%Take a kernel $\hat{\phi}\in{C}_0^\infty(\mathbb{R}^n)$ such that
%\begin{IEEEeqnarray*}{rCl}
  %\supp{\hat{\phi}}&\subseteq&\hat{B}\\[5pt]
  %\int\limits_{\hat{B}} \hat{\phi}\,d\bx & = & 1
%\end{IEEEeqnarray*} 
%Now define the kernel on B
%\begin{IEEEeqnarray*}{rCl}
  %\phi_B&=&\frac{|\hat{B}|}{|B|}\,\hat{\phi}\circ F^{-1}\\[5pt]
  %&=&\frac{1}{\det DF}\,\hat{\phi}\circ F^{-1}
%\end{IEEEeqnarray*}
%This way we have
%\begin{IEEEeqnarray*}{rClCr}
  %\int\limits_{B}\phi\,d\textbf{x}&=&
  %\int\limits_{\hat{B}}\hat{\phi}\,\,d\hat{\textbf{x}}&=&1\\[5pt]
%\end{IEEEeqnarray*}
%and also
%\begin{IEEEeqnarray*}{rClClCr}
  %\supp{\phi_B}&=&\supp{\hat{\phi}\circ F^{-1}}
  %&=&F(\supp{\hat{\phi}})&=&B.
%\end{IEEEeqnarray*}
%Then we build an averaged Taylor polynomial over $B$ with the kernel
%$\phi_B$ and it holds that
%where 
%%==============================================================================
% \begin{theorem}\label{thmErrorInterpolacionPrismas}
% Let $P$ be a right prism, and consider a local system of coordinates $x_1x_2x_3$
% such that the triangular basis of $P$ are parallel to the $x_1x_2$-coordinate
% plane. Denote by $\xi_{P,1}$ and $\xi_{P,2}$ the versors parallel to the edges
% of the triangular basis of $P$ adjacent to its maximum angle
% $\alpha_P$, $\xi_{P,3}=(0,0,1)$ and $h_{P,i}$ are the lengths of the edges of
% $P$ parallel to $\xi_{P,i}$. We assume that $h_{P,3}>ch_{P,1}$ and
% $h_{P,3}>ch_{P,2}$. Then, there exists a constant $C$ depending only on $c$
% and $\alpha_P$, such that for all $\bu\in H^1(P)$ we have
% \begin{equation}\label{interp}
% \|\bu-\boldsymbol{r}_E\bu\|_{L^2(E)} \leqslant C\left(\sum_{i=1}^3 h_{E,i}
% \|\partial_{\xi_{E,i}}\bu\|_{L^2(E)} + h_T\|\dv\bu\|_{L^2(E)}\right).
% \end{equation}
% \end{theorem}
%%==============================================================================
% subsection local_interpolation_estimates_for_prismatic_elements (end)
% section prismatic_finite_elements (end)