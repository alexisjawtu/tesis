\noindent the next step is to estimate the stability in 
an anisotropically rescaled prism. Given three positive numbers
$h_1$, $h_2$ and $h_3$ we denote
\begin{IEEEeqnarray*}{CCl}
    \yesnumber\label{tilde_prism}
    \tilde{E}   &   =   &   \tilde{T} \times \tilde{I}\\
    \tilde{T}   &   =   &   \{ 0 < \nicefrac{\tilde{x}_1}{h_1} + \nicefrac{\tilde{x}_2  }{h_2} < 1 \}\\
    \tilde{I}   &   =   &   \{ 0 < \nicefrac{\tilde{x}_3}{h_3} < 1 \}.
\end{IEEEeqnarray*}
\rescaledPrismTikz
Of course $\tilde{E} = F(\hat{E})$ where $F$ is the linear
$\mathbb{R}^3 \rightarrow \mathbb{R}^3$ transformation such as
\begin{IEEEeqnarray}{rClCl}
  \label{change_var}
  F\hat{\bx} & = & \diag{h_1}{h_2}{h_3} \hat{\bx} & = & \tilde{\bx}
\end{IEEEeqnarray}
Denote with $\tilde{\bw}_k$ the $k$--th order curl--conforming interpolation
operator over $\tilde{E}$ defined as in~(\ref{push-forward}) via the \emph{push--forward}
$F^*$. for the rest of the subsection $\tilde\bu$ will be an element
with a well defined curl--conforming interpolate.
%, namely of
%$H({\bf curl},\tilde{E})\cap H^{1/2+\delta}(\tilde{E})^3$ for 
%a positive $\delta$ with 
%${\bf curl}\,\tilde{\bu}\in L^p(\tilde{E})^3$
%for some
%$p>2$.
Write the diameter of $\tilde{E}$ as $\textit{h}$ and as
$\tilde{x}_i,\,1\leqslant i\leqslant 3$, the coordinates along the axis
in $\mathbb{R}^3$.
\begin{lemma}\label{estabLinf} There exists a positive $C$, independent
of $h_i,\,1\leqslant i\leqslant 3$ such that for all $p > 2$ and 
$\tilde{\bu}\in\wpcurl{\tilde{E}}$
\begin{IEEEeqnarray*}{rCl}
    \left\| \wkutilde \right\|_{L^\infty(\tilde{E})^3}
    & \leqslant & C \left[ |\tilde{E}|^{-\nicefrac{1}{p}} \left( \left\| \tilde{\bu} 
    \right\|_{L^p(\tilde{E})^3} +
        \sum_{i=1}^3 h_i \left\| \partial_{\tilde{x}_i}\tilde{\bu} 
        \right\|_{L^p(\tilde{E})^3} \right)\right.\\
    &   & \left.\:+\; (h_1+h_2)\, |\tilde{E}|^{-1} \left( \left\|(\curl\,\tilde{\bu})_3 
    \right\|_{L^1(\tilde{E})} + 
    \sum_{i=1}^3 h_i \left\| \partial_{\tilde{x}_i}(\curl\,\tilde{\bu})_3 
    \right\|_{L^1(\tilde{E})}\right)
    \right].
\end{IEEEeqnarray*}
{\color{BrickRed} ver las cuentas donde dice $h_1 + h_2$}
\end{lemma}
\begin{proof}
The proof of this estimate will be made componentwise
using the inequalities of 
Theorem~(\ref{thm_stab_edge}) and the vectorial bound will hold immediately.
Bounds for $(\wkutilde)_1$ and $(\wkutilde)_3$
will be established, as the bounding for $(\wkutilde)_2$ is the same as the first one.
Pulling $\wkutilde$ back to $\hat{E}$ we get by
~(\ref{piTransformado}) that $(\wkutilde)_i = 
\nicefrac{1}{h_i} (\wku)_i,\,1\leqslant i\leqslant 3$. By~(\ref{teorema_1}) and a suitable, though elementary,
change of variables dictated by~(\ref{change_var})
\begin{IEEEeqnarray*}{rCl}
  \left\| (\wkutilde)_1 \right\|_{L^\infty(\tilde{E})} & = &
    \frac{1}{h_1} \left\| (\wku)_1 \right\|_{L^\infty(\hat{E})}\\
    & \leqslant & \frac{c(\hat{E})}{h_1} \left[\|\hat{u}_1\|_{W^{1,p}(\hat{E})} + 
        \|(\curl\,\hat{\bu})_3\|_{W^{1,1}(\hat{E})}\right] \\
    & \leqslant & c(\hat{E})
  \left[
    |\tilde{E}|^{\nicefrac{-1}{p}}
    \left(
    \|\tilde{u}_1\|_{L^p(\tilde{E})} + \sum_{i=1}^3 h_i \left\|\frac{\partial\tilde{u}_1}{\partial\tilde{x}_i}
    \right\|_{L^p(\tilde{E})}
    \right)
  \right.\\
\yesnumber\label{number1}      & & \:\:+
  \left.
    h_2|\tilde{E}|^{-1}
    \left(
    \|(\curl\,\tilde{\bu})_3\|_{L^1(\tilde{E})} + 
        \sum_{i=1}^3 h_i \|\partial_{\tilde{x}_i}(\curl\,\tilde{\bu})_3\|_{L^1(\tilde{E})}
    \right)
  \right].
\end{IEEEeqnarray*}
With respect to component number three, from~(\ref{teorema_3})
\begin{IEEEeqnarray}{rCl}\label{number2}
  \left\| (\wkutilde)_3 \right\|_{L^\infty(\tilde{E})}
  & \leqslant & C|\tilde{E}|^{-\nicefrac{1}{p}}
  \left(
    \|\tilde{u}_3\|_{L^p(\tilde{E})} + \sum_{i=1}^3 h_i \|\partial_{\tilde{x}_i}\tilde{u}_3\|_{L^p(\tilde{E})}
  \right).
\end{IEEEeqnarray}
\end{proof}
\noindent With the previous bound we deduce the following
anisotropic stability estimate for the rescaled element $\tilde{E}$.
\begin{theorem} There is a $c > 0$ independent of $h_i$ such that for all
$\tilde{\bu}\in\wpcurl{\tilde{E}}$
  \begin{IEEEeqnarray*}{rCl}
    \left\| \wkutilde \right\|_{L^p(\tilde{E})}
    & \leqslant & C \left[ \left\| \tilde{\bu} \right\|_{L^p(\tilde{E})}
    + \sum_{i=1}^3 h_i \left\| \partial_{\tilde{x}_i}\tilde{\bu} \right\|_{L^p(\tilde{E})}\right.\\
    & & \left.
    \:+\;(h_1+h_2)\left(\left\|(\curl\,\tilde{\bu})_3 \right\|_{L^p(\tilde{E})}
     + \sum_{i=1}^3 h_i
     \left\| \partial_{\tilde{x}_i}(\curl\,\tilde{\bu})_3 \right\|_{L^p(\tilde{E})}\right)
  \right].
\end{IEEEeqnarray*}
\end{theorem}
\begin{proof}
    \noindent From Lemma~(\ref{estabLinf}), since $|\tilde{E}|$ is finite measured,
    the H\"older inequality tells us that, for any real $q \geqslant 1$,
    \begin{IEEEeqnarray*}{rCl}
        \|(\curl\,\tilde{\textbf{u}})_3\|_{L^1(\tilde{E})} &\leqslant&
         |\tilde{E}|^{1-\frac{1}{q}}\,\|(\curl\,\tilde{\textbf{u}})_3\|_{L^q(\tilde{E})}\\
        \|\partial_{\tilde{x}_i}(\curl\,\tilde{\textbf{u}})_3\|_{L^1(\tilde{E})} &\leqslant&
         |\tilde{E}|^{1-\frac{1}{q}}\,\|\partial_{\tilde{x}_i}(\curl\,\tilde{\textbf{u}})_3\|_{L^q(\tilde{E})}.
    \end{IEEEeqnarray*}
    So we get to
    \begin{IEEEeqnarray*}{rCl}
    \left\| (\tilde{\bw}_k\tilde{{\textbf{u}}})_1 \right\|_{L^p(\tilde{E})}
        & \leqslant & |\tilde{E}|^{\nicefrac{1}{p}}\left\| (\tilde{\bw}_k\tilde{{\textbf{u}}})_1 \right\|_{L^\infty(\tilde{E})}\\
     \mbox{(by~(\ref{number1}))\hspace{.6cm}}   & \leqslant & C
        \left[
            \|\tilde{u}_1\|_{L^p(\tilde{E})} + \sum_{i=1}^3 h_i \|\frac{\partial\tilde{u}_1}{\partial\tilde{x}_i}\|_{L^p(\tilde{E})}
        \right.\\
            & & \:\:+
        \left.
            h_2
            \left(
            \|(\curl\,\tilde{\textbf{u}})_3\|_{L^p(\tilde{E})} + 
                \sum_{i=1}^3 h_i \|\partial_{\tilde{x}_i}(\curl\,\tilde{\textbf{u}})_3\|_{L^p(\tilde{E})}
            \right)
        \right].
    \end{IEEEeqnarray*}
    Now combine this with an entirely analogous argument for component $2$ and with~(\ref{number2}) and
    the Theorem follows.
\end{proof}