\section{Local estimates for Pyramidal Virtual Elements}

\begin{theorem}
If $E$ is an isotropic tetrahedron or pyramid, then  
\begin{equation}\label{estab2}
\|\bu_I\|_{L^p(E)}\leqslant C\left(\|\bu\|_{L^p(E)}+ h_T|\bu|_{W^{1,p}(E)}\right), \qquad \forall \bu\in W^{1,p}(E),
\end{equation}
with the constant $C$ depending on the shape regularity of $E$, and $1\leqslant p$ if $E$ is a tetrahedron and $1\leqslant p\leqslant 2$ if $E$ is a pyramid.
\end{theorem}
\begin{proof}
{\color{blue}\#\#\#\#\#\#\#\# esto no ponerlo, al momento de usar tetra para
malla, citar del articulo\\.}
When $E$ is a tetrahedron, this result is contained in \cite{aadl}. So we assume that $E$ is a pyramid.\\ 
{\color{blue}\#\#\#\#\#\#\#\#\\}
We note that 
\begin{equation}\label{estab2:eq1}
\bu_I=\sum_{i=1}^5 \left(\int_{f_i}\bu\cdot \bn\right)\bv_i
\end{equation}
where $\{\bv_i\}_{i=1}^5$ is the basis of $V_h(E)$ dual to the degrees of freedom
\ref{dofs}. Denote by $f_j$, $j=1,\ldots,5$ the faces of $E$. First of all we need to estimate the $L^2$-norm of the basis functions $\bv_i$. Fixed $1\leqslant i\leqslant 5$, it follows from the proof of Lemma \ref{existenciaInterpolante} that $\bv_i=\nabla \psi$ where $\psi$ is the solution of
\begin{eqnarray*}
\Delta\psi&=&d\qquad\mbox{in }\Omega\\ \frac{\partial\psi}{\partial\bn}&=&g\qquad\mbox{on }\partial E\\ \int_E\psi&=&0
\end{eqnarray*}
with
\[
g|_{f_j}=\left\{\begin{array}{cl}\frac1{|f_i|}&\mbox{if }i=j\\0&\mbox{if }i\ne j\end{array}\right., \qquad d=\frac1{|E|}.
\]
Multiplying the first equation defining $\psi$ by $\psi$, and integrating by parts, we obtain
\begin{eqnarray*}
\|\nabla\psi\|_{L^2(E)}^2 &=& -\int_Ed\psi + \int_{\partial E}g\psi\\ &\leqslant & 
\|d\|_{L^2(E)}\|\psi\|_{L^2(E)} + \|g\|_{L^2(\partial E)}\|\psi\|_{L^2(\partial E)}.
\end{eqnarray*}
Using Poincare's and trace inequalities we have for, a constant $C$ depending on the aspect ratio of $E$ 
\begin{equation}\label{estab2:eq2}
\|\nabla\psi\|_{L^2(E)}^2\leqslant  C\left(h_E\|d\|_{L^2(E)} + h_E^\frac12 \|g\|_{L^2(\partial E)}\right) \|\nabla\psi\|_{L^2(E)}.
\end{equation}
Taking into account the definitions of $d$ and $g$ we have
\[
\|d\|_{L^2(E)}\leqslant Ch_E^{-\frac32}, \qquad \|g\|_{L^2(\partial E)}\leqslant Ch_E^{-1}
\]
and so from \eqref{estab2:eq2} we obtain
\begin{equation}\label{estab2:eq3}
\|\bv_i\|_{L^2(E)}=\|\nabla\psi\|_{L^2(E)}\leqslant Ch_E^{-\frac12}.
\end{equation}

Now, for $1\leqslant p\leqslant 2$ using H\"older's inequality and the expression \eqref{estab2:eq1} we have
\begin{eqnarray*}
\|\bu_I\|_{L^p(E)}&\le& |E|^{\frac1p-\frac12}\|\bu_I\|_{L^2(T)}\\&\le&|E|^{\frac1p-\frac12} \sum_{i=1}^5  \left|\int_{f_i}\bu\cdot\bn\right|\|\bv_i\|_{L^2(E)}.
\end{eqnarray*}
By using \eqref{estab2:eq3}, H\"older's inequality, trace inequalities and taking into account the shape-regularity of $E$ we obtain
\begin{eqnarray*}
\|\bu_I\|_{L^p(E)}&\le& C |E|^{\frac1p-\frac12} h_E^{-\frac1p}\left(\|\bu\|_{L^p(\Omega)}+h_E\|\nabla\bu\|_{L^p(E)}\right)|\partial E|^{1-\frac1p} \|\bv_i\|_{L^2(E)}\\&\le& C h_E^{3\left(\frac1p-\frac12\right)}h_E^{-\frac1p}h_E^{2\left(1-\frac1p\right)} h_E^{-\frac12}\left(\|\bu\|_{L^p(\Omega)}+h_E\|\nabla\bu\|_{L^p(E)}\right)\\&=& C \left(\|\bu\|_{L^p(\Omega)}+h_E\|\nabla\bu\|_{L^p(E)}\right)
\end{eqnarray*}


%\sum_{i=1}^5 \left|\int_{f_i}\bu\cdot\bn\right|\|\bv_i\|_{L^2(E)}\\&\le& \sum_{i=1}^5 |f_i|^\frac12\|\bu\|_{L^2(f_i)}\|\bv_i\|_{L^2(E)}\\ &\le& C|f_i|^\frac12\left(h_e^{-\frac12}\|\bu\|_{L^2(E)}+h_E^\frac12\|\nabla\bu\|_{L^2(E)}\right)\|\bv_i\|_{L^2(E)}\\ &\le& C\left(\|\bu\|_{L^2(E)}+h_E\|\nabla\bu\|_{L^2(E)}\right)

where $C$ depends on the shape regularity of $E$.
\end{proof}

\begin{proposition}\label{propErrorInterpolacionPiramidesTetraedros}
Let $E$ be a tetrahedron or a pyramid satisfying the shape-regularity property with constant $\sigma$. Then there exists a constant $C$ depending only on $\sigma$ such that 
\[
\|\bu-\bu_I\|_{L^2(E)}\leqslant C h_E|\bu|_{H^1(E)} \qquad \forall \bu\in H^1(E).
\]
\end{proposition}
\begin{proof} Let $Q\bu$ be the $L^2(E)$-projection of $u$ onto the constant fields. Then we have
\[
\bu-\bu_I = (\bu - Q\bu) + (Q\bu-\bu)_I
\]
and using the previous Lemma and a clasical estimate for the $L^2(E)$-projection error we have
\begin{eqnarray*}
\|\bu-\bu_I\|_{L^2(E)} &\le& \|\bu - Q\bu\|_{L^2(E)} + \|(\bu-Q\bu)_I\|_{L^2(E)}\\ &\le& \|\bu - Q\bu\|_{L^2(E)} + C \left(\|\bu-Q\bu\|_{L^2(E)} + h_E\|\nabla(\bu-Q\bu)\|_{L^2(E)}\right)\\ &=& C\left(\|\bu-Q\bu\|_{L^2(E)} + h_E\|\nabla\bu\|_{L^2(E)}\right)\\&\le& Ch_E\|\nabla\bu\|_{L^2(E)}
\end{eqnarray*}
as we wnated to prove.
\end{proof}


\begin{proposition}\label{propupi}
Let $E$ be a pyramid satisfying a shape regularity property with constant $\sigma$, and $\bu\in H^1(E)$. 
Then there exists a field $\bu_\pi\in W(E)$ such that
\[
\|\bu-\bu_\pi\|_{L^2(E)}\leqslant C h_E|\bu|_{H^1(E)}.
\]
(Poner directo) Then
\[
\|\bu-P^{\perp}\bu\|_{L^2(E)}\leqslant C h_E|\bu|_{H^1(E)}.
\]
\end{proposition}
\begin{proof}
We can define $\bu_\pi$ on $E$ as the $L^2(E)$-projection of $\bu$ on the space
of constant fields $P_0(E)^3\subset W(E)$. The error estimate follows from Bramble-Hilbert Lemma.  
\end{proof}
\begin{remark}
  we could replicate the results por prisms, at low order. We are not
  applying pyramidal FE interpolation, though.
\end{remark}