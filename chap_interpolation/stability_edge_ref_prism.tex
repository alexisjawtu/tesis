We are writing the following three Lemmas which state some special
behavior of the interpolation operator, mostly related to the preservation of
null components of the fields, and whose proofs consist in a smart use of the
degrees of freedom and the very definition of the operator. These Lemmas, 
although just with technical purpose, exhibit
nice properties of the interpolators.

In the present subsection $\hat\bu$ is an element
in $W^{1,p}(\hat{E})$ for $p>2$ which is a space whose elements have well
defined tangential traces on each edge of the prism $\hat{E}$.
Another possibility, as stated in Lemma $5.38$ in the page $134$ of~\cite{monk},
is to assume there are 
a positive $\delta$ and a $p>2$ such that 
$\hat\bu$ belongs to $H^{1/2+\delta}(\hat{E})^3$ and
${\bf curl}\,\bu$ belongs to $L^p(\hat{E})^3$.
For the whole section, $\hat\bw_k$ will be the $k$--th order edge 
interpolation operator on the reference
Prism determined by the element of
\emph{Definition}~\ref{edgeelement}.
\\

We will stick to the notation and indices of Tables~\ref{prismNotationTableFaces}
and~\ref{prismNotationTableEdges}.

\begin{lemma}\label{lema_PIu3_k_cualquiera} 
$(\wku)_3$ is linearly and univocally 
determined by $\hat{u}_3$.
\end{lemma}
\begin{proof} If we pay attention to the directions of the unit
tangents and normals to the edges and faces, respectively, of $\hat E$,
we realize that
the degrees of freedom which involve $(\wku)_3$ give rise only to the 
following linear equations
\begin{IEEEeqnarray}{rCrc}
\varphi_{\hat{\be}_i,p}\,(\wku) & = & \varphi_{\hat{\be}_i,p}\,(\hat{\bu}) &\quad\mbox{as in~(\ref{momentos1hcurl}) for $i$ = 3, 6, 7,}\\
\varphi_{f_1,\bq}\,(\wku) & = & \varphi_{f_1,\boldsymbol{q}}\,(\hat{\bu})
  &\quad\mbox{as in~(\ref{momentos3hcurl})}\\
\varphi_{f_2,\bq}\,(\wku) & = & \varphi_{f_2,\boldsymbol{q}}\,(\hat{\bu})
  &\quad\mbox{as in~(\ref{momentos4hcurl})}  \\
\varphi_{f_5,\bq}\,(\wku) & = & \varphi_{f_5,\boldsymbol{q}}\,(\hat{\bu})
  &\quad\mbox{as in~(\ref{momentos5hcurl})}  \\
\varphi_{\boldsymbol{r}}\,(\wku) & = & \varphi_{\boldsymbol{r}}\,(\hat{\bu})
  &\quad\mbox{as in~(\ref{momentos6hcurl})}.
\end{IEEEeqnarray}
These are 
$3k$+$3k(k-1)$+$k(k-1)(k-2)/2 = k(k+1)(k+2)/2$ equations,
just the dimension of $P_k(\hat{T})\otimes P_{k-1}(\hat{I})$, 
which is the space $(\wku)_3$ belongs to by definition.
%$\frac{k(k+1)(k+2)}{2}$. 
%libertad en los que no des\-a\-pa\-re\-ce $(\wku)_3$ son \'unicamente:
Now set all those equations equal to zero (that is, pick $u_3 = 0$) and see that
the unique solution is $(\wku)_3 = 0$.
A little more explicitly, we have:
\begin{IEEEeqnarray}{lCll}
  \label{aristas} \int_{\hat\be_i} (\wku)_3\,\hat q \, d\alpha 
  & = & 0 &\qquad \mbox{for $i$ = 3, 6 and 7, }q\in P_{k-1}(\hat\be_i)\\[5pt]
  \label{caras} \iint_{\hat f_j} (\wku)_3\,\hat q \,d\hat S
  & = & 0 &\qquad \mbox{for $j$ = 1, 2, and 5, } \hat q\in Q_{k-2,k-1}(\hat f_j)\\[5pt]
  \label{enK} \int_{\hat{E}} (\wku)_3\,\hat q_3 \, d\bx 
  & = & 0 &\qquad \mbox{for }\hat q_3\in P_{k-3,k-1}.
\end{IEEEeqnarray}
Start considering the face $\hat f_2$.
The restriction of $(\wku)_3$ to $\be_3$
is itself 
an element in $P_{k-1}({\be_3})$, 
and the same holds for $\be_6$,  
so equations~(\ref{aristas}), for $i = 3$, $6$, say that $(\wku)_3$
is identically null on those edges by which
the restriction 
$(\wku)_3|_{f_2}$, which is an element of $P_k(\hat x_1)\otimes P_{k-1}(\hat x_3)$
may be se factorized
as
$(\wku)_3|_{f_2}(\hat x_1,0,\hat x_3) = \hat x_1\,(1-\hat x_1)\,w_0(\hat x_1,\hat x_3)$,
with $w_0$ equal to some polynomial in $P_{k-2}(\hat x_1) \otimes P_{k-1}(\hat x_3)$.
Now choose $\hat q = w_0$ in the degrees of freedom~(\ref{caras}) for the face
$\hat f_2$ and it holds
\begin{IEEEeqnarray}{lClc}
	\iint_{\hat f_2} \hat{x}_1(1-\hat{x}_1)w_0(\hat{x}_1,\hat{x}_3)^2\,d\hat S & = & 0.
\end{IEEEeqnarray}
But $\hat{x}_1\,(1-\hat{x}_1)$ is almost everywhere positive over the closure
of $\hat f_2$, so 
$(\wku)_3|_{\hat f_2}$ vanishes identically.
By a completely symmetric observation we can prove 
$(\wku)_3|_{\hat f_1} \equiv 0$.\\
One more time, if we use~(\ref{aristas}) for $i =$ 6, 7,
$\hat x_1\,(1-\hat x_1)$ divides the restriction $(\wku)_3|_{\hat f_5}$, so that
there is a 
$w_1 \in P_{k-2}(\hat x_1)\otimes P_{k-1}(\hat x_3)$ for which 
$(\wku)_3|_{\hat f_5}(\hat x_1, \hat x_2, \hat x_3) = \hat{x}_1\,\displaystyle{(1-\hat{x}_1)}\,w_1(\hat{x}_1,\hat{x}_3)$.
Now equality~(\ref{caras}) for $j = 5$ implies
$(\wku)_3|_{f_5} \equiv 0$.\\
Next, since $(\wku)_3$ vanishes when restricted $\hat f_1$, $\hat f_2$ and $\hat f_5$
we get to factorize it on $\hat E$ as 
\begin{IEEEeqnarray*}{rCl}
	(\wku)_3(\hat x_1, \hat x_2, \hat x_3) 	& = 	& \hat{x}_1\,\hat{x}_2\,(1-\hat{x}_1-\hat{x}_2)\,w_3(\hat{x}_1,\hat{x}_2,\hat{x}_3),\\
									w_3		& \in 	& P_{k-3}(\hat x_1)\otimes P_{k-1}(\hat x_3).
\end{IEEEeqnarray*}
And now we evaluate the degrees of freedom~(\ref{enK}) choosing
$\hat q_3 \equiv w_3$ and conclude immediately
$(\wku)_3 \equiv 0$ on $\hat E$. 
\end{proof}
\begin{lemma}\label{lemma_PIu2_k_in_N}
\begin{itemize}
	\item []
	\item [(a)]\label{piu2_k_in_N} If $\hat\bu(\hat x_1,\hat x_2,\hat x_3) = (0, \hat u_2(\hat x_2,\hat x_3), 0)'$,
	then 
  \[
  \wku\xyz = (0, \hat\xi_2(\hat x_2,\hat x_3) ,0)'
  \]
  for some 
	$\hat\xi_2 \in P_{k-1}(\hat{x}_2) \otimes P_k(\hat{x}_3)$.
	\item [(b)]\label{piu1_k_in_N} If $\hat\bu(\hat x_1,\hat x_2,\hat x_3) = (\hat u_1(\hat x_1,\hat x_3), 0, 0)'$
	then
  \[
  \wku\xyz = (\hat\xi_1(\hat x_1,\hat x_3), 0 ,0)'
  \]
  for some
    $\hat\xi_1\in P_{k-1}(\hat{x}_1) \otimes P_k(\hat{x}_3)$.
\end{itemize}
\end{lemma}
\begin{proof} We will prove the first inequality, as the second follows
with the same ideas. In Subsection~\ref{sub:defEdgeElement} we found 
expression~(\ref{elemento_P_k}) which states
\begin{IEEEeqnarray*}{rCl}
  \wku\xyz  & = & (p_1\xyz, p_2\xyz, p_3\xyz)^t\\[4pt]
  			    & = & \begin{pmatrix}
  					        \xi_1\xyz + \hat{x}_2\,h\xyz\\
                    \yesnumber\label{expr_wku}\xi_2\xyz - \hat{x}_1\,h\xyz \\
  					        \xi_3\xyz
  				        \end{pmatrix}
\end{IEEEeqnarray*}
for
$\xi_1$ and $\xi_2$ in $P_{k-1}(\hat{f}_3) \otimes P_k(\hat{x}_3)$,
$\xi_3$ in $P_{k}(\hat{f}_3) \otimes P_{k-1}(\hat{x}_3)$,
and $h$ in $\tilde{P}_{k-1}(\hat{f}_3) \otimes P_k(\hat{x}_3)$.
Thanks to Lemma~\ref{lema_PIu3_k_cualquiera} we already know that $\xi_3 \equiv 0$,
so we are going to show
that $h \equiv 0$, $\xi_1 \equiv 0$ and that $\xi_2$ does not depend 
on $\hat{x}_1$. First, if $\hat{f}$ is either $\hat{f}_3$ o $\hat{f}_4$, then
with a direct calculation we see that $({\bf curl}\,\wku)_3 |_{\hat f}$ belongs to
$P_{k-1}(\hat{f})$. By the commutative diagram property expressed
in~(\ref{curl_commutativity}), the definition of the degrees of freedom~(\ref{momentos1hdiv})
and the interpolation operator $\br_{\hat E}$
in Definition~\ref{defi_face_element}, it holds that, if $\hat{f}$ is 
either $\hat{f}_3$ or $\hat{f}_4$, then for every $q \in P_{k-1}(\hat{f})$,
\begin{IEEEeqnarray*}{rCCCl}
  \hat\rho_{\hat{f},q}\,({\bf curl\,}\wku)
  & = & \hat\rho_{\hat{f},q} (\br_{\hat E}\,{\bf curl\,}\hat\bu) &&\\[5pt]
  & = & \iint_{\hat f} ({\bf curl\,}\hat \bu)_3\,q \,d\hat S & = & 0.	
\end{IEEEeqnarray*}
As is for time being expected, the choice $q = ({\bf curl}\,\wku)_3 |_{\hat{f}}$
yields 
\[
  ({\bf curl}\,\wku)_3 |_{\hat{f}} \equiv 0
\]
so we may write again $(\curl \wku)_3 = \hat{x}_3\,(\hat{x}_3-1)\,\hat\psi$ for
a $\hat\psi\in P_{k-1}(\hat{f}) \otimes P_{k-2}(\hat{x}_3)$.
We choose now $q=\hat\psi$ in the degrees of freedom~(\ref{momentos4hdiv}).
By the commutative diagram property
and the definition of $\br_{\hat E}$, we have 
\begin{IEEEeqnarray*}{rCCCl}
	\int_{\hat{E}} \hat{x}_3\,(\hat{x}_3-1)\,\hat\psi^2\,d\hat{\bx}
  & = &\int_{\hat{E}} (\curl\wku)_3\,\hat\psi\,d\hat{\bx}&&\\
  & = &\int_{\hat{E}} (\br_{\hat E}\curl\hat{\bu})_3\,\hat{\psi}\,d\hat{\bx}&&\\
  & = &\int_{\hat{E}} (\curl\hat{\bu})_3\,\hat{\psi}\,d\hat{\bx} & = & 0\\
\end{IEEEeqnarray*}
and it follows that
\begin{IEEEeqnarray}{rCl}
	\label{rot_3_es_0} (\curl\wku)_3 &\equiv& 0.
\end{IEEEeqnarray}
Now if we explore $({\bf curl}\,\wku)_3$ taking derivatives in  
expression~(\ref{expr_wku}) we get 
\begin{IEEEeqnarray*}{rCl}
  (\curl\wku)_3 & = & 
  \dfrac{\partial}{\partial \hat x_1}(\wku)_2 - \dfrac{\partial}{\partial \hat x_2}(\wku)_1\\[5pt]
  \label{expre_h} \yesnumber & = & -(2\,h + \hat{x}_2\,\dfrac{\partial h}{\partial \hat{x}_2} + 
	\hat{x}_1\,\dfrac{\partial h}{\partial \hat{x}_1}) + 
	\dfrac{\partial\hat\xi_2}{\partial \hat{x}_1} - \dfrac{\partial\hat\xi_1}{\partial \hat{x}_2}.
\end{IEEEeqnarray*}
Following the degrees in each terms, it holds
\begin{IEEEeqnarray*}{rCl}
  g\,:=\,2\,h + \hat{x}_2\,\dfrac{\partial h}{\partial \hat{x}_2} + 
  \hat{x}_1\,\dfrac{\partial h}{\partial \hat{x}_1}
  & \mbox{ belongs to } & \tilde{P}_{k-1}(\hat{f}_3) \otimes P_k(\hat x_3)\\[4pt]
  \dfrac{\partial\hat\xi_2}{\partial \hat{x}_1} -
  \dfrac{\partial\hat\xi_1}{\partial \hat{x}_2}
  & \mbox{ belongs to } & P_{k-2}(\hat{f}_3) \otimes P_k(\hat x_3)\mbox{,}
\end{IEEEeqnarray*}
but from this it follows necessarily that $g \equiv 0$. Now, how do the terms
of $g$ look like? Let us put
\begin{IEEEeqnarray*}{rCl}
	h\xyz &=& \sum_{\stackrel{i+j\,=\,k-1}{l\,\leqslant\,k}} \alpha_{_{i,j,l}}\,\hat{x}_1^i \hat{x}_2^j \hat{x}_3^l.
\end{IEEEeqnarray*}
Then
\begin{IEEEeqnarray*}{rCl}
  g\xyz & = & \sum_{\stackrel{i+j\,=\,k-1}{l\,\leqslant\,k}} 
  (2\alpha_{_{i,j,l}} + j\,\alpha_{_{i,j,l}} + i\,\alpha_{_{i,j,l}}) \hat{x}_1^i \hat{x}_2^j \hat{x}_3^l\\
  \yesnumber\label{h_is_zero} & = &(k+1)\,h\xyz\,=\,0,
\end{IEEEeqnarray*}
so $h \equiv 0$ too and, for now, 
\begin{IEEEeqnarray*}{rCl}
\wku\xyz &=& 
(\hat\xi_1\xyz, \hat\xi_2\xyz, 0)'. 
\end{IEEEeqnarray*}
The second--to--last task is to see that $\hat\xi_1$ vanishes identically.
We turn back to de edge degrees of freedom.
Set $\be$ equal to $\hat\be_1$ or $\hat\be_4$. Then the restriction
$\hat\xi_1|_{\be}$ belongs to $P_{k-1}(\be)$, so letting $\hat q = \hat\xi_1|_{\be}$ in
~(\ref{momentos1hcurl}) we obtain
\begin{IEEEeqnarray*}{rCCCCCl}
	0 &=& \hat\varphi_{\be,\,\hat\xi_1}\,(\hat\bu) &=&
	\hat\varphi_{\be,\,\hat\xi_1}\,(\wku) &=& \int_{\be} (\hat\xi_1)^2\,d\alpha\textrm{,}
\end{IEEEeqnarray*}
so, for some $\hat{p} \in P_{k-1}(\hat x_1)\otimes P_{k-2}(\hat x_3)$ we have
\[
  \hat\xi_1|_{\hat f_2}(\hat x_1,\hat x_3) = \hat{x}_3\,(\hat{x}_3-1)\,\hat{p}(\hat x_1,\hat x_3).
\]
Next choose $\hat{f} = \hat{f}_2$ and $\boldsymbol{q}=(0,0,\hat{p})'$ in~(\ref{momentos4hcurl}).
\begin{IEEEeqnarray*}{rCCCCCl} 
  0 & = & \hat\varphi_{\hat{f}_2,\bq}\,(\hat\bu) 
    & = & \hat\varphi_{\hat{f}_2,\bq}\,(\wku) 
    & = & \iint_{\hat{f}_2} \hat{x}_3\,(\hat{x}_3-1)\hat{p}^2\,d\hat S.
\end{IEEEeqnarray*}
It follows that $\hat\xi_1|_{\hat f_2}\equiv 0$, by which we know it exists
certain $\zeta \in P_{k-2}(\hat{f}_3)\otimes P_k(\hat{x}_3)$ satisfying
\[
\hat\xi_1\xyz = \hat{x}_2\,\zeta\xyz.
\]
Now we switch to the faces $\hat{f} = \hat{f}_3$ or $\hat{f}_4$. 
Take $\hat{\boldsymbol{q}} = (\zeta|_{\hat f},0,0)'$ in~(\ref{momentos2hcurl})
\begin{IEEEeqnarray*}{rCCCCCl}
  0 & = & \varphi_{\hat f,\hat{\boldsymbol{q}}}\,(\hat\bu) 
    & = & \varphi_{\hat f,\hat{\boldsymbol{q}}}\,(\wku) 
    & = & \iint_{\hat f} \hat{x}_2\zeta^2\,d\hat S\textrm{,}
\end{IEEEeqnarray*}
and it follows that
$\hat{x}_3\,(\hat{x}_3-1)$ divides $\zeta$. So putting this together
with the previous factorization, there is some
$r \in P_{k-2}(\hat{f}_3)\otimes P_{k-2}(\hat{x}_3)$ which satisfies.
\begin{IEEEeqnarray*}{rCl}
    \hat\xi_1\xyz &=& \hat{x}_2\,\hat{x}_3\,(\hat{x}_3-1)\,r\xyz.
\end{IEEEeqnarray*}
It remains to use the volume degrees of freedom. We could choose
 $\hat\br := (r,0,0)'$ in degree of freedom~(\ref{momentos6hcurl})
to get
\begin{IEEEeqnarray*}{rCCCCCl}
  0 & = & \hat\varphi_{\hat\br}\,(\hat{\bu}) & = & \hat\varphi_{\hat\br}\,(\wku)
    & = & \int_{\hat{E}} \hat{x}_2\,\hat{x}_3\,(\hat{x}_3-1)\,r\xyz^2\,d\hat\bx\textrm{,} 
\end{IEEEeqnarray*}
which yields, over all $\hat{E}$, $\hat\xi_1  \equiv  0$. Finally, if we
combine this last property with~(\ref{rot_3_es_0}) we prove that $\hat{\xi}_2$
does not depend on $\hat{x}_2$.
\end{proof}
\begin{lemma}\label{pi00u3} 
If $\hat{\bu}\xyz=(0,0, \hat{u}_3\xyz)^t$, then
$\wku\xyz = (0,0,\hat\xi_3\xyz)^t$ for some
$\hat\xi_3 \in {P}_k(\hat{f}_3)\otimes
{P}_{k-1}(\hat{x}_3)$.
\end{lemma}
\begin{proof} We will work again with
expression~(\ref{expr_wku}).
%but we will write with less details than in the previous two Lemmas
By expression~(\ref{expre_h}) for $(\curl\wku)_3$
and the commutativity in equation~(\ref{curl_commutativity}),
if we apply degrees of freedom~(\ref{momentos1hdiv})
to $\curl\hat\bu$ we obtain that
$(\curl\wku)_3$ vanishes on any of the horizontal faces $\hat{f}_3$ or $\hat{f}_4$ in
Table~\ref{prismNotationTableFaces}.
In other words, $(\curl\wku)_3
= \hat{x}_3\,(\hat{x}_3-1)\,\hat\psi$, 
($\hat\psi\in P_{k-1}(\hat{f}_3)\otimes P_{k-2}(\hat{x}_3)$)
and if we set $\hat{\br} := (0,0,\hat\psi)'$ in the
$H(\mbox{div})$ degrees of freedom~(\ref{momentos4hdiv})
we have
\begin{IEEEeqnarray*}{rCCCCCCCl}
0 & = & \int_{\hat{E}} (\curl\hat{\bu})_3\,\hat{\psi}
  & = & \hat\rho_{\br} (\curl\hat{\bu})
  & = & \hat\rho_{\br} (\hat{\br}_k\curl\hat{\bu})
  & = & \hat\rho_{\br} (\curl\wku)\\[4pt]
  &&&&&&& = & \int_{\hat{E}} \hat{x}_3(1-\hat{x}_3)\hat{\psi}^2\,d\bx
\end{IEEEeqnarray*}
yielding that $\hat\psi$ is identically zero, and also
$(\curl\wku)_3$ is identically zero.

From this point, if
we copy the argument in the proof of
Lemma~\ref{lemma_PIu2_k_in_N} starting with equation~(\ref{expre_h}) we arrive at
$h\equiv 0$, so we may rewrite~(\ref{expr_wku}) for the present case as
\begin{IEEEeqnarray}{rCl}
  \label{expre_pi00u3_} \wku &=&
  (\hat\xi_1,\hat\xi_2,\hat\xi_3)^t.
\end{IEEEeqnarray}
We claim that $\hat{\xi}_1\equiv\hat{\xi}_2\equiv0$.
To see this, first observe that the evaluation of the degrees of freedom
for the edges $\hat\be_1$ and $\hat\be_2$ yields
$\hat\xi_1|_{\hat\be_1} \equiv \hat\xi_2|_{\hat\be_2} \equiv 0$,
hence, evaluating the degree of freedom~(\ref{momentos2hcurl})
tangent to the face $\hat{f}_3$ two times we have
$\hat\xi_1|_{\hat{f}_3}  \equiv  \hat\xi_2|_{\hat{f}_3}  \equiv  0$.
In equal manner, if we pick $\hat\be_4$ and $\hat\be_5$, and then the 
degree of freedom tangent to $\hat{f}_4$ we obtain
$\hat\xi_1|_{\hat{f}_3} \equiv \hat\xi_2|_{\hat{f}_3} \equiv  0$.
So we proved there are polynomials $p_1$ and $p_2$ in
$P_{k-1}(\hat{f}_3)\otimes P_{k-2}(\hat{x}_3)$ which allow us to write
\begin{IEEEeqnarray*}{rCl}
  \hat\xi_1\xyz & = & \hat{x}_3(1-\hat{x}_3)p_1\xyz\\[4pt]
  \hat\xi_2\xyz & = & \hat{x}_3(1-\hat{x}_3)p_2\xyz.
\end{IEEEeqnarray*}
Take $\hat\bq := (0,0, \hat{q}_2|_{\hat{f}_2})'$ and 
evaluate the degree of freedom~(\ref{momentos3hcurl}). We have
\begin{IEEEeqnarray*}{rCCCCCl}
  0 & = & \hat\varphi_{\hat{f}_1,\hat{\bq}}\,(\hat\bu) 
    & = & \hat\varphi_{\hat{f}_1,\hat{\bq}}\,(\wku) 
    & = & \iint_{\hat{f}_1} \hat{x}_3(1-\hat{x}_3)\hat{q}_2^2\,d\hat S.
\end{IEEEeqnarray*}
Hence, there is some $\hat{r}_2\in P_{k-2}(\hat{f_3})\otimes P_{k-2}(\hat{x}_3)$
such that $\hat\xi_2 = \hat{x}_1\hat{x}_3(1-\hat{x}_3)\hat{r}_2$.
Now choose $\br = (0,\hat{r}_2,0)'$ and use degree of freedom~(\ref{momentos6hcurl})
to obtain $\int_{\hat{E}}\hat{x}_1\hat{x}_3(1-\hat{x}_3)\hat{r}_2^2\,d\hat\bx=0.$
Since 
$\hat{x}_1\hat{x}_3(1-\hat{x}_3)\hat{r}_2^2$ is almost everywhere greater than zero,
this implies
$\hat{\xi}_2 = 0$.
With the simmetric procedure starting with face $\hat f_2$ we get to prove
$\hat{\xi}_1 = 0$.
\end{proof}
Now here is our first important result.
\begin{theorem}\label{thm_stab_edge}
Given $p > 2$, $\hat{\bu} \in \wpcurl{\hat{E}}$,
\begin{IEEEeqnarray}{rCl}
\label{teorema_1} \norm{(\wku)_1}_{L^{\infty}(\hat{E})} & 
	\lesssim & \|\hat{u}_1\|_{W^{1,p}(\hat{E})} + 
	\|(\curl\hat{\bu})_3\|_{W^{1,1}(\hat{E})} \\	
\label{teorema_2} \norm{(\wku)_2}_{L^{\infty}(\hat{E})} & 
	\lesssim & \|\hat{u}_2\|_{W^{1,p}(\hat{E})} + 
	\|(\curl\hat{\bu})_3\|_{W^{1,1}(\hat{E})} \\	
\label{teorema_3} \norm{(\wku)_3}_{L^{\infty}(\hat{E})} & 
	\lesssim & \|\hat{u}_3\|_{W^{1,p}(\hat{E})}
\end{IEEEeqnarray}
where the constants in the inequalities depend only on $\hat{E}$.
\end{theorem}
\begin{proof}
The proof will rely on the three previous Lemmas, 
the triangular inequality applied on each component of 
expression~(\ref{edge_interp_explicit}) and traces inequalities or,
more precisely, the proof
of Lemma $5.38$ in the page $134$ of~\cite{monk}
and Theorem $3.9$ (\emph{Trace Theorem})
in page $43$ of
the same book.
First we will take a smooth field $\hat{\bu}$ defined on $\hat{E}$
and, by Lemma~\ref{density_wpcurl}, we will conclude the Theorem 
with a density argumentation.\\[4pt]
To prove~(\ref{teorema_1}) the idea will be to take another function
$\hat{\bw}$ such that its interpolate has the same first component
as the one of $\hat\bu$ and such that its degrees of freedom are
more easily bounded in terms of $\hat{u}_1$ and $\curl(\hat{\bu})_3$.

Let us define, for a given $\hat{\bu} \in C^\infty(\bar{\hat{E}})^3$,
$\hat{\bv}\,:\,\hat{E}\to\mathbb{R}^3$ with
\[
  \hat{\bv}\xyz = (\hat{u}_1\xyz, \hat{u}_2\xyz - \hat{u}_2(0,\hat{x}_2,\hat{x}_3), 0)'.
\]
Thanks to the Lemmas~\ref{lemma_PIu2_k_in_N} and~\ref{pi00u3} it holds
\begin{IEEEeqnarray*}{rCl}
	(\hat{\bw}_k\hat{\bv})_1 & = & (\wku)_1 - 
	\hat{\bw}_k(0, \hat{u}_2(0,\hat{x}_2,\hat{x}_3), 0)_1 -
	\hat{\bw}_k(0, 0, \hat{u}_3)_1\\
						& = & (\wku)_1\mbox{,}
\end{IEEEeqnarray*}
and we also have $(\curl\hat{\bu})_3 = (\curl\hat{\bv})_3$.
Now let us explore one by one the degrees of freedom that define
$\hat{\bw}_k\hat{\bv}$. The only edge degrees
that do not vanish directly or depend explicitly just on 
$\hat{u}_1$ are
\begin{IEEEeqnarray*}{rCl}
	\int_{\hat{\be}_8} q\,\hat{\bv}\cdot d\hat\balpha & = &
	\tfrac{1}{\sqrt{2}} \int_{\hat{\be}_8} (\hat{v}_1 - \hat{v}_2)\,q\,d\alpha\\
	\int_{\hat{\be}_9} q\,\hat{\bv}\cdot d\hat\balpha & = &
	\tfrac{1}{\sqrt{2}} \int_{\hat{\be}_9} (\hat{v}_1 - \hat{v}_2)\,q\,d\alpha
\end{IEEEeqnarray*}
for $q$ in $\pazocal{P}_{k-1}(\hat{\be}_8)$ or $\pazocal{P}_{k-1}(\hat{\be}_9)$ 
respectively. 
Pick a polynomial $q \in P_{k-1}(\hat{\be}_8)$. Since on
$\hat{\be}_8$ it is $\hat{x}_1 = 1 - \hat{x}_2$, we evaluate $q$ as
$q(\hat{x}_2)$, with $0\leqslant\hat{x}_2 \leqslant 1$. Integration
by parts over the face $\hat{f}_4$ yields
\begin{IEEEeqnarray*}{rCl}
  \iint_{\hat{f}_4} (\curl\hat{\bv})_3\,q\,d\hat S
	& = & -\iint_{\hat{f}_4} \left(\hat{v}_2\,\partial_{\hat{x}_1}q - \hat{v}_1\,
  \partial_{\hat{x}_2}q\right)\,d\hat{S}
		+ \int_{\partial \hat{f}_4} \left(\hat{v}_2\,\hat{\nu}_1 
    - \hat{v}_1\,\hat{\nu}_2\right)\,q\,d\hat\alpha\\
	& = & \iint_{\hat{f}_4} \hat{v}_1\,\partial_{\hat{x}_2}q\,d\hat S
		+ \int_{\hat{\be}_8} \left(\hat{v}_2 - \hat{v}_1\right)\,q\,d\hat\alpha + 
			\int_{\hat{\be}_4} \hat{v}_1\,q\,d\hat\alpha\mbox{,}
\end{IEEEeqnarray*}
hence
\begin{IEEEeqnarray*}{rCl}
	\hat\varphi_{\hat{\be}_8,\,q}(\hat\bv) & = &
  \dfrac{1}{\sqrt{2}} \int_{\hat{\be}_8} (\hat{v}_1 - \hat{v}_2)\,q\,d\hat\alpha\\
    & = &\tfrac{1}{\sqrt{2}} \int_{\hat{\be}_4} \hat{u}_1\,q\,d\hat\alpha - 
    \tfrac{1}{\sqrt{2}} \iint_{\hat{f}_4} (\curl\hat{\bu})_3\,q\,d\hat{S}
    + \iint_{\hat{f}_3} \hat{u}_1\,\partial_{\hat{x}_2}q\,d\hat{S}.\\
	\yesnumber\label{momentosWaristas}
    &&
\end{IEEEeqnarray*}
In a similar manner if we integrate over $\hat{f}_3 \subseteq \{ \hat{x}_3 = 0 \}$
we get
\begin{IEEEeqnarray*}{rCl}
	\hat\varphi_{\hat{\be}_9,\,q}(\hat\bv) & = & \tfrac{1}{\sqrt{2}} 
  \int_{\hat{\be}_9} (\hat{v}_1 - \hat{v}_2)\,q\,d\hat\alpha \\
	\yesnumber\label{momentosWaristas2}
  	 &=&\tfrac{1}{\sqrt{2}} \int_{\hat{\be}_1} \hat u_1\,q\,d\hat\alpha -
		 \tfrac{1}{\sqrt{2}} \iint_{\hat{f}_3} (\curl\hat{\bu})_3\,q\,d\hat{S}
		 + \iint_{\hat{f}_3} \hat{u}_1\,\partial_{\hat{x}_2}q\,d\hat{S}.
\end{IEEEeqnarray*}
If we evaluate now the face degrees of freedom, we only have to bound
those corresponding to $\hat{f}_3$, $\hat{f}_4$ and $\hat{f}_5$.
Take $\hat{q}_1$, $\hat{q}_2 \in P_{k-2}(\hat{f}_3)$ and consider $\hat{\bq} := (\hat{q}_1, \hat{q}_2, 0)$.
\begin{IEEEeqnarray}{rCl}
 	\label{cotaf3}\iint_{\hat{f}_3} \hat{\bv} \times \hat\bn \cdot \hat\bq\,d\hat{S}
 		& = & \iint_{\hat{f}_3} \hat u_1\,\hat q_2\,d\hat{S} -
    \iint_{\hat f_3} \hat{v}_2\,\hat q_1\,d\hat{S}.
\end{IEEEeqnarray}
Observe that $\hat{v}_2$ vanishes over the face $\hat{f}_1\subseteq\{\hat{x}_1=0\}$.
Now we need a polynomial $\varphi \in P_{k-1}(\hat{f}_3) $ such that 
$\partial_{\hat{x}_1} \varphi = \hat{q}_1$ and
$\varphi |_{\hat{\be}_9} = 0$; take for instance
$\varphi(\hat{x}_1,\hat{x}_2) = -\int_{\hat{x}_1}^{1-\hat{x}_2} \phi_1(t,\hat{x}_2)\,dt$. Then
\noindent{\color{BrickRed}\#\#\#\#\#\#\# seguir aca} 
\begin{IEEEeqnarray*}{rCl}
	\iint_{\hat{f}_3} (\curl\hat{\bv})_3\,\varphi\,d\hat{S} & = & -\iint_{\hat{f}_3} \left(\hat{v}_2\,\phi_1 - \hat{v}_1\,\partial_{\hat{x}_2}\varphi\right)\,d\hat{S}
		- \int_{\hat{\be}_1} \hat{v}_1\,\nu_y\,\varphi\,ds,
\end{IEEEeqnarray*}
which, together with~(\ref{cotaf3}) implies
\begin{IEEEeqnarray}{rCl}\label{momentosWcaras}
  \varphi_{\hat{f}_3,\,\hat{\bq}}(\hat{\bv})
 		& = & \iint_{\hat{f}_3} \hat{u}_1\,\phi_2\,d\hat{S} +
    \iint_{\hat{f}_3} (\curl\hat{\bu})_3\,\varphi\,d\hat{S} - 
    \iint_{\hat{f}_3} u_1\,\partial_{\hat{x}_2}\varphi\,d\hat{S}	+
    \int_{\hat{\be}_1} \hat{u}_1\,\nu_2\,\varphi\,ds.
\end{IEEEeqnarray}
If we repeated the procedure for the degree of freedom on $\hat{f}_4$, for a given 
$\bp = (p_1,p_2,0) \in P_{k-2}(\hat{f})^2\times \{0\}$ we would set 
$\psi (\hat{x}_1,\hat{x}_2) = \int_{1-\hat{x}_2}^{\hat{x}_1} p_1 (t,\hat{x}_2)\,dt$
and had
\begin{IEEEeqnarray}{rCl}\label{momentosWcaras2}
 	\varphi_{\hat{f}_4,\,\hat{\bp}}(\hat{\bv})
 		& = & - \iint_{\hat{f}_4} \hat{u}_1\,\hat{p}_2\,d\hat{S} -
    \iint_{\hat{f}_4} (\curl\hat{\bu})_3\,\psi\,d\hat{S}
    + \iint_{\hat{f}_4} \hat{u}_1\,\partial_{\hat{x}_2}\psi\,d\hat{S}	-
    \int_{\hat{\be}_4} \hat{u}_1\,\hat{\nu}_2\,\psi\,d\hat{s}.
\end{IEEEeqnarray}
For~(\ref{momentos5hcurl}) on $\hat{f}_5$, given
$\bq = (0,q_3,q_1) \in \{ 0 \} \times Q_{k-2,k-1} \times Q_{k-1,k-2}$ 
\begin{IEEEeqnarray}{rCl}\label{momentosWcaras3}
  \iint_{\hat{f}_5} \hat{\bv} \times \boldsymbol{\nu} \cdot \hat{\bq}\,d\hat{S}
    & = & \iint_{\hat{f}_5} (\hat{v}_1 - \hat{v}_2)\,\hat{q}_1\,d\hat{S}
\end{IEEEeqnarray}
Now, if $\hat{q}$ is the extension of $\hat{q}_1$ to the whole prism, then
\begin{IEEEeqnarray*}{rCl}
  \iint_{\hat{f}_5} \hat{v}_2\,\hat{q}_1\,d\hat{S} & = &
  \sqrt{2} \iint\limits_{[0,1]^2}\hat{v}_2(1-\hat{x}_2,\hat{x}_2,\hat{x}_3)\hat{q}_1(\hat{x}_2,\hat{x}_3)\,d\hat{x}_2d\hat{x}_3\\[5pt]
  &=&\sqrt{2} \iint\limits_{[0,1]^2}\int_{0}^{1-\hat{x}_2}\tfrac{\partial\hat{v}_2}{\partial{\hat{x}_1}}
  (\hat{t},\hat{x}_2,\hat{x}_3)\hat{q}(\hat{t}, \hat{x}_2,\hat{x}_3)\,d\hat{t}d\hat{x}_2d\hat{x}_3\\[5pt]
  \yesnumber\label{momentosWcaras3_}
  &=&\sqrt{2}\int_{\hat{E}} (\curl\bv)_3\hat{q}\,d\hat{\bx} + 
  \sqrt{2}\int_{\hat{E}} \tfrac{\partial\hat{v}_1}{\partial{\hat{x}_2}}\hat{q}\,d\hat{\bx}.
\end{IEEEeqnarray*}
Joining~(\ref{momentosWcaras3}) and~(\ref{momentosWcaras3_}) and using the
expression for $\hat{\bv}$ we get 
\begin{IEEEeqnarray}{rCl}\label{momentosWcaras3}
  \varphi_{\hat{f}_5,\,\hat{\bq}}(\hat{\bv})
  & = & \iint_{\hat{f}_5} \hat{u}_1\,\hat{q}_1\,d\hat{S}
  -\sqrt{2}\int_{\hat{E}} \tfrac{\partial\hat{u}_1}{\partial{\hat{x}_2}}\hat{q}\,d\hat{\bx}
  -\sqrt{2}\int_{\hat{E}} (\curl\bu)_3\hat{q}\,d\hat{\bx}. 
\end{IEEEeqnarray}
At last, we study the volume degrees of freedom. Pick
$\br = (r_1, r_2, r_3) \in (P_{k-2}(\hat{f}_3) \otimes P_{k-2}(\hat{x}_3))^{2}
\times P_{k-3}(\hat{f}_3) \otimes
P_{k-1}(\hat{x}_3)$ (cfr.~(\ref{momentos6hcurl})) 
and let $\varphi_2$ be defined in such a way that
$\varphi_2\xyz = \int_{1-\hat{x}_2}^{\hat{x}_1} 
\hat{r}_2(\hat{t},\hat{x}_2,\hat{x}_3)\,d\hat{t}$.
%, para el cual
%vale 
%$\partial_x\varphi_2 = \hat{r}_2 $
%y $\varphi_2|_{\hat{f}_5} \equiv 0$.
Green's Theorem and the fact that $\varphi_2|_{\hat{f}_5} \equiv 0$
give 
\begin{IEEEeqnarray}{rCl}\label{momentosWvolumen}
 	\int_{\hat{E}} \hat{\bv} \cdot \br\,d\bx
 	& = & \int_{\hat{E}} \hat{u}_1\,\hat{r}_1\,d\bx 
  - \int_{\hat{E}} (\curl\hat{\bu})_3\,
 	\varphi_2\,d\bx
  - \int_{\hat{E}}
  \dfrac{\partial\hat{u}_1}{\partial\hat{x}_2}\,
  \varphi_2\,d\bx.
\end{IEEEeqnarray}
%,~(\ref{momentosWcaras2}),~(\ref{momentosWcaras3})
Now we collect what has been said so far.
%%%%%%%%%%%%%%%%%%%%%%%%% equalities,,~(\ref{momentosWcaras}),~(\ref{momentosWcaras2}),~(\ref{momentosWcaras3}) and~(\ref{momentosWvolumen}).
For the edge degrees of freedom we use an inequality in page $135$ of~\cite{monk},
in the proof of Lemma 5.38, which states, for $\bu$ in the present conditions,
\begin{IEEEeqnarray}{rCl}\label{edgeTrace}
  \left|\int_{\be} \bu\cdot\boldsymbol{\tau}\,q\,d\sigma\right| 
  & \leqslant & C(q) \,\{\, \|\curl\bu\|_{L^p(\hat{E})^3}
    + \|\mbox{Tr}\,\bu\|_{L^p(\partial\hat{E})^3} \}.
\end{IEEEeqnarray}
If we put the field $(\hat{u}_1,0,0)^t$ in inequality~(\ref{edgeTrace}) then
by, H\"older's Inequality and standard traces inequalities, equation~(\ref{momentosWaristas})
and~(\ref{momentosWaristas2}) yield, for $i=8,9$,
\begin{IEEEeqnarray*}{rCl}
  \left|\varphi_{\hat{\be}_i,\,\hat{q}}(\hat\bv)\right| & \leqslant & c(\hat{q})\,
  \{\,\|\hat{u}_1\|_{W^{1,p}(\hat{E})} + \|\mbox{Tr}\,(\curl{\hat{\bu}})_3\|_{L^1(\partial\hat{E})}
  +\|\mbox{Tr}\,\hat{u}_1\|_{L^p(\partial\hat{E})}\,\}\\[5pt]
  \yesnumber\label{traceE8}
  & \leqslant & c(\hat{q})\,\{\,\|\hat{u}_1\|_{W^{1,p}(\hat{E})} + 
  \|(\curl\hat{\bu})_3\|_{W^{1,1}(\hat{E})}\,\}.
\end{IEEEeqnarray*}
If we repeat the trick for the line integral
terms in~(\ref{momentosWcaras}) and~(\ref{momentosWcaras2}) we get, for $j=3,4$,
\begin{IEEEeqnarray}{rCl}\label{traceF3}
  \left|\varphi_{\hat{f}_j,\,\hat{\bq}}(\hat\bv)\right| & \leqslant &
  c(\hat{\bq})\,\{\,
    \|\mbox{Tr}\,\hat{u}_1\|_{L^p(\partial\hat{E})} +
    \|\mbox{Tr}\,(\curl{\hat{\bu}})_3\|_{L^1(\partial\hat{E})} +
    |\hat{u}_1|_{W^{1,p}(\hat{E})}\,\}.
\end{IEEEeqnarray}
And finally, by estimates~(\ref{momentosWcaras3})--(\ref{traceF3})
and one more time H\"older's and traces inequalities,
\begin{IEEEeqnarray*}{rCCCl}
	\|(\wku)_1\|_{L^\infty(\hat{E})} & = & \|(\hat{\bw}_k\hat{\bv})_1\|_{L^\infty(\hat{E})}
  &\lesssim&\\
  \IEEEeqnarraymulticol{5}{C}{
  \sum_{i=8,9,\,p} |\varphi_{\hat{\be}_i,p}(\hat{\bv})|\,\|(\hat{\bv}_{\hat{\be}_i,p})_1\|_{L^\infty(\hat{E})} +
  \sum_{j=3,4,\,\bq} |\varphi_{\hat{f}_j,\bq}(\hat{\bv})|\,\|(\hat{\bv}_{f_j,\bq})_1\|_{L^\infty(\hat{E})} +
  \sum_{\br} |\varphi_{\br}(\hat{\bv})|\,\|(\hat{\bv}_{\br})_1\|_{L^\infty(\hat{E})}
  }\\
	& \leqslant & c(\hat{E})\,\{\,\|\hat{u}_1\|_{W^{1,p}(\hat{E})} & + &
		\|(\curl\hat{\bu})_3\|_{W^{1,1}(\hat{E})}\,\}
\end{IEEEeqnarray*}
which is the bound we wanted to prove. The same procedure applies for 
inequality~(\ref{teorema_2}).\\[7pt]
For inequality~(\ref{teorema_3}) given $\hat{\bu} \in W^{1,p}(\hat{E})^3$, define
$\hat{\bv}  =  (0,0, \hat{u}_3).$
Thanks to Lemma~\ref{lema_PIu3_k_cualquiera} we have 
$\hat{\bw}_k(\hat{\bv})_3 = (\hat{\bw}_k\hat{\bu})_3 - (\hat{\bw}_k(\hat{u}_1, \hat{u}_2, 0))_3 = (\hat{\bw}_k\hat{\bu})_3.$
By expression~(\ref{edge_interp_explicit}), taking another look at 
the unit tangent vector of the edges and unit normal vectors to the
faces, we have
\begin{IEEEeqnarray*}{rCl}
  (\hat{\boldsymbol{w}}_k\hat{\bv})_3 & = &
  \sum_{j=3,6,7\,\hat{\bp}\,\in\,\mathcal{B}_{\hat e_j}}
  \int_{\hat e_j} \hat{u}_3 \hat{p}_3\,ds \,(\hat{\bv}_{\hat{\be}_j,\hat{\bp}})_3 +
  \sum_{i=1,2,4\,q\,\in\,{\color{red}\mathcal{B}_{\hat f_i}}}
  \int_{\hat f_i} \hat{u}_3 q\,d\gamma \,(\hat{\bv}_{\hat{f}_i,q})_3\\
  &&\,+\sum_{\boldsymbol{r}\,\in\,{\color{red}\mathcal{B}_{\hat E}}}
  \int_{\hat E} \hat{u}_3 r_3\,d\bx\,(\hat{\bv}_{\boldsymbol{r}})_3.
\end{IEEEeqnarray*}
This implies, by traces inequalities and~(\ref{edgeTrace}), that
\begin{IEEEeqnarray*}{rCl}
  \norm{(\hat{\boldsymbol{w}}_k\hat{\bu})_3}_{L^{\infty}(\hat{E})}
  & \leqslant & c(\hat{E})
  \sum_{j=3,6,7\,\hat{\bp}\,\in\,\mathcal{B}_{\hat e_j}}
  \left|\int_{\hat e_j} \hat{u}_3\,\hat{p}_3\,ds\right| +
  \sum_{i=1,2,4}
  \int_{\hat f_i} |\hat{u}_3|^p\,d\gamma\\
  &&\,+ \int_{\hat E} |\hat{u}_3|^p\,d\bx\\
  &\leqslant& c(\hat{E}) \|\hat{u}_3\|_{W^{1,p}(\hat{E})}.\\[5pt]
\end{IEEEeqnarray*}
The constants in the three inequalities of this Theorem depend,
clearly, on the choice of the bases $\mathcal{B}_{\hat e_j},
\mathcal{B}_{\hat f_i}, \mathcal{B}_{\hat E}$.
\end{proof}