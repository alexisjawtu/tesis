\chapter{Interpolation}
\begin{table}[h!]
\centering
\caption{Notation}
\label{edgeFacesNotationTable}
    \begin{IEEEeqnarraybox}
    [\IEEEeqnarraystrutmode
     \IEEEeqnarraystrutsizeadd{0pt}{0pt}]{v/c/v/c/v/c/v/c/v}
        \IEEEeqnarrayrulerow\\
        \IEEEeqnarrayseprow[3pt]\\
        &\hfill\raisebox{22pt}[0pt][0pt]{}\hfill
                            & & ***
                            & & ***
                            & & *** &\\
        \IEEEeqnarrayrulerow\\
        \IEEEeqnarrayseprow[3pt]\\
        &\hfill\raisebox{30pt}[0pt][0pt]{***}\hfill& &
            \begin{IEEEeqnarraybox}{c}
            ***\\***
            \end{IEEEeqnarraybox}
        &&&&
            \begin{IEEEeqnarraybox}{c}
                ***\\***
            \end{IEEEeqnarraybox}
        &\\
        \IEEEeqnarrayseprow[3pt]\\
        \IEEEeqnarrayrulerow
    \end{IEEEeqnarraybox}
\end{table}
\begin{IEEEeqnarray*}{rClCrClCrCl}
  \hat f_1 & \subseteq &  \{\hat x_1 = 0      \} & \qquad & \hat \be_1& = & \{(\hat x_1,0,0)^t\,:\,0\leqslant\hat x_1\leqslant 1\}  &\quad& \hat \be_6& = & \{(1,0,\hat x_3)^t\,:\,0\leqslant\hat x_3\leqslant 1\} \\[4pt]
  \hat f_2 & \subseteq &  \{\hat x_2 = 0      \} & \qquad & \hat \be_2& = & \{(0,\hat x_2,0)^t\,:\,0\leqslant\hat x_2\leqslant 1\}  &\quad& \hat \be_7& = & \{(0,1,\hat x_3)^t\,:\,0\leqslant\hat x_3\leqslant 1\} \\[4pt]
  \hat f_3 & \subseteq &  \{\hat x_3 = 0      \} & \qquad & \hat \be_3& = & \{(0,0,\hat x_3)^t\,:\,0\leqslant\hat x_3\leqslant 1\}  &\quad& \hat \be_8& = & \{(\hat x_1,1-\hat x_1,1)^t\,:\,0\leqslant\hat x_1\leqslant 1\} \\[4pt]
  \hat f_4 & \subseteq &  \{\hat x_3 = 1      \} & \qquad & \hat \be_4& = & \{(\hat x_1,0,1)^t\,:\,0\leqslant\hat x_1\leqslant 1\}  &\quad& \hat \be_9& = & \{(\hat x_1,1-\hat x_1,0)^t\,:\,0\leqslant\hat x_1\leqslant 1\} \\[4pt]
  \hat f_5 & \subseteq &  \{\hat x_1+\hat x_2 = 1\}  & \qquad & \hat \be_5& = & \{(0,\hat x_2,1)^t\,:\,0\leqslant\hat x_2\leqslant 1\}  &     &           &           &
\end{IEEEeqnarray*}
\section{Stability of the Edge Element in $\hat{K}$}
In the next three Lemmas $\hat\bu$ is an element
in $H({\bf curl},\hat{E})$ and we will assume there are 
a positive $\delta$ and a $p>2$ such that 
$\hat\bu$ belongs to $H^{1/2+\delta}(\hat{E})^3$ and
${\bf curl}\,\bu$ belongs to $L^p(\hat{E})^3$.
\begin{lemma}\label{lema_PIu3_k_cualquiera} 
$(\wku)_3$ is linearly and univocally 
determined by $\hat{u}_3$.
\end{lemma}
\begin{proof} If we pay attention to the directions of the unit
tangents and normals to the edges and faces, respectively, of $\hat E$,
we realize that
the degrees of freedom which involve $\wku_3$ give rise onnlyto the 
following linear equations
\begin{IEEEeqnarray}{rCrc}
\varphi_{e_i,p}\,(\wku) & = & \varphi_{e_i,p}\,(\hat{\bu}) &\quad\mbox{for $i$ = 3, 6, 7 and }p\in\mathcal{}  \\
\varphi_{f_j,\boldsymbol{q}}\,(\wku) & = & \varphi_{f_j,\boldsymbol{q}}\,(\hat{\bu})
  &\quad\mbox{for $j$ = 1, 2, 4 and }\boldsymbol{q}\in\mathcal{}  \\
\varphi_{\boldsymbol{r}}\,(\wku) & = & \varphi_{\boldsymbol{r}}\,(\hat{\bu})
  &\quad\mbox{for }\boldsymbol{r}\in\mathcal{}.
\end{IEEEeqnarray}
These are 
$3k$+$3k(k-1)$+$k(k-1)(k-2)/2 = k(k+1)(k+2)/2$ equations,
just the dimension of $P_k(\hat{T})\otimes P_{k-1}(\hat{I})$, 
which is the space $(\wku)_3$ belongs to by definition.
%$\frac{k(k+1)(k+2)}{2}$. 
%libertad en los que no des\-a\-pa\-re\-ce $(\pi\textbf{u})_3$ son \'unicamente:
Now set all those equations to zero and see that the unique solution is $(\wku)_3 = 0$.
A little more explicitly, we have:
\noindent{\color{blue}\#\#\#\#\#\#\# } 
para cada arista $e_j$, $j = 3, 6, 7$,
\begin{IEEEeqnarray}{lClc}
  \label{aristas} \int\limits_{e_j} (\pi\textbf{u})_3 q \, ds 
  & = & 0 &\quad q\in P_{k-1}(e_j).
\end{IEEEeqnarray}
Para cada cara vertical $f=f_1$, $f_2$, $f_4$,
\begin{IEEEeqnarray}{lClc}
	\label{caras} \int\limits_{f} (\pi\textbf{u})_3 q \, 
	d\gamma & = & 0 &\quad q\in Q_{k-2,k-1}(f_{ijkl}).
\end{IEEEeqnarray}
En $\hat{K}$
\begin{IEEEeqnarray}{lClc}
	\label{enK} \int\limits_{\hat{K}} (\pi\textbf{u})_3 
	q \, d\textbf{x} & = & 0 &\quad q\in P_{k-3}(f_z) \otimes P_{k-1}(e_{xy}).
\end{IEEEeqnarray}
\noindent{\color{blue}\#\#\#\#\#\#\# }
 of freedom que estos grados de libertad se anulan todos y veamos que de esta suposici\'on
se deduce que $(\pi\textbf{u})_3 = 0$, es decir que las ecuaciones li\-nea\-les mencionadas determinan un\'ivocamente a un elemento
$(\pi\textbf{u})_3$ pertenenciente a 
$P_k(\hat{T}) \otimes P_{k-1}(\hat{I})$. 

Empezamos por ver que $(\pi\textbf{u})_3$ se anula en ciertos subconjuntos de 
$\partial \hat{K} $. La restricci\'on de $(\pi\textbf{u})_3$ a $e_j, (j=3,6,7)$
es un elemento de $P_{k-1}(e_j)$. Entonces si usamos los degrees of freedom~(\ref{aristas}) obtenemos
\begin{IEEEeqnarray}{rCl}
	\label{restriccAristas}\int\limits_{e_j} [(\pi\textbf{u})_3]^2\, ds & = & 0,
\end{IEEEeqnarray}
es decir que $(\pi\textbf{u})_3$ es id\'enticamente cero en esas aristas. Con esto la restricci\'on $(\pi\textbf{u})_3|_{f_2}$ 
que pertenece a $P_k(x)\otimes P_{k-1}(z)$ se factoriza como 
\begin{IEEEeqnarray*}{rCl}
	v(x,0,z) 	& = 	& x\,(x-1)\,w_0(x,z),\\
	w_0 		& \in 	& P_{k-2}(x)\otimes P_{k-1}(z),
\end{IEEEeqnarray*}
pero este \'ultimo es precisamente el espacio de los degrees of freedom~(\ref{caras}) con los 
cuales llegamos a
\begin{IEEEeqnarray}{lClc}
	\int\limits_{f_2} x(x-1)[w_0(x,z)]^2\,d\gamma & = & 0.
\end{IEEEeqnarray}
Entonces en $f_2$ es $w_0 \equiv 0$ y $(\pi\textbf{u})_3|_{f_2} \equiv 0$. Por simetr\'ia, el argumento para probar que
$(\pi\textbf{u})_3|_{f_1} \equiv 0$ es exactamente igual.

Si volvemos a usar las igualdades~(\ref{restriccAristas}) obtenemos que la restricc\'on de $(\pi\textbf{u})_3$ a
$f_{4} = \{ (x,y,z) \;:\; 0\leqslant x\leqslant 1,\, y = 1 - x, \, 0\leqslant z\leqslant 1 \}$ se anula si $x=1-y=0$ o bien si $x=1-y=1$;
\begin{IEEEeqnarray}{rCl}
	v(x,1-x,z) 	& = 	& x\,(1-x)\,w_1(x,z)\\
	w_1 		& \in 	& P_{k-2}(x)\otimes P_{k-1}(z).
\end{IEEEeqnarray}
Si aplicamos los degrees of freedom~(\ref{caras}) llegamos 
a que en ${f_4}$ es $w_1$ id\'enticamente cero y as\'i tambi\'en $(\pi\textbf{u})_3$.\\
Recapitulando, 
\begin{IEEEeqnarray*}{rCl}
	(\pi\textbf{u})_3|_{f_j} & \equiv & 0\quad\text{ para } j = 1,2,4.
\end{IEEEeqnarray*}
Entonces en todo $\hat{K} $ tenemos la factorizaci\'on 
\begin{IEEEeqnarray*}{rCl}
	(\pi\textbf{u})_3(x,y,z) 	& = 	& x\,y\,(1-x-y)\,w_3(x,y,z),\\
						w_3		& \in 	& P_{k-3}(x)\otimes P_{k-1}(z).
\end{IEEEeqnarray*}
Simplemente aplicando los degrees of freedom~(\ref{enK}) nos queda que $w_3 \equiv 0$, y finalmente
$(\pi \textbf{u})_3 \equiv 0$.
%	debe anularse en todo $\hat{K}^{\textrm{o}}$, y por continuidad tambi\'en en 
%$\partial\hat{K}$. Si volvemos a la expresi\'on~(\ref{piu3}) nos queda que
\end{proof}
\begin{lemma}\label{lemma_PIu2_k_in_N}
\begin{IEEEeqnarray*}{rCl}
\label{piu2_k_in_N}
	\yesnumber\pi(0, u_2(y,z), 0) & = 	& (0, \xi_2(y,z) ,0)\textrm{,}\\
\label{piu1_k_in_N}	
	\yesnumber\pi(u_1(x,z), 0, 0) & = 	& (\xi_1(x,z), 0 ,0)\textrm{,}\\
	\xi_1 				& \in 	& P_{k-1}(x) \otimes P_k(z)\textrm{,}\\
	\xi_2 				& \in 	& P_{k-1}(y) \otimes P_k(z).
\end{IEEEeqnarray*}
\end{lemma}
\begin{proof} Demostramos s\'olo la primera igualdad, porque la otra es
an\'aloga. Lo que hay que ver es que, en la expresi\'on encontrada en
~(\ref{sub:elemento_P_k}), es $h \equiv 0$, $\xi_1 \equiv 0$ y que $\xi_2$ 
no depende de $x$. Gracias al lema~\ref{lema_PIu3_k_cualquiera} ya sabemos
que $\xi_3 \equiv 0$.
As\'{\i} que veamos primero $h \equiv 0$.
Observaci\'on: si $f$ es $f_3$ o $f_4$, entonces
\begin{IEEEeqnarray}{rCl}
	(\textbf{curl}\,\pi\textbf{u})_3 |_{_{f}} & \in & P_{k-1}(x,y).	
\end{IEEEeqnarray}
Por el Lema~\ref{lema_pi_star_rot_u}, si usamos los degrees of freedom~(\ref{momentos1hdiv})
obtenemos, para todo $q \in P_{k-1}(x,y)$,
\begin{IEEEeqnarray}{rCl}
	\int\limits_{f} (\textbf{curl}\,\pi\textbf{u})_3\,q \,d\gamma & = & 
		\int\limits_{f} \textrm{rot}(\textbf{u})_3\,q \,d\gamma\\
		& = & 0.	
\end{IEEEeqnarray}
Si tomamos $q = (\textbf{curl}\,\pi\textbf{u})_3 |_{_{f}}$ tenemos
$(\textbf{curl}\,\pi\textbf{u})_3 |_{_{f}} \equiv 0$, o, de otra manera,
$z\,(z-1)$ divide a $(\textbf{curl}\,\pi\textbf{u})_3$. Escribamos
\begin{IEEEeqnarray*}{rCl}
	(\textbf{curl}\,\pi\textbf{u})_3 & = 	& z\, (z-1)\, \psi\\[6pt]
	\psi						& \in 	& P_{k-1}(x,y) \otimes P_{k-2}(z).
\end{IEEEeqnarray*}
A continuaci\'on usamos los degrees of freedom~(\ref{momentos4hdiv}) en la misma 
definici\'on de antes. Para todo $q\in P_{k-1}(x,y) \otimes P_{k-2}(z)$
\begin{IEEEeqnarray*}{rCl}
	\int\limits_{\hat{K}} z\,(z-1)\,\psi \, q\,d\textbf{x} & = & 0,
\end{IEEEeqnarray*}
as\'{\i} que tomando $q = \psi$ probamos que $\psi \equiv 0$ en $\hat{K}$,
es decir que
\begin{IEEEeqnarray}{rCl}
	\label{rot_3_es_0} (\textbf{curl}\,\pi\textbf{u})_3 &\equiv& 0.
\end{IEEEeqnarray}
Ahora veamos c\'omo es $(\textbf{curl}\,\pi\textbf{u})_3$ en t\'erminos
de la expresi\'on encontrada en~\ref{sub:elemento_P_k}.
\begin{IEEEeqnarray*}{rCl}
	(\textbf{curl}\,\pi\textbf{u})_3 & = & 
	\partial_x\,\pi(\textbf{u})_2 -	\partial_y\,\pi(\textbf{u})_1\\
	\label{expre_h} \yesnumber & = & -(2\,h + y\,\partial_y\,h + 
				x\,\partial_x\,h) +	\partial_x\,\xi_2 - \partial_y\,\xi_1,
\end{IEEEeqnarray*}
en donde, observando los grados de cada t\'ermino,
\begin{IEEEeqnarray*}{rCl}
	2\,h + y\,\partial_y\,h + x\,\partial_x\,h & \in & \tilde{P}_{k-1}(x,y)
\otimes P_k(z)\textrm{ y }\\
\partial_x\,\xi_2 - \partial_y\,\xi_1 & \in & P_{k-2}(x,y) \otimes P_k(z).
\end{IEEEeqnarray*}
De esto necesariamente sigue que 
$g := 2\,h + y\,\partial_y\,h + x\,\partial_x\,h = 0$. Ahora exploramos los 
t\'erminos de $g$. Pongamos
\[
	h(x,y,z) = \sum\limits_{i+j=k-1, l\leqslant k} \alpha_{_{i,j,l}} x^i y^j z^l.
\]
Entonces, para todo $(x,y,z)$ en $\hat{K}$
\begin{IEEEeqnarray*}{rCl}
	g(x,y,z) & = & \sum\limits_{i+j=k-1, l\leqslant k} 
(2\alpha_{_{i,j,l}} + j \alpha_{_{i,j,l}} + i \alpha_{_{i,j,l}}) x^i y^j z^l\\
	& = &(k+1)\,h(x,y,z)\\
  \yesnumber\label{h_is_zero}	& = & 0,
\end{IEEEeqnarray*}
o sea, $h \equiv 0$.
Hasta ac\'a tenemos probado que $\pi(0, u_2(y,z), 0) = 
(\xi_1(x,y,z), \xi_2(x,y,z), 0)$. Let's see that $\xi_1$ vanishes identically. Empecemos con 
los degrees of freedom sobre las aristas. Si $e$ es $e_1$ o $e_4$ entonces
\[
	\xi_1|_{e} \in P_{k-1}(x)
\]
y adem\'as, para todo $q\in P_{k-1}(x)$
\[
	\int\limits_{e} \xi_1|_{e}\,q\,ds = 0\textrm{,}
\]
con lo que llegamos a que $z\,(z-1)$ divide tambi\'en a $\xi_1|_{f_2}$. Pongamos
\begin{IEEEeqnarray}{rCl}
	\xi_1|_{f_2}(x,z) &=&z\,(z-1)\phi(x,z)\\[6pt]
	\phi &\in& P_{k-1}(x)\otimes P_{k-2}(z).
\end{IEEEeqnarray}
A continuaci\'on, si usamos los degrees of freedom en $f_2$, tenemos, para todo 
$q\in P_{k-1}(x)\otimes P_{k-2}(z)$,
\begin{IEEEeqnarray}{rCl}
	\int\limits_{f_2} z\,(z-1)\phi\,q\,d\gamma &=&0.
\end{IEEEeqnarray}
Tomando $q=\phi$ sigue que $\xi_1|_{f_2}\equiv 0$, con lo cual, para $(x,y,z)
\in \hat{K}$
\begin{IEEEeqnarray}{rCl}
	\xi_1(x,y,z) & = & y\,\rho(x,y,z)\\[6pt]
	\rho &\in&P_{k-2}(x,y)\otimes P_k(z).	
\end{IEEEeqnarray}
Ahora miramos los degrees of freedom en $f = f_3, f_4$. Vale que $\xi_1|_{f}$ pertenece 
a $P_{k-2}(x,y)$, y que para todo $q\in P_{k-2}(x,y)$
\[
	\int\limits_{f} \xi_1\,q\,d\gamma = 0\textrm{,}
\]
de donde, por tomar $q = \xi_1|_{f}$, sigue que $\xi_1|_{f} \equiv 0$. Con todo
esto tenemos que $z\,(z-1)$ divide a $\xi_1$, as\'{\i} que
\begin{IEEEeqnarray*}{rCl}
	\xi_1(x,y,x) &=&y\,z\,(z-1)\,\theta(x,y,z)\\[6pt]
	\theta &\in& P_{k-2}(x,y)\otimes P_{k-2}(z).
\end{IEEEeqnarray*}
Resta mirar los degrees of freedom de volumen aplicados a $\pi(\textbf{u})$. Para 
todo $q \in  P_{k-2}(x,y)\otimes P_{k-2}(z)$ debe ser
\begin{IEEEeqnarray*}{rCl}
\int\limits_{\hat{K}} y\,z\,(z-1)\,\theta(x,y,z)\,q\,d\textbf{x} &=&0
\textrm{,} 
\end{IEEEeqnarray*}
de donde, al tomar $q = \theta$, sigue inmediatamente que, en todo $\hat{K}$,
\begin{IEEEeqnarray}{rCl}
\label{xi1_es_0}\xi_1 & \equiv & 0.
\end{IEEEeqnarray}
Hasta ahora probamos que  
\[
	\pi(0, u_2(y,z), 0)  = 	 (0, \xi_2(x,y,z) ,0).
\]
Pero si recordamos lo que probamos en~(\ref{rot_3_es_0}) y lo combinamos con
~(\ref{xi1_es_0}), inmediatamente llegamos a 
\begin{IEEEeqnarray*}{rCCCl}
	\partial_x \pi(\textbf{u})_2 &=& \partial_x \xi_2 &\equiv& 0\textrm{,}
\end{IEEEeqnarray*}
que implica que $\xi_2$ no depende de $x$, y que $\pi(0, u_2(y,z), 0)$ tiene 
la forma que quer\'{\i}amos.
\end{proof}
\begin{lemma}\label{pi00u3} Si $\pi$ es el interpolador determinado por el elemento de la
\emph{Definici\'on}~(\ref{edgeelement}), entonces
\begin{IEEEeqnarray}{rCl}
	\pi(0,0, u_3)& = & (0,0,\xi_3)\textrm{,}\\
	\nonumber		\xi_3 & \in & \p{k}{k-1}.
\end{IEEEeqnarray}
\end{lemma}
\begin{proof} La demostraci\'on ser\'a muy parecida a la del Lema~(\ref{lemma_PIu2_k_in_N}). Recordemos
la expresi\'on que encontramos en la Subsecci\'on~\ref{sub:elemento_P_k}; tenemos
\begin{IEEEeqnarray}{rCl}
	\label{expre_pi00u3} \pi(0,0,u_3) &=& (\xi_1 + y\,h,\xi_2 - x\,h, \xi_3).	
\end{IEEEeqnarray}
Empezamos por ver que $h$ se anula. Sea $f$ cualquiera de las dos caras horizontales del 
prisma de referencia. Usamos la expresi\'on~(\ref{expre_h}) para 
$(\textbf{curl}\,\pi\textbf{u})_3 \in \p{k-1}{k}$. Gracias al
Lema~\ref{lema_pi_star_rot_u} y a las igualdades~(\ref{momentos1hdiv}) vale
\begin{IEEEeqnarray}{rCl}
	(\textbf{curl}\,\pi\textbf{u})_3 |_{f}&\equiv&0.
\end{IEEEeqnarray}
Entonces los polinomios $z$ y $z-1$ dividen a $(\textbf{curl}\,\pi\textbf{u})_3=
z\,(z-1)\,\psi$, ($\psi\in P_{k-1}(x,y)\otimes P_{k-2}(z)$). Ahora vamos a los
degrees of freedom~(\ref{momentos4hdiv}) para obtener finalmente que $\psi$ es id\'enticamente cero. Es decir
que, en todo $\hat{K}$, $(\textbf{curl}\,\pi\textbf{u})_3 \equiv 0$ y, siguiendo
el argumento en la demostraci\'on del Lema~\ref{lemma_PIu2_k_in_N} probamos que $h$ es id\'enticamente
cero. As\'{\i} que podemos reescribir la expresi\'on~(\ref{expre_pi00u3}) y poner
\begin{IEEEeqnarray}{rCl}
	\label{expre_pi00u3_} \pi(0,0,u_3) &=& (\xi_1,\xi_2, \xi_3)\\
	\nonumber\xi_1, \xi_2&\in& \p{k-1}{k}.
\end{IEEEeqnarray}
Resta ver que $\xi_1$ y $\xi_2$ se anulan. Con la misma idea, si consideramos
las caras $f_1$ con normal $\boldsymbol{\nu}=(-1,0,0) $ y $f_2$ con normal 
$\boldsymbol{\nu}=(0,-1,0)$, entonces las igualdades~(\ref{momentos1hcurl}) 
para las aristas $e_1, e_4$ junto con las igualdades~(\ref{momentos4hcurl}) por un lado,
y las igualdades~(\ref{momentos1hcurl}) para las aristas $e_2, e_5$ y~(\ref{momentos3hcurl})
por el otro, implican, respectivamente, que $y$ divide a $\xi_1$ y $x$ divide a $\xi_2$. Si 
continuamos con los degrees of freedom sobre las dos caras horizontales~(\ref{momentos2hcurl})
obtenemos tambi\'en que $z\,(z-1)$ los divide a ambos. Finalmente, si aplicamos las
igualdades~(\ref{momentos6hcurl}) probamos que $\xi_1 = \xi_2 \equiv 0$ en todo $\hat{K}$.
En conclusi\'on, $\pi(0,0, u_3)$ tiene la forma deseada.
\end{proof}
\begin{theorem}\label{thm_stab_edge}
Dados $p > 2$, $\hat{\emph{\textbf{u}}} \in \wpcurl{\hat{K}}$, si $\hat{\pi}$ es 
el operador de interpolaci\'on determinado por el elemento de la Definici\'on~(\ref{edgeelement}),
entonces
\begin{IEEEeqnarray}{rCl}
\label{teorema_1} \norm{\hat{\pi}(\hat{\emph{\textbf{u}}})_1}_{L^{\infty}(\hat{E})} & 
	\lesssim & \|\hat{u}_1\|_{W^{1,p}(\hat{E})} + 
	\|\emph{\textbf{curl}}(\hat{\emph{\textbf{u}}})_3\|_{{\color{red} W^{1,1}(\hat{E})}} \\	
\label{teorema_2} \norm{\hat{\pi}(\hat{\emph{\textbf{u}}})_2}_{L^{\infty}(\hat{E})} & 
	\lesssim & \|\hat{u}_2\|_{W^{1,p}(\hat{E})} + 
	\|\emph{\textbf{curl}}(\hat{\emph{\textbf{u}}})_3\|_{{\color{red} W^{1,1}(\hat{E})}} \\	
\label{teorema_3} \norm{\hat{\pi}(\hat{\emph{\textbf{u}}})_3}_{L^{\infty}(\hat{E})} & 
	\lesssim & \|\hat{u}_3\|_{W^{1,p}(\hat{E})}.
\end{IEEEeqnarray}
\end{theorem}
\begin{proof}
The proof will rely on the three previous Lemmas, 
the triangular inequality applied on each component of 
expression~(\ref{edge_interp_explicit}) and traces inequalities.

First we will take a smooth field $\hat{\bu}$ defined on $\hat{E}$
and, by Lemma~(\ref{lemaDensidad}), we will conclude the Theorem 
with a density argumentation.\\[5pt]


Las dos primeras se 
demuestran de forma parecida; probamos~(\ref{teorema_1}). La idea es tomar una funci\'on otra,
$\hat{\textbf{w}}$, tal que su interpolada tenga igual primera coordenada que la de 
$\hat{\bu}$,
pero que tenga degrees of freedom m\'as c\'omodos de acotar en t\'erminos exclusivamente de 
$\hat{u}_1$ y $\textbf{curl}(\hat{\bu})_3$.
Definimos, dada $\hat{\bu} \in W^{1,p}(\hat{K})^3$,
\begin{IEEEeqnarray*}{rCl}
	\hat{\textbf{w}} & = & (\hat{u}_1, \hat{u}_2 - \hat{u}_2(0,y,z), 0).
\end{IEEEeqnarray*}
Como dec\'{\i}amos, observemos primero que gracias a los lemas~(\ref{lemma_PIu2_k_in_N}) 
y~(\ref{pi00u3}) es
\begin{IEEEeqnarray*}{rCl}
	\hat{\pi}(\hat{\textbf{w}})_1 & = & (\hat{\pi}\hat{\bu})_1 - 
	\hat{\pi}(0, \hat{u}_2(0,y,z), 0)_1 -
	\hat{\pi}(0, 0, \hat{u}_3)_1\\
						& = & (\hat{\pi}\hat{\bu})_1.
\end{IEEEeqnarray*}
Ahora exploremos uno por uno los degrees of freedom que definen a $\pi(\hat{\textbf{w}})$. Los \'unicos
degrees of freedom sobre aristas que no se anulan o que no dependen expl\'{\i}citamente s\'olo de $\textrm{u}_1$
son
\begin{IEEEeqnarray*}{rCl}
	\int\limits_{e_8} \hat{\textbf{w}} \cdot \boldsymbol{\tau}\,\phi\,ds & = &
	\tfrac{1}{\sqrt{2}} \int\limits_{e_8} (w_1 - w_2)\,\phi\,ds\\
	\int\limits_{e_9} \hat{\textbf{w}} \cdot \boldsymbol{\tau}\,\phi\,ds & = &
	\tfrac{1}{\sqrt{2}} \int\limits_{e_9} (w_1 - w_2)\,\phi\,ds
\end{IEEEeqnarray*}
para $q\in \mathcal{P}_{k-1}$. Para el momento sobre $e_8$ hacemos partes en la cara $f_4
\subseteq \{ z=1 \}$. Tomemos un polinomio $\phi \in \mathcal{P}_{k-1}$ sobre $e_8$, que puede
verse como $\phi(y)$, dado que en $e_8$ es $x = 1 - y$.
\begin{IEEEeqnarray*}{rCl}
	\int\limits_{f_4} \textbf{curl}(\hat{\bu})_3\,\phi\,d\gamma & = &
	\int\limits_{f_4} \textbf{curl}(\hat{\textbf{w}})_3\,\phi\,d\gamma\\
	& = & -\int\limits_{f_4} \left(w_2\,\partial_x\phi - w_1\,\partial_y\phi\right)\,d\gamma
		+ \int\limits_{\partial f_4} \left(w_2\,\nu_x - w_1\,\nu_y\right)\,\phi\,ds\\
	& = &  \int\limits_{f_4} w_1\,\partial_y\phi\,d\gamma
		+ \int\limits_{e_8} \left(w_2 - w_1\right)\,\phi\,ds + 
			\int\limits_{e_4} w_1\,\phi\,ds.
\end{IEEEeqnarray*}
Es decir
\begin{IEEEeqnarray}{rCl}\label{momentosWaristas}
	\tfrac{1}{\sqrt{2}} \int\limits_{e_8} (w_1 - w_2)\,\phi\,ds & = &
		 \tfrac{1}{\sqrt{2}} \int\limits_{e_4} u_1\,\phi\,ds - 
		 \tfrac{1}{\sqrt{2}} \int\limits_{f_4} \textbf{curl}(\hat{\bu})_3\,\phi\,d\gamma
		 + \int\limits_{f_3} u_1\,\partial_y\phi\,d\gamma.
\end{IEEEeqnarray}
De igual manera hacemos partes en $f_3 \subseteq \{ z=0 \}$ para obtener
\begin{IEEEeqnarray}{rCl}\label{momentosWaristas2}
	\tfrac{1}{\sqrt{2}} \int\limits_{e_9} (w_1 - w_2)\,\phi\,ds & = &
		 \tfrac{1}{\sqrt{2}} \int\limits_{e_1} u_1\,\phi\,ds - 
		 \tfrac{1}{\sqrt{2}} \int\limits_{f_3} \textbf{curl}(\hat{\bu})_3\,\phi\,d\gamma
		 + \int\limits_{f_3} u_1\,\partial_y\phi\,d\gamma.
\end{IEEEeqnarray}
Ahora miramos los degrees of freedom sobre las caras. Tambi\'en aqu\'{\i} hacemos notar que s\'olo hace falta
trabajar con los degrees of freedom sobre las dos caras horizontales $f_3$ y $f_4$ y sobre la cara $f_5$.
Tomemos $\phi_1$, $\phi_2 \in \mathcal{P}_{k-2}(x,y)$, $\boldsymbol{\phi} = (\phi_1, \phi_2, 0)$.
\begin{IEEEeqnarray}{rCl}
 	\label{cotaf3}\int\limits_{f_3} \hat{\textbf{w}} \times \boldsymbol{\nu} \cdot \boldsymbol{\phi}\,d\gamma
 		& = & \int\limits_{f_3} u_1\,\phi_2\,d\gamma + \int\limits_{f_3} w_2\,\phi_1\,d\gamma.
\end{IEEEeqnarray}
Ahora sea un polinomio $\varphi_1 \in \mathcal{P}_{k-1}(x,y) $ tal que 
\begin{IEEEeqnarray*}{rCl}
	\partial_x \varphi_1 & = & \phi_1\textrm{,}\\
	\varphi_1 |_{e_9} 	 & \equiv & 0\textrm{;}
\end{IEEEeqnarray*}
tomemos por ejemplo $\varphi_1 = -\int_{x}^{1-y} \phi_1(t,y)\,dt$. Entonces
\begin{IEEEeqnarray*}{rCl}
	\int\limits_{f_3} \textbf{curl}(\hat{\bu})_3\,\varphi_1\,d\gamma & = & -\int\limits_{f_3} \left(w_2\,\phi_1 - w_1\,\partial_y\varphi_1\right)\,d\gamma
		+ \int\limits_{e_1} w_1\,\nu_y\,\varphi_1\,ds,
\end{IEEEeqnarray*}
y de esto con junto con~(\ref{cotaf3}) sigue que
\begin{IEEEeqnarray}{rCl}\label{momentosWcaras}
 	\nonumber\int\limits_{f_3} \hat{\textbf{w}} \times \boldsymbol{\nu} \cdot \boldsymbol{\phi}\,d\gamma
 		& = & \int\limits_{f_3} u_1\,\phi_2\,d\gamma - \int\limits_{f_3} \textbf{curl}(\hat{\bu})_3\,\varphi_1\,d\gamma\\
 		& 	& + \int\limits_{f_3} u_1\,\partial_y\varphi_1\,d\gamma	+ \int\limits_{e_1} u_1\,\nu_y\,\varphi_1\,ds.
\end{IEEEeqnarray}
Si repetimos lo anterior para el momento en $f_4$, obtenemos 
\begin{IEEEeqnarray}{rCl}\label{momentosWcaras2}
 	\nonumber\int\limits_{f_4} \hat{\textbf{w}} \times \boldsymbol{\nu} \cdot \boldsymbol{\phi}\,d\gamma
 		& = & \int\limits_{f_4} u_1\,\phi_2\,d\gamma - \int\limits_{f_4} \textbf{curl}(\hat{\bu})_3\,\varphi_1\,d\gamma\\
 		& 	& + \int\limits_{f_4} u_1\,\partial_y\varphi_1\,d\gamma	+ \int\limits_{e_4} u_1\,\nu_y\,\varphi_1\,ds\textrm{,}
\end{IEEEeqnarray}
y con la misma técnica para $f_5$, si dado $\textbf{q} = (0,q_3,q_1) \in \{ 0 \} \times Q_{k-2,k-1} \times Q_{k-1,k-2}$ 
tomamos $\varphi_1(x,z)=-\int\limits_{x}^{1} q_1(t,z)\,dt$, entonces obtenemos
\begin{IEEEeqnarray}{rCl}\label{momentosWcaras3}
 	\nonumber\int\limits_{f_4} \hat{\textbf{w}} \times \boldsymbol{\nu} \cdot \textbf{q}\,d\gamma
 		& = & \int\limits_{f_5} u_1\,q_1\,d\gamma + \int\limits_{f_5} \textbf{curl}(\hat{\bu})_3\,\varphi_1\,d\gamma\\
 		& 	& - \int\limits_{f_5} u_1\,\partial_y\varphi_1\,d\gamma	+ \int\limits_{e_4} u_1\,\varphi_1\,ds.
\end{IEEEeqnarray}
Finalmente, estudiamos los degrees of freedom de volumen. Tomemos
$\boldsymbol{\phi} = (\phi_1, \phi_2, \phi_3) \in (P_{k-2}(x,y) \otimes P_{k-2}(z))^{\color{red}2}
\times P_{k-3}(x,y) \otimes
P_{k-1}(z)$ (cfr.~(\ref{momentos6hcurl})) y sea $\varphi_2 = - \int\limits_{x}^{1-y} \phi_2(t,y,z)\,dt$, para el cual vale 
$\partial_x\varphi_2 = \phi_2 $ y $\varphi_2|_{f_5} \equiv 0$. Entonces, por partes,
\begin{IEEEeqnarray}{rCl}\label{momentosWvolumen}
 	\nonumber\int\limits_{\hat{K}} \hat{\textbf{w}} \cdot \boldsymbol{\phi}\,d\textbf{x}
 		& = & \int\limits_{\hat{K}} u_1\,\phi_1\,d\textbf{x} -\int\limits_{\hat{K}} \textbf{curl}(\hat{\bu})_3\,
 		\varphi_2,d\textbf{x}\\
 		& 	& + \int\limits_{\hat{K}} u_1\,\partial_y\varphi_2\,d\textbf{x}	+ 
 		\int\limits_{f_2} u_1\,\varphi_2\,d\gamma.
\end{IEEEeqnarray} %,~(\ref{momentosWcaras2}),~(\ref{momentosWcaras3})
Ahora recolectamos lo que dicen las igualdades~(\ref{momentosWaristas}),~(\ref{momentosWaristas2}),
y~(\ref{momentosWcaras})--(\ref{momentosWvolumen}), para hacer simplemente
%\begin{IEEEeqnarray*}{rCl}
%	\|\hat{\pi}(\hat{\textbf{w}})\|_{L^\infty(\hat{K})} & = 
%	 & 
%	 & \leqslant & C \left(\|\hat{u}_1\|_{W^{1,p}(\hat{K})} +
%		\|{\textbf{curl}}({\hat{\bu}})_3\|_{W^{1,1}(\hat{K})}\right)
%\end{IEEEeqnarray*}
\begin{IEEEeqnarray*}{rCl}
	\|(\hat{\pi}\hat{\bu})_1\|_{L^\infty(\hat{K})} & = 
	&\|(\hat{\pi}\hat{\textbf{w}})_1\|_{L^\infty(\hat{K})}\\
	& \leqslant & C \left\|\sum F_{i}(\hat{\textbf{w}}, \textbf{q}_i)\,\hat{\textbf{v}}_i\right\|_{L^\infty(\hat{K})}\\
	& \leqslant & C \sum \left|F_{i}(\hat{\textbf{w}}, \textbf{q}_i)\right|\,
		\left\|\hat{\textbf{v}}_i\right\|_{L^\infty(\hat{K})}\\
	& \leqslant & C \left(\|\hat{u}_1\|_{W^{1,p}(\hat{K})} +
		\|{\textbf{curl}}({\hat{\bu}})_3\|_{{\color{red} W^{1,1}(\hat{K})}}\right),
\end{IEEEeqnarray*}
que es la desigualdad a la que quer\'iamos llegar.\\[5pt]
\noindent For inequality~(\ref{teorema_3}) given $\hat{\bu} \in W^{1,p}(\hat{K})^3$, define
$\hat{\bv}  =  (0,0, \hat{u}_3).$
Thanks to Lemma~(\ref{lema_PIu3_k_cualquiera}) we have 
$\hat{\bw}_k(\hat{\bv})_3 = (\hat{\bw}_k\hat{\bu})_3 - (\hat{\bw}_k(\hat{u}_1, \hat{u}_2, 0))_3 = (\hat{\bw}_k\hat{\bu})_3.$
Apply now expression~(\ref{edge_interp_explicit}) to $\bv$ writing the degrees of
freedom more explicitly. Taking another look at 
the unit tangent vector of the edges and unit normal vectors to the faces we have to
consider only a few not null terms ({\color{red}recall the numbering of the edges and faces}).
\begin{IEEEeqnarray*}{rCl}
  (\hat{\boldsymbol{w}}_k\hat{\bv})_3 & = &
  \sum_{j=3,6,7\,p\,\in\,{\color{red}\mathcal{B}_{\hat e_j}}} \int\limits_{\hat e_j} \hat{u}_3 p\,ds         \,(\hat{\bv}_{e_j,p})_3 +
  \sum_{i=1,2,4\,q\,\in\,{\color{red}\mathcal{B}_{\hat f_i}}} \int\limits_{\hat f_i} \hat{u}_3 q\,d\gamma    \,(\hat{\bv}_{f_i,q})_3 +
  \sum_{\boldsymbol{r}\,\in\,{\color{red}\mathcal{B}_{\hat E}}}
  \int\limits_{\hat E} \hat{u}_3 r_3\,d\textbf{x}\,(\hat{\bv}_{\boldsymbol{r}})_3.
\end{IEEEeqnarray*}
And now apply the triangular inequality and the proof
of Lemma $5.38$ in the page $134$ of Monk~\cite{monk}
and Theorem $3.9$ (\emph{Trace Theorem})
in page $43$ of
the same book, to get \noindent{\color{blue}\#\#\#\#\#\#\# revisar los exponentes hoelder y todo...
porque en monk pone $H^{1/2+\delta}$} 
\begin{IEEEeqnarray*}{rCl}
  \norm{(\hat{\boldsymbol{w}}_k\hat{\bu})_3}_{L^{\infty}(\hat{E})} =
  \norm{(\hat{\boldsymbol{w}}_k\hat{\bv})_3}_{L^{\infty}(\hat{E})} & 
  \leqslant & c\,\|\hat{u}_3\|_{W^{1,p}(\hat{E})}
\end{IEEEeqnarray*}
where the quantity $c$ depends, clearly, on the choice of the bases $\mathcal{B}_{\hat e_j},
\mathcal{B}_{\hat f_i}, \mathcal{B}_{\hat E}$.
\end{proof}
\section{Stability of the generalized Raviart-Thomas element in $\hat{E}$} % (fold)
The exposition in this section will be there $H(\mbox{div})$--conforming analogue
of the exposition in the previous section.\\
\noindent In the next three Lemmas $\hat\bu$ is an element
in $H(\mbox{div},\hat{E})$ and we will assume there is
a positive $\delta$ such that $\hat\bu$ belongs to $H^{1/2+\delta}(\hat{E})^3$.
$\hat\br_k$ will be, for the whole section, the $k$--th order face 
interpolation operator on the reference
Prism determined by the element of
\emph{Definition}~(\ref{defi_h_div_conforme}).
\label{stability_of_rt_element_in_hat_k}
\begin{lemma}\label{lemmaRT3zero}
$(\rku)_3$ is linearly and univocally determined by $\hat{u}_3$.
\end{lemma}
%\noindent{\color{blue}\#\#\#\#\#\#\# de aqui a la sgte marca, chequear si las expresiones
%~(\ref{exprPrt}) son correctas 
%  From the definition of the discrete space, and the notation involved, it follows that for each
%  $\textbf{p} = (p_1, p_2, p_3) \in  P_{\hat{E}}$ there are unique $q_1, q_2 \in 
%    \mathcal{P}_{k-1}(x,y)\otimes\mathcal{P}_{k-1}(z)$, $q_3 \in \mathcal{P}_{k-1}(x,y)\otimes\mathcal{P}_{k}(z)$, and
%    $h \in \tilde{\mathcal{P}}_{k-1}(x,y)\otimes\mathcal{P}_{k-1}(z)$ such that
%    \begin{IEEEeqnarray*}{rCl}
%                                    p_1(x, y, z) & = & q_1(x, y, z) + x\,h(x, y, z)\\
%        \label{exprPrt} \yesnumber  p_2(x, y, z) & = & q_2(x, y, z) + y\,h(x, y, z)\\
%                                    p_3(x, y, z) & = & q_3(x, y, z)
%    \end{IEEEeqnarray*} 
%}
%\noindent{\color{blue}\#\#\#\#\#\#\# }\\[5pt]
\begin{proof}
By the unisolvence of the Finite Element~(\ref{defi_h_div_conforme})
$(\rku)_3$ is determined by the following linear equations.
\begin{IEEEeqnarray}{rCrl}
\label{rku_3_1}
\rho_{f_j,q}\,(\rku) & = & \rho_{f_j,q}\,(\hat{\bu})
  &\quad\mbox{for $j$ = 3, 4 and }q\in\mathcal{P}_{\hat{f}} \\
\label{rku_3_2}
\rho_{\br}\,(\rku) & = & \rho_{\br}\,(\hat{\bu})
  &\quad\mbox{for }\br\in\mathcal{P}_{\hat{E}}\mbox{ with }r_1 = r_2 = 0.
\end{IEEEeqnarray}
We have $\nicefrac{k(k+1)^2}{2}$ equations, which is the 
number of independent coefficients in $(\rku)_3$.
Now set $u_3 = 0$, which makes the right hand side of all the equations in~(\ref{rku_3_1})
and~(\ref{rku_3_2}) equal to zero.
Take $\hat f$ to be either $\hat{f}_3$ or $\hat{f}_4$. Put $q = (\rku)_3|_{\hat{f}}$ in~(\ref{rku_3_1}).
It yields
\begin{IEEEeqnarray*}{rCl}
  \int\limits_{\hat{E}} ((\rku)_3)^2\,d\bx & = & 0
\end{IEEEeqnarray*}
so there is a polinomial $\hat{q}_3\in(\mathcal{P}_{\hat{E}})_3$ such that
$(\rku)_3 \xyz = \hat{x}_3(\hat{x}_3-1)\,\hat{q}_3\xyz$
and the Lemma will follow once we apply the degrees of freedom~(\ref{rku_3_2})
with $\br$ set equal to $(0,0,\hat{q}_3)^t$. 
\end{proof}
{\color{blue}\#\#\#\#\#\#\#\# CONTINUE HERE.}
\begin{lemma}\label{lemma_u1_u2} If $\hat\bu\in P_{\hat E}$ is such that $\hat{u}_1 = 0$
and $\hat{u}_2 = 0$, then $(\hat{\br}_k\hat{\bu})_1 = 0$
and $(\hat{\br}_k\hat{\textbf{\emph{u}}})_2 = 0$.

\begin{itemize}
  \item []
  \item [(a)]\label{piu2_k_in_N} If $\hat\bu(\hat x_1,\hat x_2,\hat x_3) = (0, \hat u_2(\hat y,\hat z), 0)^t$,
  then $\wku\xyz = (0, \hat\xi_2(y,z) ,0)^t$ for some 
  $\hat\xi_2 \in P_{k-1}(\hat{x}_2) \otimes P_k(\hat{x}_3)$.
  \item [(b)]\label{piu1_k_in_N} If $\hat\bu(\hat x_1,\hat x_2,\hat x_3) = (\hat u_1(\hat x,\hat z), 0, 0)^t$
  then $\wku\xyz = (\hat\xi_1(x,z), 0 ,0)^t$ for some
    $\hat\xi_1\in P_{k-1}(\hat{x}_1) \otimes P_k(\hat{x}_3)$.
\end{itemize}


\end{lemma}
\begin{proof} We use again the expression~(\ref{exprPrt}) for $\hat{\br}_k\hat{\bu} =
  (\hat{p}_1, \hat{p}_2, \hat{p}_3) \in  P_{\hat{E}}$. Take an arbitrary
  $\hat{q}\in\mathcal{P}_{k-1}(\hat f_1)\otimes\mathcal P_{k-1}(\hat z)$, the trick
  will be to apply Green's Theorem to the field
  $(\hat{p}_1, \hat{p}_2, 0)^t$
  . By the surface degrees of freedom~(\ref{momentos2hdiv})
  and the volume degrees of freedom~(\ref{momentos3hdiv}) we have
  \begin{IEEEeqnarray*}{rCl}
    \int\limits_{\hat{E}} \text{div}(\hat{p}_1, \hat{p}_2, 0)^t\,\hat{q}\,d\hat{\textbf{x}}&=&
    \int\limits_{\partial\hat{E}} (\hat{p}_1, \hat{p}_2, 0)^t\cdot\boldsymbol{\hat\nu}\,\hat{q}\,d\gamma
    - \int\limits_{\hat{E}} (\hat{p}_1, \hat{p}_2, 0)^t\cdot\nabla \hat{q}\,d\hat{\textbf{x}}\\[5pt]
    & = &
    \int\limits_{{\color{red}\hat{f}}x+y=1} (\hat{u}_1 + \hat{u}_2) \hat{q}_{|_{{\color{red}\hat f x+y} = 1}}\,d\gamma
    - \int\limits_{{\color{red}\hat{f}}x=0} \hat{u}_1 \hat{q}_{|_{{\color{red}\hat f x=0}}}\,d\gamma\\[5pt]
    &&\,- \int\limits_{{\color{red}\hat{f}}y=0} \hat{u}_2 \hat{q}_{|_{{\color{red}\hat f y=0}}}\,d\gamma
    - \int\limits_{\hat{E}} \hat{u}_1\partial_{\hat x}\hat{q} 
      + \hat{u}_2 \partial_{\hat y}\hat{q}\,d\hat{\textbf{x}}\,=\,0.
  \end{IEEEeqnarray*}
  And since $\text{div}(\hat{p}_1, \hat{p}_2, 0)^t$ also belongs to
  $\pp{k-1}{k-1}$, we just proved it vanishes on all $\hat{E}$.\\[3pt]
  In a similar way as done in Lemma~(\ref{lemma_PIu2_k_in_N}), eq.~(\ref{h_is_zero}),
  we get to prove $h \equiv 0$ in expression~(\ref{exprPrt}). It means we may
  assume
  \begin{IEEEeqnarray*}{rCl}
    \hat{p}_1, \hat{p}_2 & \in & \mathcal{P}_{k-1}(\hat{f}_1)\otimes\mathcal{P}_{k-1}(\hat{z}).
  \end{IEEEeqnarray*}
  Now it is convenient to use again the degrees of freedom on the faces normal to $(-1, 0, 0)$ and $(0, -1, 0)$.
  The conditions
  \begin{IEEEeqnarray*}{rCl}
      \int\limits_{\hat f_1} \hat{p}_1\hat q\,d\hat\gamma & = & 0\qquad\mbox{for all }\hat{q}\in\mathcal{P}_{k-1}(\hat f_1)
  \end{IEEEeqnarray*}
  \begin{IEEEeqnarray*}{rCl}
      \int\limits_{\hat f_2} \hat{p}_2\hat q\,d\hat\gamma & = & 0\qquad\mbox{for all }\hat{q}\in\mathcal{P}_{k-1}(\hat f_2)
  \end{IEEEeqnarray*}
  imply that $\hat{x}_i$ divides $\hat{p}_i$ for both $i=1,2$. In other words, 
  there are $\phi$ and $\psi$ in $\pp{k-2}{k-1}$ such that
  \begin{IEEEeqnarray*}{rCl}
    \hat{p}_1 & = & \hat{x}\,\phi\\
    \hat{p}_2 & = & \hat{y}\,\psi.
  \end{IEEEeqnarray*}
  But finally, if we evaluate the degrees of freedom~(\ref{momentos3hdiv}),
  $\hat{p}_1 = (\hat{\br}_k\hat{\bu})_1$ and 
  $\hat{p}_2 = (\hat{\br}_k\hat{\bu})_2$ can be no other that
  constantly null over all $\hat{E}$. 
\end{proof}
\begin{lemma} Let $\hat{\bu}$ be in $P_{\hat E}$.\\[5pt]
If $\hat{\bu}(\hat x_1,\hat x_2,\hat x_3) = (0, \hat u_2(\hat x_1,\hat x_3), 0)$,
then 
$\hat{\br}_k \hat{\bu}(\hat x_1,\hat x_2,\hat x_3) =
(0, \hat{r}_2(\hat x_1, \hat x_3), 0)$ for certain $\hat r_2\in (P_{\hat E})_2$.\\[3pt]
If $\hat{\bu}(\hat x_1,\hat x_2,\hat x_3) = (u_1(\hat x_2,\hat x_3), 0, 0)$,
then $\br_k \hat{\bu}(\hat x_1,\hat x_2,\hat x_3) =
(\hat{r}_1(\hat x_2,\hat x_3), 0, 0)$ for certain $\hat r_1\in (P_{\hat E})_1$.
\end{lemma}
\begin{proof} Let us prove the first one of the two claims. The second one 
  has an entirely analogous proof. By Lemma~(\ref{lemmaRT3zero}) we get
  $(\hat{\br}_k\hat{\boldsymbol{u}})_3 = 0$.
  The nullity of $\dv\hat{\bu}$ and the commutative
  diagram property~(\ref{lema_pi_star_rot_u}) give us %$\dv((\hat{\br}_k\hat{\bu})_1,(\hat{\br}_k\hat{\bu})_2, 0) = 0$.
  $\dv\hat{\br}_k\hat{\bu} = 0$.
  This last fact, together with the result of the evaluation of the 
  degrees of freedom~(\ref{momentos2hdiv})
  on the face $\hat f_1 \subseteq \{x_1=0\}$
  and~(\ref{momentos3hdiv}) over $\hat E$, implies, as we have seen in
  Lemma~(\ref{lemma_u1_u2}), that $(\hat{\br}_k\hat{\boldsymbol{u}})_1 = 0$.
  And if look again at 
  $\dv\hat{\br}_k\hat{\bu} = {\partial\hat{\br}_k\hat{\bu}}/{\partial\hat x_2} = 0$
  we have that $(\hat{\br}_k\hat{\boldsymbol{u}})_2$ does not depend on $\hat x_2$.
\end{proof}
\noindent It is time to state and prove the Theorem that was the purpose of this section.
\begin{theorem}\label{thm_stab_div}
If $\hat{\br}_k$ is the  
interpolation operator determined by the Finite Element of degree $k$ in
definition~(\ref{defi_h_div_conforme}), then
for all $\hat{\bu} \in W^{1,1}(\hat{E})$
\begin{IEEEeqnarray}{rCl}
\label{teoremaDiv_1} \norm{(\hat{\br}_k\hat{\bu})_1}_{L^{\infty}(\hat{E})} & 
    \lesssim & \|\hat{u}_1\|_{W^{1,1}(\hat{E})} + 
    \|\emph{div}(\hat{u}_1, \hat{u}_2, 0)\|_{L^{1}(\hat{E})} \\ 
\label{teoremaDiv_2} \norm{(\hat{\br}_k\hat{\bu})_2}_{L^{\infty}(\hat{E})} & 
    \lesssim & \|\hat{u}_2\|_{W^{1,1}(\hat{E})} + 
    \|\emph{div}(\hat{u}_1, \hat{u}_2, 0)\|_{L^{1}(\hat{E})} \\ 
\label{teoremaDiv_3} \norm{(\hat{\br}_k\hat{\bu})_3}_{L^{\infty}(\hat{E})} & 
    \lesssim & \|\hat{u}_3\|_{W^{1,1}(\hat{E})}.
\end{IEEEeqnarray}
\end{theorem}
\begin{proof} 
According to expression~(\ref{face_interp_explicit}) \dots

{\color{blue}\#\#\#\#\#\#\#\# HERE. Tal vez me convenga ocuparme primero
del teorema correspondiente para H(curl) porque hay muchas cosas
que ac'a ser'an 'como dijimos en el caso H(curl) ... etc ...'}
The proof is based on the last three Lemmas.
Given $\hat{\bu} \in W^{1,1}(\hat{E})$ let
$\hat{\bv} = (u_1, u_2 - u_2(x,0,z), 0)$. Then $(\pi \textbf{v})_1 = (\pi \textbf{u})_1$.
\\[5pt]
The DOF's.\\[5pt]
Some of them vanish or depend exclusively on $u_1$. The others:
Let $f$ be the face contained in $\{x+y = 1\}$. Take $q\in\mathcal{Q}_{k-1,k-1,k-1}$.
\begin{IEEEeqnarray*}{rCl}
    \int\limits_{f} \pi\textbf{v}\cdot\boldsymbol{\nu}\,q\,d\gamma &=&
    \int\limits_{f} (v_1 + v_2)\,q\,d\gamma.
\end{IEEEeqnarray*}
Observe that the restriction of $q$ to $f$ belongs to $\mathcal{P}_{k-1}(x)\otimes
\mathcal{P}_{k-1}(z)$. Now
\begin{IEEEeqnarray*}{rCl}
    \int\limits_{\hat{E}} \text{div} (u_1,u_2,0) \,q\,d\textbf{x} &=&
    \int\limits_{\hat{E}} \text{div}\,\textbf{v} \,q\,d\textbf{x}\\
    & = & -\int\limits_{\hat{E}} (u_1, v_2, 0)^t\cdot\nabla q\,d{\textbf{x}} + 
            \int\limits_{\partial_{\hat{E}}} (u_1, v_2, 0)^t\cdot\boldsymbol{\nu}\,q\,d\gamma\\
    & = & -\int\limits_{\hat{E}} u_1\partial_xq \,d{\textbf{x}} - \int\limits_{f_1}u_1\,q\,d\gamma
        + \int\limits_{f} (v_1 + v_2)\,q\,d\gamma,
\end{IEEEeqnarray*}
where $f_1$ is the face contained in the plane $x = 0$. So we have
\begin{IEEEeqnarray*}{rCl}
    \int\limits_{f} (v_1 + v_2)\,q\,d\gamma & = & \int\limits_{\hat{E}} \text{div} (u_1,u_2,0)
    \,q\,d\textbf{x} + \int\limits_{\hat{E}} u_1\partial_xq \,d{\textbf{x}} +
    \int\limits_{f_1}u_1\,q\,d\gamma.
\end{IEEEeqnarray*}
For the volume degrees of freedom~(\ref{momentos3hdiv}) take $q_2 \in P_{k-2}(x,y) \otimes P_{k-1}(z)$. Since
\begin{IEEEeqnarray*}{rCl}
    v_2(x,y,z) & = &\int\limits_0^{y} \partial_y u_2(x,t,z) \,dt,
\end{IEEEeqnarray*}
we may do
\begin{IEEEeqnarray*}{rCl}
    \int\limits_{\hat{E}} v_2\,q_2  & = &\int\limits_0^1\int\limits_0^1\int\limits_0^{1-x}
    \int\limits_0^{y}
      \partial_y u_2(x,t,z) \,dt \,q_2 (x,y,z)\,dy\,dx\,dz\\
        & = &\int\limits_0^1\int\limits_0^1\int\limits_0^{1-x}\int\limits_0^{y}
                \partial_y u_2(x,t,z) \,q_2 (x,y,z)\,dt\,dy\,dx\,dz\\
        & = &\int\limits_0^1\int\limits_0^1\int\limits_0^{1-x}\int\limits_t^{1-x}
                \partial_y u_2(x,t,z) \,q_2 (x,y,z)\,dy\,dt\,dx\,dz\\
        & = &\int\limits_0^1\int\limits_0^1\int\limits_0^{1-x}\partial_y u_2(x,t,z)
                \int\limits_t^{1-x}\,q_2 (x,y,z)\,dy\,dt\,dx\,dz\\
        & = &\int\limits_0^1\int\limits_0^1\int\limits_0^{1-x}\partial_y u_2(x,t,z)\,\phi (x,t,z)\,dt\,dx\,dz\\
& = &\int\limits_{\hat{E}} (\partial_xu_1 + \partial_y u_2)\,\phi\,d{\textbf{x}}
    - \int\limits_{\hat{E}} \partial_xu_1\,\phi\,d{\textbf{x}}
\end{IEEEeqnarray*}
(for some $\phi \in \mathcal{P}_{k-1}(x,y)\otimes\mathcal{P}_{k-1}(z)$), as we wanted. 
\end{proof}
Theorems~(\ref{thm_stab_edge}) and~(\ref{thm_stab_div}) show that the interpolations
determined by the Finite Elements~(\ref{edgeelement}) and~(\ref{defi_h_div_conforme})
are anisotropically stable, in the sense that the image of a field $\hat\bu$ under
the linear operator depends not only continously on $\hat\bu$, but also with a
\emph{componentwise} bound, with perhaps and additional curl or divergence term, respectively.

% section stability_of_rt_element_in_hat_k (end)