\chapter{Local Interpolation}
We prove anisotropic local interpolation error estimates for two different operators
and use this result to estimate the global approximation error.

Edge elements are one example of $\textrm{H}(\textbf{curl})-$conforming elements, and they were
defined to determine a natural interpolation operator for fields with continuous tangential components.
Similarly, the other elements considered are $\textrm{H}(\text{div})-$conforming, and we will use them 
to interpolate fields with continuous tangential components. These are the cases of the
solutions of Stokes' equations, the solutions of time harmonic Maxwell's equations, and the vectorial
variable of the following problem, which will be our application.

\section{Prismatic Finite Elements} % (fold)
\label{sec:prismatic_finite_elements}
\subsection{An\-iso\-tropic Stability Estimates for $H(\Div)$--Conforming Finite Elements
on Prisms} % (fold)
In the present subsection $\hat\bu$ will be an element
of $W^{1,1}(\hat E)$ in which elements have well defined
normal traces over the faces of $\hat{E}$, another possibility
being, as mentioned in~\cite{monk}, Lemma $5.15$, page $120$, to assume
there is
a positive $\delta$ such that $\hat\bu$ belongs to
$H^{1/2+\delta}(\hat{E})^3$.
For the whole subsection, $\hat\br_{\hat E}$ will be the $k$--th order face 
interpolation operator on the reference
Prism determined by the element of
Definition~\ref{defi_h_div_conforme}.
\label{stability_of_rt_element_in_hat_k}
\begin{lemma}\label{lemmaRT3zero}
$(\rku)_3$ is linearly and univocally determined by $\hat{u}_3$.
\end{lemma}
\begin{proof}
By the unisolvence of the finite element in Definition~\ref{defi_h_div_conforme},
$(\rku)_3$ is determined by the following linear equations.
\begin{IEEEeqnarray}{rCrl}
\label{rku_3_1}
\hat\rho_{\hat f_j,\hat q}\,(\rku) & = & \hat\rho_{\hat f_j,\hat q}\,(\hat{\bu})
  &\quad\mbox{for $j$ = 3, 4 and }\hat q\in P_{k-1}({\hat{f}_j}) \\
\label{rku_3_2}
\hat\rho_{\hat\br}\,(\rku) & = & \hat\rho_{\hat\br}\,(\hat{\bu})
  &  \quad\mbox{for }\hat\br =(0, 0, \hat r_3)\mbox{, }\hat r_3 \in P_{k-1,k-2}.
\end{IEEEeqnarray}
We have $\nicefrac{k(k+1)^2}{2}$ independent equations, which is the 
number of independent coefficients in $(\rku)_3$.
Now set $\hat u_3 = 0$, which makes the right hand side of all the equations in~(\ref{rku_3_1})
and~(\ref{rku_3_2}) equal to zero.
Take $\hat f$ to be either $\hat{f}_3$ or $\hat{f}_4$. Put 
$\hat q = (\rku)_3|_{\hat{f}}$ in~(\ref{rku_3_1}) recalling~(\ref{prismaticSpace}).
It yields
\begin{IEEEeqnarray*}{rCl}
  \iint_{\hat{f}} \{(\rku)_3\}^2\,d\hat{S} & = & 0.
\end{IEEEeqnarray*}
so there is a polinomial $\hat{q}_3\in {P}_{k-1,k-1}$ such that
$(\rku)_3 \xyz = \hat{x}_3(\hat{x}_3-1)\,\hat{q}_3\xyz$
and the Lemma will follow once we apply the degrees of freedom~(\ref{rku_3_2})
with the test polynomial $\hat\br$ set equal to $(0,0,\hat{q}_3)'$. 
\end{proof}
\begin{lemma}\label{lemma_u1_u2} If $\hat\bu\xyz = (0,0, \hat{u}_3\xyz)'$,
then $\rku\xyz = (0,0,\hat{\xi}_3\xyz)'$ for some $\hat{\xi}_3\in
P_{k-1}(\hat{f}_3)\otimes P_k(\hat{x}_3)$.
\end{lemma}
\begin{proof}
First take an explicit expression for
$\rku = (\hat{p}_1, \hat{p}_2, \hat{p}_3)' \in  P_{\hat{E}}$ as 
\begin{IEEEeqnarray*}{rCl}
  \hat{p}_1\xyz & = & \hat{q}_1\xyz + \hat{x}_1\,\hat{h}\xyz\\
  \label{exprPrt}\yesnumber
  \hat{p}_2\xyz & = & \hat{q}_2\xyz + \hat{x}_2\,\hat{h}\xyz\\
  \hat{p}_3\xyz & = & \hat{q}_3\xyz
\end{IEEEeqnarray*}
for unique $\hat{q}_1, \hat{q}_2 \in {P}_{k-1}(\hat{f}_3)
\otimes{P}_{k-1}(\hat{x}_3)$,
$\hat{q}_3 \in {P}_{k-1}(\hat{f}_3)\otimes{P}_{k}(\hat{x}_3)$,
and
$\hat{h} \in \tilde{{P}}_{k-1}(\hat{f}_3)\otimes{P}_{k-1}(\hat{x}_3)$.
Next, take an arbitrary $\hat{q}\in{P}_{k-1}(\hat f_1)\otimes P_{k-1}(\hat z)$.
The trick will be to apply Green's Theorem to the field
$(\hat{p}_1, \hat{p}_2, 0)'$ and the scalar $\hat{q}$. By the surface degrees of
freedom~(\ref{momentos2hdiv}) and the volume degrees of freedom~(\ref{momentos3hdiv}) we have
  \begin{IEEEeqnarray*}{rClCl}
    \int_{\hat{E}} \mbox{div}(\hat{p}_1, \hat{p}_2, 0)'\,\hat{q}\,d\hat{\bx}&=&
    \int_{\partial\hat{E}} (\hat{p}_1, \hat{p}_2, 0)'\cdot\hat\bn\,\hat{q}\,d\hat{S}
    - \int_{\hat{E}} (\hat{p}_1, \hat{p}_2, 0)'\cdot\nabla \hat{q}\,d\hat{\bx}&&\\[5pt]
    & = &
    \int_{\hat{f}_5} (\hat{u}_1 + \hat{u}_2)
    \hat{q}_{|_{\hat{f}_5 = 1}}\,d\hat{S}
    - \int_{\hat{f}_1} \hat{u}_1 \hat{q}_{|_{\hat{f}_1}}\,d\hat{S}&&\\[5pt]
    &&\,- \int_{\hat{f}_2} \hat{u}_2 \hat{q}_{|_{\hat{f}_2}}\,d\hat{S}
    - \int_{\hat{E}} \hat{u}_1
    \tfrac{\partial\hat{q}}{\partial{\hat{x}_1}} 
      + \hat{u}_2\tfrac{\partial\hat{q}}{\partial{\hat{x}_2}}\,d\hat{\bx} & = & 0.
  \end{IEEEeqnarray*}
  And since $\mbox{div}(\hat{p}_1, \hat{p}_2, 0)'$ also belongs to
  $P_{k-1,k-1}$, we just established it vanishes on all $\hat{E}$.\\[3pt]
  Now the nullity of $\mbox{div}(\hat{p}_1, \hat{p}_2, 0)'$ implies at once that
  $\hat{h} \equiv 0$ in expression~(\ref{exprPrt}) (it can be deduced directly
  derivating the polynomials and observing the degrees of the terms; cfr. 
  proof of Lemma~\ref{lemma_PIu2_k_in_N} onwards). 
  This means we may
  assume that $\hat{p}_1$ and $\hat{p}_2$ belong to 
  ${P}_{k-1}(\hat{f}_3)\otimes{P}_{k-1}(\hat{x}_3)$.
  Now it is convenient to use again the degrees of freedom on the
  faces normal to $(-1, 0, 0)'$ and $(0, -1, 0)'$.
  The conditions
  \begin{IEEEeqnarray*}{rCCCll}
    \hat\rho_{\hat f_i\,\hat q}(\rku) & = & \iint_{\hat f_i} \hat{p}_i\hat q\,d\hat S
    & = & 0\qquad\mbox{for all }\hat{q}\in{P}_{k-1}(\hat f_i)\mbox{, \,}&1\leqslant i\leqslant 2
  \end{IEEEeqnarray*}
  ensure that $\hat{x}_i$ divides $\hat{p}_i$ for both $i=1$ and $2$.
%  In other words, there are $\phi$ and $\psi$ in $\pp{k-2}{k-1}$ such that
%  \begin{IEEEeqnarray*}{rCl}
%    \hat{p}_1 & = & \hat{x}\,\phi\\
%    \hat{p}_2 & = & \hat{y}\,\psi.
%  \end{IEEEeqnarray*}
But finally, if we evaluate the degrees of freedom~(\ref{momentos3hdiv}),
we see that  $\hat{p}_1 = (\rku)_1$ and 
$\hat{p}_2 = (\rku)_2$ can be no other that
constantly null over all $\hat{E}$. 
\end{proof}
\begin{lemma}
\begin{itemize}
  \item []
  %\label{piu2_k_in_N}
  \item [(a)] If $\hat\bu(\hat x_1,\hat x_2,\hat x_3) =
  (0, \hat{u}_2(\hat x_1,\hat x_3), 0)'$,
  then $\rku\xyz = (0, \hat\xi_2(\hat x_1, \hat x_3) ,0)'$ for some 
  $\hat\xi_2 \in P_{k-1}(\hat{x}_2) \otimes P_k(\hat{x}_3)$.
  \item [(b)] If $\hat\bu(\hat x_1,\hat x_2,\hat x_3) = 
  (\hat{u}_1(\hat x_2,\hat x_3), 0, 0)'$
  then $\rku\xyz = (\hat\xi_1(\hat x_2,\hat x_3), 0 ,0)'$ for some
    $\hat\xi_1\in P_{k-1}(\hat{x}_1) \otimes P_k(\hat{x}_3)$.
\end{itemize}
\end{lemma}
\begin{proof} Let us prove the first one of the two claims. The second one 
  has an analogous proof. By Lemma~\ref{lemmaRT3zero} we get
  $(\rku)_3 = 0$.
  The nullity of $\dv\hat{\bu}$ and the commutative
  diagram property~(\ref{div_commutativity}) give us
  $\dv\rku = 0$.
  This last fact, together with the result of the evaluation of the 
  degrees of freedom~(\ref{momentos2hdiv})
  on the face $\hat f_1 \subseteq \{x_1=0\}$
  and~(\ref{momentos3hdiv}) over $\hat E$, implies, as we have seen in
  Lemma~\ref{lemma_u1_u2}, that $(\rku)_1 = 0$.
  And if we look again at 
  $\dv\rku = {\partial\rku}/{\partial\hat x_2} = 0$
  we have that $(\rku)_2$ does not depend on $\hat x_2$.
\end{proof}
\noindent It is time to state and prove the Theorem that was the purpose of this section.
\begin{theorem}\label{thm_stab_div} Given $\hat{\bu} \in W^{1,1}(\hat{E})$
\begin{IEEEeqnarray}{rCl}
\label{teoremaDiv_1} \norm{(\rku)_1}_{L^{\infty}(\hat{E})} & 
    \lesssim & \|\hat{u}_1\|_{W^{1,1}(\hat{E})} + 
    \|\emph{div}(\hat{u}_1, \hat{u}_2, 0)\|_{L^{1}(\hat{E})} \\ 
\label{teoremaDiv_2} \norm{(\rku)_2}_{L^{\infty}(\hat{E})} & 
    \lesssim & \|\hat{u}_2\|_{W^{1,1}(\hat{E})} + 
    \|\emph{div}(\hat{u}_1, \hat{u}_2, 0)\|_{L^{1}(\hat{E})} \\ 
\label{teoremaDiv_3} \norm{(\rku)_3}_{L^{\infty}(\hat{E})} & 
    \lesssim & \|\hat{u}_3\|_{W^{1,1}(\hat{E})}
\end{IEEEeqnarray}
where the constants in the inequalities depend olny on $\hat{E}$.
\end{theorem}
\begin{proof}      %%[Proof of Theorem~\ref{thm_stab_div}]
The proof is based on the
last three Lemmas. By Proposition~\ref{density_wpcurl} we can state the estimate for
a smooth field $\hat{\bu}$ and then finish the proof with a density argument.\\[4pt]
Take %again $\omega$ as the closure of $\hat{E}$ and 
$\hat\bu\in\pazocal{C}^\infty(\bar{\hat{E}})^3$. For 
the first inequality. Set
$\hat{\bv} = (\hat{u}_1, \hat{u}_2 - \hat{u}_2(\hat{x}_1,0,\hat{x}_3), 0)$.
Then $({\br}_{\hat E}\hat\bv)_1 = (\rku)_1$.
By evaluating the degrees of freedom of $\hat{\bv}$ we observe
that some of them vanish or depend exclusively on $\hat{u}_1$ in the way
we need them to depend on $\hat{u}_1$. As for the others,
pick first                                                %{P}_{\hat{f}_5} = 
$\hat{q}_0 \in
P_{k-1}(\hat{x}_1)\otimes P_{k-1}(\hat{x}_3)$
and extend it as the same polynomial  
$\hat{q}$ to $Q_{k-1,k-1,k-1}$.
\begin{IEEEeqnarray*}{rCl}
  \hat\rho_{\hat{f}_5,\,\hat{q}_0} (\hat{\bv})
  & = & \iint_{\hat{f}_5} \hat{u}_1\,\hat{q}_0\,d\hat{S} + 
  \sqrt{2}\iint_{[0,1]^2} \hat{q}_0(\hat{x}_1,\hat{x}_3)
  \int_0^{1-\hat{x}_1}\tfrac{\partial \hat{v}_2}{\partial\hat{x}_2}
    (\hat{x}_1,\hat{t},\hat{x}_3)\,d\hat{t}d\hat{x}_1d\hat{x}_3\\[5pt]    
  & = & \iint_{\hat{f}_5} \hat{u}_1\,\hat{q}_0\,d\hat{S} + 
  \sqrt{2}\int_{\hat{E}} \hat{q}\tfrac{\partial \hat{v}_2}{\partial\hat{x}_2}\,d\hat{\bx}\\[5pt]    
  & = & \iint_{\hat{f}_5} \hat{u}_1\,\hat{q}_0\,d\hat{S} + 
  \sqrt{2}\int_{\hat{E}} \hat{q}\,\{\,\mbox{div} (\hat{u}_1,\hat{u}_2,0)' -
    \tfrac{\partial \hat{u}_1}{\partial\hat{x}_1}\,\}\,d\hat{\bx}.
\end{IEEEeqnarray*}
For the volume degrees of freedom~(\ref{momentos3hdiv}) take $\hat{q}_2
\in P_{k-2,k-1}$. Write 
\[
\hat{v}_2\xyz = \int_0^{\hat{x}_2} 
\tfrac{\partial\hat{u}_2}{\partial\hat{x}_2}(\hat{x}_1,\hat{t},\hat{x}_3)\,d\hat{t}
\]
and do
\begin{IEEEeqnarray*}{rCl}
  \int_{\hat{E}} \hat{v}_2\,\hat{q}_2 
  & = &\int\limits_0^1\int\limits_0^1\int\limits_0^{1-\hat{x}_1}
  \int\limits_0^{\hat{x}_2}
    \tfrac{\partial \hat{u}_2}{\partial \hat{x}_2} 
    (\hat{x}_1,\hat{t},\hat{x}_3)\,d\hat{t}\,\hat{q}_2\xyz\,d\hat{x}_2\,d\hat{x}_1\,d\hat{x}_3\\
  & = &\int\limits_0^1\int\limits_0^1\int\limits_0^{1-\hat{x}_1}\int\limits_0^{\hat{x}_2}
        \tfrac{\partial\hat{u}_2}{\partial\hat{x}_2}(\hat{x}_1,\hat{t},\hat{x}_3)
        \,\hat{q}_2 \xyz\,d\hat{t}\,d\hat{x}_2\,d\hat{x}_1\,d\hat{x}_3\\
  & = &\int\limits_0^1\int\limits_0^1\int\limits_0^{1-\hat{x}_1}\int\limits_{\hat{t}}^{1-\hat{x}_1}
        \tfrac{\partial\hat{u}_2}{\partial\hat{x}_2}(\hat{x}_1,\hat{t},\hat{x}_3)\,\hat{q}_2\xyz\,
        d\hat{x}_2\,d\hat{t}\,d\hat{x}_1\,d\hat{x}_3\\
  & = &\int\limits_0^1\int\limits_0^1\int\limits_0^{1-\hat{x}_1}
        \tfrac{\partial \hat{u}_2}{\partial \hat{x}_2}(\hat{x}_1,\hat{t},\hat{x}_3)
        \int\limits_{\hat{t}}^{1-\hat{x}_1}\,\hat{q}_2\xyz\,d\hat{x}_2\,d\hat{t}\,d\hat{x}_1\,d\hat{x}_3\\
  & = &\int\limits_0^1\int\limits_0^1\int\limits_0^{1-\hat{x}_1}
  \tfrac{\partial\hat{u}_2}{\partial\hat{x}_2}(\hat{x}_1,\hat{t},\hat{x}_3)\,
       \hat{\phi} (\hat{x}_1,\hat{t},\hat{x}_3)\,d\hat{t}\,d\hat{x}_1\,d\hat{x}_3\\
& = &\int_{\hat{E}} \mbox{div} (\hat{u}_1,\hat{u}_2,0)'\,\hat{\phi}\,d\hat{\bx}
    - \int_{\hat{E}}\tfrac{\partial\hat{u}_1}{\partial\hat{x}_1}\,\hat{\phi}\,d\hat{\bx}
\end{IEEEeqnarray*}
(for some $\hat{\phi} \in  P_{k-1}(\hat{f}_3)\otimes P_{k-1}(\hat{x}_3)$),
which is what we needed. The inequality~(\ref{teoremaDiv_2}) is proved in the same way.
For inequality~(\ref{teoremaDiv_3}) we can do
\begin{IEEEeqnarray*}{rCl}
  (\rku)_3 & = & (\hat{\br}_k(0,0,\hat{u}_3)')_3\\[4pt]
  & = & \sum_{i=3,4;\,\hat{\bq}}
  \iint_{\hat f_i} \hat{u}_3 \hat{q}_3\,d\hat{S} \,(\hat{\bv}_{\hat{f}_i,\hat{\bq}})_3
    +\sum_{\hat{\br}}
  \int_{\hat E} \hat{u}_3\,\hat{r}_3\,d\hat{\bx}\,(\hat{\bv}_{\hat{\br}})_3.
\end{IEEEeqnarray*}
Then, by standard results for traces in Sobolev spaces,
\begin{IEEEeqnarray*}{rCl}
  \|(\rku)_3\|_{L^\infty(\hat{E})} 
  & \leqslant & C(\hat{E}) \left\{
   \sum_{i=3,4}
     \int\limits_{\hat f_i} |\hat{u}_3|\,d\hat{\gamma}
   + \int\limits_{\hat E} |\hat{u}_3|\,d\hat{\bx}
  \right\}\\
  &\leqslant& C(\hat{E}) (\|\hat{u}_3|_{\partial\hat{E}}\|_{L^1(\partial\hat{E})} + 
    \|\hat{u}_3\|_{L^1(\hat{E})})\\
  &\leqslant& C(\hat{E}) \|\hat{u}_3\|_{W^{1,1}(\hat{E})}.
\end{IEEEeqnarray*}
\end{proof}
Theorem~\ref{thm_stab_div} shows that the interpolation
determined by the finite element in Definition~\ref{defi_h_div_conforme}
is anisotropically stable, in the sense that the image of a field $\hat\bu$ under
the linear operator depends not only continously on $\hat\bu$, but also with a
\emph{componentwise} bound, with perhaps and additional
divergence term. We refer the reader to Theorem~\ref{thm_stab_edge} onwards to
note the same property for the $\bcurl$--conforming case.\\

The next step is to estimate the stability in 
an anisotropically rescaled prism. 
Given three positive numbers
$h_1$, $h_2$ and $h_3$ we denote
\begin{IEEEeqnarray*}{CCl}
    \yesnumber\label{tilde_prism}
    \tilde{E}   &   =   &   \tilde{T} \times \tilde{I}
\end{IEEEeqnarray*}
where
\begin{IEEEeqnarray*}{rCl}
    \tilde{T}   &   =   &   \{ 0 < \nicefrac{\tilde{x}_1}{h_1} + \nicefrac{\tilde{x}_2  }{h_2} < 1 \}\\
    \tilde{I}   &   =   &   \{ 0 < \nicefrac{\tilde{x}_3}{h_3} < 1 \}.
\end{IEEEeqnarray*}
\rescaledPrismTikz
Of course $\tilde{E} = F(\hat{E})$ where $F$ is the linear
$\mathbb{R}^3 \rightarrow \mathbb{R}^3$ transformation such that
\begin{IEEEeqnarray}{rClCl}
  \label{change_var}
  F\hat{\bx} & = & \diag{h_1}{h_2}{h_3} \hat{\bx} & = & \tilde{\bx}.
\end{IEEEeqnarray}
\begin{theorem} \label{thmStabilityKtildeRT}
There is $C > 0$, independent of $h_1$, $h_2$ and $h_3$, s.t. for all $p \geqslant 1$ and 
  $\tilde{\bu}\in W^{1,p}(\tilde{E})$
  \begin{IEEEeqnarray*}{rCl}
    \left\| \rkutilde \right\|_{L^p(\tilde{E})}
    & \leqslant & C \big( \left\| \tilde{\bu} \right\|_{L^p(\tilde{E})}
    + \sum_{i=1}^3 h_i \| \tfrac{\partial\tilde{\bu}}{\partial\tilde{x}_i} \|_{L^p(\tilde{E})}\\
    &&\qquad+ \max\{h_1,h_2\}\|{\dv}(\tilde{u}_1, \tilde{u}_2, 0) \,\|_{L^p(\tilde{E})}\big).
  \end{IEEEeqnarray*}
\end{theorem}
\begin{remark}\label{auxlabel4}
When it comes to estimate in terms of the data $f$ of a problem we
will assume $h_3 \geqslant C\max\{h_1,h_2\}$ so that 
Theorem~\ref{thmStabilityKtildeRT} implies
  \begin{IEEEeqnarray*}{rCl}
    \left\| \rkutilde \right\|_{L^p(\tilde{E})}
    &\leqslant& C \left( \left\| \tilde{\bu} \right\|_{L^p(\tilde{E})}
    + \sum_{i=1}^3 h_i \| \tfrac{\partial\tilde{\bu}}{\partial\tilde{x}_i} \|_{L^p(\tilde{E})}
    + h_{\tilde E}\left\|{\dv}\tilde\bu\right\|_{L^p(\tilde{E})}\right).
  \end{IEEEeqnarray*}
  This is not a restriction since we needed the prisms to be 
  elongated exactly along the direction paralell to the cuadrilateral faces.
\end{remark}
\begin{proof}[Proof of Theorem~\ref{thmStabilityKtildeRT}]
Pick $p\geqslant 1$. If we pull 
$\tilde{\bu}$ back to $\hat{E}$ we get the relation
\begin{IEEEeqnarray}{rCl}\label{pull1}
  \hat{\bu}(\hat{\bx}) & = & (det\,DF)DF^{-1}\tilde{\bu}(F\hat{\bx})
\end{IEEEeqnarray}
\begin{IEEEeqnarray}{rCl}
  D\hat{\bu}(\hat{\bx}) & = & \diag{h_2\,h_3}{h_1\,h_3}{h_1\,h_2}\cdot
  D\tilde{\bu}(F\hat{\bx})\cdot\diag{h_1}{h_2}{h_3}
\end{IEEEeqnarray}
and by~(\ref{div_interp_commutes})
\begin{IEEEeqnarray}{rCl}\label{pull2}
  (\det DF)DF^{-1}\rkutilde(F(\hat{\bx})) & = & \rku(\hat{\bx}).
\end{IEEEeqnarray}
With expressions~(\ref{pull1}) and~(\ref{pull2}) and 
stability inequality~(\ref{teoremaDiv_1}) 
plus H\"older's inequality we obtain 
\begin{IEEEeqnarray*}{rCl}
  \|(\rkutilde)_1\|_{L^{\infty}(\tilde{E})} & = &
  (h_2\,h_3)^{-1}
  \|(\rku)_1\|_{L^{\infty}(\hat{E})}\\[7pt]
\IEEEeqnarraymulticol{3}{r}{
\begin{IEEEeqnarraybox*}{rcL}
\qquad&\leqslant&(h_2\,h_3)^{-1}\left(\|\hat{u}_1\|_{W^{1,1}(\hat{E})}
  +\|\dvg(\hat{u}_1,\hat{u}_2,0)'\|_{L^{1}(\hat{E})}\right)\\[7pt]
  &=& 
  (h_2\,h_3)^{-1}
  \left(
    \int_{\hat{E}}\left|\hat{u}_1\right|\,d\hat{\bx}
    +\sum_{i=1}^3\int_{\hat{E}}\left|\tfrac{\partial\hat{u}_1}{\partial\hat{x}_i}\right|\,d\hat{\bx}
    +\int_{\hat{E}}\left|\tfrac{\partial\hat{u}_1}{\partial\hat{x}_1} + \tfrac{\partial\hat{u}_2}{\partial\hat{x}_2}\right|
    \,d\hat{\bx}
  \right)\\[7pt]
  &=&(\det DF)^{-1}\left(
  \int_{\tilde{E}}|\tilde{u}_1|\,d\tilde{\bx}
  +\sum_{i=1}^3h_i\int_{\tilde{E}}|\tfrac{\partial\tilde{u}_1}{\partial\tilde{x}_i}|\,d\tilde{\bx}
  +h_1\int_{\tilde{E}}|\tfrac{\partial\tilde{u}_1}{\partial\tilde{x}_1} + \tfrac{\partial\tilde{u}_2}{\partial\tilde{x}_2}|
  \,d\tilde{\bx}\right)\\[7pt]
  &=&(2|\tilde{E}|)^{-1}\left(
  \|\tilde{u}_1\|_{L^1(\tilde{E})}
  +\sum_{i=1}^3h_i\|\tfrac{\partial\tilde{u}_1}{\partial\tilde{x}_i}\|_{L^1(\tilde{E})}
  +h_1\|\dvg(\tilde{u}_1,\tilde{u}_2,0)\|_{L^1(\tilde{E})}\right)\\[7pt]
  &\leqslant&(2|\tilde{E}|^{\nicefrac{1}{p}})^{-1}\left(
  \|\tilde{u}_1\|_{L^p(\tilde{E})}
  +\sum_{i=1}^3h_i\|\tfrac{\partial\tilde{u}_1}{\partial\tilde{x}_i}\|_{L^p(\tilde{E})}
  +h_1\|\dvg(\tilde{u}_1,\tilde{u}_2,0)\|_{L^p(\tilde{E})}\right).
\end{IEEEeqnarraybox*}
}
\end{IEEEeqnarray*}
Now
\begin{IEEEeqnarray*}{rCl}
  \|(\rkutilde)_1\|_{L^{p}(\tilde{E})}
  &\leqslant&
  |\tilde{E}|^{1/p}\|(\tilde{\br}_k\tilde{\boldsymbol{u}})_1\|_{L^{\infty}(\tilde{E})}\\
  &\lesssim&\|\tilde{u}_1\|_{L^p(\tilde{E})}
  +\sum_{i=1}^3h_i\|\tfrac{\partial\tilde{u}_1}{\partial\tilde{x}_i}\|_{L^p(\tilde{E})}
  +h_1\|\dvg(\tilde{u}_1,\tilde{u}_2,0)\|_{L^p(\tilde{E})},
\end{IEEEeqnarray*}
and again, the symmetric inequality holds for component two. For component three,
stability inequality~(\ref{teoremaDiv_3}) gives us
\begin{IEEEeqnarray*}{rCl}
  \|(\rkutilde)_3\|_{L^{\infty}(\tilde{E})} & = &({h_1h_2})^{-1}
  \|(\rku)_3\|_{L^{\infty}(\hat{E})}\\[6pt]
  &\leqslant&{C}({h_1h_2})^{-1}\,\|\hat{u}_3\|_{W^{1,1}(\hat{E})}\\[6pt]
  &=&C\,|\tilde{E}|^{-1}\,\left[\|\tilde{u}_3\|_{L^1(\tilde{E})} +
    \sum_{i=1,2,3} h_i\,\|\tfrac{\partial\tilde{u}_3}{\partial\tilde{x}_i}\|_{L^1(\tilde{E})}\right]\\[6pt]
  &\leqslant& {C}\,|\tilde{E}|^{-\nicefrac{1}{p}}\,\left[\|\tilde{u}_3\|_{L^p(\tilde{E})} +
    \sum_{i=1,2,3} h_i\,\|\tfrac{\partial\tilde{u}_3}{\partial\tilde{x}_i}\|_{L^p(\tilde{E})}\right]
\end{IEEEeqnarray*}
so, immediately,
\begin{IEEEeqnarray}{rCl} \label{aux_label18}
  \|(\rkutilde)_3\|_{L^{p}(\tilde{E})}
  &\leqslant& C\,\left(
  \|\tilde{u}_3\|_{L^p(\tilde{E})} +
    \sum_{i=1,2,3} h_i\,\|\tfrac{\partial\tilde{u}_3}{\partial\tilde{x}_i}\|_{L^p(\tilde{E})}
  \right)
\end{IEEEeqnarray}
and the sum of the three estimates yields the Theorem.
\end{proof}
%===============================================================================
% \begin{lemma}\label{L6} Let $P$ be a right prism. There exists a constant $C$ depending only on $\alpha_P$ such that for all $\bu$ in $W^{1,1}(P)$ we have
% \begin{multline}\label{estabL1}
% \|\bu_I\|_{L^1(P)} \leqslant C\Bigg(\|\bu\|_{L^1(P)} + \sum_{i=1}^3 h_{i,P}\|\partial_{\xi_{P,i}}\bu\|_{L^1(P)}\\ + \max\{h_{P,1},h_{P,2}\}\|\mbox{div\,}(u_1,u_2,0)\|_{L^1(P)}\Bigg).
% \end{multline}
% \end{lemma}
% \begin{proof} Using the notation introduced above for the vertices of $P$, suppose that $v_0$ is the vertex with the maximum angle of the triangle $v_0v_1v_2$. Let $\tilde P$ be a prism with vertices at $(0,0,0)$, $(h_{P,1},0,0)$, $(0,h_{P,2},0)$, $(0,0,h_{P,3})$, $(h_{P,1},0,h_{P,3})$ and $(0,h_{P,2},h_{P,3})$. Then by standard rescaling arguments using the Piola Transform we can prove from Lemma \eqref{L5} that there exists a constant $C$ such that for all $\bu\in W^{1,1}(\tilde P)$ we have
% \begin{eqnarray*}
% \|\tilde\bu_I\|_{L^1(\tilde P)}&\le& C\Bigg(\|\tilde\bu\|_{L^1(\tilde P)} + \sum_{i=1}^3h_{P,i}\|\partial_{x_i}\tilde\bu\|_{L^1(\tilde P)}\\&&\qquad + 
% \max\{h_{P,1},h_{P,2}\}\|\dv(\tilde u_1,\tilde u_2,0)\|_{L^1(\tilde P)}\Bigg).
% \end{eqnarray*}
% Let $B$ be the matrix with columns $\xi_{P,1}$, $\xi_{P,2}$ and $\xi_{P,3}$ (note $B$ has the form \eqref{matrix} and $\xi_{P,3}=(0,0,1)$). Then the map $F(\tilde{\bf x})=B\tilde{\bf x}+v_0$ sends $\tilde P$ onto $P$. Then, again by a change of variables, we obtain from the previous estimate, that for all $\bu\in W^{1,1}(P)$ it holds
% \begin{eqnarray*}
% \|\bu_I\|_{L^1(P)}&\le& C\|B\|\|B^{-1}\|\bigg(\|\bu\|_{L^1(P)} + \sum_{i=1}^3h_{P,i}\|\partial_{\xi_{P,i}}\bu\|_{L^1(P)}\\ &&  \qquad +\max\{h_{P,1},h_{P,2}\}\frac1{\|B^{-1}\|}\|\dv(u_1,u_2,0)\|_{L^1(P)}\bigg). 
% \end{eqnarray*}
% Then the proof concludes by noting that $\|B\|\leqslant C$ and $\|B^{-1}\|\sim \sin\alpha_P$.
% \end{proof}
% 
% \begin{remark} Stability estimates in $L^p$-norm, $p>1$, can by proved analogously.
% In particular, from  \eqref{estabL1}, using an inverse inequality on the left hand side, and Cauchy-Schwarz inequality on the right hand side, we obtain under assumptions of Lemma \ref{L6}
% \begin{multline}\label{estabL2}
% \|\bu_I\|_{L^2(P)} \leqslant C\Bigg(\|\bu\|_{L^2(P)} + \sum_{i=1}^3 h_{i,P}\|\partial_{\xi_{P,i}}\bu\|_{L^2(P)}\\ + \max\{h_{P,1},h_{P,2}\}\|\mbox{div\,}(u_1,u_2,0)\|_{L^2(P)}\Bigg)
% \end{multline}
% \end{remark}
%===============================================================================
\subsection{Local Interpolation Estimates for Prismatic Elements} % (fold)
\label{sub:local_interpolation_estimates_for_prismatic_elements}
We will state scaling consequences
of inequalities~(\ref{aux_label19}). Some of them will be used here
and the rest will be used in Chapter~\ref{auxlabel202}.
%=========
%For a non--negative integer $m$ let $\partial^m f$ denote the sum of the absolute values of all the derivatives of order $m$ of $f$.
%=========
\begin{remark}\label{aux_label28} Recall the vector averaged Taylor polynomial $\Qbb_{m,E}(\cdot)$
defined in~\eqref{auxlabel210}.
If $(\,\cdot\,)\,\hat{}\,$ denotes any of transformations~(\ref{transfHcurl})
and~(\ref{transfDiv}) then it holds
\begin{IEEEeqnarray*}{rCl}
  \Qbb_{m,\hat{E}}\hat{\bw} & = & (\Qbb_{m,E}\,\bw)\,\hat{}.
\end{IEEEeqnarray*}
\end{remark}
\begin{lemma}\label{aux_label20}
Let $\tilde E$ be the rescaled reference prism in Figure~\ref{rescaled_prism}
and denote its diameter by $h$.
Given $p\geqslant 1$, $m\geqslant 0$ and
$\tilde\bu \in W^{m+1,p}(\tilde E)$,  then
for $m \geqslant 0$ and $p \geqslant 1$ the following items hold.
\begin{enumerate}
%  \item 
%  For any component $1\leqslant i\leqslant 3$
%  \begin{IEEEeqnarray}{rCl}\label{aux_label30} 
%    \|\tilde  u_i - \tilde\Qb_{m,\tilde E}\,\tilde u_i \|_{L^p(\tilde E)}& \leqslant &
%      C \sum_{j+k+l=m+1} h_1^jh_2^kh_3^l 
%      \left\|\tfrac{\tilde\partial^{m+1}\tilde u_i}
%      {\partial\tilde x_1^j\partial\tilde x_2^k\partial\tilde x_3^l}
%      \right\|_{L^p(\tilde E)}
%  \end{IEEEeqnarray}
  \item 
  For any component $1\leqslant i\leqslant 3$, for any $\bbeta$ with 
  $|\bbeta|\leqslant m+1$ 
  \begin{IEEEeqnarray}{rCl}\label{aux_label30} 
    \|\partial^{\bbeta}(\tilde  u_i - \tilde\Qb_{m,\tilde E}\,\tilde u_i)\|_{L^p(\tilde E)}
    & \leqslant & C 
    \sum_{|\balpha|=m-|\bbeta|+1} \bh^{\balpha+\bbeta} 
      \left\|\partial^{\,\balpha + \bbeta}\tilde u_i\right\|_{L^p(\tilde E)}
  \end{IEEEeqnarray}
  \item 
  For any component $1\leqslant l\leqslant 3$ 
  \begin{IEEEeqnarray}{rCl}\label{aux_label31}
    \left\|\tfrac{\partial}{\partial \tilde x_i}(\tilde u_l-\tilde\Qb_{m,\tilde E}\,\tilde u_l)\right\|_{L^p(\tilde E)}
    &\leqslant&C\sum_{|\alpha| = m} \bh^{\balpha} 
    \left\|\tfrac{\partial}{\partial\tilde x_i} (\partial^{\alpha}\tilde u_l)
    \right\|_{L^p(\tilde E)}
  \end{IEEEeqnarray}
  %%%===========================================================================
  % \begin{IEEEeqnarray}{rCl}\label{aux_label31}
  %   \left \|\tfrac{\partial}{\partial \tilde x_i}(\tilde\bu-\tilde\bq)\right\|_{L^p(\tilde E)}
  %   &\leqslant&C\sum_{|\alpha| = m} \bh^\alpha 
  %   \left \|\tfrac{\partial}{\partial\tilde x_i} (\partial^{\alpha}\tilde\bu)
  %   \right\|_{L^p(\tilde E)}
  % \end{IEEEeqnarray} 
  %%%===========================================================================
  \item For any component $1\leqslant i \leqslant 3$,
  if $m \geqslant 1$ and $p \geqslant 1$
  %%=============================================================================
  %\begin{IEEEeqnarray}{rCl}
  %  \label{aux_label24__}
  %  \|\curl(\tilde\bu-\tilde\bq)\|_{L^p(\tilde E)}&\leqslant&
  %  C\,h^m
  %  \|\partial^m\curl\tilde\bu\|_{L^p(\tilde E)}\\[5pt]
  %  \label{aux_label25__}
  %  \|\tfrac{\partial}{\partial_{\tilde x_i}}\curl(\tilde\bu-\tilde\bq)\|_{L^p(\tilde E)}&\leqslant&
  %  C\,h^{m-1}
  %  \|\partial^m\curl\tilde\bu\|_{L^p(\tilde E)}
  %\end{IEEEeqnarray}
  %%=============================================================================
  \begin{IEEEeqnarray}{rCl}
    \label{aux_label24}
    \|\curl(\tilde\bu- \tilde\Qbb_{m,\tilde{E}}\tilde{\bu})_i\|_{L^p(\tilde E)}&\leqslant&
    C\,\sum_{j+k+l=m}  h_1^jh_2^kh_3^l
    \left\|\tfrac{\tilde\partial^m(\curl\tilde\bu)_i}{\partial\tilde x_1^j\partial\tilde x_2^k\partial\tilde x_3^l}
    \right\|_{L^p(\tilde E)}
  \end{IEEEeqnarray}
  \item For any component $1\leqslant i \leqslant 3$ and any $1\leqslant j \leqslant 3$
  \begin{IEEEeqnarray}{rCl}
    \label{aux_label25}
    \left\|\tfrac{\partial\curl(\tilde\bu-\tilde\bq)_i}{\partial\tilde x_j}
    \right\|_{L^p(\tilde E)}
      &\leqslant& 
        C\,\sum_{j+k+l=m-1}  h_1^jh_2^kh_3^l
          \left\|\tfrac{\tilde\partial^{m-1}\tilde\partial(\tilde\curl\tilde\bu)_i}
                 {\partial\tilde x_1^j\partial\tilde x_2^k\partial\tilde x_3^l\partial\tilde x_j}
          \right\|_{L^p(\tilde E)}
  \end{IEEEeqnarray}
  \item For the divergence it holds
  \begin{IEEEeqnarray}{rCl}\label{aux_label23}
   \|\tilde{\dv}(\tilde\bu-\tilde\Qbb_{m,\tilde{E}}\tilde{\bu})\|_{L^p(\tilde E)}&\leqslant&
   C\sum_{j+k+l=m} h_1^jh_2^kh_3^l
   \left\|
    \tfrac{\tilde \partial^m\tilde{\text{div}}\tilde\bu}{\partial\tilde x_1^j\partial\tilde x_2^k\partial\tilde x_3^l}
   \right\|_{L^p(\tilde E)}
  \end{IEEEeqnarray}
\end{enumerate}
where C depends only on $m$, $\sigma$ (cfr. Theorem~\ref{aux_label21})
and the reference element.
\end{lemma}
\begin{proof}[Proof of Lemma~\ref{aux_label20}]
We will use  Lemma~\ref{aux_label40} rescaling $\hat{E}$ so that $d = 1$ and then
pushing forward to $\tilde E$. In fact, by~\eqref{aux_label29} for any multi--index 
$\balpha$ we have
  \begin{IEEEeqnarray}{rCl}\label{aux_label41}
    h_1^{\alpha_1}h_2^{\alpha_2}h_3^{\alpha_3}\tilde\partial^\alpha\tilde u_i (\tilde\bx)
    & = & (\nicefrac{1}{h_i}\hat\partial^{\alpha}\hat u_i)(F_{\tilde E}^{-1}\tilde\bx)
  \end{IEEEeqnarray}
  so, to prove~\eqref{aux_label30},
  \begin{IEEEeqnarray*}{rCl}
  \|\partial^{\bbeta}(\tilde  u_i - \tilde\Qb_{m,\tilde E}\,\tilde u_i)\|^p_{L^p(\tilde E)}
  & = & \|\frac{1}{h_i}(\partial^{\bbeta} u_i - \Qb_{m-|\bbeta|, E}\,\partial^{\bbeta}u_i)\circ F_{\tilde E}^{-1}\|^p_{L^p(\tilde E)} \\[4pt]
  & = & \frac{|\det M_E|}{h_i^p} \| \partial^{\bbeta}  u_i - \Qb_{m-|\bbeta|, E}\, \partial^{\bbeta} u_i\|^p_{L^p(\hat E)} \\[4pt]
  & \leqslant & \frac{C|\det M_E|}{h_i^p} |\partial^{\bbeta} u_i|^p_{m-|\bbeta|+1,p,\hat E} \\[4pt]
  & = & C|\det M_E| \sum_{|\balpha  |=m-|\bbeta|+1} \left\|
   \nicefrac{1}{h_i} \partial^{\balpha+\bbeta} u_i\right\|^p_{L^p(\hat{E})} \\[4pt]
  (\mbox{by~(\ref{aux_label41})})\qquad& = &
  C \sum_{|\balpha  |=m-|\bbeta|+1} 
  (\bh^{\balpha+\bbeta})^p \left\| \tilde\partial^{\balpha+\bbeta}\tilde u_i
    \right\|^p_{L^p(\tilde E)}.
\end{IEEEeqnarray*}
To prove~(\ref{aux_label24}), by a straightforward manipulation we have
\begin{IEEEeqnarray*}{rCl}
  \tilde\curl \tilde \Qbb_{m,\tilde E}\tilde\bu & = & 
  \tilde \Qbb_{m-1,\tilde E} \tilde\curl \tilde\bu.
\end{IEEEeqnarray*}
Then, componentwise, by~(\ref{aux_label30}), 
\begin{IEEEeqnarray*}{rCl}
  \|\tilde\curl(\tilde\bu-\tilde \Qbb_{m,\tilde E}\tilde\bu)_i\|_{L^p(\tilde E)}& = &
    \|(\tilde\curl\tilde\bu)_i - \tilde\Qb_{m-1}(\tilde\curl\tilde\bu)_i\|_{L^p(\tilde E)}\\[4pt]
  &\leqslant& C \sum_{j+k+l=m} h_1^jh_2^kh_3^l 
  \left\|\frac{\tilde\partial^m(\tilde\curl\tilde\bu)_3}
         {\partial\tilde x_1^j\partial\tilde x_2^k\partial\tilde x_3^l}\right\|_{L^p(\tilde E)}.
\end{IEEEeqnarray*}
Now~(\ref{aux_label25}) is an easy consecuence because
\begin{IEEEeqnarray*}{rCl}
\left\|\frac{\partial\curl(\tilde\bu-\tilde\bq)_i}{\partial\tilde x_j}\right\|_{L^p(\tilde E)}
    & = & 
\left\|\frac{\partial(\tilde\curl\tilde\bu)_i}{\partial\tilde x_j} - 
  \frac{\partial\tilde\Qb_{m-1,\tilde E}(\tilde\curl\tilde\bu)_i}{\partial\tilde x_j}
\right\|_{L^p(\tilde E)} \\
& = & 
\left\|\frac{\partial(\tilde\curl\tilde\bu)_i}{\partial\tilde x_j} - 
  \tilde\Qb_{m-2,\tilde E}\frac{\partial(\tilde\curl\tilde\bu)_i}{\partial\tilde x_j}
\right\|_{L^p(\tilde E)} \\
    &\leqslant& 
      C\,\sum_{j+k+l=m-1}  h_1^jh_2^kh_3^l
        \left\|\frac{\tilde\partial^{m-1}\tilde\partial(\tilde\curl\tilde\bu)_i}
               {\partial\tilde x_1^j\partial\tilde x_2^k\partial\tilde x_3^l\partial\tilde x_j}
        \right\|_{L^p(\tilde E)}.
\end{IEEEeqnarray*}
Equality~\eqref{aux_label23} follows in a simpler way.
\end{proof}
We arrived at one of the main results in this Thesis which is the an\-iso\-tropic
local interpolation error estimates for div--conforming elements on prisms of any order.
As we said earlier, in Chapter~\ref{auxlabel202} we will arrive
at the $\bcurl$--conforming analogue.
\\[5pt]

%% Sean un prisma obl\'{\i}cuo irregular $E$, $k\in\mathbb{N}$, el operador de interpolaci\'on 
%% $\bw_E$ de grado $k$ determinado por el elemento en la Definición~(\ref{edgeelement}), y
%% $p>2$. 
%% such that (\noindent{\color{BrickRed}hip. prisma}).
We will adopt the following notations. For a prism $E$, 
$h$ will denote its diameter and, for $1\leqslant i \leqslant 3$ ,
$\xi_i$ will denote unitary vectors with the directions
of the three edges $\be_i$ sharing a vertex $\bx_E$ of $E$ whose lengths are
$h_i$ and $\xi_3$ will be the particular direction of
the edge which is common to two quadrilateral faces. Recall that, for 
$\bh = (h_1,h_2,h_3)'$, $\bh^{\balpha}$ means 
$h_1^{\alpha_1}h_2^{\alpha_2}h_3^{\alpha_3}$
and 
$\partial^{\balpha} = \frac{\partial^{|{\balpha}|}}
{\partial\xi_1^{\alpha_1}\xi_2^{\alpha_2}\xi_3^{\alpha_3}}.$

Following the proof of Theorem 6.2 in~\cite{aadl} and adding the information 
of~\eqref{aux_label23},
from Remark~\ref{auxlabel4} we derive the following Theorem.
\begin{theorem}\label{aux_label46}
Let $k\in\mathbb{N}_0$ and $p \geqslant 1$.
Let $E$ be a prism whose triangular
faces have greatest angle less than $c_0$.
There exists $C > 0$ and three edges $\be_i$ of $E$ incident to a common vertex
$\bx_E$ such that for all $\bu\in W^{m + 1,p}(E)^3$
and $m\leqslant k$,
\begin{IEEEeqnarray}{C}\nonumber
  \|\bu-\br_E \bu\|_{L^p(E)} \leqslant C \left\{
  \sum_{|{\balpha}|=m+1}\bh^{\balpha} \|\partial^{\balpha}\bu\|_{L^p(E)} +
  h_E\sum_{|\balpha| = m}
  	\bh^{\balpha}\|\partial^{\balpha}\text{div} \,\bu\|_{L^p(E)} \right\}.\\[4pt]
  \label{aux_label39}
\end{IEEEeqnarray}
\end{theorem}

%
%Take a kernel $\hat{\phi}\in{C}_0^\infty(\mathbb{R}^n)$ such that
%\begin{IEEEeqnarray*}{rCl}
  %\supp{\hat{\phi}}&\subseteq&\hat{B}\\[5pt]
  %\int\limits_{\hat{B}} \hat{\phi}\,d\bx & = & 1
%\end{IEEEeqnarray*} 
%Now define the kernel on B
%\begin{IEEEeqnarray*}{rCl}
  %\phi_B&=&\frac{|\hat{B}|}{|B|}\,\hat{\phi}\circ F^{-1}\\[5pt]
  %&=&\frac{1}{\det DF}\,\hat{\phi}\circ F^{-1}
%\end{IEEEeqnarray*}
%This way we have
%\begin{IEEEeqnarray*}{rClCr}
  %\int\limits_{B}\phi\,d\textbf{x}&=&
  %\int\limits_{\hat{B}}\hat{\phi}\,\,d\hat{\textbf{x}}&=&1\\[5pt]
%\end{IEEEeqnarray*}
%and also
%\begin{IEEEeqnarray*}{rClClCr}
  %\supp{\phi_B}&=&\supp{\hat{\phi}\circ F^{-1}}
  %&=&F(\supp{\hat{\phi}})&=&B.
%\end{IEEEeqnarray*}
%Then we build an averaged Taylor polynomial over $B$ with the kernel
%$\phi_B$ and it holds that
%where 
%%==============================================================================
% \begin{theorem}\label{thmErrorInterpolacionPrismas}
% Let $P$ be a right prism, and consider a local system of coordinates $x_1x_2x_3$
% such that the triangular basis of $P$ are parallel to the $x_1x_2$-coordinate
% plane. Denote by $\xi_{P,1}$ and $\xi_{P,2}$ the versors parallel to the edges
% of the triangular basis of $P$ adjacent to its maximum angle
% $\alpha_P$, $\xi_{P,3}=(0,0,1)$ and $h_{P,i}$ are the lengths of the edges of
% $P$ parallel to $\xi_{P,i}$. We assume that $h_{P,3}>ch_{P,1}$ and
% $h_{P,3}>ch_{P,2}$. Then, there exists a constant $C$ depending only on $c$
% and $\alpha_P$, such that for all $\bu\in H^1(P)$ we have
% \begin{equation}\label{interp}
% \|\bu-\boldsymbol{r}_E\bu\|_{L^2(E)} \leqslant C\left(\sum_{i=1}^3 h_{E,i}
% \|\partial_{\xi_{E,i}}\bu\|_{L^2(E)} + h_T\|\dv\bu\|_{L^2(E)}\right).
% \end{equation}
% \end{theorem}
%%==============================================================================
% subsection local_interpolation_estimates_for_prismatic_elements (end)
% section prismatic_finite_elements (end)

\section{Pyramidal Finite Elements} % (fold)
\label{sec:pyramidal_finite_elements}




$\hat{E}$ will be the reference pyramid  in Figure~\ref{reference_pyramid}.

\subsection{Edge Elements} % (fold)
\label{sub:edge_elements}
The proof is based on mere calculation. 
Let us recall the shape funtions in Table~(\ref{shape_edge_table}) and
start with $\bu$ of the form $(u_1,0,0)'$. After computations we have
\begin{IEEEeqnarray*}{rCl}
	\nabla\times\bu &=& (0, \frac{{\s\partial} u_1}{{\s\partial} x_3},-\frac{{\s\partial} u_1}{{\s\partial} x_2})\\[5pt]
	\wku	&=& [{\s\int_{\hat{\be}_1}\bu\cdot\btau\, ds}]\bgamma_1 +
				[{\s\int_{\hat{\be}_3}\bu\cdot\btau\, ds}]\bgamma_3 + 
				[{\s\int_{\hat{\be}_6}\bu\cdot\btau\, ds}]\bgamma_6 + 
				[{\s\int_{\hat{\be}_8}\bu\cdot\btau\, ds}]\bgamma_8\\[5pt]
			&=& \alpha_1(\hat\bu)\hat\bgamma_1 + 
				\alpha_3(\hat\bu)\hat\bgamma_3 + 
				\alpha_6(\hat\bu)\hat\bgamma_6 + 
				\alpha_8(\hat\bu)\hat\bgamma_8
\end{IEEEeqnarray*}
\begin{IEEEeqnarray*}{rCl}
  (\wku)_1(x,y,z) 
    &  = & \alpha_1(\hat\bu)(1-z-y)+ 
	  \alpha_3(\hat\bu)y+ 
	  \alpha_6(\hat\bu)(-z+\frac{yz}{1-z})+ 
	  \alpha_8(\hat\bu)(-\frac{yz}{1-z})\\
	& = & \alpha_1(\hat\bu) - (\alpha_1 + \alpha_6)(\hat\bu)\,z+ 
	  (\alpha_3 - \alpha_1)(\hat\bu)\,y + (\alpha_6-\alpha_8)(\hat\bu)\,\frac{yz}{1-z}.
\end{IEEEeqnarray*}
Now we explore the new coefficients separately. The tangential component of $\hat\bu$
along $\hat\be_5$ equals zero, and this is an argument we are using repeatedly in the forthcoming
computations.\\[4pt]
\noindent By Stokes' Theorem we have
\begin{IEEEeqnarray*}{rCl}
  (\alpha_1+\alpha_6)(\hat\bu)
  	& = & \int_{\hat{\be}_1}\hat\bu\cdot\hat\btau_1\, ds +
  											\int_{\hat{\be}_6}\hat\bu\cdot\hat\btau_6\, ds -
  											\int_{\hat{\be}_5}\hat\bu\cdot\hat\btau_5\, ds \\[5pt]
  	& = & \int_{\hat{f}_1} \nabla\times\hat\bu\cdot\hat\bn\,d\gamma \\[5pt]
  	& = & -\int_{\hat{f}_1} \frac{{\s\partial} u_1}{{\s\partial} x_3}\,d\gamma.
\end{IEEEeqnarray*}
Next,
\begin{IEEEeqnarray*}{rCl}
	(\alpha_3-\alpha_1)(\hat\bu) & = & (\alpha_3-\alpha_2-\alpha_1+\alpha_4)(\hat\bu)\\
	& = & - \int_{\partial\hat{f}_5}\hat{\bu}\cdot\hat\btau\,d\hat{s}\\
	&=& -\iint_{\hat{f}_5}\nabla\times\hat{\bu}\cdot\hat{\bn}_5\,d\gamma\\
	&=&	 \iint_{\hat{f}_5}\frac{{\s\partial} \hat{u}_1}{{\s\partial} \hat{x}_2}\,d\gamma.
\end{IEEEeqnarray*}
And
\begin{IEEEeqnarray*}{rCl}
  (\alpha_6-\alpha_8)(\hat\bu) & = & \left(\int_{\hat\be_6}\hat\bu\cdot\hat\btau_6\,ds-
    \int_{\hat\be_8}\hat\bu\cdot\hat\btau_6\,ds-
	\int_{\hat\be_2}\hat\bu\cdot\hat\btau_6\,ds\right)\\[5pt]
	& = &-\int_{\partial\hat{f}_3}\hat\bu\cdot\hat\btau\,ds  
	  =  -\int_{\hat{f}_3}\nabla\times\hat\bu\cdot\bn\,d\gamma
	  =   \int_{\hat{f}_3}\frac{{\s\partial} \hat u_1}{{\s\partial} x_2}\,d\gamma.
\end{IEEEeqnarray*}
By exactly the last computation,
\begin{IEEEeqnarray*}{rCl}
  (\wku)_2 & = & (\alpha_6-\alpha_8)(\hat\bu)\frac{xz}{1-z}
  = \int_{\hat{f}_3}\frac{{\s\partial}\hat u_1}{{\s\partial} x_2}\,d\gamma\,\frac{xz}{1-z}.
\end{IEEEeqnarray*}
Next,
\begin{IEEEeqnarray*}{rCl}
	(\wku)_3 & = &     \alpha_1(\hat\bu)\left(x-\frac{xy}{1-z}\right) + \alpha_3(\hat\bu)\frac{xy}{1-z}\\[6pt]
			 &   &\,+\,\alpha_6(\hat\bu)\left(x-\frac{xy}{1-z}+\frac{xyz}{(1-z)^2}\right)
		            +  \alpha_8(\hat\bu)\left(\frac{xy}{1-z}-\frac{xyz}{(1-z)^2}\right)\\[6pt]
			 & = &  (\alpha_1 + \alpha_6)(\hat\bu)\,x +
			 		(\alpha_3-\alpha_1+\alpha_8-\alpha_6)(\hat\bu)\,\frac{xy}{1-z}\\[6pt]
			 &   &\,+ (\alpha_6-\alpha_8)(\hat\bu)\,\frac{xyz}{(1-z)^2}.
\end{IEEEeqnarray*}
As $\hat\bu$ has zero tangential component along $\be_5$ and $\be_7$,
\begin{IEEEeqnarray*}{rCl}
  (\alpha_3-\alpha_1+\alpha_8-\alpha_6)(\hat\bu)&=&
  (\alpha_3-\alpha_7+\alpha_8)(\hat\bu)+(\alpha_5-\alpha_6-\alpha_1)(\hat\bu)\\[8pt]
  &=&-\int_{\partial\hat{f}_4}\hat\bu\cdot\hat\btau\,ds
   -\int_{\partial\hat{f}_1}\hat\bu\cdot\hat\btau\,ds\\[8pt]
  &=&-\iint_{\hat{f}_1}\nabla\times\hat\bu\cdot\hat\bn\,d\gamma
   -\iint_{\hat{f}_4}\nabla\times\hat\bu\cdot\hat\bn\,d\gamma\\[8pt]
\yesnumber\label{pyr_edge_one}
  &=&\iint_{\hat{f}_1}(\nabla\times\hat\bu)_2\,d\gamma\\[8pt]
  & &\quad-2^{-\nicefrac{1}{2}}\iint_{\hat{f}_4}[(\nabla\times\hat\bu)_2 + (\nabla\times\hat\bu)_3]\,d\gamma.\\[8pt]
  &=&\iint_{\hat{f}_1}\frac{\partial\hat{u}_1}{\partial\hat{x}_3}\,d\hat\gamma
  -2^{-\nicefrac{1}{2}}\iint_{\hat{f}_4}[\frac{\partial\hat{u}_1}{\partial\hat{x}_3}
   + \frac{\partial\hat{u}_1}{\partial\hat{x}_2}]\,d\hat\gamma.
\end{IEEEeqnarray*}
\noindent Now, as expected, we switch to $\bu = (0,u_2,0)'$. In this case we have
\begin{IEEEeqnarray*}{rCl}
  \wku     & = & \alpha_2(\hat{\bu})\,\bgamma_2 +
	\alpha_4(\hat{\bu})\,\bgamma_4+ \alpha_7(\hat{\bu})\,\bgamma_7+\alpha_8(\hat{\bu})\,\bgamma_8.\\
  (\wku)_1 & = &(\alpha_7-\alpha_8)(\hat{\bu})\,\frac{yz}{1-z}\\
  		   & = &(\alpha_7-\alpha_8 - \alpha_3)(\hat{\bu})\,\frac{yz}{1-z}\\
  		   & = &\int_{\partial\hat{f}_4}\hat{\bu}\cdot\btau\,d\hat{s}\,\frac{yz}{1-z}\\
  		   & = &\iint_{\hat{f}_4} \nabla\times\hat\bu\cdot\,d\gamma\,\frac{yz}{1-z}
  		     =  \iint_{\hat{f}_4} \frac{\partial\hat{u}_2}{\partial\hat{x}_1}\,d\gamma\,\frac{yz}{1-z}.
\end{IEEEeqnarray*}
For the next component,
\begin{IEEEeqnarray*}{rCl}
	(\wku)_2 & = &\alpha_4(\hat\bu) + (\alpha_2-\alpha_4)(\hat\bu)\,x -
	(\alpha_4+\alpha_7)(\hat\bu)\,z + (\alpha_7-\alpha_8)(\hat\bu)\,\frac{xz}{1-z}.\\
	(\alpha_2-\alpha_4)(\hat\bu) & = & (\alpha_2-\alpha_3-\alpha_4+\alpha_1)(\hat\bu)\\
  &=&-\int_{\partial\hat{f}_5}\hat\bu\cdot\hat\btau\,d\hat{s}\\
  &=&-\iint_{\hat{f}_5}\nabla\times\hat{\bu}\cdot\hat\bn_5\,d\gamma
   =  \iint_{\hat{f}_5}\frac{\partial\hat{u}_2}{\partial\hat{x}_1}\,d\gamma.
\end{IEEEeqnarray*}
\begin{IEEEeqnarray*}{rCl}
  (\alpha_4+\alpha_7)(\hat\bu) & = & 
  (\alpha_4+\alpha_7-\alpha_5)(\hat\bu)\\
  &=& - \int_{\partial\hat{f}_2} \hat\bu\cdot\hat\btau\,d\hat{s}\\
  &=& -\iint_{\hat{f}_2}\nabla\times\hat\bu\cdot\hat\bn\,d\gamma~=~
      -\iint_{\hat{f}_2}\frac{{\s\partial} u_2}{{\s\partial} x_3}.
\end{IEEEeqnarray*}
And for the third one, {\color{red}\#\#\#\#\#\#\#\# aca pruebo una manera de acomodar. ver otras después de
armar la tabla para acotar.}
\begin{IEEEeqnarray*}{rCl}
	(\wku)_3&=&(\alpha_4+\alpha_7)(\hat\bu)\,y + (\alpha_2-\alpha_4-\alpha_7+\alpha_8)(\hat\bu)\,\frac{xy}{1-z}\\[4pt]
	& &\,+\,(\alpha_7-\alpha_8)(\hat\bu)\,\frac{xyz}{(1-z)^2}.\\[8pt]
	&=&(\alpha_2 - \alpha_4)(\hat\bu)\,\frac{xy}{1-z} + (\alpha_4+\alpha_7)(\hat\bu)\,y\\[8pt]
	& &-(\alpha_7-\alpha_8)(\hat\bu)\,\frac{xy}{(1-z)^2}.
\end{IEEEeqnarray*}
{\color{blue}\#\#\#\#\#\#\#\# continue here.}
Finally for $\hat\bu = (0,0,\hat u_3)$ it is
$\wku = \alpha_5(\hat\bu)\bgamma_5 + 
		\alpha_6(\hat\bu)\bgamma_6 + 
		\alpha_7(\hat\bu)\bgamma_7 +
		\alpha_8(\hat\bu)\bgamma_8. $
\begin{IEEEeqnarray*}{rCl}
	(\pi\bu)_1 & = &(\alpha_5-\alpha_6)z+
		(-\alpha_5+\alpha_6+\alpha_7-\alpha_8)\frac{yz}{1-z}.\\
	\alpha_5-\alpha_6 & = &-\int\limits_{\mathcal{C}_{125}}\bu\cdot\btau\,d\sigma\\
	&=&-\iint\limits_{[a_1,a_2,a_5]}\nabla\times\bu\cdot\bn\,dS
\end{IEEEeqnarray*}
($\bn$ is the outer normal $(0,-1,0)$)
Similarly:
\begin{IEEEeqnarray*}{rCl} 	
	\alpha_7-\alpha_8 & = & 
	\iint\limits_{[a_3,a_5,a_4]}\nabla\times\bu\cdot\bn\,dS
\end{IEEEeqnarray*}
so {\color{red}($\nabla\times\bu\cdot(0,1,1)/\sqrt{2} = -\frac{{\s\partial} u_3}{{\s\partial} x_1}$)}
\begin{IEEEeqnarray*}{rCl}
	-\alpha_5+\alpha_6+\alpha_7-\alpha_8 & = & 
	\iint\limits_{[a_1,a_2,a_5]}\nabla\times\bu\cdot\bn\,dS
	+\iint\limits_{[a_3,a_5,a_4]}\nabla\times\bu\cdot\bn\,dS\\
	&=&\int\limits_{0}^{1}
	\int\limits_{0}^{1-t}\frac{{\s\partial} u_3}{{\s\partial} x_1} (s,0,t) \,ds\,dt
	-\int\limits_{0}^{1}
	 \int\limits_{0}^{1-t}\frac{{\s\partial} u_3}{{\s\partial} x_1} (s,1-t,t) \,ds\,dt\\
	&=&-\int\limits_{0}^{1}
	\int\limits_{0}^{1-t}
	\int\limits_{0}^{1-t}\frac{{\s\partial}^2u_3}{{\s\partial} x_2{\s\partial} x_1}(s,r,t)\,dr\,ds\,dt\\
	&=&-\iiint\limits_{\hat{P}}\frac{{\s\partial}^2u_3}{{\s\partial} x_2{\s\partial} x_1}\,dV.
\end{IEEEeqnarray*}
For now
\begin{IEEEeqnarray*}{rCl}
	(\pi\bu)_1 & = & -z\int\limits_{0}^{1-t}\frac{{\s\partial} u_3}{{\s\partial} x_1}\,ds\,dt
	-\frac{yz}{1-z}\iiint\limits_{\hat{P}}
		\frac{{\s\partial}^2u_3}{{\s\partial} x_2{\s\partial} x_1}\,dV.
\end{IEEEeqnarray*}
\begin{IEEEeqnarray*}{rCl}
	(\pi\bu)_2& = &(\alpha_5-\alpha_7)z
				+(-\alpha_5+\alpha_6+\alpha_7-\alpha_8)\frac{xz}{1-z}\\[5pt]
\end{IEEEeqnarray*}
\begin{IEEEeqnarray*}{rCl}
	\alpha_5-\alpha_7& = & \iint\limits_{[a_1,a_5,a_3]}\nabla\times\bu\cdot\bn\,dS\\
	& = &-\int\limits_{0}^{1}\int\limits_{0}^{1-t} \frac{{\s\partial} u_3}{{\s\partial} x_2}(0,s,t)\,ds\,dt
\end{IEEEeqnarray*}
\begin{IEEEeqnarray*}{rCl}
	(\pi\bu)_2& = &
	-z\int\limits_{0}^{1}\int\limits_{0}^{1-t}\frac{{\s\partial} u_3}{{\s\partial} x_2}(0,s,t)\,ds\,dt
	+\frac{xz}{1-z}\iiint\limits_{\hat{P}}
	\frac{{\s\partial}^2u_3}{{\s\partial} x_2{\s\partial} x_1}\,dV.
\end{IEEEeqnarray*}
\begin{IEEEeqnarray*}{rCl}
	(\pi\bu)_3&=& \alpha_5 + x(-\alpha_5+\alpha_6)+y(-\alpha_5+\alpha_7)\\[5pt]
	&&\,+(\alpha_5-\alpha_6-\alpha_7+\alpha_8)\frac{xy}{1-z}
	+(-\alpha_5+\alpha_6+\alpha_7-\alpha_8)\frac{xyz}{(1-z)^2}.
\end{IEEEeqnarray*}
\begin{IEEEeqnarray*}{rCl}
	\alpha_5&=&\int\limits_{0}^1u_3(0,0,t)\,dt.
\end{IEEEeqnarray*}
Joinig everything
\begin{IEEEeqnarray*}{rCl}
	(\pi\bu)_3 & = & \int\limits_{0}^1u_3(0,0,t)\,dt
				+ x \int\limits_{0}^{1}\int\limits_{0}^{1-t}
						\frac{{\s\partial} u_3}{{\s\partial} x_1} (s,0,t) \,ds\,dt
				+ y \int\limits_{0}^{1}\int\limits_{0}^{1-t}
						\frac{{\s\partial} u_3}{{\s\partial} x_2}(0,s,t) \,ds\,dt\\
				&&\,+ \frac{xyz}{(1-z)^2}\iiint\limits_{\hat{P}}
				\frac{{\s\partial}^2u_3}{{\s\partial} x_2{\s\partial} x_1}\,dV.
				-\frac{xz}{1-z}\iiint\limits_{\hat{P}}
				\frac{{\s\partial}^2u_3}{{\s\partial} x_2{\s\partial} x_1}\,dV.
\end{IEEEeqnarray*}
All together, for an $\bu=(u_1,u_2,u_3)$.
\begin{IEEEeqnarray*}{rCl}
	(\pi\bu)_1 & = & \int\limits_{0}^{1}u_1(t,0,0)\,dt + 
	z \int\limits_0^1\int\limits_0^{1-t}
	\frac{{\s\partial} u_1}{{\s\partial} x_3}(t,0,s)\,dsdt +
	y \int\limits_0^1\int\limits_0^{1}
	\frac{{\s\partial} u_1}{{\s\partial} x_2}(t,s,0)\,dsdt\\
	&&\,+\frac{yz}{1-z} \int\limits_0^1\int\limits_0^{1-t}
	\frac{{\s\partial} u_1}{{\s\partial} x_2}(1-t,s,t)\,dsdt +
	\frac{yz}{1-z} \int\limits_0^1\int\limits_0^{1-t}
	\frac{{\s\partial} u_2}{{\s\partial} x_1}(s,1-t,t)\,dsdt\\
	&&\,-z\int\limits_0^1\int\limits_0^{1-t}
	\frac{{\s\partial} u_3}{{\s\partial} x_1}(s,0,t)\,dsdt +
	\frac{yz}{1-z} \iiint\limits_{\hat{P}}
	\frac{{\s\partial}^2u_3}{{\s\partial} x_2{\s\partial} x_1}\,dV.
\end{IEEEeqnarray*}
\begin{IEEEeqnarray*}{rCl}
	(\pi\bu)_2 & = & \int\limits_{0}^{1}u_2(0,t,0)\,dt + 
	z \int\limits_0^1\int\limits_0^{1-t}
	\frac{{\s\partial} u_2}{{\s\partial} x_3}(0,t,s)\,dsdt +
	x \int\limits_0^1\int\limits_0^{1}
	\frac{{\s\partial} u_2}{{\s\partial} x_1}(s,t,0)\,dsdt\\
	&&\,+\frac{xz}{1-z} \int\limits_0^1\int\limits_0^{1-t}
	\frac{{\s\partial} u_1}{{\s\partial} x_2}(1-t,s,t)\,dsdt +
	\frac{xz}{1-z} \int\limits_0^1\int\limits_0^{1-t}
	\frac{{\s\partial} u_2}{{\s\partial} x_1}(s,1-t,t)\,dsdt\\
	&&\,-z\int\limits_0^1\int\limits_0^{1-t}
	\frac{{\s\partial} u_3}{{\s\partial} x_2}(0,s,t)\,dsdt +
	\frac{xz}{1-z} \iiint\limits_{\hat{P}}
	\frac{{\s\partial}^2u_3}{{\s\partial} x_2{\s\partial} x_1}\,dV.
\end{IEEEeqnarray*}
\begin{IEEEeqnarray*}{rCl}
	(\pi\bu)_3 & = & \int\limits_{0}^{1}u_3(0,0,t)\,dt + 
	x \int\limits_0^1\int\limits_0^{1-t}
	\frac{{\s\partial} u_3}{{\s\partial} x_1}(s,0,t)\,dsdt -
	x \int\limits_0^1\int\limits_0^{1-t}
	\frac{{\s\partial} u_1}{{\s\partial} x_3}(t,0,s)\,dsdt\\
	&&\,+y \int\limits_0^1\int\limits_0^{1-t}
	\frac{{\s\partial} u_3}{{\s\partial} x_2}(0,s,t)\,dsdt -
	y \int\limits_0^1\int\limits_0^{1-t}
	\frac{{\s\partial} u_2}{{\s\partial} x_3}(0,t,s)\,dsdt\\
	&&\,+\frac{xy}{1-z} \int\limits_0^1\int\limits_t^{1}
	\frac{{\s\partial} u_1}{{\s\partial} x_2}(t,s,0)\,dsdt +
	\frac{xy}{1-z} \int\limits_0^1\int\limits_0^{t}
	\frac{{\s\partial} u_2}{{\s\partial} x_1}(t,s,1-t)\,dsdt\\
	&&\,+\frac{xyz}{(1-z)^2} \int\limits_0^1\int\limits_0^{1-t}
	\frac{{\s\partial} u_2}{{\s\partial} x_1}(s,1-t,t)\,dsdt
	+\frac{xyz}{(1-z)^2} \int\limits_0^1\int\limits_0^{1-t}
	\frac{{\s\partial} u_1}{{\s\partial} x_2}(1-t,s,t)\,dsdt\\
	&&\,-\frac{xy}{1-z}
	\int\limits_{0}^{1}
	\int\limits_{0}^{t}
	\int\limits_{0}^{1-t}
	\frac{{\s\partial}^2u_1}{{\s\partial}x_2{\s\partial}x_3}(t,s,r)\,dr\,ds\,dt
	-\frac{xy}{1-z}
	\int\limits_{0}^{1}
	\int\limits_{0}^{t}
	\int\limits_{0}^{t}
	\frac{{\s\partial}^2u_2}{{\s\partial}x_1{\s\partial}x_3}(r,s,1-t)\,dr\,ds\,dt\\
	&&\,
	+\frac{xyz}{(1-z)^2} \iiint\limits_{\hat{P}}
	\frac{{\s\partial}^2 u_3}{{\s\partial} x_1{\s\partial} x_2}\,dV
	-\frac{xz}{1-z} \iiint\limits_{\hat{P}}
	\frac{{\s\partial}^2 u_3}{{\s\partial} x_1{\s\partial} x_2}\,dV
\end{IEEEeqnarray*}
{\color{red}controlar contra el papel y contra los signos
según las normales exteriores}

	% &=&-\int\limits_0^1\int\limits_0^{1-t}\frac{{\s\partial} u_3}{{\s\partial} x_1}(s,0,t)\,dsdt.


%	(\pi\bu)_2 & = &\alpha_4 + (\alpha_2-\alpha_4)x -
%	(\alpha_4+\alpha_7)z + (\alpha_7-\alpha_8)\frac{xz}{1-z}.\\
%	\alpha_4 = \int\limits_{0}^{1}u_2(0,t,0)\,dt\\
%	\alpha_2-\alpha_4 & = & \int\limits_{0}^{1} u_2(1,t,0)-u_2(0,t,0)\,dt\\
%		&=&\int\limits_{0}^{1}\int\limits_{0}^{1}\frac{{\s\partial} u_2}{{\s\partial} x_1}(s,t,0)\,dsdt\\
%	\alpha_4+\alpha_7 & = & \int\limits_0^1 u_2(0,t,0)-u_2(0,t,1-t)\,dt\\
%		& = & \int\limits_0^1\int\limits_0^{1-t}\frac{{\s\partial} u_2}{{\s\partial} x_3}(0,t,s)\,dsdt.\\
%	\alpha_7-\alpha_8&=&\int\limits_{0}^{1} u_2(1-t,1-t,t)-u_2(0,1-t,t)\,dt\\
%		&=&\int\limits_{0}^{1}\int\limits_{0}^{1-t}\frac{{\s\partial} u_2}{{\s\partial} x_1}(s,1-t,t)\,dsdt.

% subsection edge_elements (end)
\subsection{Face Elements} % (fold)
\label{sub:face_elements}

\paragraph{Stability $\hat{E}$} 
\label{par:stability_hat}
\begin{theorem}
\noindent{\color{blue}\#\#\#\#\#\#\# cambiar H1 por W1p a la derecha.
corregir en las estimaciones de la demostración porque
está todo con H1} 
\begin{IEEEeqnarray*}{rCl}
  \|(\rku)_1\|_{\scriptscriptstyle{L^p(\hat{E})}}
  &\lesssim& \|\hat u_1\|_{\scriptscriptstyle{W^{1,p}(\hat{E})}} +
    \|\dv \bu\|_{\scriptscriptstyle{L^p}(\hat{E})} + 
    \left\|\hat{u}_3\right\|_{\scriptscriptstyle{H^1}(\hat{E})}\\[12pt]
  \|(\rku)_2\|_{\scriptscriptstyle{L^p(\hat{E})}}
  &\lesssim& \|\hat u_2\|_{\scriptscriptstyle{W^{1,p}(\hat{E})}} +
    \|\dv \bu\|_{\scriptscriptstyle{L^p}(\hat{E})} + 
    \left\|\hat{u}_3\right\|_{\scriptscriptstyle{H^1}(\hat{E})}\\[12pt]
  \|(\rku)_3\|_{\scriptscriptstyle{L^p(\hat{E})}} & \lesssim & 
    \|u_3\|_{\scriptscriptstyle{W^{1,p}(\hat{E})}} +
    \|\dv \bu\|_{\scriptscriptstyle{L^p}(\hat{E})}.
\end{IEEEeqnarray*}
\end{theorem}

\begin{proof}
We will use the notation of Table~\ref{shape_edge_table} for the 
shape functions and Tables~\ref{pyramidNotationTableFaces} and
~\ref{pyramidNotationTableEdges} for the border of the pyramid. The variables 
in the local coordinate system of $\hat E$ are $x,y$ and $z$.\\[5pt]
Consider the case $\hat{\bu} = (\hat{u}_1,0,0)'$ and compute it's interpolate. 
\begin{IEEEeqnarray*}{rCl}
  \rku & = & ({\scriptstyle\iint_{\hat{f}_2} \bu \cdot \hat\bn_2\,d\gamma})\,\bzeta_2 + 
         \iint_{\hat{f}_3} \bu \cdot \hat\bn_3\,d\gamma\,\bzeta_3\\[4pt]
       & = & \alpha_2(\hat\bu)\,\bzeta_2 + \alpha_3(\hat\bu)\,\bzeta_3.
\end{IEEEeqnarray*}
Then 
\begin{IEEEeqnarray*}{rCl}
  (\rku)_1\xyz & = & -2\alpha_2(\hat\bu) + 
    (\alpha_2(\hat\bu)+\alpha_3(\hat\bu))\,\frac{2x-xz}{1-z}\\[4pt]
    & = & -2{\iint_{\hat{f}_2} \hat{\bu} \cdot \hat\bn_2\,d\gamma} + 
          ({\iint_{\hat{f}_2} \hat{\bu} \cdot \hat\bn_2\,d\gamma}+
                  {\iint_{\hat{f}_3} \hat{\bu} \cdot \hat\bn_3\,d\gamma})\frac{2x-xz}{1-z}\\[4pt]
    & = & -2{\iint_{\hat{f}_2} \hat{\bu} \cdot \hat\bn_2\,d\gamma} + 
          {\iint_{\partial\hat{E}} \hat{\bu} \cdot \hat\bn\,d\gamma}\frac{2x-xz}{1-z}\\[4pt]
    & = & -2{\iint_{\hat{f}_2} \hat{\bu} \cdot \hat\bn_2\,d\gamma} + 
            {\int_{\hat{E}} \dv\hat{\bu} \,d\hat{\boldsymbol{x}}}\frac{2x-xz}{1-z}.\\[8pt]
  (\rku)_2\xyz & = & -(\alpha_2(\hat\bu)+\alpha_3(\hat\bu))\,\frac{yz}{1-z}\\[4pt]
    & = & -{\int_{\hat{E}} \dv\hat{\bu} \,d\hat{\boldsymbol{x}}}\frac{yz}{1-z}.
\end{IEEEeqnarray*}
Switch to $\hat{\bu} = (0,\hat{u}_2,0)'$.
\begin{IEEEeqnarray*}{rCl}
  \rku & = & ({\scriptstyle\iint_{\hat{f}_1} \bu \cdot \hat\bn_1\,d\gamma})\,\bzeta_1 + 
         \iint_{\hat{f}_4} \bu \cdot \hat\bn_4\,d\gamma\,\bzeta_4\\[4pt]
       & = & \alpha_1(\hat\bu)\,\bzeta_1 + \alpha_4(\hat\bu)\,\bzeta_4.
\end{IEEEeqnarray*}
Then
\begin{IEEEeqnarray*}{rCl}
  (\rku)_1\xyz & = & -(\alpha_1(\hat\bu)+\alpha_4(\hat\bu))\,\frac{xz}{1-z}\\[4pt]
    & = & -{\int_{\hat{E}} \dv\hat{\bu} \,d\hat{\boldsymbol{x}}}\frac{xz}{1-z}.\\[8pt]
  (\rku)_2\xyz & = & -2\alpha_1(\hat\bu) + 
  (\alpha_1(\hat\bu)+\alpha_4(\hat\bu))\,\frac{2y-yz}{1-z}\\[4pt]
    & = & -2{\iint_{\hat{f}_1} \hat{\bu} \cdot \hat\bn_1\,d\gamma} + 
            {\iint_{\partial\hat{E}} \hat{\bu} \cdot \hat\bn\,d\gamma}\frac{2y-yz}{1-z}\\[4pt]
    & = & -2{\iint_{\hat{f}_1} \hat{\bu} \cdot \hat\bn_1\,d\gamma} + 
            {\int_{\hat{E}} \dv\hat{\bu} \,d\hat{\boldsymbol{x}}}\frac{2y-yz}{1-z}.\\[8pt]
\end{IEEEeqnarray*}
Switch to $\hat{\bu} = (0,0,\hat{u}_3)'$.
\begin{IEEEeqnarray*}{rCl}
  \rku & = & ({\scriptstyle\iint_{\hat{f}_3} \bu \cdot \hat\bn_3\,d\gamma})\,\bzeta_3 + 
         \iint_{\hat{f}_4} \bu \cdot \hat\bn_4\,d\gamma\,\bzeta_4 + 
         \iint_{\hat{f}_5} \bu \cdot \hat\bn_5\,d\gamma\,\bzeta_5\\[4pt]
       & = & \alpha_3(\hat\bu)\,\bzeta_3 + \alpha_4(\hat\bu)\,\bzeta_4
       + \alpha_5(\hat\bu)\,\bzeta_5.
\end{IEEEeqnarray*}
Then
\begin{IEEEeqnarray*}{rCl}
  (\rku)_1\xyz & = & (\alpha_3(\hat\bu)+\alpha_5(\hat\bu))x
  + \alpha_3(\hat\bu) \frac{x}{1-z} - \alpha_4\frac{xz}{1-z}.
\end{IEEEeqnarray*}
Now observe
\begin{IEEEeqnarray*}{rCl}
  (\alpha_3(\hat\bu)+\alpha_5(\hat\bu)) & = & 
    {\iint_{\partial\hat{E}} \hat{\bu} \cdot \hat\bn\,d\gamma} - 
      {\iint_{\hat{f}_4} \hat{\bu} \cdot \hat\bn_4\,d\gamma} \\[4pt]
  & = & {\int_{\hat{E}} \dv\hat{\bu}\,d\hat{\boldsymbol{x}}} - 
        \alpha_4(\hat{\bu})
\end{IEEEeqnarray*}
and
\begin{IEEEeqnarray*}{rCl}
  \alpha_3(\hat\bu)-\alpha_4(\hat\bu) & = & 
  {\iint_{\hat{f}_3} \bu \cdot \hat\bn_3\,d\gamma} - 
  {\iint_{\hat{f}_4} \bu \cdot \hat\bn_4\,d\gamma} \\[4pt]
  & = & \int_{0}^{1}\int_{0}^{x} \hat{u}_3(x,y,1-x)\,dydx - 
        \int_{0}^{1}\int_{0}^{y} \hat{u}_3(x,y,1-y)\,dxdy\mbox{,}
\end{IEEEeqnarray*}
so
\begin{IEEEeqnarray*}{rCl}
  (\rku)_1\xyz & = & {\int_{\hat{E}} \dv\hat{\bu}\,d\hat{\boldsymbol{x}}}\,x +\\[4pt]
  \IEEEeqnarraymulticol{3}{r}{\qquad(\int_{0}^{1}\int_{0}^{x} \hat{u}_3(x,y,1-x)\,dydx - 
                          \int_{0}^{1}\int_{0}^{y} \hat{u}_3(x,y,1-y)\,dxdy)
                        \frac{x}{1-z}.}
\end{IEEEeqnarray*}
In a completely similar fashion
\begin{IEEEeqnarray*}{rCl}
  (\rku)_2\xyz & = & {\int_{\hat{E}} \dv\hat{\bu}\,d\hat{\boldsymbol{x}}}\,y\,+
  ({\iint_{\hat{f}_3} \bu \cdot \hat\bn_3\,d\gamma} - 
   {\iint_{\hat{f}_4} \bu \cdot \hat\bn_4\,d\gamma})\frac{y}{1-z}.\\[4pt]
               & = & {\int_{\hat{E}} \dv\hat{\bu}\,d\hat{\boldsymbol{x}}}\,y\,+\\[4pt]
  \IEEEeqnarraymulticol{3}{r}{\qquad(\int_{0}^{1}\int_{0}^{x} \hat{u}_3(x,y,1-x)\,dydx - 
                          \int_{0}^{1}\int_{0}^{y} \hat{u}_3(x,y,1-y)\,dxdy)
                        \frac{y}{1-z}.}
\end{IEEEeqnarray*}
We collect every term obtained so far for the first and second components in
Table~\ref{terms_table}.
\begin{table}[!h]
    \centering  
    \caption{Terms\\[4pt]$q(s,t) = \frac{2s-st}{1-t},\,r(s,t) = \frac{st}{1-t}$}
    \label{terms_table}
    \begin{IEEEeqnarraybox*}
    [\IEEEeqnarraystrutmode
    \IEEEeqnarraystrutsizeadd{2pt}{12pt}]{v/c/v/c/v/c/v/}
        \IEEEeqnarrayrulerow\\
        \IEEEeqnarrayseprow[5pt]\\
        & & & (\rku)_1 & & (\rku)_2 & \\
        \IEEEeqnarrayrulerow\\
        \IEEEeqnarrayseprow[5pt]\\
        & (\hat{u}_1,0,0)' & &
          \begin{IEEEeqnarraybox*}{l}
            -2{\iint_{\hat{f}_2} \hat{\bu} \cdot \hat\bn_2\,d\gamma}\\ + 
            {q(x,z)\int_{\hat{E}} \dv\hat{\bu} \,d\hat{\boldsymbol{x}}}
          \end{IEEEeqnarraybox*}
        & &
          -r(y,z){\int_{\hat{E}} \dv\hat{\bu} \,d\hat{\boldsymbol{x}}} &\\
        \IEEEeqnarrayrulerow\\
        \IEEEeqnarrayseprow[5pt]\\
        & (0,\hat{u}_2,0)' & & 
          -r(x,z){\int_{\hat{E}} \dv\hat{\bu} \,d\hat{\boldsymbol{x}}} 
        & & 
          \begin{IEEEeqnarraybox*}{l}
            -2{\iint_{\hat{f}_1} \hat{\bu} \cdot \hat\bn_1\,d\gamma}\\ + 
            {q(x,z)\int_{\hat{E}} \dv\hat{\bu} \,d\hat{\boldsymbol{x}}}
          \end{IEEEeqnarraybox*}
        &\\
        \IEEEeqnarrayrulerow\\
        \IEEEeqnarrayseprow[5pt]\\
        & (0,0,\hat{u}_3)' & & 
          \begin{IEEEeqnarraybox*}{l}
            x\int_{\hat{E}} \dv\hat{\bu}\,d\hat{\boldsymbol{x}} \\[5pt] +\, 
            ({\iint_{\hat{f}_3} \bu \cdot \hat\bn_3\,d\gamma}
             \\[5pt] 
             -{\iint_{\hat{f}_4} \bu \cdot \hat\bn_4\,d\gamma})r(x,z)
          \end{IEEEeqnarraybox*}
         & & 
          \begin{IEEEeqnarraybox*}{l}
            y\int_{\hat{E}} \dv\hat{\bu}\,d\hat{\boldsymbol{x}}\\[5pt] +\, 
              ({\iint_{\hat{f}_4} \bu \cdot \hat\bn_4\,d\gamma}
             \\[5pt] 
             -{\iint_{\hat{f}_3} \bu \cdot \hat\bn_3\,d\gamma})r(y,z)
          \end{IEEEeqnarraybox*}
        &\\\IEEEeqnarrayrulerow
    \end{IEEEeqnarraybox*}
\end{table}
Lastly, for any $\hat\bu$
\begin{IEEEeqnarray*}{rCl}
    (\rku)_3\xyz & = &   z\sum_{i=1}^4\iint_{\hat{f}_i} \hat{\bu}\cdot\hat{\bn}_i\,d\gamma
                      + (z-1) \iint_{\hat{f}_5}\hat{\bu}\cdot\hat{\bn}_5\,d\gamma\\[5pt]
                 & = & z\iint_{\partial\hat{E}} \hat{\bu}\cdot\hat{\bn} - \iint_{f_5}
                 \hat{\bu}\cdot\hat{\bn}_5\,d\gamma\\[5pt]
   \yesnumber\label{term_rk3}
                 & = &\hat{x}_3\int\limits_{\hat{E}} \mbox{div}\,\hat{\bu}\,d\hat{\bx} 
     + \iint\limits_{\hat{f}_5} \hat{u}_3\,d\hat{\gamma}
\end{IEEEeqnarray*}
Now we bound each term 
\begin{IEEEeqnarray*}{rCl}
  (\rku)_1 & = & (\br_{\hat{E}}(\hat{u}_1,0,0)')_1 + 
                 (\br_{\hat{E}}(0,\hat{u}_2,0)')_1 + 
                 (\br_{\hat{E}}(0,0,\hat{u}_3)')_1\\[5pt]
  & = & -2\iint_{\hat{f}_2}\hat{u}_1\,d\gamma +
  \frac{2x}{1-z}\int_{\hat{E}} \frac{\partial\hat{u}_1}{\partial\hat{x}_1}\,d\hat{\bx} -
  \frac{xz}{1-z}\int_{\hat{E}} \frac{\partial\hat{u}_1}{\partial\hat{x}_1}\,d\hat{\bx} \\[5pt]
  & & \,- \frac{xz}{1-z}\int_{\hat{E}} \frac{\partial\hat{u}_2}{\partial\hat{x}_2}\,d\hat{\bx}
  + \left( x + \frac{xz}{1-z}-\frac{xz}{1-z} \right)
  \int_{\hat{E}} \frac{\partial\hat{u}_3}{\partial\hat{x}_3}\,d\hat{\bx}\\[5pt]
  &   &\, + \left({\iint_{\hat{f}_3} \hat{u}_3\,d\gamma}
        - {\iint_{\hat{f}_4} \hat{u}_3\,d\gamma}\right)\frac{xz}{1-z}\\[5pt]
  & = & -2\iint_{\hat{f}_2}\hat{u}_1\,d\gamma +
  \frac{2x}{1-z}\int_{\hat{E}}\frac{\partial\hat{u}_1}{\partial\hat{x}_1}\,d\hat{\bx} -
  \frac{xz}{1-z}\int_{\hat{E}}\dv\hat{\bu}\,d\hat{\bx}\\[5pt]
  \yesnumber\label{face_integrals}
  &  & \,+\frac{x}{1-z}\int_{\hat{E}}\frac{\partial\hat{u}_3}{\partial\hat{x}_3}\,d\hat{\bx}
  + \left({\iint_{\hat{f}_3} \hat{u}_3\,d\gamma}
        - {\iint_{\hat{f}_4} \hat{u}_3\,d\gamma}\right)\frac{xz}{1-z}
\end{IEEEeqnarray*}
For the surface integrals in~(\ref{face_integrals}), by Theorem 3.9 in~\cite{monk}, page 43,
\begin{IEEEeqnarray*}{rCl}
  \left|\iint_{\hat{f}_2} \hat{u}_1\,d\gamma\right| 
  & \leqslant & 2^{-1/4}\,\|\mbox{Tr}_{\hat{f}_2}\hat{u}_1\|_{L^2(\hat{f}_3)} \\[5pt]
  & \leqslant & C\,\|\mbox{Tr\,}\hat{u}_1\|_{H^{\delta}(\partial\hat{E})} \\[5pt]
  & \leqslant & C\,\|\hat{u}_1\|_{H^{1/2+\delta}(\hat{E})}. \\[5pt]
  & \leqslant & C\,\|\hat{u}_1\|_{H^{1}(\hat{E})}.
\end{IEEEeqnarray*}
and similarly
\begin{IEEEeqnarray*}{rCl}
  \left|\iint_{\hat{f}_3} \hat{u}_3\,d\gamma - \iint_{\hat{f}_4} \hat{u}_3\,d\gamma\right| 
  & \leqslant & C\,\|\hat{u}_3\|_{H^{1}(\hat{E})}
\end{IEEEeqnarray*}
all of which leads to
\begin{IEEEeqnarray*}{rCl}
  \|(\rku)_1\|_{L^{\infty}(\hat{E})} & \leqslant & C_{\hat{E}} 
  \left[ 
    \|\hat{u}_1\|_{H^{1}(\hat{E})} + 
    \|\dv\hat{\bu}\|_{L^{2}(\hat{E})} + 
    \|\hat{u}_3\|_{H^{1}(\hat{E})}
  \right].
\end{IEEEeqnarray*}
Copying the argument for the second component
\begin{IEEEeqnarray*}{rCl}
  \|(\rku)_1\|_{L^{\infty}(\hat{E})} & \leqslant & C_{\hat{E}} 
  \left[ 
    \|\hat{u}_1\|_{H^{1}(\hat{E})} + 
    \|\dv\hat{\bu}\|_{L^{2}(\hat{E})} + 
    \|\hat{u}_3\|_{H^{1}(\hat{E})}
  \right].
\end{IEEEeqnarray*}
From~(\ref{term_rk3}) we deduce
\begin{IEEEeqnarray*}{rCl}
  \|(\rku)_3\|_{\scriptscriptstyle{L^\infty(\hat{E})}} & \leqslant & C_{\hat{E}}
    \left[\|u_3\|_{\scriptscriptstyle{H^{1}(\hat{E})}} +
    \|\dv \bu\|_{\scriptscriptstyle{L^2}(\hat{E})}\right].
\end{IEEEeqnarray*}
The quantity $C_{\hat{E}}$ depends only on the supremum of the (fixed)
basis shape functions of Table~\ref{shape_face_table} over the pyramid.
\end{proof}
% subsection face_elements (end)
\subsection{Local Interpolation Estimates for Pyramidal Finite Elements} % (fold)
\label{sub:local_interpolation_estimates_for_pyramidal_elements}
decir que permitimos pirámides elongadas perpendicularmente a la base
$h_3\geqslant C\min\{h_1,h_2\}$
Verificar si es esto o $h3 >= max (h1, h2)$\\
poner tres dibujos con casos $h1=h2<h3$; $h1<h2<h3$; $h2<h1<h3$

% subsection local_interpolation_estimates_for_pyramidal_elements (end)

% section pyramidal_finite_elements (end) 

\section{Local estimates for Pyramidal Virtual Elements}

\begin{theorem}
If $E$ is an isotropic tetrahedron or pyramid, then  
\begin{equation}\label{estab2}
\|\bu_I\|_{L^p(E)}\leqslant C\left(\|\bu\|_{L^p(E)}+ h_T|\bu|_{W^{1,p}(E)}\right), \qquad \forall \bu\in W^{1,p}(E),
\end{equation}
with the constant $C$ depending on the shape regularity of $E$, and $1\leqslant p$ if $E$ is a tetrahedron and $1\leqslant p\leqslant 2$ if $E$ is a pyramid.
\end{theorem}
\begin{proof}
{\color{blue}\#\#\#\#\#\#\#\# esto no ponerlo, al momento de usar tetra para
malla, citar del articulo\\.}
When $E$ is a tetrahedron, this result is contained in \cite{aadl}. So we assume that $E$ is a pyramid.\\ 
{\color{blue}\#\#\#\#\#\#\#\#\\}
We note that 
\begin{equation}\label{estab2:eq1}
\bu_I=\sum_{i=1}^5 \left(\int_{f_i}\bu\cdot \bn\right)\bv_i
\end{equation}
where $\{\bv_i\}_{i=1}^5$ is the basis of $V_h(E)$ dual to the degrees of freedom
\ref{dofs}. Denote by $f_j$, $j=1,\ldots,5$ the faces of $E$. First of all we need to estimate the $L^2$-norm of the basis functions $\bv_i$. Fixed $1\leqslant i\leqslant 5$, it follows from the proof of Lemma \ref{existenciaInterpolante} that $\bv_i=\nabla \psi$ where $\psi$ is the solution of
\begin{eqnarray*}
\Delta\psi&=&d\qquad\mbox{in }\Omega\\ \frac{\partial\psi}{\partial\bn}&=&g\qquad\mbox{on }\partial E\\ \int_E\psi&=&0
\end{eqnarray*}
with
\[
g|_{f_j}=\left\{\begin{array}{cl}\frac1{|f_i|}&\mbox{if }i=j\\0&\mbox{if }i\ne j\end{array}\right., \qquad d=\frac1{|E|}.
\]
Multiplying the first equation defining $\psi$ by $\psi$, and integrating by parts, we obtain
\begin{eqnarray*}
\|\nabla\psi\|_{L^2(E)}^2 &=& -\int_Ed\psi + \int_{\partial E}g\psi\\ &\leqslant & 
\|d\|_{L^2(E)}\|\psi\|_{L^2(E)} + \|g\|_{L^2(\partial E)}\|\psi\|_{L^2(\partial E)}.
\end{eqnarray*}
Using Poincare's and trace inequalities we have for, a constant $C$ depending on the aspect ratio of $E$ 
\begin{equation}\label{estab2:eq2}
\|\nabla\psi\|_{L^2(E)}^2\leqslant  C\left(h_E\|d\|_{L^2(E)} + h_E^\frac12 \|g\|_{L^2(\partial E)}\right) \|\nabla\psi\|_{L^2(E)}.
\end{equation}
Taking into account the definitions of $d$ and $g$ we have
\[
\|d\|_{L^2(E)}\leqslant Ch_E^{-\frac32}, \qquad \|g\|_{L^2(\partial E)}\leqslant Ch_E^{-1}
\]
and so from \eqref{estab2:eq2} we obtain
\begin{equation}\label{estab2:eq3}
\|\bv_i\|_{L^2(E)}=\|\nabla\psi\|_{L^2(E)}\leqslant Ch_E^{-\frac12}.
\end{equation}

Now, for $1\leqslant p\leqslant 2$ using H\"older's inequality and the expression \eqref{estab2:eq1} we have
\begin{eqnarray*}
\|\bu_I\|_{L^p(E)}&\le& |E|^{\frac1p-\frac12}\|\bu_I\|_{L^2(T)}\\&\le&|E|^{\frac1p-\frac12} \sum_{i=1}^5  \left|\int_{f_i}\bu\cdot\bn\right|\|\bv_i\|_{L^2(E)}.
\end{eqnarray*}
By using \eqref{estab2:eq3}, H\"older's inequality, trace inequalities and taking into account the shape-regularity of $E$ we obtain
\begin{eqnarray*}
\|\bu_I\|_{L^p(E)}&\le& C |E|^{\frac1p-\frac12} h_E^{-\frac1p}\left(\|\bu\|_{L^p(\Omega)}+h_E\|\nabla\bu\|_{L^p(E)}\right)|\partial E|^{1-\frac1p} \|\bv_i\|_{L^2(E)}\\&\le& C h_E^{3\left(\frac1p-\frac12\right)}h_E^{-\frac1p}h_E^{2\left(1-\frac1p\right)} h_E^{-\frac12}\left(\|\bu\|_{L^p(\Omega)}+h_E\|\nabla\bu\|_{L^p(E)}\right)\\&=& C \left(\|\bu\|_{L^p(\Omega)}+h_E\|\nabla\bu\|_{L^p(E)}\right)
\end{eqnarray*}


%\sum_{i=1}^5 \left|\int_{f_i}\bu\cdot\bn\right|\|\bv_i\|_{L^2(E)}\\&\le& \sum_{i=1}^5 |f_i|^\frac12\|\bu\|_{L^2(f_i)}\|\bv_i\|_{L^2(E)}\\ &\le& C|f_i|^\frac12\left(h_e^{-\frac12}\|\bu\|_{L^2(E)}+h_E^\frac12\|\nabla\bu\|_{L^2(E)}\right)\|\bv_i\|_{L^2(E)}\\ &\le& C\left(\|\bu\|_{L^2(E)}+h_E\|\nabla\bu\|_{L^2(E)}\right)

where $C$ depends on the shape regularity of $E$.
\end{proof}

\begin{proposition}\label{propErrorInterpolacionPiramidesTetraedros}
Let $E$ be a tetrahedron or a pyramid satisfying the shape-regularity property with constant $\sigma$. Then there exists a constant $C$ depending only on $\sigma$ such that 
\[
\|\bu-\bu_I\|_{L^2(E)}\leqslant C h_E|\bu|_{H^1(E)} \qquad \forall \bu\in H^1(E).
\]
\end{proposition}
\begin{proof} Let $Q\bu$ be the $L^2(E)$-projection of $u$ onto the constant fields. Then we have
\[
\bu-\bu_I = (\bu - Q\bu) + (Q\bu-\bu)_I
\]
and using the previous Lemma and a clasical estimate for the $L^2(E)$-projection error we have
\begin{eqnarray*}
\|\bu-\bu_I\|_{L^2(E)} &\le& \|\bu - Q\bu\|_{L^2(E)} + \|(\bu-Q\bu)_I\|_{L^2(E)}\\ &\le& \|\bu - Q\bu\|_{L^2(E)} + C \left(\|\bu-Q\bu\|_{L^2(E)} + h_E\|\nabla(\bu-Q\bu)\|_{L^2(E)}\right)\\ &=& C\left(\|\bu-Q\bu\|_{L^2(E)} + h_E\|\nabla\bu\|_{L^2(E)}\right)\\&\le& Ch_E\|\nabla\bu\|_{L^2(E)}
\end{eqnarray*}
as we wnated to prove.
\end{proof}


\begin{proposition}\label{propupi}
Let $E$ be a pyramid satisfying a shape regularity property with constant $\sigma$, and $\bu\in H^1(E)$. 
Then there exists a field $\bu_\pi\in W(E)$ such that
\[
\|\bu-\bu_\pi\|_{L^2(E)}\leqslant C h_E|\bu|_{H^1(E)}.
\]
(Poner directo) Then
\[
\|\bu-P^{\perp}\bu\|_{L^2(E)}\leqslant C h_E|\bu|_{H^1(E)}.
\]
\end{proposition}
\begin{proof}
We can define $\bu_\pi$ on $E$ as the $L^2(E)$-projection of $\bu$ on the space
of constant fields $P_0(E)^3\subset W(E)$. The error estimate follows from Bramble-Hilbert Lemma.  
\end{proof}
\begin{remark}
  we could replicate the results por prisms, at low order. We are not
  applying pyramidal FE interpolation, though.
\end{remark}
