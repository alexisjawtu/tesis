Following the proof of Theorem 6.2 in~\cite{aadl} and adding the information 
of~\eqref{aux_label23},
from Remark~\ref{auxlabel4} we derive the following Theorem.
\begin{theorem}\label{aux_label46}
Let $k\in\mathbb{N}_0$ and $p \geqslant 1$.
Let $E$ be an oblique prism (cfr. Figure~\ref{auxlabel417}).
There exists $C > 0$, which depends on the
greatest angle of the triangular faces and the angles on
the cuadrilateral faces of $E$, and three edges $\be_i$ of $E$ incident
to a common vertex
$\bx_E$ such that for all $\bu\in W^{m + 1,p}(E)^3$
and $m\leqslant k$,
\begin{IEEEeqnarray}{C}\nonumber
  \|\bu-\br_E \bu\|_{L^p(E)} \leqslant C \left\{
  \sum_{|{\balpha}|=m+1}\bh^{\balpha} \|\partial^{\balpha}\bu\|_{L^p(E)} +
  h_E\sum_{|\balpha| = m}
  	\bh^{\balpha}\|\partial^{\balpha}\text{div} \,\bu\|_{L^p(E)} \right\}.\\[4pt]
  \label{aux_label39}
\end{IEEEeqnarray}
\end{theorem}
\begin{figure}[!ht]
	\centering
	\begin{tikzpicture}[scale = 0.5]
		\draw[->] (0, 0, 0) -- (0, 0, 3);	
		\draw[->] (0, 0, 0) -- (3, 0, 0);
		\draw[->] (0, 0, 0) -- (0, 3, 0);
		\node (x1) at (2,0,0) [below] {$_{_1}$};
		\node (x2) at (0,2,0) [left] {$_{_1}$};
		\node (x3) at (0,0,2) [left] {$_{_1}$};
		\draw (0, 0, 2) -- (2, 0, 0);
		\draw (0, 2, 2) -- (2, 2, 0);
		\draw (2, 0, 0) -- (2, 2, 0);
		\draw (0, 0, 2) -- (0, 2, 2);
		\draw (0,2,2) -- (0,2,0) ;
		\draw (0,2,0) -- (2,2,0);

		\draw[xshift = 6cm, ->] (0, 0) -- (0, 0, 3);	
		\draw[xshift = 6cm, ->] (0, 0) -- (3,0);
		\draw[xshift = 6cm, ->] (0, 0) --  (0, 5, 0);
		\draw[xshift = 6cm] (0, 0, 1) -- (2, 0, 0); 
		\draw[xshift = 6cm] (0, 3.5, 1) -- (2, 3.5, 0); 
		\draw[xshift = 6cm] (2, 0, 0) -- node[label={right:$_{_{h_3}}$}] { } (2, 3.5, 0);
		\draw[xshift = 6cm] (0, 0, 1) -- (0, 3.5, 1); 
		\draw[xshift = 6cm] (0,3.5,1) -- node[label={left:$_{_{h_1}}$}] { }  (0,3.5,0) ;
		\draw[xshift = 6cm] (0,3.5,0) -- node[label={above:$_{_{h_2}}$}] { } (2,3.5,0);

		%va con transformacion afin
		\draw[xshift = 11.5cm] (0, 0, 0) -- (0, 0, 1)
			node (h1) [left, align=center, midway] {$_{_{h_1}}$} ;	
		\draw[xshift = 11.5cm] (0, 0, 0) -- (2, 0, 0);
		\draw[xshift = 11.5cm] (0, 0, 1) -- (2, 0, 0);
		\draw[xshift = 11.5cm] (0, 0, 0) -- ($1.5652*(2,1,0)$);
		\draw[xshift = 11.5cm] ($1.5652*(2,1,0) + (0,0,1)$)--($1.5652*(2,1,0)$);
		\draw[xshift = 11.5cm] ($1.5652*(2,1,0)$) -- ($1.5652*(2,1,0) + (2,0,0)$)
			node (h2) [above, align=center, midway] {$_{_{h_2}}$};
		\draw[xshift = 11.5cm] (0,0,1) -- ($1.5652*(2,1,0) + (0,0,1)$);
		\draw[xshift = 11.5cm] (2,0,0) -- ($1.5652*(2,1,0) + (2,0,0)$)
			node (h3) [below right, align=center, midway] {$_{_{h_3}}$};
		\draw[xshift = 11.5cm] ($1.5652*(2,1,0) + (0,0,1)$) -- 
		($1.5652*(2,1,0) + (2,0,0)$);		

		\draw[xshift = .5cm]   (0,-2.8,0) node (KRef) 		{$\hat E$};
		\draw[xshift = .5cm]   (1.5,-2.8,0) node (KRefFict) 	{};
		\draw[xshift = 5cm]   (0,-2.8,0) node (KRefFict2) 	{};

		\draw[xshift = 6.5cm] (0,-2.8,0) node (KTilde) 	{$\tilde E$};
		\draw[xshift = 8cm] (0,-2.8,0) node (KTildeFict) 	{};
		\draw[xshift = 11cm] (0,-2.8,0) node (KTildeFict2)	{};
		
		\draw[xshift = 12.5cm] (0,-2.8,0) node (KPhysical) 	{$E$};
		
		\draw[->] (KRefFict) -- (KRefFict2);
		\draw[->] (KTildeFict) -- (KTildeFict2);
	\end{tikzpicture}
	\caption{Affine transformations of the reference prism.}
	\label{auxlabel417}
\end{figure}
%\vspace{0.5cm}
%\begin{IEEEeqnarray*}{LLLLL}
%	\hat{K}\hspace{1cm} & \longrightarrow \hspace{1cm} & \tilde{K} \hspace{1cm} 
%	& \longrightarrow \hspace{1cm} & K \hspace{1cm}
%\end{IEEEeqnarray*}	