$\hat\bu$ is an element of $W^{1,p}(\hat(E))$ for a $p$ greater than $2$.
The proof is based on mere calculation. 
Let us recall the shape funtions in Table~\ref{shape_edge_table} and
start with $\bu$ of the form $(u_1,0,0)'$. After computations we have
\begin{IEEEeqnarray*}{rCl}
	\nabla\times\bu &=& (0, \frac{{\s\partial} u_1}{{\s\partial} x_3},-\frac{{\s\partial} u_1}{{\s\partial} x_2})\\[5pt]
	\wku	&=& [{\s\int_{\hat{\be}_1}\bu\cdot\btau\, ds}]\bgamma_1 +
				[{\s\int_{\hat{\be}_3}\bu\cdot\btau\, ds}]\bgamma_3 + 
				[{\s\int_{\hat{\be}_6}\bu\cdot\btau\, ds}]\bgamma_6 + 
				[{\s\int_{\hat{\be}_8}\bu\cdot\btau\, ds}]\bgamma_8\\[5pt]
			&=& \alpha_1(\hat\bu)\hat\bgamma_1 + 
				\alpha_3(\hat\bu)\hat\bgamma_3 + 
				\alpha_6(\hat\bu)\hat\bgamma_6 + 
				\alpha_8(\hat\bu)\hat\bgamma_8
\end{IEEEeqnarray*}
\begin{IEEEeqnarray*}{rCl}
  (\wku)_1(x,y,z) 
    &  = & \alpha_1(\hat\bu)(1-z-y)+ 
	  \alpha_3(\hat\bu)y+ 
	  \alpha_6(\hat\bu)(-z+\frac{yz}{1-z})+ 
	  \alpha_8(\hat\bu)(-\frac{yz}{1-z})\\
	& = & \alpha_1(\hat\bu) - (\alpha_1 + \alpha_6)(\hat\bu)\,z+ 
	  (\alpha_3 - \alpha_1)(\hat\bu)\,y + (\alpha_6-\alpha_8)(\hat\bu)\,\frac{yz}{1-z}.
\end{IEEEeqnarray*}
Now we explore the new coefficients separately. The tangential component of $\hat\bu$
along $\hat\be_5$ equals zero, and this is an argument we are using repeatedly in the forthcoming
computations.\\[4pt]
\noindent By Stokes' Theorem we have
\begin{IEEEeqnarray*}{rCl}
  (\alpha_1+\alpha_6)(\hat\bu)
  	& = & \int_{\hat{\be}_1}\hat\bu\cdot\hat\btau_1\, ds +
  											\int_{\hat{\be}_6}\hat\bu\cdot\hat\btau_6\, ds -
  											\int_{\hat{\be}_5}\hat\bu\cdot\hat\btau_5\, ds \\[5pt]
  	& = & \int_{\hat{f}_1} \nabla\times\hat\bu\cdot\hat\bn\,d\gamma \\[5pt]
  	& = & -\int_{\hat{f}_1} \frac{{\s\partial} u_1}{{\s\partial} x_3}\,d\gamma.
\end{IEEEeqnarray*}
Next,
\begin{IEEEeqnarray*}{rCl}
	(\alpha_3-\alpha_1)(\hat\bu) & = & (\alpha_3-\alpha_2-\alpha_1+\alpha_4)(\hat\bu)\\
	& = & - \int_{\partial\hat{f}_5}\hat{\bu}\cdot\hat\btau\,d\hat{s}\\
	&=& -\iint_{\hat{f}_5}\nabla\times\hat{\bu}\cdot\hat{\bn}_5\,d\gamma\\
	&=&	 \iint_{\hat{f}_5}\frac{{\s\partial} \hat{u}_1}{{\s\partial} \hat{x}_2}\,d\gamma.
\end{IEEEeqnarray*}
And
\begin{IEEEeqnarray*}{rCl}
  (\alpha_6-\alpha_8)(\hat\bu) & = & \left(\int_{\hat\be_6}\hat\bu\cdot\hat\btau_6\,ds-
    \int_{\hat\be_8}\hat\bu\cdot\hat\btau_6\,ds-
	\int_{\hat\be_2}\hat\bu\cdot\hat\btau_6\,ds\right)\\[5pt]
	& = &-\int_{\partial\hat{f}_3}\hat\bu\cdot\hat\btau\,ds  
	  =  -\int_{\hat{f}_3}\nabla\times\hat\bu\cdot\bn\,d\gamma
	  =   \int_{\hat{f}_3}\frac{{\s\partial} \hat u_1}{{\s\partial} x_2}\,d\gamma.
\end{IEEEeqnarray*}
By exactly the last computation,
\begin{IEEEeqnarray*}{rCl}
  (\wku)_2 & = & (\alpha_6-\alpha_8)(\hat\bu)\frac{xz}{1-z}
  = \int_{\hat{f}_3}\frac{{\s\partial}\hat u_1}{{\s\partial} x_2}\,d\gamma\,\frac{xz}{1-z}.
\end{IEEEeqnarray*}
Next,
\begin{IEEEeqnarray*}{rCl}
	(\wku)_3 & = &     \alpha_1(\hat\bu)\left(x-\frac{xy}{1-z}\right) + \alpha_3(\hat\bu)\frac{xy}{1-z}\\[6pt]
			 &   &\,+\,\alpha_6(\hat\bu)\left(x-\frac{xy}{1-z}+\frac{xyz}{(1-z)^2}\right)
		            +  \alpha_8(\hat\bu)\left(\frac{xy}{1-z}-\frac{xyz}{(1-z)^2}\right)\\[6pt]
			 & = &  (\alpha_1 + \alpha_6)(\hat\bu)\,x +
			 		(\alpha_3-\alpha_1+\alpha_8-\alpha_6)(\hat\bu)\,\frac{xy}{1-z}\\[6pt]
			 &   &\,+ (\alpha_6-\alpha_8)(\hat\bu)\,\frac{xyz}{(1-z)^2}.
\end{IEEEeqnarray*}
As $\hat\bu$ has zero tangential component along $\be_5$ and $\be_7$,
\begin{IEEEeqnarray*}{rCl}
  (\alpha_3-\alpha_1+\alpha_8-\alpha_6)(\hat\bu)&=&
  (\alpha_3-\alpha_7+\alpha_8)(\hat\bu)+(\alpha_5-\alpha_6-\alpha_1)(\hat\bu)\\[8pt]
  &=&-\int_{\partial\hat{f}_4}\hat\bu\cdot\hat\btau\,ds
   -\int_{\partial\hat{f}_1}\hat\bu\cdot\hat\btau\,ds\\[8pt]
  &=&-\iint_{\hat{f}_1}\nabla\times\hat\bu\cdot\hat\bn\,d\gamma
   -\iint_{\hat{f}_4}\nabla\times\hat\bu\cdot\hat\bn\,d\gamma\\[8pt]
\yesnumber\label{pyr_edge_one}
  &=&\iint_{\hat{f}_1}(\nabla\times\hat\bu)_2\,d\gamma\\[8pt]
  & &\quad-2^{-\nicefrac{1}{2}}\iint_{\hat{f}_4}[(\nabla\times\hat\bu)_2 + (\nabla\times\hat\bu)_3]\,d\gamma.\\[8pt]
  &=&\iint_{\hat{f}_1}\frac{\partial\hat{u}_1}{\partial\hat{x}_3}\,d\hat\gamma
  -2^{-\nicefrac{1}{2}}\iint_{\hat{f}_4}[\frac{\partial\hat{u}_1}{\partial\hat{x}_3}
   + \frac{\partial\hat{u}_1}{\partial\hat{x}_2}]\,d\hat\gamma.
\end{IEEEeqnarray*}
\noindent Now, as expected, we switch to $\bu = (0,u_2,0)'$. In this case we have
\begin{IEEEeqnarray*}{rCl}
  \wku     & = & \alpha_2(\hat{\bu})\,\bgamma_2 +
	\alpha_4(\hat{\bu})\,\bgamma_4+ \alpha_7(\hat{\bu})\,\bgamma_7+\alpha_8(\hat{\bu})\,\bgamma_8.\\
  (\wku)_1 & = &(\alpha_7-\alpha_8)(\hat{\bu})\,\frac{yz}{1-z}\\
  		   & = &(\alpha_7-\alpha_8 - \alpha_3)(\hat{\bu})\,\frac{yz}{1-z}\\
  		   & = &\int_{\partial\hat{f}_4}\hat{\bu}\cdot\btau\,d\hat{s}\,\frac{yz}{1-z}\\
  		   & = &\iint_{\hat{f}_4} \nabla\times\hat\bu\cdot\,d\gamma\,\frac{yz}{1-z}
  		     =  \iint_{\hat{f}_4} \frac{\partial\hat{u}_2}{\partial\hat{x}_1}\,d\gamma\,\frac{yz}{1-z}.
\end{IEEEeqnarray*}
For the next component,
\begin{IEEEeqnarray*}{rCl}
	(\wku)_2 & = &\alpha_4(\hat\bu) + (\alpha_2-\alpha_4)(\hat\bu)\,x -
	(\alpha_4+\alpha_7)(\hat\bu)\,z + (\alpha_7-\alpha_8)(\hat\bu)\,\frac{xz}{1-z}.\\
	(\alpha_2-\alpha_4)(\hat\bu) & = & (\alpha_2-\alpha_3-\alpha_4+\alpha_1)(\hat\bu)\\
  &=&-\int_{\partial\hat{f}_5}\hat\bu\cdot\hat\btau\,d\hat{s}\\
  &=&-\iint_{\hat{f}_5}\nabla\times\hat{\bu}\cdot\hat\bn_5\,d\gamma
   =  \iint_{\hat{f}_5}\frac{\partial\hat{u}_2}{\partial\hat{x}_1}\,d\gamma.
\end{IEEEeqnarray*}
\begin{IEEEeqnarray*}{rCl}
  (\alpha_4+\alpha_7)(\hat\bu) & = & 
  (\alpha_4+\alpha_7-\alpha_5)(\hat\bu)\\
  &=& - \int_{\partial\hat{f}_2} \hat\bu\cdot\hat\btau\,d\hat{s}\\
  &=& -\iint_{\hat{f}_2}\nabla\times\hat\bu\cdot\hat\bn\,d\gamma~=~
      -\iint_{\hat{f}_2}\frac{{\s\partial} u_2}{{\s\partial} x_3}.
\end{IEEEeqnarray*}
And for the third one, {\color{red}\#\#\#\#\#\#\#\# aca pruebo una manera de acomodar. ver otras después de
armar la tabla para acotar.}
\begin{IEEEeqnarray*}{rCl}
	(\wku)_3&=&(\alpha_4+\alpha_7)(\hat\bu)\,y + (\alpha_2-\alpha_4-\alpha_7+\alpha_8)(\hat\bu)\,\frac{xy}{1-z}\\[4pt]
	& &\,+\,(\alpha_7-\alpha_8)(\hat\bu)\,\frac{xyz}{(1-z)^2}.\\[8pt]
	&=&(\alpha_2 - \alpha_4)(\hat\bu)\,\frac{xy}{1-z} + (\alpha_4+\alpha_7)(\hat\bu)\,y\\[8pt]
	& &-(\alpha_7-\alpha_8)(\hat\bu)\,\frac{xy}{(1-z)^2}
\end{IEEEeqnarray*}
but expressions for $(\alpha_2 - \alpha_4)(\cdot)$, $(\alpha_4+\alpha_7)(\cdot)$ and
$(\alpha_7-\alpha_8)(\cdot)$ were already stated.\\[4pt] 
{\color{green}Finally for $\hat\bu = (0,0,\hat u_3)'$} it is
$\wku = \alpha_5(\hat\bu)\bgamma_5 + 
		\alpha_6(\hat\bu)\bgamma_6 + 
		\alpha_7(\hat\bu)\bgamma_7 +
		\alpha_8(\hat\bu)\bgamma_8.$\\[4pt]
First component:
\begin{IEEEeqnarray*}{rCl}
	(\wku)_1 & = &(\alpha_5-\alpha_6)(\hat\bu)z+
		(-\alpha_5+\alpha_6+\alpha_7-\alpha_8)(\hat\bu)\,\frac{yz}{1-z}.\ok\\
	(\alpha_5-\alpha_6)(\hat\bu) & = & (\alpha_5-\alpha_6-\alpha_1)(\hat\bu) \ok\\
	&=&-\iint\limits_{\hat f_1}\nabla\times\hat\bu\cdot\hat\bn\,dS\ok
\end{IEEEeqnarray*}
On the other hand and analogously
\begin{IEEEeqnarray*}{rCl} 	
	\alpha_7-\alpha_8 & = &	\iint\limits_{\hat{f}_4}\nabla\times\hat\bu\cdot\hat\bn\,dS\ok
\end{IEEEeqnarray*}
% &=&-\iint\limits_{\hat f_1}\partial_1 \hat{u}_3\,dS.
so \noindent{\color{blue} cambia lo de region tipo I y eso? estoy haciendo
$0\leqslant t_2 \leqslant 1; 0\leqslant t_1\leqslant 1-t_2$} 
\begin{IEEEeqnarray*}{rCl}
  (-\alpha_5+\alpha_6+\alpha_7-\alpha_8)(\hat\bu) & = & 
  \iint\limits_{\hat{f}_1}\nabla\times\bu\cdot\bn\,dS
  +\iint\limits_{\hat{f}_4}\nabla\times\bu\cdot\bn\,dS\ok\\
&=&
  \int\limits_{\mathbb{D}_{\hat f_1}}\frac{{\s\partial} u_3}{{\s\partial} x_1}
  (\Phi_{\hat f_1}(t_1,t_2))\,dt_1dt_2
  -\int\limits_{\mathbb{D}_{\hat f_4}}\frac{{\s\partial} u_3}{{\s\partial} x_1}
  (\Phi_{\hat f_4}(t_1,t_2))\,dt_1dt_2\ok\\
&=&
  \int_0^1\int_0^{1-t_2} 
  \left[\frac{{\s\partial} u_3}{{\s\partial} x_1}(t_1,0,t_2)
    - \frac{{\s\partial} u_3}{{\s\partial} x_1}(t_1,1-t_2,t_2)\right]
  \,dt_1dt_2\ok\\
&=&
  -\int_0^1\int_0^{1-t_2}\int_0^{1-t_2} 
  \frac{{\s\partial^2} u_3}{{\s\partial} x_2{\s\partial} x_1}(t_1,s,t_2)
  \,dsdt_1dt_2\ok\\[6pt]
&=&-\int\limits_{\hat{E}}\frac{{\s\partial}^2u_3}{{\s\partial} x_2{\s\partial} x_1}
\,d\hat\bx\ok.
\end{IEEEeqnarray*}
For now
\begin{IEEEeqnarray*}{rCl}
	(\wku)_1 & = & -z\iint\limits_{\hat f_1}\frac{{\s\partial} u_3}{{\s\partial} x_1}\,dS
	-\frac{yz}{1-z}\int\limits_{\hat{E}}
		\frac{{\s\partial}^2u_3}{{\s\partial} x_2{\s\partial} x_1}\,d\hat\bx.\ok
\end{IEEEeqnarray*}
Regarding the second component, it is the simmetric case.
\begin{IEEEeqnarray*}{rCl}
  (\wku)_2& = &(\alpha_5-\alpha_7)(\hat\bu)z
		+(-\alpha_5+\alpha_6+\alpha_7-\alpha_8)(\hat\bu)\frac{xz}{1-z}\\
	& = &z\iint\limits_{\hat f_2}\nabla\times\bu\cdot\bn\,dS
	-\frac{xz}{1-z}\int\limits_{\hat{E}}
	\frac{{\s\partial}^2u_3}{{\s\partial} x_2{\s\partial} x_1}\,d\hat\bx\\
	& = &-z\iint\limits_{\hat f_2}\frac{{\s\partial} u_3}{{\s\partial} x_2}\,dS
		-\frac{xz}{1-z}\int\limits_{\hat{E}}
	\frac{{\s\partial}^2u_3}{{\s\partial} x_2{\s\partial} x_1}\,d\hat\bx.\ok
\end{IEEEeqnarray*}
{\color{blue}\#\#\#\#\#\#\#\# continue here.}
\begin{IEEEeqnarray*}{rCl}
	(\pi\bu)_3&=& \alpha_5 + x(-\alpha_5+\alpha_6)+y(-\alpha_5+\alpha_7)\\[5pt]
	&&\,+(\alpha_5-\alpha_6-\alpha_7+\alpha_8)\frac{xy}{1-z}
	+(-\alpha_5+\alpha_6+\alpha_7-\alpha_8)\frac{xyz}{(1-z)^2}.
\end{IEEEeqnarray*}
\begin{IEEEeqnarray*}{rCl}
	\alpha_5&=&\int\limits_{0}^1u_3(0,0,t)\,dt.
\end{IEEEeqnarray*}
Joinig everything
\begin{IEEEeqnarray*}{rCl}
	(\pi\bu)_3 & = & \int\limits_{0}^1u_3(0,0,t)\,dt
				+ x \int\limits_{0}^{1}\int\limits_{0}^{1-t}
						\frac{{\s\partial} u_3}{{\s\partial} x_1} (s,0,t) \,ds\,dt
				+ y \int\limits_{0}^{1}\int\limits_{0}^{1-t}
						\frac{{\s\partial} u_3}{{\s\partial} x_2}(0,s,t) \,ds\,dt\\
				&&\,+ \frac{xyz}{(1-z)^2}\iiint\limits_{\hat{P}}
				\frac{{\s\partial}^2u_3}{{\s\partial} x_2{\s\partial} x_1}\,dV.
				-\frac{xz}{1-z}\iiint\limits_{\hat{P}}
				\frac{{\s\partial}^2u_3}{{\s\partial} x_2{\s\partial} x_1}\,dV.
\end{IEEEeqnarray*}
All together, for an $\bu=(u_1,u_2,u_3)$.
\begin{IEEEeqnarray*}{rCl}
	(\pi\bu)_1 & = & \int\limits_{0}^{1}u_1(t,0,0)\,dt + 
	z \int\limits_0^1\int\limits_0^{1-t}
	\frac{{\s\partial} u_1}{{\s\partial} x_3}(t,0,s)\,dsdt +
	y \int\limits_0^1\int\limits_0^{1}
	\frac{{\s\partial} u_1}{{\s\partial} x_2}(t,s,0)\,dsdt\\
	&&\,+\frac{yz}{1-z} \int\limits_0^1\int\limits_0^{1-t}
	\frac{{\s\partial} u_1}{{\s\partial} x_2}(1-t,s,t)\,dsdt +
	\frac{yz}{1-z} \int\limits_0^1\int\limits_0^{1-t}
	\frac{{\s\partial} u_2}{{\s\partial} x_1}(s,1-t,t)\,dsdt\\
	&&\,-z\int\limits_0^1\int\limits_0^{1-t}
	\frac{{\s\partial} u_3}{{\s\partial} x_1}(s,0,t)\,dsdt +
	\frac{yz}{1-z} \iiint\limits_{\hat{P}}
	\frac{{\s\partial}^2u_3}{{\s\partial} x_2{\s\partial} x_1}\,dV.
\end{IEEEeqnarray*}
\begin{IEEEeqnarray*}{rCl}
	(\pi\bu)_2 & = & \int\limits_{0}^{1}u_2(0,t,0)\,dt + 
	z \int\limits_0^1\int\limits_0^{1-t}
	\frac{{\s\partial} u_2}{{\s\partial} x_3}(0,t,s)\,dsdt +
	x \int\limits_0^1\int\limits_0^{1}
	\frac{{\s\partial} u_2}{{\s\partial} x_1}(s,t,0)\,dsdt\\
	&&\,+\frac{xz}{1-z} \int\limits_0^1\int\limits_0^{1-t}
	\frac{{\s\partial} u_1}{{\s\partial} x_2}(1-t,s,t)\,dsdt +
	\frac{xz}{1-z} \int\limits_0^1\int\limits_0^{1-t}
	\frac{{\s\partial} u_2}{{\s\partial} x_1}(s,1-t,t)\,dsdt\\
	&&\,-z\int\limits_0^1\int\limits_0^{1-t}
	\frac{{\s\partial} u_3}{{\s\partial} x_2}(0,s,t)\,dsdt +
	\frac{xz}{1-z} \iiint\limits_{\hat{P}}
	\frac{{\s\partial}^2u_3}{{\s\partial} x_2{\s\partial} x_1}\,dV.
\end{IEEEeqnarray*}
\begin{IEEEeqnarray*}{rCl}
	(\pi\bu)_3 & = & \int\limits_{0}^{1}u_3(0,0,t)\,dt + 
	x \int\limits_0^1\int\limits_0^{1-t}
	\frac{{\s\partial} u_3}{{\s\partial} x_1}(s,0,t)\,dsdt -
	x \int\limits_0^1\int\limits_0^{1-t}
	\frac{{\s\partial} u_1}{{\s\partial} x_3}(t,0,s)\,dsdt\\
	&&\,+y \int\limits_0^1\int\limits_0^{1-t}
	\frac{{\s\partial} u_3}{{\s\partial} x_2}(0,s,t)\,dsdt -
	y \int\limits_0^1\int\limits_0^{1-t}
	\frac{{\s\partial} u_2}{{\s\partial} x_3}(0,t,s)\,dsdt\\
	&&\,+\frac{xy}{1-z} \int\limits_0^1\int\limits_t^{1}
	\frac{{\s\partial} u_1}{{\s\partial} x_2}(t,s,0)\,dsdt +
	\frac{xy}{1-z} \int\limits_0^1\int\limits_0^{t}
	\frac{{\s\partial} u_2}{{\s\partial} x_1}(t,s,1-t)\,dsdt\\
	&&\,+\frac{xyz}{(1-z)^2} \int\limits_0^1\int\limits_0^{1-t}
	\frac{{\s\partial} u_2}{{\s\partial} x_1}(s,1-t,t)\,dsdt
	+\frac{xyz}{(1-z)^2} \int\limits_0^1\int\limits_0^{1-t}
	\frac{{\s\partial} u_1}{{\s\partial} x_2}(1-t,s,t)\,dsdt\\
	&&\,-\frac{xy}{1-z}
	\int\limits_{0}^{1}
	\int\limits_{0}^{t}
	\int\limits_{0}^{1-t}
	\frac{{\s\partial}^2u_1}{{\s\partial}x_2{\s\partial}x_3}(t,s,r)\,dr\,ds\,dt
	-\frac{xy}{1-z}
	\int\limits_{0}^{1}
	\int\limits_{0}^{t}
	\int\limits_{0}^{t}
	\frac{{\s\partial}^2u_2}{{\s\partial}x_1{\s\partial}x_3}(r,s,1-t)\,dr\,ds\,dt\\
	&&\,
	+\frac{xyz}{(1-z)^2} \iiint\limits_{\hat{P}}
	\frac{{\s\partial}^2 u_3}{{\s\partial} x_1{\s\partial} x_2}\,dV
	-\frac{xz}{1-z} \iiint\limits_{\hat{P}}
	\frac{{\s\partial}^2 u_3}{{\s\partial} x_1{\s\partial} x_2}\,dV
\end{IEEEeqnarray*}
{\color{red}controlar contra el papel y contra los signos
según las normales exteriores}

	% &=&-\int\limits_0^1\int\limits_0^{1-t}\frac{{\s\partial} u_3}{{\s\partial} x_1}(s,0,t)\,dsdt.


%	(\pi\bu)_2 & = &\alpha_4 + (\alpha_2-\alpha_4)x -
%	(\alpha_4+\alpha_7)z + (\alpha_7-\alpha_8)\frac{xz}{1-z}.\\
%	\alpha_4 = \int\limits_{0}^{1}u_2(0,t,0)\,dt\\
%	\alpha_2-\alpha_4 & = & \int\limits_{0}^{1} u_2(1,t,0)-u_2(0,t,0)\,dt\\
%		&=&\int\limits_{0}^{1}\int\limits_{0}^{1}\frac{{\s\partial} u_2}{{\s\partial} x_1}(s,t,0)\,dsdt\\
%	\alpha_4+\alpha_7 & = & \int\limits_0^1 u_2(0,t,0)-u_2(0,t,1-t)\,dt\\
%		& = & \int\limits_0^1\int\limits_0^{1-t}\frac{{\s\partial} u_2}{{\s\partial} x_3}(0,t,s)\,dsdt.\\
%	\alpha_7-\alpha_8&=&\int\limits_{0}^{1} u_2(1-t,1-t,t)-u_2(0,1-t,t)\,dt\\
%		&=&\int\limits_{0}^{1}\int\limits_{0}^{1-t}\frac{{\s\partial} u_2}{{\s\partial} x_1}(s,1-t,t)\,dsdt.
