\begin{theorem} \label{aux_label53}
  {\color{Orange}\#\#\#\#\#\#\#\# Ariel:
    ir a la parte~(\ref{aux_label42}) para establecer un enunciado}
\end{theorem}
\begin{proof}
$\hat\bu$ is an element of $W^{1,p}(\hat(E))$ for a $p$ greater than $2$.
Let us recall the shape funtions in Table~\ref{shape_edge_table} and
start with $\bu$ of the form $(u_1,0,0)'$. After computations we have
\begin{IEEEeqnarray*}{rCl}
	\nabla\times\bu &=& (0, \frac{{\s\partial} u_1}{{\s\partial} x_3},-\frac{{\s\partial} u_1}{{\s\partial} x_2})\\[5pt]
	\wku	&=& [{\s\int_{\hat{\be}_1}\bu\cdot\btau\, ds}]\bgamma_1 +
				[{\s\int_{\hat{\be}_3}\bu\cdot\btau\, ds}]\bgamma_3 + 
				[{\s\int_{\hat{\be}_6}\bu\cdot\btau\, ds}]\bgamma_6 + 
				[{\s\int_{\hat{\be}_8}\bu\cdot\btau\, ds}]\bgamma_8\\[5pt]
			&=& \alpha_1(\hat\bu)\hat\bgamma_1 + 
				\alpha_3(\hat\bu)\hat\bgamma_3 + 
				\alpha_6(\hat\bu)\hat\bgamma_6 + 
				\alpha_8(\hat\bu)\hat\bgamma_8
\end{IEEEeqnarray*}
\begin{IEEEeqnarray*}{rCl}
  (\wku)_1(x,y,z) 
    &  = & \alpha_1(\hat\bu)(1-z-y)+ 
	  \alpha_3(\hat\bu)y+ 
	  \alpha_6(\hat\bu)(-z+\frac{yz}{1-z})+ 
	  \alpha_8(\hat\bu)(-\frac{yz}{1-z})\\
	& = & \alpha_1(\hat\bu) - (\alpha_1 + \alpha_6)(\hat\bu)\,z+ 
	  (\alpha_3 - \alpha_1)(\hat\bu)\,y + (\alpha_6-\alpha_8)(\hat\bu)\,\frac{yz}{1-z}.
\end{IEEEeqnarray*}
Now we explore the new coefficients separately. As the tangential component of $\hat\bu$
along $\hat\be_5$ equals zero, and this is an argument we are using repeatedly in the forthcoming
computations, we may write, by Stokes' Theorem,
\begin{IEEEeqnarray*}{rCl}
  (\alpha_1+\alpha_6)(\hat\bu)
  	& = & \int_{\hat{\be}_1}\hat\bu\cdot d\hat{\balpha}_1 \\
    &&\mbox{\color{brown}cambiar a una notacion como la de aca arriba. simil apostol o marsden.}\\
    &&\mbox{\color{brown}ver, por ejemplo, apostol pag 319 expresion 37)}\\
    &&      +	\int_{\hat{\be}_6}\hat\bu\cdot\hat\btau_6\, ds -
  			\int_{\hat{\be}_5}\hat\bu\cdot\hat\btau_5\, ds \\[5pt]
  	& = & \iint_{\hat{f}_1} \nabla\times\hat\bu\cdot\hat\bn\,dS \\[5pt]
  	& = & -\iint_{\hat{f}_1} \frac{{\s\partial} u_1}{{\s\partial} x_3}\,dS.
\end{IEEEeqnarray*}
Next,
\begin{IEEEeqnarray*}{rCl}
	(\alpha_3-\alpha_1)(\hat\bu) & = & (\alpha_3-\alpha_2-\alpha_1+\alpha_4)(\hat\bu)\\
	& = & - \int_{\partial\hat{f}_5}\hat{\bu}\cdot\hat\btau\,d\hat{s}\\
	&=& -\iint_{\hat{f}_5}\nabla\times\hat{\bu}\cdot\hat{\bn}_5\,d\gamma\\
	&=&	 \iint_{\hat{f}_5}\frac{{\s\partial} \hat{u}_1}{{\s\partial} \hat{x}_2}\,d\gamma.
\end{IEEEeqnarray*}
And
\begin{IEEEeqnarray*}{rCl}
  (\alpha_6-\alpha_8)(\hat\bu) & = & \int_{\hat\be_6}\hat\bu\cdot\hat\btau_6\,ds-
    \int_{\hat\be_8}\hat\bu\cdot\hat\btau_6\,ds-
	\int_{\hat\be_2}\hat\bu\cdot\hat\btau_6\,ds\\[5pt]
	& = &-\int_{\partial\hat{f}_3}\hat\bu\cdot\hat\btau\,ds  
	  =  -\int_{\hat{f}_3}\nabla\times\hat\bu\cdot\hat\bn_3\,dS
	  =   2^{-\nicefrac12}\iint_{\hat{f}_3}\frac{{\s\partial} \hat u_1}{{\s\partial} x_2}\,dS.
\end{IEEEeqnarray*}
So in this case in which $\hat\bu$ has null first and second components
\begin{IEEEeqnarray}{rCl}\label{first_a}
	\nonumber
  (\wku)_1 & = & \int_{\hat\be_1}\hat u_1\,d\alpha + 
                z\iint_{\hat f_1} \frac{{\s\partial}\hat u_1}{{\s\partial} x_3}\,dS +
                y\iint_{\hat f_5} \frac{{\s\partial}\hat u_1}{{\s\partial} x_2}\,dS\\
           &&\,+\frac{yz}{1-z}\,2^{-\nicefrac12}\iint_{\hat f_3} \frac{{\s\partial}\hat u_1}{{\s\partial} x_2}\,dS.
\end{IEEEeqnarray}
By exactly the last computation,
\begin{IEEEeqnarray}{rCl}\label{second_a}
  (\wku)_2 & = & (\alpha_6-\alpha_8)(\hat\bu)\frac{xz}{1-z}
  = \iint_{\hat{f}_3}\frac{{\s\partial}\hat u_1}{{\s\partial} x_2}\,dS\,\frac{xz}{1-z}.
\end{IEEEeqnarray}
Next,
\begin{IEEEeqnarray*}{rCl}
	(\wku)_3 & = &     \alpha_1(\hat\bu)\left(x-\frac{xy}{1-z}\right) + \alpha_3(\hat\bu)\frac{xy}{1-z}\\[6pt]
			 &   &\,+\,\alpha_6(\hat\bu)\left(x-\frac{xy}{1-z}+\frac{xyz}{(1-z)^2}\right)
		            +  \alpha_8(\hat\bu)\left(\frac{xy}{1-z}-\frac{xyz}{(1-z)^2}\right)\\[6pt]
			 & = &  (\alpha_1 + \alpha_6)(\hat\bu)\,x +
			 		(\alpha_3-\alpha_1+\alpha_8-\alpha_6)(\hat\bu)\,\frac{xy}{1-z}\\[6pt]
			 &   &\,+ (\alpha_6-\alpha_8)(\hat\bu)\,\frac{xyz}{(1-z)^2}.
\end{IEEEeqnarray*}
As $\hat\bu$ has zero tangential component along $\be_5$ and $\be_7$,
\begin{IEEEeqnarray*}{rCl}
  (\alpha_3-\alpha_1+\alpha_8-\alpha_6)(\hat\bu)&=&
  (\alpha_3-\alpha_7+\alpha_8)(\hat\bu)+(\alpha_5-\alpha_6-\alpha_1)(\hat\bu)\\[8pt]
  &=&-\int_{\partial\hat{f}_4}\hat\bu\cdot\hat\btau\,ds
   -\int_{\partial\hat{f}_1}\hat\bu\cdot\hat\btau\,ds\\[8pt]
  &=&-\iint_{\hat{f}_1}\nabla\times\hat\bu\cdot\hat\bn\,d\gamma
   -\iint_{\hat{f}_4}\nabla\times\hat\bu\cdot\hat\bn\,d\gamma\\[8pt]
%\yesnumber\label{pyr_edge_one}
  &=&\iint_{\hat{f}_1}(\nabla\times\hat\bu)_2\,d\gamma\\[8pt]
  & &\quad-2^{-\nicefrac{1}{2}}\iint_{\hat{f}_4}[(\nabla\times\hat\bu)_2 + (\nabla\times\hat\bu)_3]\,d\gamma.\\[8pt]
  &=&\iint_{\hat{f}_1}\frac{\partial\hat{u}_1}{\partial\hat{x}_3}\,dS
  -2^{-\nicefrac{1}{2}}\iint_{\hat{f}_4}[\frac{\partial\hat{u}_1}{\partial\hat{x}_3}
   + \frac{\partial\hat{u}_1}{\partial\hat{x}_2}]\,dS.
\end{IEEEeqnarray*}
We write down this component:
\begin{IEEEeqnarray}{rCl}\label{third_a}
	\nonumber
  (\wku)_3 & = & 
    -x\,\iint_{\hat{f}_1} \frac{{\s\partial} u_1}{{\s\partial} x_3}\,dS
    +\frac{xyz}{(1-z)^2}\,2^{-\nicefrac12}\iint_{\hat{f}_3}\frac{{\s\partial} \hat u_1}{{\s\partial} x_2}\,dS\\[8pt]
    &&\,+\frac{xy}{1-z}
     \left\{\iint_{\hat{f}_1}\frac{\partial\hat{u}_1}{\partial\hat{x}_3}\,dS
      -2^{-\nicefrac{1}{2}}\iint_{\hat{f}_4}[\frac{\partial\hat{u}_1}{\partial\hat{x}_3}
     +\frac{\partial\hat{u}_1}{\partial\hat{x}_2}]\,dS\right\}
\end{IEEEeqnarray}
\noindent Now, as expected, we switch to $\bu = (0,u_2,0)'$. In this case we have
\begin{IEEEeqnarray*}{rCl}
  \wku     & = & \alpha_2(\hat{\bu})\,\bgamma_2 +
	\alpha_4(\hat{\bu})\,\bgamma_4+ \alpha_7(\hat{\bu})\,\bgamma_7+\alpha_8(\hat{\bu})\,\bgamma_8.\\
  (\wku)_1 & = &(\alpha_7-\alpha_8)(\hat{\bu})\,\frac{yz}{1-z}\\
  		   & = &(\alpha_7-\alpha_8 - \alpha_3)(\hat{\bu})\,\frac{yz}{1-z}\\
  		   & = &\int_{\partial\hat{f}_4}\hat{\bu}\cdot\btau\,d\hat{s}\,\frac{yz}{1-z}\\
  		   \yesnumber\label{first_b}
  		   & = &\iint_{\hat{f}_4} \nabla\times\hat\bu\cdot\hat\bn_4\,dS\,\frac{yz}{1-z}
  		  \, = \,2^{-\nicefrac12} \iint_{\hat{f}_4} \frac{\partial\hat{u}_2}{\partial\hat{x}_1}\,dS\,\frac{yz}{1-z}.
\end{IEEEeqnarray*}
For the next component,
\begin{IEEEeqnarray*}{rCl}
	(\wku)_2 & = &\alpha_4(\hat\bu) + (\alpha_2-\alpha_4)(\hat\bu)\,x -
	(\alpha_4+\alpha_7)(\hat\bu)\,z + (\alpha_7-\alpha_8)(\hat\bu)\,\frac{xz}{1-z}.\\
	(\alpha_2-\alpha_4)(\hat\bu) & = & (\alpha_2-\alpha_3-\alpha_4+\alpha_1)(\hat\bu)\\
  &=&-\int_{\partial\hat{f}_5}\hat\bu\cdot\hat\btau\,d\hat{s}\\
  &=&-\iint_{\hat{f}_5}\nabla\times\hat{\bu}\cdot\hat\bn_5\,d\gamma
   =  \iint_{\hat{f}_5}\frac{\partial\hat{u}_2}{\partial\hat{x}_1}\,d\gamma.\\
  (\alpha_4+\alpha_7)(\hat\bu) & = & 
  (\alpha_4+\alpha_7-\alpha_5)(\hat\bu)\\
  &=& - \int_{\partial\hat{f}_2} \hat\bu\cdot\hat\btau\,d\hat{s}\\
  &=& -\iint_{\hat{f}_2}\nabla\times\hat\bu\cdot\hat\bn\,d\gamma~=~
      -\iint_{\hat{f}_2}\frac{{\s\partial} u_2}{{\s\partial} x_3}\,dS\mbox{,}
\end{IEEEeqnarray*}
and we write down this second component of the interpolate
\begin{IEEEeqnarray}{rCl}
  \nonumber
  (\wku)_2& = & \int_{\hat\be_4}\hat u_2\,d\alpha
               +x\iint_{\hat{f}_5}\frac{\partial\hat{u}_2}{\partial\hat{x}_1}\,dS.
               +z\iint_{\hat{f}_2}\frac{{\s\partial} u_2}{{\s\partial} x_3}\,dS\\
\label{second_b}
&&\,+\frac{xz}{1-z}\,2^{-\nicefrac12} \iint_{\hat{f}_4} \frac{\partial\hat{u}_2}{\partial\hat{x}_1}\,dS
\end{IEEEeqnarray}
And for the third one,
\begin{IEEEeqnarray*}{rCl}
	(\wku)_3&=&(\alpha_4+\alpha_7)(\hat\bu)\,y + (\alpha_2-\alpha_4-\alpha_7+\alpha_8)(\hat\bu)\,\frac{xy}{1-z}\\[4pt]
	& &\,+\,(\alpha_7-\alpha_8)(\hat\bu)\,\frac{xyz}{(1-z)^2}.\\[8pt]
	&=&(\alpha_2 - \alpha_4)(\hat\bu)\,\frac{xy}{1-z} + (\alpha_4+\alpha_7)(\hat\bu)\,y\\[8pt]
	& &-(\alpha_7-\alpha_8)(\hat\bu)\,\frac{xy}{(1-z)^2}
\end{IEEEeqnarray*}
but expressions for $(\alpha_2 - \alpha_4)(\cdot)$, $(\alpha_4+\alpha_7)(\cdot)$ and
$(\alpha_7-\alpha_8)(\cdot)$ were already stated, so we have
\begin{IEEEeqnarray}{rCl}
  \nonumber
  (\wku)_3 & = & 
\frac{xy}{1-z}\,\iint_{\hat{f}_5}\frac{\partial\hat{u}_2}{\partial\hat{x}_1}\,dS 
-y\,\iint_{\hat{f}_2}\frac{{\s\partial} u_2}{{\s\partial} x_3}\,dS\\[8pt]
  \label{third_b}
  & &-2^{-\nicefrac12} \iint_{\hat{f}_4} \frac{\partial\hat{u}_2}{\partial\hat{x}_1}\,dS\,\frac{xy}{(1-z)^2}.
\end{IEEEeqnarray}
\noindent{Finally for $\hat\bu = (0,0,\hat u_3)'$} it is
$\wku = \alpha_5(\hat\bu)\bgamma_5 + 
		\alpha_6(\hat\bu)\bgamma_6 + 
		\alpha_7(\hat\bu)\bgamma_7 +
		\alpha_8(\hat\bu)\bgamma_8$ and $\nabla\times\hat\bu = (\partial_2\hat{u}_3,
		-\partial_1\hat{u}_3,0)'.$\\[7pt]
First component of the interpolate:
\begin{IEEEeqnarray*}{rCl}
	(\wku)_1 & = &(\alpha_5-\alpha_6)(\hat\bu)z+
		(-\alpha_5+\alpha_6+\alpha_7-\alpha_8)(\hat\bu)\,\frac{yz}{1-z}.\\
	(\alpha_5-\alpha_6)(\hat\bu) & = & (\alpha_5-\alpha_6-\alpha_1)(\hat\bu) \\
	&=&-\iint\limits_{\hat f_1}\nabla\times\hat\bu\cdot\hat\bn\,dS
\end{IEEEeqnarray*}
On the other hand and analogously
\begin{IEEEeqnarray*}{rCl} 	
	\alpha_7-\alpha_8 & = &	\iint\limits_{\hat{f}_4}\nabla\times\hat\bu\cdot\hat\bn\,dS
\end{IEEEeqnarray*}
% &=&-\iint\limits_{\hat f_1}\partial_1 \hat{u}_3\,dS.
so \noindent{\color{blue} cambia lo de region tipo I y eso? estoy haciendo
$0\leqslant t_2 \leqslant 1; 0\leqslant t_1\leqslant 1-t_2$} 
\begin{IEEEeqnarray*}{rCl}
  (-\alpha_5+\alpha_6+\alpha_7-\alpha_8)(\hat\bu) & = & 
  \iint\limits_{\hat{f}_1}\nabla\times\bu\cdot\bn\,dS
  +\iint\limits_{\hat{f}_4}\nabla\times\bu\cdot\bn\,dS\\
&=&
  \int\limits_{\mathbb{D}_{\hat f_1}}\frac{{\s\partial} u_3}{{\s\partial} x_1}
  (\Phi_{\hat f_1}(t_1,t_2))\,dt_1dt_2
  -\int\limits_{\mathbb{D}_{\hat f_4}}\frac{{\s\partial} u_3}{{\s\partial} x_1}
  (\Phi_{\hat f_4}(t_1,t_2))\,dt_1dt_2\\
&=&
  \int_0^1\int_0^{1-t_2} 
  \left[\frac{{\s\partial} u_3}{{\s\partial} x_1}(t_1,0,t_2)
    - \frac{{\s\partial} u_3}{{\s\partial} x_1}(t_1,1-t_2,t_2)\right]
  \,dt_1dt_2\\
&=&
  -\int_0^1\int_0^{1-t_2}\int_0^{1-t_2} 
  \frac{{\s\partial^2} u_3}{{\s\partial} x_2{\s\partial} x_1}(t_1,s,t_2)
  \,dsdt_1dt_2\\[6pt]
&=&-\int\limits_{\hat{E}}\frac{{\s\partial}^2u_3}{{\s\partial} x_2{\s\partial} x_1}
\,d\hat\bx.
\end{IEEEeqnarray*}
For now
\begin{IEEEeqnarray}{rCl}\label{first_c}
	(\wku)_1 & = & -z\iint\limits_{\hat f_1}\frac{{\s\partial} u_3}{{\s\partial} x_1}\,dS
	-\frac{yz}{1-z}\int\limits_{\hat{E}}
		\frac{{\s\partial}^2u_3}{{\s\partial} x_2{\s\partial} x_1}\,d\hat\bx.
\end{IEEEeqnarray}
Regarding the second component, it is the symmetrical case.
\begin{IEEEeqnarray*}{rCl}
  (\wku)_2& = &(\alpha_5-\alpha_7)(\hat\bu)z
		+(-\alpha_5+\alpha_6+\alpha_7-\alpha_8)(\hat\bu)\frac{xz}{1-z}\\
	& = &z\iint\limits_{\hat f_2}\nabla\times\bu\cdot\bn\,dS
	-\frac{xz}{1-z}\int\limits_{\hat{E}}
	\frac{{\s\partial}^2u_3}{{\s\partial} x_2{\s\partial} x_1}\,d\hat\bx\\
	\yesnumber\label{second_c}
	& = &-z\iint\limits_{\hat f_2}\frac{{\s\partial} u_3}{{\s\partial} {x_2}}\,dS
		-\frac{xz}{1-z}\int\limits_{\hat{E}}
	\frac{{\s\partial}^2u_3}{{\s\partial} x_2{\s\partial} x_1}\,d\hat\bx.
\end{IEEEeqnarray*}
For the third component denote $\xi(x,y,z) = 
  \frac{xyz}{(1-z)^2}-\frac{xz}{1-z}$. Then
\begin{IEEEeqnarray*}{rCl}
  (\wku)_3& = & \alpha_5(\hat\bu) + x\,(\alpha_6-\alpha_5)(\hat\bu)+
  y\,(\alpha_7-\alpha_5)(\hat\bu)\\[5pt]
  && \,+\,\xi(x,y,z)\,(\alpha_6-\alpha_5+\alpha_7-\alpha_8)(\hat\bu).\\[5pt]
  \yesnumber\label{third_c}
  &=& \int\limits_{\hat\be_5}\hat\bu\cdot\hat\tau\,ds + 
  y\,\iint\limits_{\hat{f}_2}\frac{{\s\partial} u_3}{{\s\partial} {x_2}}\,dS +
  x\,\iint\limits_{\hat{f}_1}\frac{{\s\partial} u_3}{{\s\partial} {x_1}}\,dS
	-\xi(x,y,z)\int\limits_{\hat{E}}
	  \frac{{\s\partial}^2u_3}{{\s\partial} x_2{\s\partial} x_1}\,d\hat\bx.
\end{IEEEeqnarray*}
%\begin{IEEEeqnarray*}{rCl}
%	(\wku)_3 & = & \int\limits_{0}^1u_3(0,0,t)\,dt
%				+ x \int\limits_{0}^{1}\int\limits_{0}^{1-t}
%						\frac{{\s\partial} u_3}{{\s\partial} x_1} (s,0,t) \,ds\,dt
%				+ y \int\limits_{0}^{1}\int\limits_{0}^{1-t}
%						\frac{{\s\partial} u_3}{{\s\partial} x_2}(0,s,t) \,ds\,dt\\
%				&&\,\xi\int\limits_{\hat{E}}
%				\frac{{\s\partial}^2u_3}{{\s\partial} x_2{\s\partial} x_1}\,d\hat\bx.
%\end{IEEEeqnarray*}
All together, for a $\hat\bu=(\hat u_1,\hat u_2,\hat u_3)'$, if we combine 
what was obtained in~(\ref{first_a})--(\ref{third_c}), then
\begin{IEEEeqnarray*}{rCl}
  (\wku)_1 & = & 
    \int_{\hat\be_1}u_1\,d\alpha + 
  z \iint_{\hat f_1}(\nabla\times\hat\bu)_2\,dS +
  y \iint_{{\hat f_5}}\frac{{\s\partial} u_1}{{\s\partial} x_2}\,dS\\[6pt]
    &&\,
+\frac{yz}{1-z} \iint_{\hat f_3} \frac{{\s\partial} u_1}{{\s\partial} x_2}\,dS +
 \frac{yz}{1-z} \iint_{\hat f_4} \frac{{\s\partial} u_2}{{\s\partial} x_1}\,dS\\[6pt]
    &&\,
-\frac{yz}{1-z} \int\limits_{\hat{E}}\frac{{\s\partial}^2u_3}{{\s\partial} x_2{\s\partial} x_1}\,d\hat\bx.\\[12pt]
    (\wku)_2 & = & \int_{\hat\be_4}u_2\,d\alpha - 
    z \iint_{\hat f_2}(\nabla\times\hat\bu)_1\,dS +
    x \iint_{\hat f_5}\frac{{\s\partial} u_2}{{\s\partial} x_1}\,dS\\
    &&\,+\frac{xz}{1-z} \iint_{\hat f_3}
    \frac{{\s\partial} u_1}{{\s\partial} x_2}\,dS +
    \frac{xz}{1-z} \iint_{\hat f_4}
    \frac{{\s\partial} u_2}{{\s\partial} x_1}\,dS\\
    &&\,+\frac{xz}{1-z} \int\limits_{\hat{E}}
    \frac{{\s\partial}^2u_3}{{\s\partial} x_2{\s\partial} x_1}\,d\hat\bx.\\[12pt]
  (\wku)_3 & = & \int_{\hat\be_5}u_3\,d\alpha - 
    x \iint_{\hat{f}_1} (\nabla\times\hat\bu)_2\,dS +
    y \iint_{\hat{f}_2} (\nabla\times\hat\bu)_1\,dS\\[8pt]
  &&\,+\frac{xy}{1-z}
\left\{
  \iint_{\hat{f}_5}\frac{{\s\partial} u_2}{{\s\partial} x_1}\,dS+
  \iint_{\hat{f}_1}\frac{{\s\partial} u_1}{{\s\partial} x_3}\,dS-
  2^{-\nicefrac12}\iint_{\hat{f}_4}\frac{{\s\partial} u_1}{{\s\partial} x_3}\,dS
\right.\\[8pt]
  &&\,-
\left.
  2^{-\nicefrac12}\iint_{\hat{f}_4}\frac{{\s\partial} u_1}{{\s\partial} x_2}\,dS
\right\}-
\frac{xy}{(1-z)^2}
2^{-\nicefrac12}\iint_{\hat{f}_4}\frac{{\s\partial} u_2}{{\s\partial} x_1}\,dS\\[8pt]
\yesnumber\label{aux_label42}
&&\,+
\frac{xyz}{(1-z)^2}
2^{-\nicefrac12}\iint_{\hat{f}_3}\frac{{\s\partial} u_1}{{\s\partial} x_2}\,dS+
\xi(x,y,z)\,
\int\limits_{\hat{E}}
  \frac{{\s\partial}^2 u_3}{{\s\partial} x_1{\s\partial} x_2}\,d\hat\bx.
\end{IEEEeqnarray*}
\end{proof}
%%==========================================================================
%{\color{brown}
%    \begin{IEEEeqnarray*}{rCl}
%        (\wku)_1 & = & \int\limits_{0}^{1}u_1(t,0,0)\,dt + 
%        z \int\limits_0^1\int\limits_0^{1-t_1}
%        \frac{{\s\partial} u_1}{{\s\partial} x_3}(t_1,0,t_2)\,dt_2dt_1 +
%        y \int_{{\hat f_5}}
%        \frac{{\s\partial} u_1}{{\s\partial} x_2}\,dS\\
%        &&\,+\frac{yz}{1-z} \int\limits_0^1\int\limits_0^{1-t}
%        \frac{{\s\partial} u_1}{{\s\partial} x_2}(1-t,s,t)\,dsdt +
%        \frac{yz}{1-z} \int\limits_0^1\int\limits_0^{1-t}
%        \frac{{\s\partial} u_2}{{\s\partial} x_1}(s,1-t,t)\,dsdt\\
%        &&\,-z\int\limits_0^1\int\limits_0^{1-t_1}
%        \frac{{\s\partial} u_3}{{\s\partial} x_1}(t_1,0,t_2)\,dt_2dt_1 -
%        \frac{yz}{1-z} \int\limits_{\hat{E}}
%        \frac{{\s\partial}^2u_3}{{\s\partial} x_2{\s\partial} x_1}\,dV.
%    \end{IEEEeqnarray*}
%}
%%=========================================================================

%%===================================================================
%{\color{brown}
%\begin{IEEEeqnarray*}{rCl}
%    (\wku)_2 & = & \int\limits_{0}^{1}u_2(0,t,0)\,dt + 
%    z \int\limits_0^1\int\limits_0^{1-t}
%    \frac{{\s\partial} u_2}{{\s\partial} x_3}(0,t,s)\,dsdt +
%    x \int\limits_0^1\int\limits_0^{1}
%    \frac{{\s\partial} u_2}{{\s\partial} x_1}(s,t,0)\,dsdt\\
%    &&\,+\frac{xz}{1-z} \int\limits_0^1\int\limits_0^{1-t}
%    \frac{{\s\partial} u_1}{{\s\partial} x_2}(1-t,s,t)\,dsdt +
%    \frac{xz}{1-z} \int\limits_0^1\int\limits_0^{1-t}
%    \frac{{\s\partial} u_2}{{\s\partial} x_1}(s,1-t,t)\,dsdt\\
%    &&\,-z\int\limits_0^1\int\limits_0^{1-t}
%    \frac{{\s\partial} u_3}{{\s\partial} x_2}(0,s,t)\,dsdt +
%    \frac{xz}{1-z} \int\limits_{\hat{P}}
%    \frac{{\s\partial}^2u_3}{{\s\partial} x_2{\s\partial} x_1}\,dV.
%\end{IEEEeqnarray*}
%}
%%==================================================================

%%===========================================================================
%{\color{brown}
%\begin{IEEEeqnarray*}{rCl}
%        (\wku)_3 & = & \int\limits_{0}^{1}u_3(0,0,t)\,dt + 
%        x \int\limits_0^1\int\limits_0^{1-t}
%        \frac{{\s\partial} u_3}{{\s\partial} x_1}(s,0,t)\,dsdt -
%        x \int\limits_0^1\int\limits_0^{1-t}
%        \frac{{\s\partial} u_1}{{\s\partial} x_3}(t,0,s)\,dsdt\\
%        &&\,+y \int\limits_0^1\int\limits_0^{1-t}
%        \frac{{\s\partial} u_3}{{\s\partial} x_2}(0,s,t)\,dsdt -
%        y \int\limits_0^1\int\limits_0^{1-t}
%        \frac{{\s\partial} u_2}{{\s\partial} x_3}(0,t,s)\,dsdt\\
%        &&\,+\frac{xy}{1-z} \int\limits_0^1\int\limits_t^{1}
%        \frac{{\s\partial} u_1}{{\s\partial} x_2}(t,s,0)\,dsdt +
%        \frac{xy}{1-z} \int\limits_0^1\int\limits_0^{t}
%        \frac{{\s\partial} u_2}{{\s\partial} x_1}(t,s,1-t)\,dsdt\\
%        &&\,+\frac{xyz}{(1-z)^2} \int\limits_0^1\int\limits_0^{1-t}
%        \frac{{\s\partial} u_2}{{\s\partial} x_1}(s,1-t,t)\,dsdt
%        +\frac{xyz}{(1-z)^2} \int\limits_0^1\int\limits_0^{1-t}
%        \frac{{\s\partial} u_1}{{\s\partial} x_2}(1-t,s,t)\,dsdt\\
%        &&\,-\frac{xy}{1-z}
%        \int\limits_{0}^{1}
%        \int\limits_{0}^{t}
%        \int\limits_{0}^{1-t}
%        \frac{{\s\partial}^2u_1}{{\s\partial}x_2{\s\partial}x_3}(t,s,r)\,dr\,ds\,dt
%        -\frac{xy}{1-z}
%        \int\limits_{0}^{1}
%        \int\limits_{0}^{t}
%        \int\limits_{0}^{t}
%        \frac{{\s\partial}^2u_2}{{\s\partial}x_1{\s\partial}x_3}(r,s,1-t)\,dr\,ds\,dt\\
%        &&\,
%        +\frac{xyz}{(1-z)^2} \int\limits_{\hat{P}}
%        \frac{{\s\partial}^2 u_3}{{\s\partial} x_1{\s\partial} x_2}\,dV
%        -\frac{xz}{1-z} \int\limits_{\hat{P}}
%        \frac{{\s\partial}^2 u_3}{{\s\partial} x_1{\s\partial} x_2}\,dV
%    \end{IEEEeqnarray*}
%    }
%%======================================================================================

% &=&-\int\limits_0^1\int\limits_0^{1-t}\frac{{\s\partial} u_3}{{\s\partial} x_1}(s,0,t)\,dsdt.

%	(\pi\bu)_2 & = &\alpha_4 + (\alpha_2-\alpha_4)x -
%	(\alpha_4+\alpha_7)z + (\alpha_7-\alpha_8)\frac{xz}{1-z}.\\
%	\alpha_4 = \int\limits_{0}^{1}u_2(0,t,0)\,dt\\
%	\alpha_2-\alpha_4 & = & \int\limits_{0}^{1} u_2(1,t,0)-u_2(0,t,0)\,dt\\
%		&=&\int\limits_{0}^{1}\int\limits_{0}^{1}\frac{{\s\partial} u_2}{{\s\partial} x_1}(s,t,0)\,dsdt\\
%	\alpha_4+\alpha_7 & = & \int\limits_0^1 u_2(0,t,0)-u_2(0,t,1-t)\,dt\\
%		& = & \int\limits_0^1\int\limits_0^{1-t}\frac{{\s\partial} u_2}{{\s\partial} x_3}(0,t,s)\,dsdt.\\
%	\alpha_7-\alpha_8&=&\int\limits_{0}^{1} u_2(1-t,1-t,t)-u_2(0,1-t,t)\,dt\\
%		&=&\int\limits_{0}^{1}\int\limits_{0}^{1-t}\frac{{\s\partial} u_2}{{\s\partial} x_1}(s,1-t,t)\,dsdt.
