\chapter{Introduction}
\section{Motivation of the Problems and Previous Results} % (fold)
Input of using prisms combined with tetrahedra, with gaps filled with pyramids.
We say that a tetrahedron $T$ satisfies the ``regular vertex property'' with a
constant $c > 0$ (written RVP($c$)) if $T$ has a vertex $\bx_T$ such that,
if $M_T$ is the matrix made up of the unitary vectors in the directions
of the edges sharing $\bx_T$ as columns, then $|\det M_T| > c$.

If we mesh using only tetrahedra we will incur in using sequences of tetrahedra
wich do not fullfil uniformly the RVP($c$) for any positive constant $c$.

{\color{Orange}\#\#\#\#\#\#\#\# seguir aca.}


In other words, deficiency in using $\pazocal{F}_2$ class tetrahedra: we don't
have $\textit{h}$ order with the asymptotic relation 
$\textit{h}\sim N_{\Th}^{-\nicefrac13}$ preserved because we use strongly
that
\begin{IEEEeqnarray*}{rCl}
  \|\bu-\br_{\sss k,T}\bu\|_{\sss L^2(T)}& \leqslant & \sum_{1\leqslant i\leqslant 3} h_i \|{\s\partial_{\xi_i}}\bu\|_{\sss L^2(T)}
  	+ h_T\|\dv \bu\|_{\sss L^2(T)}
\end{IEEEeqnarray*}
for the family $\pazocal{F}_1$ while, for the family $\pazocal{F}_2$, the sharp
estimation is
\begin{IEEEeqnarray*}{rCl}
  \|\bu-\br_{\sss k,T}\bu\|_{\sss L^2(T)}& \leqslant & h_T \sum_{1\leqslant i\leqslant 3}
  \|{\s\partial_{\xi_i}}\bu\|_{\sss L^2(T)}.
\end{IEEEeqnarray*}
In the latter case, a big derivative in the direction $\xi_i$ will not be 
compensated with a small $h_i$, so we necessarily have to make the diameter
$h_T$ small and refine the meshes in all the directions.



{\color{Orange}\#\#\#\#\#\#\#\# intro vems and Motivation}
% section intro (end)