\chapter*{Introduction}
\addcontentsline{toc}{chapter}{Introduction} \markboth{INTRODUCTION}{}

A physical object is anisotropic if it exhibits vectorial physical
properties with different magnitudes when measured in different directions. In the case of
a numerical method we say that the family of meshes used is anisotropic if it 
contains sequences of elements such that the decrease order of their sizes along
one coordinate direction is higher than along other independent directions. 
We will formalize and 
illustrate this  in 
Chapter~\ref{auxlabel207}.

In several situations in mixed finite element approximations the use of an\-isot\-ropic
meshes 
is needed. This is the case for instance when dealing with
the Poisson equation 
in a polyhedron $\Omega\subseteq\mathbb{R}^3$ with concave edges and vertices, which, introducing the 
vectorial variable $\bu=-\nabla p$ can 
be written in mixed form as 
\begin{equation}\label{mf} 
\left\{\begin{IEEEeqnarraybox*}[][c]{,r/c/c/c/l,}
	&&&&\\
	\bu     & = & \nabla p   & \qquad & \mbox{in }\Omega\\
	-\dv\bu & = &        f   & \qquad & \mbox{in }\Omega\qquad(f\in L^2(\Omega))\\
	p       & = & 0          & \qquad & \mbox{on }\partial\Omega.\\
	&&&&%
	\end{IEEEeqnarraybox*}\right.
\end{equation}
The standard mixed 
variational formulation is to find $\bu\in H(\Div,\Omega)$ and $p\in L^2(\Omega)$ 
such that
\begin{equation}\label{P}
	\begin{IEEEeqnarraybox*}[][c]{l/c/r/C/l}
	\forall \bv\in H(\Div,\Omega)  & \qquad & a(\bu,\bv) + b(\bv,p)   & = & 0    \\
	\forall q\in   L^2(\Omega)	   & \qquad &    		   b(\bu,q)   & = & (-f,q)
	\end{IEEEeqnarraybox*}
\end{equation}
with
\[
a(\bv,\bw)=\int_\Omega\bv\cdot\bw\,d\bx,\qquad b(\bv,q)=\int_\Omega q\,\mbox{div\,}\bv\,d\bx.
\]
%Of course, the problem for the $\mbox{div}(a\nabla)$ operator can be similarly treated.
With $\Omega$ being non--convex, the solution $\bu$ of~\eqref{P} is not in $H^1(\Omega)$ in the general case
due to vertex and edges singularities. 
In particular, close to concave edges, the solution is expected to be 
more regular along the direction of that edge than 
transversally to it (see~\cite{apelNicaise}), 
and that is why the mesh has to be accordingly graded and 
refined in 
order to recover the optimal order of convergence of the sequence of numerical
solutions
with respect to the number of degrees of freedom (see~\cite{alw, apelNicaise}). 
Those meshes contain elements which are arbitrarily elongated in the direction 
of the concave edges. If we tried to construct such a sequence of meshes using
only tetrahedra we would incur in using subfamilies of them which do
not fullfill certain conditions needed for the anisotropic analysis, and
there is where our proposed method with elements of different geometries becomes
meaningful. 

To explain the latter further and the main result of the present Thesis we 
put the following
definitions and related previous results.

First, we say that a tetrahedron $T$ satisfies the \emph{regular vertex property}
with a
constant $\bar{c} > 0$ (written $T\in \pazocal{RVP}(\bar c)$) if $T$ has
a vertex $\bx_T$ such that,
if $M_T$ is the matrix made up with the unitary vectors in the directions
of the edges sharing $\bx_T$ as columns, then $|\det M_T| > \bar{c}$.

A less restrictive geometrical property is the following. 
We say that a tetrahedron $T$ satisfies the  {\it maximum angle condition} 
with paramenter $\bar\alpha$
(writen $T\in\pazocal{MAC}(\bar\alpha$))  if the angles of the faces of 
$T$ and between faces are
less than $\bar\alpha$. 

With these two notions in mind we need to start with the following result 
from~\cite{aadl}.
If $T$ is a tetrahedron in $\pazocal{RVP}(\bar c)$ and $\br_{\sss k,T}$ 
is the Raviart-Thomas interpolation 
operator (see~\cite{nedelec2, MR0483555}), then there is a positive $C(\bar c)$
such that for all  
$\bu\in H^1(T)^3$
\begin{IEEEeqnarray}{rCl}\label{rvp}
  \|\bu-\br_{\sss k,T}\bu\|_{\sss L^2(T)^3}& \leqslant & C 
    \left\{\sum_{1\leqslant i\leqslant 3} h_i \|{\s\partial_{\xi_i}}\bu\|_{\sss L^2(T)^3}
	  + h_T \|\dv\bu\|_{\sss L^2(T)}\right\}
\end{IEEEeqnarray}
where we wrote 
$\xi_i$ for the local coordinates with origin at the regular vertex and 
$h_i$ for the lengths of the edges incident to it
and $h_T$ for the diameter of $T$.

On the other hand, if $T\in\pazocal{MAC}(\bar\alpha)$ then there exists
a positive $C(\bar\alpha)$
depending only on $\bar\alpha$ such that for all  
$\bu\in H^1(T)^3$
it holds
\begin{IEEEeqnarray}{rCl}\label{mac}
  \|\bu-\br_{\sss k,T}\bu\|_{\sss L^2(T)^3}& \leqslant & C h_T \sum_{1\leqslant i\leqslant 3}
  \|{\s\partial_{\xi_i}}\bu\|_{\sss L^2(T)^3}
\end{IEEEeqnarray}
($\xi_i$ are local coordinates at a vertex of $T$, but we don't necessarily 
have a regular one).

As an observation we point out that inequality~\eqref{mac} is strictly 
weaker than~\eqref{rvp}, since there are elements  
satisfying MAC($\bar\alpha$) for a fixed $\bar\alpha$ with arbitrarily 
small $\pazocal{RVP}$ parameter $\bar c$, making the constant $C(\bar c)$ in~\eqref{rvp} 
to degenerate. An example is 
depicted in Figure~\ref{fig:tetraedros}. Furthermore,
as stated 
in~\cite{aadl} by means of a counterexample, 
inequality~\eqref{rvp} can't be proved under the maximum angle condition only. 

Now returning to clarify what was said in the second paragraph, 
it is possible to construct anisotropic graded meshes to tacle numerically
elliptic problems in domains with singularities, consisting those meshes 
exclusively of tetrahedral elements all satisfying a $\pazocal{MAC}(\bar\alpha)$ condition for 
a fixed $\bar\alpha<\pi$, but unfortunately those 
elements do not satisfy an $\pazocal{RVP}(\bar c)$ condition for any uniform positive parameter
$\bar{c}$. That is 
due to the existence of tetrahedra with bounded 
maximal angle but poor regular vertex constant. Consequently, interpolation 
error estimate~\eqref{rvp} 
can not be globally used to estimate the error approximation, but~\eqref{mac} 
has to be taken, and therefore
the anisotropic properties of the meshes may give no profit. In other words, 
if we look at both inequalities, in the second case a \emph{big derivative} 
in the 
direction $\xi_i$ would not necessarily be 
compensated with a small $h_i$, so we would have to make the diameter
$h_T$ smaller and refine the meshes in all the directions, which is what we want
to avoid, and we would lose $\textit{h}$ order of convergence of the numerical error 
with the asymptotic relation 
$\textit{h}\sim N_{\Th}^{-\nicefrac13}$ (here $\Th$ is a mesh and $N_{\Th}$ is its
cardinal; the concrete meaning of the meshing parameter $\textit{h}$, 
which isn't the diameter of the elements, as it is in the case of uniform meshes, 
will
be made clear in Subsection~\ref{meshes} when we propose our meshing procedure).

\tetsTikz

An idea to overcome the mentioned difficulty, for the case of $\Omega$ being a 
cylindrical polyhedral domain, 
was proposed in~\cite{MR1866274}. In  this case, when $f$ is in $L^2(\Omega)$, 
the solution may exhibit only singularities along concave edges.
Then the authors proposed a lowest order mixed Raviart--Thomas method on graded 
anisotropic meshes made up of triangular right prisms and they proved optimal error 
estimates by means of adequate anisotropic interpolation results. 
In this way, tetrahedra which do not satisfy a
uniform regular vertex property are avoided.

Also in~\cite{MR1866274} a mixed Raviart--Thomas method is 
proposed on the tetrahedral 
anisotropic graded mesh which is obtained by splitting the prismatic elements 
into three tetrahedra. Of course, these kind of meshes contain the bad elements 
which are avoided with the prismatic ones, and in order to obtain optimal 
approximation error estimates, the price payed is to require additional regularity 
on the right hand side $f$, precisely, it has to belong to a weighted Sobolev space.

One of the results presented in this thesis extends
the result in~\cite{MR1866274} since our method was seeked in order to be able 
to deal with the mixed approximation of~\eqref{mf} for $f\in L^2(\Omega)$,
and also for a general polyhedral domain $\Omega$. As such domains not always 
admit a partition by means of right prisms and since 
we also would like to avoid to require more regularity to $f$ as mentioned in 
the previous paragraph, we propose a 
discretization based on hybrid meshes that consists of prisms combined with 
tetrahedra, with 
interelemental gaps filled with pyramids. We obtain 
the approximation error estimate
\begin{equation}\label{auxlabel401}
 \|\bu-\bu_{\textit h}\|_{L^2(\Omega)^3} + \|p-p_{\textit h}\|_{L^2(\Omega)} 
 \leqslant C\,\textit{h}\|f\|_{L^2(\Omega)}
\end{equation}
with $\bu_{\textit h}$ and $p_{\textit h}$ being the approximations of the 
solutions $\bu$ and $p$ of~\eqref{mf} by means of weighted Sobolev norms
treating the singularities in a localized manner. 
Firstly we introduced and analyzed a new 
combined Finite and Virtual 
Element Method on a polyhedron which is fully analysed when the meshes are made 
up of the mentioned polyhedra. Local 
interpolation error estimates for the discrete element spaces are 
optimal and anisotropic on anisotropic right prisms, which can be
used to obtain optimal approximation error estimates when the 
solution has edge or vertex singularities, as pointed out at the beginning of 
this introduction, by using suitably graded and adapted meshes which 
allow for anisotropic elements, and which only includes tetrahedra with a
uniform positive regular vertex constant, precisely avoiding the use of tetrahedra
with arbitrary small $\pazocal{RVP}$ constant.

Not only we proved the result for an abstract family of meshes with the previous 
properties, but also we show a general meshing procedure which starts by isolating and 
classifying the singularities (cfr. Theorem~\ref{thm_regularity}), and so we prove the 
existence of such 
a family of anisotropic graded hybrid meshes by 
constructing it.

The 
discrete spaces $V_{\textit h}$ 
corresponding 
to the proposed meshes
were obtained fulfilling
the following conditions, suitable for discretizations in $\mathbb{R}^3$.
\begin{enumerate}
\item \label{punto1} Conformity: 
The space $V_{\textit h}$ is $H(\mbox{div})$-conforming.
\item \label{punto2} Optimal and anisotropic approximation properties: 
Optimal interpolation error estimates are valid even on this family of 
meshes which do not satisfy the standard shape--regularity condition 
(cfr. Definition~\ref{auxlabel403} and see~\cite{ciarlet}).
\item \label{punto3} Domain generality: The space is well defined on 
conforming meshes without restricting the 
considered domains to few special polyhedra. 
\end{enumerate}
As suggested in~\cite{bfm} for the 2D case, we present $V_{\textit h}$ as a 
virtual element space which 
locally coincides with the original lowest order 3D Finite Element 
Raviart-Thomas space on
tetrahedra and right prisms, and naturally extends it to pyramidal elements. 
In particular, normal components of the discrete functions are constant on the 
faces of the 
elements, fitting well across different shape's elements. 
In this way requirement~\ref{punto1} is verified. 
One advantage of this presentation, is that the definition of the local spaces 
is independent of the 
geometry of the element, see Section~\ref{auxlabel290}. 
As we already commented, item~\ref{punto3} in the previous list is fulfilled
by incorporating the pyramids of parallelogram basis to a grid with triangular
right prisms and tetrahedra.
With respect to point~\ref{punto2}, we present optimal anisotropic local stability
and  
interpolation error estimates in several parts although not all of them
are used in the application, in some cases because the functional spaces 
they are used for are not involved in the  mixed model elliptic problem~\eqref{P}
and in other cases because our meshing strategy requires not all the element
types to present anisotropy.

Meshes with more general polyhedral--shaped elements can be considered. 
That is one of the Virtual Elements Method's main features indeed. 
But we decided to restrict
ourselves to few 
shapes since our main objective is to allow for meshes with anisotropic elements, 
and with uniformly valid anisotropic 
estimates. As we said, this was fulfilled by allowing elongated right prisms, 
since the local space for
those elements becomes known (cfr. Proposition~\ref{vem_equal_fem}). One 
difficulty when other shapes are considered (like
oblique prisms, for instance) is that the local virtual element space is not 
preserved by 
\emph{push--forward} transformations based on affine mappings
(we require a vanishing ${\bf curl}$ property that is not preserved in this 
situation), and so the standard rescaling arguments are hard to use 
(see~\ref{auxlabel290}). However, this is not a limitation on the polyhedra
that we can treat with our procedure, since we can restrict ourselves
to right prisms by duplicating the elements of the first mesh in the worst case,
and so augmenting the number of elements by a constant factor.

Our method presents different number of degrees of freedom through the 
mesh, as prisms, tetrahedra and pyramids have different number of faces, and also 
allows for different planar geometry for the degrees of freedom, since we are 
dealing
with triangular and rectangular faces at the same time. As a consequence, our 
method is in some sense a generalization of the classical Virtual Element Methods,
in which all the elements have the same arbitrary geometry, and each element yields
the same number of degrees of freedom, all of the same type (for example, 
evaluation at the same points on the boundary of each element).

Hybrid meshes that include tetrahedral and prismatic (and even hexahedral) 
elements may be needed to satisfy the demands of a specific problem geometry 
(complex regions) or to reach efficient calculations. If these meshes are 
to avoid hanging nodes then they will in general contain pyramids, see 
for instance~\cite{owenSaigal}. Several articles contain introduction and analysis
of conforming finite elements on pyramids; some of them are~\cite{bergot} 
for $H^1$--elements and 
\cite{gh99, Nigam-2012} for $H(\mbox{div})$-- and $H({\bf curl})$--elements, 
the first one for 
lowest order and the second one for higher order. 
In~\cite{Nigam-2012} the authors prove that it is not possible to construct 
useful $H^1$--finite 
elements on pyramids using  polynomial functions only. In the $H(\mbox{div})$ 
case, it is explained also that all the spaces 
constructed in the literature contain non--polynomial functions.

We finish this introduction mentioning other results obtained which we regard
as additional results because they weren't used
for the main problem
of the Thesis and can be seen as extensions of some theoretical results.

We proved local anisotropic 
stability and
interpolation error estimates for arbitrary order Prismatic Finite Elements in both
the curl--conforming and div--conforming classes of elements. Our method uses only
the least order case of this estimates, an we leave a \emph{high order} version
of the method for further research.

Lastly, we obtained and present also local anisotropic stability
and interpolation error estimates for the lowest order Pyramidal
Finite Elements constructed in~\cite{gh99, Nigam-2012} for both
the $\bcurl$--conforming and div--conforming classes of elements. As we show
explicitly in Chapter~\ref{auxlabel202}, the shape functions 
spanning these Finite Element spaces are not polynomial and are singular, yet bounded,
in the reference pyramid. This is a reason why we considered that
our combined FE--VE approach presents the advantage of avoiding the evaluation
of functions with those properties in computer implementations.   

% section intro (end)