\chapter*{Introducci\'on}
\addcontentsline{toc}{chapter}{Introducci\'on} \markboth{INTRODUCCION}{}

A combined Finite and Virtual Element Method is introduced for the 
mixed approximation of a 
simple model problem for the Laplace operator on a polyhedron. The 
method is fully analysed when the meshes are made up of triangularly
right prisms, pyramids and tetrahedra. The local discrete 
spaces coincide with the lowest order Raviart-Thomas 
spaces on tetrahedral and triangularly right prismatic elements, and 
extend them to pyramidal elements. The discrete scheme 
is well posed and optimal error estimates are proved on meshes which 
allow for anisotropic elements. In particular, local 
interpolation error estimates for the discrete element space are 
optimal and anisotropic on anisotropic right prisms, which can be
used to obtain optimal approximation error estimates when the 
solution has edge or vertex singularities by using suitably adapted meshes.   

The main result of the present Thesis in motivated by the following
definitions and related previous results.

We say that a tetrahedron $T$ satisfies the ``regular vertex property'' with a
constant $\bar{c} > 0$ (written $T\in \pazocal{RVP}(\bar c)$) if $T$ has
 a vertex $\bx_T$ such that,
if $M_T$ is the matrix made up of the unitary vectors in the directions
of the edges sharing $\bx_T$ as columns, then $|\det M_T| > \bar{c}$.

A less restrictive geometrical property is the following. 
 We say that a tetrahedron $T$ satisfies the  {\it maximum angle condition} with paramenter $\bar\alpha$
(writen $T\in\pazocal{MAC}(\bar\alpha$))  if the angles of the faces of 
$T$ and between faces are 
less than $\bar\alpha$. 

If we mesh using only tetrahedra we will incur in using subfamilies of tetrahedra
wich do not fullfil uniformly the $\pazocal{RVP}(\bar c)$ condition for any positive constant $\bar c$.

In other words, deficiency in using $\pazocal{F}_2$ class tetrahedra: we don't
have $\textit{h}$ order with the asymptotic relation 
$\textit{h}\sim N_{\Th}^{-\nicefrac13}$ preserved because we strongly use 
that if $T\in \pazocal{RVP}(\bar c)$ and $\br_{\sss k,T}$ is the Raviart-Thomas interpolation 
operator (\cite{nedelec2, MR0483555}), then there is a positive $C(\bar c)$
\begin{IEEEeqnarray*}{rCl}
  \|\bu-\br_{\sss k,T}\bu\|_{\sss L^2(T)^3}& \leqslant & \sum_{1\leqslant i\leqslant 3} h_i \|{\s\partial_{\xi_i}}\bu\|_{\sss L^2(T)^3}
  	+ h_T\|\dv \bu\|_{\sss L^2(T)}
\end{IEEEeqnarray*}
for the family $\pazocal{F}_1$ (see~(\cite{aadl}) while, for the family $\pazocal{F}_2$, the sharp
estimation is
 that we know that there exists a constant $C(\bar\alpha)$
 depending only on $\bar\alpha$ such that for all $T\in\pazocal{MAC}(\bar\alpha)$ 
 for all $\bu\in H^1(T)^3$
 it holds
\begin{IEEEeqnarray*}{rCl}
  \|\bu-\br_{\sss k,T}\bu\|_{\sss L^2(T)^3}& \leqslant & h_T \sum_{1\leqslant i\leqslant 3}
  \|{\s\partial_{\xi_i}}\bu\|_{\sss L^2(T)^3}.
\end{IEEEeqnarray*}
In the latter case, a big derivative in the direction $\xi_i$ will not necessarily be 
compensated with a small $h_i$, so we  have to make the diameter
$h_T$ small and refine the meshes in all the directions.

Inequality \eqref{mac} is weaker than \eqref{rvp}, since there are elements 
satisfying MAC($\bar\alpha$) for a fixed $\bar\alpha$ with arbitrarily 
small $\pazocal{RVP}$ parameter $\bar c$, making to degenerate  
the constant $C(\bar c)$ in \eqref{rvp} (cfr.~\cite{aadl}). Furthermore,
as stated 
in~\cite{aadl} by means of a counterexample, 
inequality~\eqref{rvp} can't be proved under the maximum angle condition only. 


In several situations in mixed finite element approximations the use of meshes with narrow elements is needed. This is the case for instance when dealing with the Poisson equation 
\begin{eqnarray}\label{ecuacion}
-\Delta p &=& f\qquad \mbox{on }\Omega\\\nonumber p&=&0\qquad \mbox{in }\partial\Omega,
\end{eqnarray}
in a polyhedron $\Omega$ with concave edges, which, introducing the vectorial variable $\bu=-\nabla p$ can be written as
\begin{equation}\label{mf} \left\{\begin{array}{rcl}
\bu&=&-\nabla p\qquad \mbox{in }\Omega\\
\dv\bu&=&f\qquad \mbox{in }\Omega\\
p&=&0\qquad\mbox{on }\partial\Omega.\end{array}\right.
\end{equation}
In this case,  $\bu$ is in general not in $H^1$ due to vertex and edges 
singularities. In particular, close to concave edges, the solution is expected to be more regular in its direction than 
transversally to it, and consequently the mesh has to be accordingly refined in order to recover optimal order of convergence 
with respect to the number of degrees of freedom 
\cite{alw, apelNicaise}. Those meshes contain elements arbitrarily 
elongated in the direction of concave edges (direction in which the solution is more regular). It is possibly to construct this
kind of meshes with tetrahedral elements all satisfying MAC($\bar\alpha$) for $\bar\alpha<\pi$ fixed, but unfortunately those 
elements do not satisfy $\pazocal{RVP}$ with a parameter uniformly far from $0$. That is 
due to the presence of tetrahedra with bounded 
maximal angle but poor regular vertex constant, see Figure \ref{fig:tetraedros}. So, interpolation error estimate \eqref{rvp} 
can not be globally used to estimate the error approximation, but \eqref{mac} has to be taken, and consequently, 
the anisotropic properties of the meshes may give no profit.  

\tetsTikz

An idea to overcome this difficulty, just for the case of $\Omega$ being a cylindrical polyhedral domain, 
was proposed in \cite{MR1866274}. In  this case, when $f$ is in $L^2(\Omega)$, 
the solution may exhibit only singularities along concave edges.
Then the authors proposed a lowest order mixed Raviart--Thomas method on graded 
anisotropic meshes made up of triangular right prisms and they proved optimal error 
estimates by means of adequate anisotropic interpolation results. 
In this way, tetrahedra which do not satisfy an uniform regular vertex property are avoided. 

Also in \cite{MR1866274} a mixed Raviart--Thomas method on the tetrahedral 
anisotropic graded mesh which is obtained by splitting the prismatic elements 
into three tetrahedra. Of course, these kind of meshes contain the bad elements 
which are avoided with the prismatic ones. However, in order to obtain optimal 
approximation error estimates, the price payed is to require additional regularity 
on the right hand side, precisely, it has to belong to a weighted Sobolev space.

Input of using prisms combined with tetrahedra, with gaps filled with pyramids.

One of the results presented in this thesis is an extension 
of the result in~\cite{MR1866274} in order to be able to deal with the
mixed approximation of \eqref{ecuacion} with $f\in L^2(\Omega)$ in general polyhedral domains. 
Since such domains not always admit a partition by means of right prisms and since 
we also would like to avoid to require more regularity to $f$ as mentioned in the previous paragraph, we propose a 
discretization based on hybrid meshes. As in \cite{MR1866274}, we obtain the estimate
\begin{equation}\label{jl2: aim}
 \|\bu-\bu_h\|_{L^2(\Omega)} + \|p-p_h\|_{L^2(\Omega)} \le Ch\|f\|_{L^2(\Omega)}.
\end{equation}
with $\bu_h$ and $p_h$ being the approximations of the solutions $\bu$ and $p$ of \eqref{ecuacion}. 
To do that, firstly we introduce and analyze a new finite/virtual element space $V_h$ in 
$\mathbb R^3$ with the following conditions:  
\begin{enumerate}
\item \label{punto1} Conformity: The space $V_h$ must be $H(\mbox{div})$-conforming.
\item \label{punto2} Optimal and aniowenSaigaltropic approximation properties: Optimal interpolation error estimates have to be valid even on families of meshes which bergot no satisfy the standard shape-regularity condition \cite{ciarlet}.
\item \label{punto3} Domain generality: The space has to be well defined on conforming (without hanging nodes) bergot meshes without restricting the considered domains to few special polyhedra. 
\end{enumerate}
As suggested in \cite{bfm} for the 2d case, we present $V_h$ as a virtual element space on a conforming 
polyhedral mesh, which 
locally coincides with the original lowest order 3d Raviart-Thomas space on
tetrahedra and right prisms, and naturally extends it in the case of pyramidal elements. 
In particular, normal components of the discrete functions are constants on the faces of the 
elements, fitting well across different shape's elements. In this way requirement~\ref{punto1} is verified. 
One advantage of this presentation, is that the definition of the local spaces is independent of the 
geometry of the element, see Section \ref{discrete_spaces_in_chap_virtuals}. 

In order to satisfy condition \ref{punto2} and taking into account what we 
remarked at the beginning of this Section, we need to avoid the use of a kind of anisotropic 
tetrahedra. Following \cite{MR1866274} we allow for arbitrarily anisotropic (triangularly) 
right prisms for which optimal anisotropic local interpolation error are proved. But the 
use of only right prisms would restrict too much the domains which can be considered, 
and because of that we further allow for tetrahedral elements, and pyramids of parallelogram basis
in order to glue right prisms and tetrahedra together. 
Our interpolation error estimates depend on the aspect ratio of the pyramids 
and, for simplicity, also of tetrahedra, and so we are implicitly imposing 
that this kind of elements must be uniformly isotropic. However we do not 
lose generality, since meshes adapted to general singularities in polyhedra can be constructed,
as we show at the end of the article, satisfying this requirement. 
Then conditions \ref{punto2} and \ref{punto3} are also satisfied.  

Hybrid meshes including tetrahedral and prismatic (and even hexahedral) 
elements may be needed to satisfy the demands of a specific problem geometry 
(complex regions) or to reach efficient calculations. If these meshes are 
to avoid hanging nodes then they will in general contain pyramids, see 
for instance \cite{owenSaigal}. Several articles have introduced and analysed 
conforming finite elements on pyramids, some of them are \cite{bergot} for $H^1$-elements, 
\cite{gh99, Nigam-2012} for $H(\mbox{div})$- and $H({\bf curl})$-elements, the first one for 
lowest order and the second one for higher order. 
In \cite{Nigam-2012} it is proved that it is not possible to construct useful $H^1$-finite 
elements on pyramids using only polynomial functions. In the $H(\mbox{div})$ 
case, it is explained also in the same article that all the spaces 
constructed in the literature contain non--polynomial functions.

In this Thesis we obtained and present local anisotropic stability
and interpolation error estimates for the lowest order Pyramidal
Finite Elements constructed in~\cite{gh99, Nigam-2012} for both
the $\bcurl$--conforming and div--conforming classes of elements. As we show
explicitly in Chapter~\ref{auxlabel202}, the shape functions 
spanning these Finite Element spaces are not polynomial and are singular, yet bounded,
in the reference pyramid. This is a reason why we considered to move
to our combined FE--VE approach better, in this case from the implementation
point of view. 

We consider the model elliptic problem of the mixed
formulation \eqref{ecuacion}  with 
$f\in L^2(\Omega)$ and $\Omega$ a polyhedral domain. Its mixed 
variational formulation can be written as: to find $\bu\in V$ and $p\in Q$ 
such that
\begin{eqnarray}\label{P}
a(\bu,\bv) - b(\bv,p) & = & 0 \qquad \forall \bv\in V\\ \nonumber b(\bu,q) &
= & (f,q)\qquad \forall q\in Q,  
\end{eqnarray}
with
\[
a(\bv,\bw)=\int_\Omega\bv\cdot\bw,\qquad b(\bv,q)=\int_\Omega
q\,\mbox{div\,}\bv
\]
and
\[
V=H(\dv,\Omega), \qquad Q=L^2(\Omega).
\]
Of course, the problem for the $\mbox{div}(a\nabla)$ operator can be similarly treated. 

% The proposed scheme gives optimal error estimates when the solution $p$ is in $H^2(\Omega)$. We do not deal in this paper with the more challenging case when $p$ is not so regular, which would require an extent development of too fine estimates involving weighted norms. However, we develop here all the numerical tools to deal with it in a forthcoming article.

Meshes with more general polyhedral-shaped elements can be considered. 
Indeed that is one of the VEM's main features. But we decided to restrict ourselves to few (but without loss of generality) 
shapes since our main objective is to allow for meshes with anisotropic elements, and with uniformly valid anisotropic 
estimates. This is fulfilled by allowing right prisms, since the local space for those elements becomes known \cite{nedelec2} 
and that allows to obtain stability and interpolation error estimates. One difficulty when other shapes are considered (like 
oblique prisms, for instance) is that the local VEM space is not preserved by Piola transformation (the vanishing ${\bf curl}$ 
property is not preserved), and so standard rescaling arguments are hard to use.  


% section intro (end)