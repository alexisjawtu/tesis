\chapter*{Introducci\'on}
\addcontentsline{toc}{chapter}{Introducci\'on} \markboth{INTRODUCCION}{}
Un objeto f\'isico se dice anis\'otropo si exhibe 
propiedades f\'isicas vectoriales con magnitudes
distintas cuando se las mide en direcciones 
distintas (cfr.~\cite{jamet}). En
el caso de un m\'etodo num\'erico diremos que la familia
de mallas usada es anis\'otropa si contiene sucesiones 
de elementos tales que el orden de decrecimiento de
sus tamaños a lo largo de una direcci\'on es superior
al correspondiente a otras direcciones independientes.
Formalizaremos e ilustraremos esto en el 
Cap\'itulo~\ref{auxlabel207}.

En varias situaciones en aproximaciones por elementos 
finitos mixtos es necesario el uso de mallas anis\'otropas.
Este es el caso, por ejemplo, cuando se trata con la 
ecuaci\'on de Poisson en un poliedro 
$\Omega\subseteq\mathbb{R}^3$ con aristas y v\'ertices
c\'oncavos, la cual, introduciendo la variable vectorial
$\bu=\nabla p$ puede escribirse en forma mixta como
\begin{equation}\label{mfSpanish} 
\left\{\begin{IEEEeqnarraybox*}[][c]{,r/c/c/c/l,}
	&&&&\\
	\bu     & = & \nabla p   & \qquad & \mbox{in }\Omega\\
	-\dv\bu & = &        f   & \qquad & \mbox{in }\Omega\qquad(f\in L^2(\Omega))\\
	p       & = & 0          & \qquad & \mbox{on }\partial\Omega.\\
	&&&&%
	\end{IEEEeqnarraybox*}\right.
\end{equation}
La formulaci\'on variacional mixta est\'andar es hallar
$\bu\in H(\Div,\Omega)$ and $p\in L^2(\Omega)$ 
tales que
\begin{equation}\label{PSpanish}
	\begin{IEEEeqnarraybox*}[][c]{l/c/r/C/l}
	\forall \bv\in H(\Div,\Omega)  & \qquad & a(\bu,\bv) + b(\bv,p)   & = & 0    \\
	\forall q\in   L^2(\Omega)	   & \qquad &    		   b(\bu,q)   & = & (-f,q)
	\end{IEEEeqnarraybox*}
\end{equation}
en donde
\[
a(\bv,\bw)=\int_\Omega\bv\cdot\bw\,d\bx,\qquad b(\bv,q)=\int_\Omega q\,\mbox{div\,}\bv\,d\bx.
\]
Dado que $\Omega$ no es convexo, la soluci\'on $\bu$ 
de~\eqref{PSpanish} no pertenece a $H^1(\Omega)$ en
el caso general por tener singularidades de arista y 
v\'ertice. En particular, cerca de aristas c\'oncavas,
se espera que la soluci\'on sea m\'as regular a lo largo
de la direcci\'on de esa arista que transversalmente a ella
(ver~\cite{apelNicaise}), y es por eso que las mallas
tienen que ser adecuadamente graduadas y refinadas
para recuperar el orden \'optimo de convergencia de 
la sucesi\'on de soluciones num\'ericas con respecto al 
n\'umero de grados de libertad (ver~\cite{alw,apelNicaise}).
Ese tipo de mallas contiene elementos que son tan
estrechos como se quiera en direcci\'on ortogonal
a las aristas c\'oncavas, con tal de tomarlos lo 
suficientemente cerca de ellas. Si tom\'aramos 
una sucesi\'on de mallas como esta hecha \'unicamente con
tetraedros, incurrir\'iamos en el uso de subfamilias de 
estos que no verifican ciertas condiciones que son 
necesarias para el an\'alisis anis\'otropo, y es aqu\'i donde
nuestro m\'etodo propuesto  con elementos de geometr\'ias
diferentes cobra sentido.

Para aclarar lo \'ultimo y el principal resultado de esta
Tesis (cfr.~\cite{alexisAriel}) ponemos las siguientes definiciones y resultados
previos relacionados.

Primero, $T$ satisface la \emph{propiedad del
v\'ertice regular}
con una 
constante $\bar{c} > 0$ (escrito $T\in \pazocal{RVP}(\bar c)$) si
$T$ tiene un v\'ertice $\bx_T$ tal que,
tomando $M_T$ como la  matriz cuyas columnas
son los vectores unitarios en las direcciones
de las aristas que comparten a $\bx_T$, entonces
$|\det M_T| > \bar{c}$.

Una propiedad geom\'etrica menos restrictiva  es
la siguiente. 
Diremos que un tetraedro $T$ satisface la
\emph{propiedad del \'angulo m\'aximo}
con par\'ametro $\bar\alpha$
(escrito $T\in\pazocal{MAC}(\bar\alpha$))
si los \'angulos de las caras de $T$
y entre caras son menores a $\bar\alpha$. 

Con estas dos nociones en mente, citamos el siguiente
resultado de~\cite{aadl}. Si $T$ es un tetraedro en 
$\pazocal{RVP}(\bar c)$ y $\br_{\sss k,T}$ 
es el interpolador de Raviart-Thomas
(ver~\cite{nedelec2, MR0483555}), entonces existe una
$C>0$ que depende solamente de $\bar{c}$ tal que para toda  
$\bu\in H^1(T)^3$
\begin{IEEEeqnarray}{rCl}\label{rvpspanish}
  \|\bu-\br_{\sss k,T}\bu\|_{\sss L^2(T)^3}& \leqslant & C 
    \left\{\sum_{1\leqslant i\leqslant 3} h_i \|{\s\partial_{\xi_i}}\bu\|_{\sss L^2(T)^3}
	  + h_T \|\dv\bu\|_{\sss L^2(T)}\right\}
\end{IEEEeqnarray}
donde escribimos
$\xi_i$ para las coordenadas locales con origen
en el v\'ertice regular
y $h_i$ para las longitudes
de las aristas incidentes a \'el y 
$h_T$ para el di\'ametro de $T$.

Por otro lado, si $T\in\pazocal{MAC}(\bar\alpha)$ entonces
existe una $C>0$ que depende solamente de $\bar\alpha$
tal que 
para toda $\bu\in H^1(T)^3$
vale
\begin{IEEEeqnarray}{rCl}\label{macspanish}
  \|\bu-\br_{\sss k,T}\bu\|_{\sss L^2(T)^3}& \leqslant & C h_T \sum_{1\leqslant i\leqslant 3}
  \|{\s\partial_{\xi_i}}\bu\|_{\sss L^2(T)^3}
\end{IEEEeqnarray}
($\xi_i$ son coordenadas locales a partir de
un v\'ertice de $T$, pero
no necesariamente tenemos uno regular).

Observamos que la desigualdad~\eqref{macspanish} es estrictamente
m\'as d\'ebil que~\eqref{rvpspanish}, dado que hay elementos
que satisfacen la condici\'on $\pazocal{MAC}(\bar\alpha)$
para $\bar\alpha$ fijo, pero con par\'ametro $\pazocal{RVP}$
arbitrariamente peque\~no, haciendo que se degenere la
cantidad $C(\bar c)$ en~\eqref{rvpspanish}. Ponemos un ejemplo
en la Figura~\ref{tetraedrosSpanish}. Adem\'as, como est\'a
dicho en~\cite{aadl} mediante un contraejemplo, 
la desigualdad~\eqref{rvpspanish} no puede ser probada bajo
condici\'on de \'angulo m\'aximo solamente.

Volviendo al segundo p\'arrafo, lo que dec\'iamos es que 
es posible construir mallas graduadas anis\'otropas para
problemas el\'ipticos en dominios con singularidades que
consistan solamente de tetraedros que satisfacen 
condic\'on  $\pazocal{MAC}(\bar\alpha)$ para $\alpha<\pi$,
pero estos elementos no satisfacen la condici\'on de $\pazocal{RVP}(\bar c)$
para ning\'un par\'ametro uniforme positivo $\bar c$.
En consecuencia, la estimaci\'on 
del error de interpolaci\'on~\eqref{rvpspanish} no puede ser 
usada de manera global y en cambio~\eqref{macspanish} debe ser tomada,
y por esto las propiedades de anisotrop\'ia de las mallas
pueden no traer ninguna ventaja. En otras palabras,
en la segunda desigualdad una \emph{derivada grande} en
la direcci\'on $\xi_i$ no estar\'ia necesariamente compensada
por un $h_i$ peque\~no, as\'i que tendr\'iamos que hacer
peque\~no al di\'ametro $h_T$ y refinar las mallas en
todas las direcciones, que es lo que queremos evitar, y
perder\'iamos orden $\pazocal{O}(\textit{h})$ en la 
convergencia del error num\'erico con la relaci\'on 
asint\'otica $\textit{h}\sim N_{\Th}^{-\nicefrac13}$
 (aqu\'i $\Th$ es una malla y $N_{\Th}$ es su 
cardinal; el significado concreto meaning 
del par\'ametro de mallado $\textit{h}$, que no es
el di\'ametro de sus elementos, como s\'i lo es en el caso
de las mallas uniformes, se har\'a claro en 
la Subsecci\'on~\ref{meshes} cuando propongamos nuestro proceso
de mallado).

\tetsTikzSpanish

Una idea para sobrellevar la mencionada dificultad, para
el caso en que $\Omega$ es un dominio cil\'indrico 
poliedral, fue propuesta en~\cite{MR1866274}. En este caso,
cuando $f$ est\'a en $L^2(\Omega)$, la soluci\'on puede
exhibir solamente singularidades a lo largo de aristas
c\'oncavas. Los autores proponen un m\'etodo mixto de 
Raviart--Thomas en mallas graduadas de prismas triangulares
y probaron estimaciones \'optimas de error por medio de 
resultados de interpolaci\'on con anisotrop\'ia y de este modo
los tetraedros que no satisfacen una propiedad 
$\pazocal{RVP}$ uniforme son evitados. Tambi\'en proponen
un m\'etodo similar con mallas de tetraedros anis\'otropos
 graduados obtenido subdividiendo cada prisma entre tres
 tetraedros. Por supuesto, estas mallas contienen a los 
tetraedros malosm, y para obtener estimaciones
de error \'optimas el precio pagado es el de requerir m\'as
regularidad al dato $f$, m\'as precisamente, debe pertenecer
a un espacio de Sobolev con pesos.

Uno de los resultados presentados en esta Tesis extiende
los resultados de~\cite{MR1866274} puesto que nuestro
m\'etodo fue buscado con el prop\'osito de poder
lidiar con la aproximaci\'on mixta de~\eqref{mfSpanish}
para $f\in L^2(\Omega)$, y tambi\'en para un poliedro
cualquiera $\Omega$. Como estos dominios no necesariamente
admiten una partici\'on en t\'erminos de prismas rectos y como
tambi\'en nosotros quisi\'eramos evitar pedir
m\'as regularidad al dato $f$, proponemos una discretiza\-ci\'on
basada en mallas h\'ibridas consistentes en prismas combinados
con tetraedros, con brechas interelementales rellenadas
con pir\'amides. Obtenemos la estimaci\'on del error de
aproximaci\'on
\begin{equation}\label{auxlabel410}
 \|\bu-\bu_{\textit h}\|_{L^2(\Omega)^3} + \|p-p_{\textit h}\|_{L^2(\Omega)} 
 \leqslant C\,\textit{h}\|f\|_{L^2(\Omega)}{\mbox{,}}
\end{equation}
en donde $\bu_{\textit h}$ y $p_{\textit h}$ son las
aproximaciones de las soluciones $\bu$ y $p$, por medio
de estimaciones con normas de Sobolev con pesos 
(para $\bu$) tratando las singularidades de manera 
localizada. Primero introducimos y analizamos por completo
un nuevo
m\'etodo combinado de Elementos Finitos y Virtuales
en un poliedro cuando las mallas son construidas con
los mencionados poliedros elementales. Las estimaciones
de interpolaci\'on locales obtenidas 
para los espacios discretos son
\'optimas y anis\'otropas en prismas anis\'otropos, lo que 
nos permite obtener estimaciones \'optimas de error de
aproximaci\'on cuando la soluci\'on tiene tanto singularidades
de aristas como de v\'ertices, usando
mallas graduadas adecuadamente que incluyen
elementos anis\'otropos, y que solamente incluyen tetraedros
con un par\'ametro de condici\'on $\pazocal{RVP}$ uniforme positivo.

No solamente probamos el resultado para una familia abstracta
de tales mallas con las mencionadas propiedades,
 sino que tambi\'en mostramos un proceso general de mallado
 que comienza aislando y clasificando las singularidades
del dominio (cfr. Teorema~\ref{thm_regularity}), y
con eso probamos la existencia de una familia de mallas
como la requerimos, mediante su construcci\'on.

Los espacios discretos $V_{\textit h}$ correspondientes
a las mallas propuestas fueron obtenidos satisfaciendo
las siguientes condiciones, adecuadas para una discretiza\-ci\'on
en $\mathbb{R}^3$.
\begin{enumerate}
	\item \label{auxlabel411} Conformidad.
	\item \label{auxlabel412} Propiedades de aproximaci\'on anis\'otropas y \'optimas.
	\item \label{auxlabel413} Generalidad de dominio.
\end{enumerate}
Tal como es sugerido en~\cite{bfm} para el caso $2D$,
presentamos $V_{\textit h}$ como un espacio de elementos 
virtuales que coincide localmente con el espacio original
$3D$ de Raviart--Thomas en tetraedros y prismas, y lo
extiende de manera natural a elementos piramidales. 
En particular, las componentes normales de las
funciones discretas son constantes sobre las caras de
los elementos, coincidiendo de manera conforme a trav\'es 
de caras comunes a elementos de distinta geometr\'ia. De
este modo el requerimiento~\ref{auxlabel411} fue satisfecho.
Una ventaja de esta presentaci\'on es que la definici\'on
de los espacios locales es independiente de la geometr\'ia 
del elemento, ver Secci\'on~\ref{auxlabel290}. Como ya comentamos,
el \'item~\ref{auxlabel413} es verificado mediante la 
incorporaci\'on de pir\'amides de base paralelogramo.
Con respecto al punto~\ref{auxlabel412}, presentamos
estimaciones \'optimas de estabilidad y de interpolaci\'on
anis\'otropas en varias partes aunque no todas ellas
son usadas en la aplicaci\'on final, en algunos casos
porque los espacios funcionales para los cuales son usadas
no se ven involucrados en el problema 
modelo~\eqref{PSpanish} y en otros casos porque
nuestro esquema de mallado funciona sin requerir
que los elementos de todos los tipos presenten 
anisotrop\'ia.

Tambi\'en podr\'ian ser consideradas mallas con elementos
de geometr\'ia m\'as general. De hecho esta es una 
de las principales propiedades de los m\'etodos
de elementos virtuales. Nosotros decidimos
restringirnos a unas pocas geometr\'ias elementales
pues nuestro principal objetivo fue admitir 
mallas anis\'otropas, y con estimaciones anis\'otropas
uniformemente v\'alidas. Una dificultad que aparece
cuando se consideran otras geometr\'ias elementales 
(como prismas obl\'icuos) es que los espacios virtuales
locales no necesariamente se ve preservado 
por transformaciones \emph{push--forward} basadas 
en aplicaciones afines (nosotros 
usamos una propiedad de $\bcurl$ nulo que no
es invariante en esta situaci\'on), con
la consecuencia de que los argumentos de
reescalado est\'andar son dif\'iciles de usar
(ver Secci\'on~\ref{auxlabel290}). Sin embargo, esto no deviene
en una limitaci\'on en los dominios poliedrales que
podemos tratar con nuestro m\'etodo, pues podemos
restringirnos a usar prismas rectos duplicando, en el
peor de los casos, el n\'umero de elementos de la
primera malla, y as\'i aumentando el n\'umero de elementos
en una malla cualquiera posterior en un factor constante.

Nuestro m\'etodo presenta distinta cantidad de grados
de libertad a trav\'es de la malla, pues prismas, tetraedros
y pir\'amides tienen distinta cantidad de caras, y tambi\'en
admite distintas geometr\'ias para los grados de libertad pues
estamos trabajando con caras triangulares y cuadril\'ateras
al mismo tiempo. Como consecuencia de esto, nuestro m\'etodo
es en cierto sentido una generalizaci\'on de los
m\'etodos de elementos virtuales cl\'asicos, en los cuales todos
los elementos tienen la misma geometr\'ia arbitraria, y de 
cada elemento de toma el mismo n\'umero de grados de libertad, todos
del mismo tipo (por ejemplo, evaluaci\'on en los mismos puntos en
el borde de cada elemento).

Las mallas h\'ibridas que incluyen tetraedros y prismas (y hasta
hexaedros) pueden ser necesarias para satisfaces las demandas
de la geometr\'ia espec\'ifica de un problema (regiones no triviales)
o para alcanzar c\'alculos eficientes. Si se requiere que estas mallas
eviten nodos colgantes entonces a menudo incluir\'an pir\'amides; ver
por ejemplo~\cite{owenSaigal}. Varios art\'iculos contienen
la construcci\'on de elementos finitos conformes en pir\'amides,
algunos de los cuales son~\cite{bergot} 
para elementos $H^1$ y~\cite{gh99, Nigam-2012}
para elementos $H(\mbox{div})$ y $H({\bf curl})$,
el primero para orden bajo y el segundo para orden arbitrario.
En~\cite{Nigam-2012} los autores prueban que no es posible
construir elementos finitos $H^1$ \'utiles en pir\'amides
usando solamente funciones polinomiales y muestran que, 
en el caso $H(\mbox{div})$,
todos los espacios construidos en la literatura contienen
funciones no polinomiales.

Queremos finalizar esta introducci\'on mencionando otros
resultados que obtuvimos, presentados como resultados
adicionales porque no fueron usados para el problema principal 
de la Tesis, y que pueden ser vistos como extensiones 
de ciertos resultados te\'oricos.

Probamos estimaciones locales anis\'otropas 
de estabilidad y de error 
de interpolaci\'on para elementos finitos prism\'aticos
tanto para las clases de elementos 
$\bcurl$--conformes como para las div--conformes.
Nuestro m\'etodo usa solo el caso de orden bajo de las 
estimaciones en $H(\Div)$ y dejamos una versi\'on 
de orden alto del m\'etodo para investigaciones futuras.

Finalmente, obtuvimos y presentamos estimaciones locales anis\'otropas 
de estabilidad y de error 
de interpolaci\'on para los elementos finitos piramidales
introducidos en~\cite{gh99, Nigam-2012} 
tanto para las clases de elementos 
$\bcurl$--conformes como para las div--conformes.
Como mostramos en el Cap\'itulo~\ref{auxlabel202}, las funciones
de forma que generan estos espacios son racionales y singulares,
aunque acotadas, en la pir\'amide de referencia. Por esta 
raz\'on consideramos que nuestro m\'etodo combinado FE--VE
presenta una ventaja, la de evitar las evaluaciones de 
funciones con esas propiedades en implementaciones en
computadoras.
% section intro (end)