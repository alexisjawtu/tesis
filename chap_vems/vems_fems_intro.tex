\begin{chapter}[vems--fems]{vems--fems}
It is our intention to propose a Virtual Elements Method (VEM) scheme
as a generalization of $H(\text{div})$--conforming Finite Elements
in meshes consisting of polyhedra of arbitrary kind.
We deal with tetrahedra, triangular prisms and pyramids and, in the presence of
the latter, our VEM scheme is put in the framework of non-polynomial Finite Elements.\\[5pt]
Let be given an open \emph{non--convex} domain $\Omega\subseteq\mathbb{R}^3$ with
Lipschitz--continuous boundary
consisting of planar faces and define $V:=H(\mbox{div},\Omega)$ and $Q:=L^2(\Omega)$.
Let us consider the following continuous problem.\\[5pt]
With 
\begin{IEEEeqnarray*}{rClCrCl}
	a(\bv,\bw) & = & \forma{v}{w} &\quad\mbox{and}\quad& b(\bv,q) & = & \formb{v}{q}
\end{IEEEeqnarray*}
find $\bu\in V$ and $p\in Q$ such that for every $\bv\in V$ and every $q\in Q$
\begin{equation}\label{mixedContinuous}
  \makebox[0pt]{
    \begin{minipage}{\linewidth}
  	  \begin{IEEEeqnarray*}{rCl}
  		a(\bu,\bv) + b(\bv,p) & = & 0\\[5pt]
  				   - b(\bu,q) & = & (f,q).
  	  \end{IEEEeqnarray*}
    \end{minipage}}
  \tag{MC}
\end{equation}
We start with a polihedral triangulation $\Th$ of $\Omega$ to define the 
virtual spaces $V_h$ and $Q_h$ as discretizations of $V$ and $Q$ respectively.\\[5pt]
For $E\in\Th$ the local space of vector fields will be
\begin{eqnarray*}
  V_h(E)&=&\Big\{\bv\in H(\mbox{div},E)\cap H(\textbf{curl},E)\,:\,\\
  &&\qquad \bv\cdot\bn|_f\in \mathcal P_0(f) \,\,\mbox{for all face $f$ of }E, \\
  && \qquad\dv\bv\in \mathcal P_0(E) \mbox{ and } \curl\bv = 0 \Big\}
\end{eqnarray*}
and the  global space $V_h$ will consist of functions defined piecewise with the former
local spaces:
\begin{IEEEeqnarray*}{ccrCl}
V_h&=&V_h(\Th)&:=&\Big\{\bv\in H(\dv,\Omega): \bv|_E\in V_h(E), \mbox{for all element }
E\in\Th\Big\}
\end{IEEEeqnarray*}
The scalar discrete space we will consider is
\begin{IEEEeqnarray}{rCl}
  Q_h & = & \mathcal{P}_0(\Th)
\end{IEEEeqnarray}
meaning the functions that constant on each element of $\Th$.

The condition in the definitions of these spaces suffice to construct an interpolation
operator, which is a key object both in Finite Elements as in Virtual Elements.
Let us take the degrees of freedom
\begin{IEEEeqnarray}{rCl}\label{dofs}
  \int\limits_f \bv\cdot\bn\,d\gamma & \qquad\mbox{ for all face $f$ of } & \Th.
\end{IEEEeqnarray}
With \emph{faces of $\Th$} we mean the family of all faces forming the boundary
consistently in the case of neighbour elements.
\begin{lemma}\label{unisolvence} Given a polyhedron $E\in\Th$, the degrees
  of freedom~(\ref{dofs}) corresponding to the faces of $E$ are unisolvent
  in $V_h(E)$.
\end{lemma}
\begin{proof} \emph{Existence.} Let $n_{f,E}$ the number of faces of $E$ and
take real numbers $\{\alpha_i\}_{i=1}^{n_{f,E}}$. Let $g$ the  piecewise constant
function on $\partial E$ satisfying, for all $i$, %face $f$ of $E$
\[
  \int\limits_{f_i} g = \alpha_i
\]
and let $d$ be the constant function in $E$ such that
\[
 \int\limits_E d\,d\textbf{x} = \sum \alpha_i.
\]
Then we consider the auxiliary problem of seeking a solution of
\begin{IEEEeqnarray}{rClrCl}
  \label{aux_prob}
  \Delta \phi & = & d \quad \mbox{in $E$,} \qquad & 
  \frac{\partial \phi}{\partial \bn}& = &g \quad \mbox{on }\partial E.
\end{IEEEeqnarray}
{\color{red}By definition we obtain the compatibility condition}
\begin{IEEEeqnarray*}{rCl}
  \int\limits_E d\,d\textbf{x}& = & \int\limits_{\partial E} g\,d\gamma    
\end{IEEEeqnarray*}
so the solution $\phi$ to the problem~(\ref{aux_prob}) exists. Now
we take $\bu:=\nabla \phi$ and  it holds immediately $\dv\bu$ is constant in $E$,
$\bu\cdot\bn$ is constant on each face of $\partial E$ and $\curl\bu = 0$. So
$\bu$ lies in $V_h(E)$ and also for all $1\leqslant i\leqslant n_{f,E}$ $\int_{f_i} \bu\cdot\bn\,d\gamma = \alpha_i$.\\[4pt]
\emph{Uniqueness.} Suppose that $\bv\in V_h(E)$ has vanishing
degrees of freedom. Condition $\curl\bv=0$ implies
$\bv=\nabla\phi$ for certain $\phi$. Now, since $\dv\bv$ is constant on $E$, the
relations 
\begin{IEEEeqnarray*}{rCl}
   0 & = & \int\limits_{\partial P}\bv\cdot\bn\,d\gamma 
\end{IEEEeqnarray*} %& = & \int\limits_P\dv\bv\,d\textbf{x}
imply $\dv\bv=0$ by Green Theorem. Then, the potential $\phi$ satisfies
\begin{IEEEeqnarray*}{rClrCl}
  \Delta \phi & = & 0 \quad \mbox{in $E$,} \qquad & 
  \frac{\partial \phi}{\partial \bn}& = &0 \quad \mbox{on }\partial E
\end{IEEEeqnarray*}
{\color{red} which means it is a constant}, and it follows $\bv=0$.
\end{proof}
With this Lemma already {\color{red}proven} we are able to consider
an $H(\text{div})$--like local interpolation operator well defined.
\begin{corollary}
  For every $\bv\in H^1(E)^3$ there exists a $V_h(E)$--interpolant $\bv_I$
  defined as the unique element in $V_h(E)$ such that for every face $f$ of $E$
    \begin{IEEEeqnarray*}{rCl}
      \int\limits_f \bv_I\cdot\bn\,d\gamma & = & \int\limits_f \bv\cdot\bn\,d\gamma       
    \end{IEEEeqnarray*}
\end{corollary}
\begin{lemma} Consider the projection $P_0$ onto the constants on $E$. It holds
\begin{IEEEeqnarray*}{rCl}
  \dv\bv_I & = & P_0\,\dv\bv.
\end{IEEEeqnarray*}
\end{lemma}
\noindent{\color{blue}\#\#\#\#\#\#\# desde ac\'a est\'a sin controlar} 
\begin{theorem}
The solutions $\bu$ and $p$ of problem ~(\ref{mixedContinuous}) satisfy
\[
p\in H^1(\Omega)
\] 
and for each $\ell$
\[
\bu=\bu_r + \bu_s
\]
with $\bu_r\in H^1(\Omega)$ and
\[
\bu_s\cdot \xi_i\in V^{1,2}_{\beta,\delta}(\Lambda_\ell), \quad i=1,2, \qquad \bu_s\cdot\xi_3\in V^{1,2}_{\beta,0}(\Lambda_\ell)
\]
where $\xi_i$, $i=1,2,3$, are the directions of three concurrent edges of $\Lambda_\ell$ with $\xi_3$ being the direction of the singular edge if it exists in $\Omega_\ell$, and $\beta,\delta\ge0$ satisfying $\beta>\frac12-\lambda_v^{(\ell)}$ and $\delta>1-\lambda_e^{(\ell)}$, $v$ and $e$ being the singular vertex and edge, respectively, {\color{red}mas lindo}if they exist.
\end{theorem} 
\paragraph{discretized forms} % (fold)
\label{par:discretized_forms}
\[
\noindent{\color{blue}\#\#\#\#\#\#\# \mbox{¿hace falta?}} 
  a(\bv,\bw)=\sum_{E\in\mathcal T_h}a^E(\bv,\bw), \qquad
  b(\bv,q)=\sum_{E\in\mathcal T_h}b^E(\bv,q).
\]
Next, together with the finite dimensional spaces already defined,
 discretized version of the bilinear forms.\\
The evaluation of the form $b(\cdot,\cdot)$ in $(\bv,q)\in V_h\times Q_h$ can
be computed using the degrees of freedom~(\ref{dofs}) applied to $\bv$. For if $q\in Q_h$, then
\[
  b(\bv,q)=\int_\Omega q\,\dv\bv = \sum_{E\in\mathcal T_h}
  \int_Eq\,\dv\bv = \sum_{E\in\mathcal T_h}\int_{\partial E}q \bv\cdot\bn.
\]
so in the case of the form $b_h(\bv,q)$ we put simply
\[
  b_h(\bv,q) = b(\bv,q) =
  \sum_{E\in\mathcal T_h}b^E(\bv,q)\mbox{,}
\]
For the bilinear form $a(,)$, we can decompose it as
\[
  a(\bv,\bw)=\sum_{E\in\mathcal T_h}a^E(\bv,\bw)
\]
but this is not sufficient \noindent{\color{blue}\#\#\#\#\#\#\# por que?}. The following construction
is done in order to stabilize the oper. 
\[
\hat V(E) = \left\{\hat \bv\in V_h(E): \hat \bv = \nabla \hat q_2,
\mbox{ for some } \hat q_2\in \mathcal P_2(E)\right\} = V_h(E)\cap\nabla\mathcal{P}_2(E).
\]
By an inspection of the Finite Elements defined in Chapter~\ref{chap_prelim}
we obtain immediately the following
\begin{remark}\label{vem_equal_fem} If $E$ is a tetrahedron or a prism, then $V_h(E) = \hat V(E) = RT_0(E)$
(cfr.~(\ref{sub:definition_of_the_h_div_element_on_prisms})
and~(\ref{sub:definition_of_the_h_div_element_on_tetrahedra}) both
in the lower case).
\end{remark}
Again, if $\bv\in V_h(E)$ and $\hat\bu\in\hat V(E)$, $a^E(\hat\bu,\bv)$ can be 
computed using the degrees of freedom of $\bv$, because it holds
\begin{IEEEeqnarray*}{rCCCl}
a^E(\hat\bu,\bv) &=& \int_E\hat\bu\cdot\bv &=& \int_E\nabla \hat q_2\cdot\bv\\
                 &=&-\int_E\hat q_2 \dv\bv &+& \int_{\partial E}\hat q_2\bv\cdot\bn.
\end{IEEEeqnarray*}
This means we can consider an auxiliary projection  operator $\hat\pi^E$
from $H(\dv,E)$ onto $\hat V(E)$ defined by
\begin{IEEEeqnarray}{rCl}\label{projection}
  a^E(\bv-\hat\pi^E\bv,\hat\bw) = 0 &\qquad& \mbox{for all }\hat\bw\in\hat V(E)
\end{IEEEeqnarray}
fully computably in terms of the degrees of freedom of $\bv$.\\[5pt]
One fact to note
about the projection $\hat\pi$ in~(\ref{projection}) is that, by Remark~(\ref{vem_equal_fem}),
it coincides with the identity $I\,:\,\hat V(E)\to\hat V(E)$ whenever the element $E\in\Th$
is a Prism or a Tetrahedron, so that, immediately
\begin{remark}\label{ah_equal_a} If $E$ is a tetrahedron or a prism, then
  $a_h^E(\bv,\bw)=a^E(\bv,\bw)$ for all $\bv,\bw \in V_h(E)$.
\end{remark}
We need the following last object to complete the discretization of $a(\cdot,\cdot)$.
Thanks to Lemma~(\ref{unisolvence}) we can consider a basis $B=\{\bv_{i,E}\}$
of $V_h(E)$ dual to the functionals~(\ref{dofs}), that is, for every
$1\leqslant i,j\leqslant n_{f,E}$
\begin{IEEEeqnarray}{rCl}
  \int\limits_{f_j} \bv_{i,E}\cdot\bn\,d\gamma & = & \delta_{i,j}.
\end{IEEEeqnarray}
What we are considering is the inner product
$\mathcal S^E(\bv,\bw)$ of the coordinates
of the fields in $V_h(E)\times V_h(E)$ with respect 
to this local dual basis $B$.\\
An important thing about $\mathcal S^E$:
\begin{remark}\label{equivalence} For all $\bv\in V_h(E)$ and all $E\in\mathcal T_h$
  \[
    S^E(\bv,\bv)\sim c_E a^E(\bv,\bv)
  \]
and $c_E$ depends only on the shape regularity of $E$.
\noindent{\color{blue}\#\#\#\#\#\#\# lo siguiente un poco mejor?} 
In particular, {\color{blue}if the elements belong to a regular
family, then $c_E=O(1)$.}
\end{remark}

Finally, the local discrete form $a_h^E$ is defined by the following equation:
\[
  a^E_h(\bv,\bw) := a^E(\hat\pi^E\bv,\hat\pi^E\bw) + 
  \mathcal S^E((I-\hat\pi^E)\bv,(I-\hat\pi^E)\bw)
\]
for $\bv$ and $\bw$ in $V_h(E)$. Now that we have the local forms, the discrete global form $a_h$ is,
as always,
\[
  a_h(\bv,\bw) = \sum_{E\in \mathcal T_h} a_h^E(\bv|_E,\bw|_E),
    \quad\mbox{ for } \bv \mbox{ and } \bw \in V_h.
\]
\paragraph{discrete $\inf--\sup$ condition.} % (fold)
\label{par:discrete_inf_sup}
\begin{lemma} For all $E\in\Th$, all $\hat\bu\in\hat V(E)$ and all $\bv\in V_h(E)$
\begin{IEEEeqnarray}{rCcCl} 
a_h^E(\hat\bu,\bv) & = & a^E(\hat\bu,\bv)       & &\label{L1}\\
ca^E(\bv,\bv)      & \leqslant & a_h^E(\bv,\bv) & \leqslant & Ca^E(\bv,\bv)\label{L2}
\end{IEEEeqnarray}
\end{lemma}
\begin{proof} The relation $\hat\bu=\hat\pi^E\bu$, condition~(\ref{projection})
and the symmetry of $a^E$ imply
\[
  a_h^E(\hat\bu,\bv)=a^E(\hat\pi^E\hat\bu,\hat\pi^E\bv)=a^E(\hat\pi^E\hat\bu,\bv)=a^E(\hat\bu,\bv)
\]
which proves~(\ref{L1}).
To prove~(\ref{L2}), if $E$ is not a Pyramid this property is already a consequence
of~(\ref{ah_equal_a}). In the case of a Pyramid, firstly a simple computation yields
\begin{IEEEeqnarray}{rCcCl}
  \label{comput}
  a^E(\bv,\bv)&=&a^E(\bv-\hat\pi^E\bv,\bv-\hat\pi^E\bv)&+&a^E(\hat\pi^E\bv,\hat\pi^E\bv).
\end{IEEEeqnarray}
Remark~(\ref{equivalence}) and~(\ref{comput}) give
\begin{IEEEeqnarray*}{rCl}
a_h^E(\bv,\bv) &\leqslant& a^E(\hat\pi\bv,\hat\pi\bv) + Ca^E(\bv-\hat\pi\bv,\bv-\hat\pi\bv) \\[5pt]
               &\leqslant& \max\{1,C\}a^E(\bv,\bv).
\end{IEEEeqnarray*}
In a similar manner, it holds
\[
  a_h^E(\bv,\bv)\geqslant C a^E(\bv,\bv).
\]
\end{proof}
\noindent If we consider the following kernel
\begin{IEEEeqnarray*}{rCl}
  \mathcal K_k & = & \{\bv_h\in V_h: b(\bv_h,q)=0\,\,\forall q\in Q_h\} \\[4pt]
               & = & \{\bv_h\in V_h: \dv\bv_h=0\}
\end{IEEEeqnarray*}
then the previuos Lemma implies the following corollary.
\begin{corollary}
  $a_h$ is coercitive over $\mathcal K_k$ and the coercivity constant is independent
  of $h$.  
\end{corollary}
\begin{lemma} There exists a constant $\beta^*>0$ depending only on $\Omega$
such that for all $q^*\in Q_h$ there exists $\bw_h^*\in V_h$ such
that
\[
\dv\bw_h^*=q^*, \qquad \beta^*\|\bw_h^*\|_{L^2(\Omega)}\leqslant\|q^*\|_{L^2(\Omega)}
\]
\end{lemma}
\begin{proof} Start considering the infinite dimensional version
of the statement. There is, in fact, a constant $\beta^*$ such that
for each every $q\in L^2(\Omega)$ there exists 
$\bw^*\in [H^1_0]^3$ with $\dv\bw^*=q$ such that
\[
  \beta^*\|\bw^*\|_{H^1}\leqslant \|q\|_{L^2(\Omega)}.
\]

{\color{blue} HERE }

Now, take $\bw_h^*=\bw^*_I$ the $V_h$-interpolant of $\bw^*$. Then
we see that
\[
\|\bw_I^*\|_{L^2}\le
(1+Ch)\|\bw^*\|_{H^1}\le\frac{1+Ch}{\beta^*}\|q\|_{L^2},
\]
where we used stability estimates for $RT_0$ in the case of
prismatic and tetrahedral (possibly anisotropic) elements and the
stability estimate for VEM on pyramids.

On the other hand, since $q\in Q_h$,
\[
\dv\bw_I^*=P_0\dv\bw^*=P_0q=q.
\]
\end{proof}

% paragraph discrete_inf_sup (end)
\end{chapter}