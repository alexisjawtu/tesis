\begin{chapter}[Virtual Elements]{Virtual Elements}
\section*{Introducci\'on al cap\'itulo}
Los Elementos Virtuales son definidos en t\'erminos de un dado problema. 
Es nestra intenci\'on
proponer un esquema para un m\'etodo combinado 
de elementos Finitos/Virtuales
como una generalizaci\'on de los Elementos Finitos $H(\Div)$--conformes
en mallas consistentes en
poliedros de geome\-tr\'ia arbitraria. En este
Cap\'itulo presentamos un desarrollo de los espacios, las formas
bilineales y las propiedades concernientes a los  
elementos virtuales.
Trabajamos con tetraedros, prismas triangulares y pir\'amides y,
en presencia de estas \'ultimas,
nuestro esquema FEM/VEM se ubica en
el marco de los Elementos Finitos no--polinomiales.

El m\'etodo de elementos virtuales (VEM) 
fue introducido recientemente~\cite{MR2997471}
como una generalizaci\'on de elementos $H^1$--conformes 
para elementos de geome\-tr\'ia arbitraria y como una
generalizaci\'on de las Diferencias Finitas Mim\'eticas
a grado arbitrario de precisi\'on. Una extensi\'on de 
la discretizaci\'on de campos vectoriales $H(\Div)$--conformes
y aproximaciones por elementos finitos mixtos
fue propuesta en~\cite{bfm} en dimensi\'on 2. 
Adem\'as, en~\cite{MR3507271} un VEM mixto
fue analizado
para la aproximaci\'on de 
problemas lineales el\'ipticos con coeficientes variables.

Un espacio de elementos virtuales puede contener funciones
no \emph{polinomiales a trozos}
y principalmente contiene funciones
que son desconocidas y no pueden ser evaluadas en ning\'un punto.
En el VEM los espacios y los grados de libertad son tomados de tal manera
que la matriz de rigidez elemental puede ser computada
sin conocer estas funciones no polinomiales, en nuestro caso gracias 
al Teorema de la Divergencia.

\section*{Introduction to the chapter}
The Virtual Elements are customarly defined in terms of a given problem. Our
intention 
is to propose a combined Finite/Virtual Elements Method scheme
as a generalization of $H(\Div)$--conforming Finite Elements
in meshes consisting of polyhedra of arbitrary kind. In this Chapter we develop
the spaces, bilinear forms and properties concerning 
virtual elements.
We deal with tetrahedra, triangular prisms and pyramids and, in the presence of
the latter, our FEM/VEM scheme is put in the framework of 
non--polynomial Finite Elements.

The virtual element method (VEM) has been recently introduced \cite{MR2997471} 
as a generalization of $H^1$-conforming finite elements to arbitrary 
element--geometry 
and as a generalization of Mimetic Finite Differences to arbitrary degree 
of
accuracy and arbitrary continuity properties. An extension to 
the discretization of $H(\Div)$--conforming vector fields and mixed finite 
element 
approximations has been proposed in~\cite{bfm} 
in the two dimensional case. Furthermore, in \cite{MR3507271} a mixed VEM 
has been
analysed for the approximation of general linear elliptic problems with variable
coefficients.  
The virtual element space can contain non piecewise polynomial
functions and mainly functions which are a priori unknown, 
in the sense that they
can't be evaluated in any point. In the VEM approach, the space and the degrees
of freedom are taken in such a way that the elementary stiffness matrix can be 
computed without actually computing these non--polynomial functions, but just using 
the degrees of freedom. In this respect, a key
point in this approach and particularly in our case is that, 
given an element $E$, if $\bu=\nabla q_2$ for a known 
polynomial $q_2$, then for a field $\bv$ the quantity 
\[
\int_E\bu\cdot\bv\,d\bx
\]
can be computed if $\mbox{div\,}\bv$ on $E$ and the outer normal 
component $\bv\cdot{\bf n}$ of $\bv$ on 
$\partial E$ are known polynomials, by using the divergence Theorem.
%In Section~\ref{auxlabel290} the discrete spaces for the discrete mixed 
%formulation are defined together with the 
%variational discrete forms. The discrete form, $a_h$, requires of projections 
%onto subspaces of the local discrete 
%fields. Those projections become the identity in the case of tetrahedral or 
%prismatic elements, 
%but for pyramids, they require some additional analysis. 
%In Section \ref{liee} the interpolation on the virtual element spaces is 
%considered and 
%interpolation error estimates are proved under different shape assumptions. Also a 
%discrete $\inf--\sup$ property is proved, which is used in Section \ref{appr err} to 
%give an abstract approximation error result. 
\section{Definition and Construction}
\label{auxlabel290}
First we recall Problem~\ref{weakMixedContinuous} and add some notation for the 
bilinear forms.
Let  $V:=H(\Div, \Omega)$, $Q:=L^2(\Omega)$ and
\begin{IEEEeqnarray*}{rClCrCl}
  a(\bv,\bw) & = & \forma{v}{w} &\quad\mbox{and}\quad& b(\bv,q) & = & \formb{v}{q}.
\end{IEEEeqnarray*}
Then Problem~\ref{weakMixedContinuous} reads to
find $\bu\in V$ and $p\in Q$ such that for every $\bv\in V$ and every $q\in Q$
\begin{IEEEeqnarray*}{rCcCl}                          % Let be given an open \emph{non--convex} domain $\Omega\subseteq\mathbb{R}^3$ with
  a(\bu,\bv) & + &b(\bv,p) & = & 0\\[5pt]                % Lipschitz--continuous boundary
	       	   &   &b(\bu,q) & = & (-f,q).                     % consisting of planar faces and define 
\end{IEEEeqnarray*}                                   % Let us consider the following continuous problem.\\[5pt]
We start supposing we are given with a conforming polihedral triangulation $\Th$ of $\Omega$ to define the 
virtual spaces $V_h$ and $Q_h$ as discretizations of $V$ and $Q$ respectively. For now 
we will suppose the aspect ratios of tetrahedra and pyramids are bounded
in terms of a constant independent of $\textit{h}$. However, in Chapter~\ref{auxLabel100}
we will construct a mesh using the three types of polihedra mentioned before, and
we will see that the last assumtion is not a restriction.
We will assume a strictly decreasing parameter $\textit{h}$ tending to zero 
to mitigate the abuse of notation present in expressions like 
``$\bu_{\textit{h}}$'' and ``$\Th$''.
Then we are able to write, for example, $\bu_{\textit{h}}$ or 
$\bu_{\Th}$ indistinctly.

For $E\in\Th$ the local space of virtual vector fields will be
\begin{IEEEeqnarray*}{rCl}
  V_h(E)& = &\Big\{\bv\in H(\Div,E)\cap H(\bcurl,E)\,:\,\\
  \yesnumber\label{vhE}
   & & \qquad \bv\cdot\bn|_f\in P_0(f) \,\,\mbox{for all face $f$ of }E, \\
   & & \qquad \dv\bv  \in P_0(E) \mbox{ and } \curl\bv = 0 \Big\}
\end{IEEEeqnarray*}
and the  global space $V_h$ will consist of functions defined piecewise with the former
local spaces:
\begin{IEEEeqnarray*}{ccrCl}
  V_h & = & V_h(\Th) & := & \Big\{\bv\in H(\Div,\Omega): \bv|_E\in V_h(E),
  \mbox{for all element } E\in\Th\Big\}.
\end{IEEEeqnarray*}
The  condition on the $\curl$ is put because we are
considering gradients, as we will see later.
The scalar discrete space we will consider is
\begin{IEEEeqnarray}{rCl}
  Q_h & = & {P}_0(\Th)
\end{IEEEeqnarray}
meaning the functions that are constant on each element of $\Th$. As expected,
with $V_h$ we consider the $H(\Div,\Omega)$ norm, 
$\|\bv\|^2_{V_h} = \|\bv\|^2_{L^2(\Omega)} + \|\dv\bv\|^2_{L^2(\Omega)}$,
and with $Q_h$ we consider the $L^2(\Omega)$ norm.
As Problem~\ref{weakMixedContinuous}  has solution in $H(\Div,\Omega)$
it suggests to use a face--like  operator as virtual interpolator. For that reason,
let the degrees of freedom be
\begin{IEEEeqnarray}{rCl}\label{dofs}
  \iint_f \bv\cdot\bn\,dS & \qquad\mbox{ for all face $f$ of } & \Th.
\end{IEEEeqnarray}
With \emph{faces of $\Th$} we mean the family of all faces forming the boundary
of the elements, with $\bn$ being a unit normal vector, chosen for each face among the
two possibilities in the case of
neighboring elements. 

Something that has been already established for Finite Elements in
Chapter~\ref{aux_label43} must be proved for our Virtual Elements.
\begin{lemma}\label{unisolvence} Given a polyhedron $E\in\Th$, the degrees
  of freedom~(\ref{dofs}) corresponding to the faces of $E$ are unisolvent
  in $V_h(E)$.
\end{lemma}
\begin{proof} \emph{Existence.} Let $n_{f,E}$ be the number of faces of $E$ and
take real numbers $\{\alpha_i\}_{i=1}^{n_{f,E}}$. Let $g$ be the  piecewise constant
function on $\partial E$ satisfying, for all $1\leqslant i\leqslant n_{f,E}$, %face $f$ of $E$
\[
  \iint_{f_i} g\,dS = \alpha_i
\]
and let $d$ be the constant function in $E$ such that
\[
 \int_E d\,d\bx = \sum_i \alpha_i.
\]
As this proof is performed locally on a fixed element $E$, we can fix the degrees
of freedom in~\eqref{dofs} so that the normal vector points outward.
Then we consider the auxiliary problem of seeking a solution of
\begin{IEEEeqnarray}{rClrCl}
  \label{aux_prob}
  \Delta \phi & = & d \quad \mbox{in $E$,} \qquad & 
  \frac{\partial \phi}{\partial \bn}& = &g \quad \mbox{on }\partial E.
\end{IEEEeqnarray}
By definition we obtain the compatibility condition
\begin{IEEEeqnarray*}{rCl}
   \int_E d\,d\bx& = & \iint_{\partial E} g\,dS
\end{IEEEeqnarray*}
so the solution $\phi$ to the problem~(\ref{aux_prob}) exists. Now
we take $\bu:=\nabla \phi$ and  it holds immediately that $\dv\bu$ is constant in $E$,
$\bu\cdot\bn$ is constant on each face of $\partial E$ and $\curl\bu = 0$. So
$\bu$ lies in $V_h(E)$ and also for all $1\leqslant i\leqslant n_{f,E}$ 
$\iint_{f_i} \bu\cdot\bn\,dS = \alpha_i$.\\[4pt]
\emph{Uniqueness.} Suppose that $\bv\in V_h(E)$ has vanishing
degrees of freedom. Condition $\curl\bv=0$ implies
$\bv=\nabla\phi$ for certain $\phi$. Now, since $\dv\bv$ is constant on $E$, the
relation
\begin{IEEEeqnarray*}{rCl}
   0 & = & \int_{\partial P}\bv\cdot\bn\,dS
\end{IEEEeqnarray*} %& = & \int\limit s_P\dv\bv\,d\bx
implies $\dv\bv=0$ by Green Theorem. Then, the potential $\phi$ satisfies
\begin{IEEEeqnarray*}{rClrCl}
  \Delta \phi & = & 0 \quad \mbox{in $E$,} \qquad & 
  \frac{\partial \phi}{\partial \bn}& = &0 \quad \mbox{on }\partial E
\end{IEEEeqnarray*}
which means it is a constant, and it follows $\bv=0$.
\end{proof}
With this Lemma already demonstrated we are able to consider
an $H(\Div)$--like local interpolation operator well defined.
\begin{corollary} \label{interpolant}
  For every $\bv\in H^1(E)^3$ there exists a $V_h(E)$--interpolant $I\bv$
  defined as the unique element in $V_h(E)$ such that for every face $f$ of $E$
    \begin{IEEEeqnarray*}{rCl}
      \iint_f I\bv\cdot\bn\,dS & = & \iint_f \bv\cdot\bn\,dS.       
    \end{IEEEeqnarray*}
\end{corollary}
\begin{lemma} \label{p0_projection} Consider the projection $P_0$ onto the constants on $E$. It holds
\begin{IEEEeqnarray*}{rCl}
  \dv I\bv & = & P_0\,\dv\bv.
\end{IEEEeqnarray*}
\begin{proof}
  \begin{IEEEeqnarray*}{rCl}
    \int_E(\dv I\bv - \dv\bv)\,d\bx& = & \int_E\dv (I\bv - \bv)\,d\bx\\
    & = & \iint_{\partial E}(I\bv - \bv)\cdot\bn\,dS\\
    & = & \sum_{f\subseteq \partial E} \iint_{f}(I\bv - \bv)\cdot\bn\,dS = 0
  \end{IEEEeqnarray*}
\end{proof}
\end{lemma}
\begin{proposition}\label{vem_equal_fem}
Recall the space $D_k$ introduced in~(\ref{dk}). Then
\begin{enumerate}
  \item 
if $E$ is a right prism, we have
\begin{IEEEeqnarray}{rCl}\label{d1p1}
  V_{\textit{h}}(E) & = & D_1(x,y)\times P_1(z)\mbox{,}
\end{IEEEeqnarray}
with $x,y,z$ being the variables in a cartesian system of coordinates in which
the $z$--axis is orthogonal to the planes containing the triangular faces of $E$.
  \item 
If $E$ is a tetrahedron, we have
\begin{IEEEeqnarray}{rCl}\label{p03}
  V_{\textit{h}}(E) & = & P_0^3(\bx) + P_0\bx\mbox{,}
\end{IEEEeqnarray}
with $\bx = (x,y,z)'$ being the vector of variables in a cartesian system of
coordinates.
\end{enumerate}
\end{proposition}
\begin{proof}
  In the case of a prism $E$, the space $D_1(x,y)\otimes P_1(z)$ can be written as
  \[
    \{\bv \in P_1(E)\,:\, \bv=(a+\gamma x,b+\gamma y,c+d z),
        a,b,c,d,\gamma\in\mathbb{R})\}
  \]
  and then we see immediately that it has the same dimension 
  as $V_{\textit{h}}(E)$ which has dimension five. Finally, recalling that 
  the $\dv$ and $\curl$ operators
  can be computed in the chosen local variables, given $\bv\in D_1(x,y)\otimes P_1(z)$
  we can compute $\bv|_{f}\cdot\bn_f$ (on each face $f$ of $E$), $\dv\bv$ and $\curl\bv$
  to verify explicitly that the space on the right side of~(\ref{d1p1}) fulfills the 
  definition of $V_h(E)$ given in~(\ref{vhE}).
  Exactlty the same argument works for the case of a tetrahedron $E$, in which we
  deal with two vectorial spaces of dimension four that are such that 
  every element in
  \[
    P_0^3(\bx) + P_0\bx=\{\bv \in P_1(E)\,:\,
    \bv=(a+\gamma x,b+\gamma y,c+\gamma z)',
    a,b,c,\gamma\in\mathbb{R}\}
  \]
  fulfills the conditions defining $V_{\textit{h}}(E)$.
\end{proof}
\begin{remark}
  By Proposition~\ref{vem_equal_fem} the spaces $V_h(E)$ coincide with the
  lowest order $H(\Div)$--conforming local Finite Element spaces
  introduced in~(\ref{prismaticSpace})
  for the prismatic case and in~(\ref{tetrahedralSpace}) for the tetrahedral case.
\end{remark}
\begin{lemma}
  The definition of the lowest order $H(\Div)$ Finite Elements on
  right prisms or tetrahedra is independent of
the choice of the cartesian axes (as long as the $z$-axis is
perpendicular to the trianguar faces in case of prisms).
\end{lemma}
\begin{proof}
  This can be proved by hand. Let us prove the case of a prism $E$. Let 
  $(x,y,z)'$ and $(x',y',z')'$ be two cartesian coordinate systems satisfying the
  required properties. For them there is a change of coordinates such as
  \begin{IEEEeqnarray*}{rCl}
    x & = & p+\alpha x'-\beta  y' \\
    y & = & q+\beta x' -\alpha y' \\
    z & = & r+z'
  \end{IEEEeqnarray*}
for $\alpha$ and $\beta$ satisfying $\alpha^2+\beta^2 = 1$.
Let 
$\bv$ be an element in $V_{\textit{h}}(E)$. So
\begin{IEEEeqnarray*}{rCl}
\bv(x,y,z) & = & (a+\gamma x,b+\gamma y,c+d z)'\\
&=&((a+\gamma p) + \gamma(\alpha x' - \beta y'),\\
&& \,\,        (b+\gamma q) + \gamma(\beta x'+\alpha y'),\\
 && \,\,       (c+dr)+dz')'.
\end{IEEEeqnarray*}
Then, the components of $\bv$ in the new coordinate versors are
\begin{IEEEeqnarray*}{c}
 \begin{cases}
  \begin{IEEEeqnarraybox*}{lClCl}
    \bv\cdot(\alpha,\beta,0)' & = &
     (\alpha a + \beta b + \gamma(\alpha p + \beta q )) +\gamma x' & =: & a' + \gamma x' \\
    \bv\cdot(-\beta,\alpha,0)' & = &
     (-\beta a + \alpha b + \gamma(-\beta p + \alpha q )) +\gamma y' & =: & b' + \gamma y' \\
    \bv\cdot(0,0,1)' & = &
     (c+dr)+d z' & =: & c' + dz'.
  \end{IEEEeqnarraybox*}
 \end{cases}
\end{IEEEeqnarray*}
It follows that, in the $x'y'z'$ system,
\[
  \bv(x',y',z') = (a' + \gamma x', b' + \gamma y', c' + dz')'
  \in D_1(x',y')\otimes P_1(z').
\]
\end{proof}
%\paragraph{discretized forms} % (fold)
%\label{par:discretized_forms}
%\[
%\noindent{\color{blue}\#\#\#\#\#\#\# \mbox{¿hace falta?}} 
%  a(\bv,\bw)=\sum_{E\in\Th}a^E(\bv,\bw), \qquad
%  b(\bv,q)=\sum_{E\in\Th}b^E(\bv,q).
%\]
Next, together with the finite dimensional spaces already defined,
we present discretized version of the bilinear forms. The evaluation of
the form $b(\cdot,\cdot)$ at a pair $(\bv,q)\in V_h\times Q_h$ can
be computed using the degrees of freedom~(\ref{dofs}) applied to $\bv$. For if $q\in Q_h$, then
\[
  b(\bv,q)=\int_\Omega q\,\dv\bv\,d\bx = \sum_{E\in\Th}
  \int_E q\,\dv\bv\,d\bx = \sum_{E\in\Th}\iint_{\partial E}q \bv\cdot\bn\,dS.
\]
So in the case of the form $b_h(\cdot,\cdot)$ we put simply
\begin{IEEEeqnarray}{rCcCl}\label{aux_label44}
  b_h(\bv,q) & = & b(\bv,q) & = &
  \sum_{E\in\Th}b^E(\bv,q)\mbox{,}
\end{IEEEeqnarray}
where $b^E(\bv,q) = \int_E q\,\dv\bv\,d\bx$.
For the bilinear form $a(\cdot,\cdot)$, we can decompose it as
\[
  a(\bv,\bw)=\sum_{E\in\Th}\int_E\bv\cdot\bw\,d\bx
\]
but this is not computable in terms of the 
degrees of freedom. The following construction
is done in order to get a calculable discrete form and
also we will introduce a term for the
stability.\\[4pt]
For each element $E\in\Th$, let the space $W(E)$
be defined by
\begin{IEEEeqnarray*}{rCl}
  W(E) & = & \left\{ \bw\in V_h(E):  \bw = \nabla  q_2,
\mbox{ for some }  q_2\in P_2(E)\right\}\\[5pt]
       & = & V_h(E)\cap\nabla {P}_2(E)\mbox{,}
\end{IEEEeqnarray*}
and with these we consider $W(\Th) = \{{\bw} : {\bw}|_{E} \in W(E)
\mbox{ for each }E\in\Th\}$. Inspecting arbitrary elements
in~(\ref{d1p1}) and~(\ref{p03}) yields the following result.
\begin{lemma} When $E\in\Th$ is a tetrahedron or 
a prism, then $W(E) = V_{\textit{h}}(E)$.  
\end{lemma}
Again, observe that if $\bv\in V_h(E)$ and $\bw\in W(E)$, $a^E(\bw,\bv)$ can be 
computed using the degrees of freedom of $\bv$, because it holds
\begin{IEEEeqnarray*}{rCCCl}
a^E(\bw,\bv) &:=& \int_E\bw\cdot\bv\,d\bx &=& \int_E\nabla q_2\cdot\bv\,d\bx\\
                 &=&-\int_E q_2 \dv\bv\,d\bx &+& \iint_{\partial E} q_2\bv\cdot\bn\,dS.
\end{IEEEeqnarray*}
This means we can consider an auxiliary projection  operator $\pi^E$
from $H(\Div,E)$ onto $W(E)$ defined by
\begin{IEEEeqnarray}{rCl}\label{projection}
  a^E(\bv-\pi^E\bv,\bw) = 0 &\qquad& \mbox{for all }\bw\in W(E)
\end{IEEEeqnarray}
fully computable in terms of the degrees of freedom of $\bv$.

We need the following last object to complete the discretization of $a(\cdot,\cdot)$.
Thanks to Lemma~\ref{unisolvence} we can consider a basis $B=\{\bv_{i,E}\}$
of $V_h(E)$ dual to the functionals~(\ref{dofs}), that is, for every
$1\leqslant i,j\leqslant n_{f,E}$
\begin{IEEEeqnarray}{rCl}
  \iint_{f_j} \bv_{i,E}\cdot\bn_j\,dS & = & \delta_{i,j}.
\end{IEEEeqnarray}
What we are considering is the inner product
$\langle(\bv)_B,(\bw)_B\rangle$ of the coordinates
of the fields $\bv,\bw\in V_h(E)$ with respect 
to this local dual basis $B$.

Finally, the local discrete form $a_h^E$ is stated in the following definition.
\begin{defi} Given an element $E\in\Th$, for $\bv$ and $\bw$ in $V_{\textit{h}}(E)$
\begin{IEEEeqnarray}{rCl}\label{discreteLocal_a}
  a^E_{\textit{h}}(\bv,\bw) &:=& a^E(\pi^E\bv,\pi^E\bw) + 
  h_E^{-1}\langle(\bv-\pi^E\bv)_B,(\bw-\pi^E\bw)_B\rangle
\end{IEEEeqnarray}  
where $h_E$ is the diameter of $E$.
The discrete global form $a_h$ is defined by the following equation.
\begin{IEEEeqnarray}{rCl} \label{discreteGlobal_a}
  a_h(\bv,\bw) & = & \sum_{E\in \Th} a_h^E(\bv|_E,\bw|_E),
    \quad\mbox{ for } \bv \mbox{ and } \bw \in V_h.
\end{IEEEeqnarray}
\end{defi}
One fact to note
about the projection $\pi^E$ in~(\ref{projection}) is that, by
Proposition~\ref{vem_equal_fem},
it coincides with the identity $I:W(E)\to W(E)$ whenever the element $E\in\Th$
is a Prism or a Tetrahedron so that, immediately, we have the following
property.
\begin{remark}\label{ah_equal_a} If $E$ is a tetrahedron or a prism, then
  $a_h^E(\bv,\bw)=a^E(\bv,\bw)$ for all $\bv,\bw \in V_h(E)$.
\end{remark}
Now we state a key result concerning the stability of the
discrete form $a^E_{\textit{h}}$ and which explains the form of the
term $h_E^{-1}\langle\,\cdot\,,\cdot\,\rangle$ in~(\ref{discreteLocal_a}).
In what follows we analyse this term in expression~(\ref{discreteLocal_a})
when $E$ is a pyramid.
Let $B_E=\{\bv_i\,:\,\,1\leqslant i\leqslant 5\}$  denote the dual basis of the degrees of freedom~\eqref{dofs}
%, that is,
%\[
%\bv_i\in V_h(E), \qquad \mbox{and}\qquad \int_{f_j}\bv_i\cdot\bn =\delta_{i,j}, \quad 1\leqslant i,j\leqslant 5,
%\]
and let $f_j$, $1\leqslant j\leqslant 5$, denote the faces of $E$. Then 
$\bv\in V_h(E)$ has a unique expression as
\begin{IEEEeqnarray}{rCcCl}\label{auxlabel2}
\bv & = & \sum_{i=1}^5 a_i \bv_i
& = &
\sum_{i=1}^5\left\{\iint_{f_i}\bv\cdot\bn\,dS\right\}\bv_i.
\end{IEEEeqnarray}
Let $\vertiii{\cdot}_E$ denote the following norm on $V_h(E)$
\begin{IEEEeqnarray}{rCl}\label{auxlabel}
  \vertiii{\bv}_E^2 & := & \frac1{h_E}\sum_{i=1}^5 a_i^2.
\end{IEEEeqnarray}
Since the space is finite dimensional, for certain constants $c_E$ and $C_E$ we have
\begin{equation}\label{equiv0}
c_E\|\bv\|_{L^2(E)}\leqslant \vertiii{\bv}_E\leqslant C_E\|\bv\|_{L^2(E)}\mbox{,} 
\end{equation}
for all $\bv\in V_h(E)$. Now the purpose of the next Proposition is to prove that $C_E$ and $c_E$ can be
taken depending only on the aspect ratio of $E$.% \textcolor{red}{% {\bf Assumption (A).} We assume that if $\mbox{diam\,}E=1$, constants $c$ and $C$ above depends only on the shape-regularity of $E$.% }
\begin{proposition}
\label{stabilizing_term}
Let $E$ be a pyramid, and consider the basis $B_E$ of
$V_h(E)$ used in~(\ref{auxlabel2}), and the associated discrete norm $\vertiii{\cdot}_E$ introduced
in~(\ref{auxlabel}). Then there exist constants $C_E$ and $c_E$ depending only on the aspect
ratio of $E$ such that~\eqref{equiv0} holds true for all $\bv\in V_h$.
\end{proposition}
\begin{proof} First we note that if $\bv\in V_h(E)$ is given by $\bv = \sum_{i=1}^5 a_i\bv_i$
then there is a function $\phi$ of zero mean over $E$ for which $\bv = \nabla\phi$ and also
$\Delta\phi = d$ on $E$, $\frac{\partial\phi}{\partial\bn}=g$ on $\partial E$
for
\begin{equation}\label{ai}
g|_{f_i}=\frac{a_i}{|f_i|},\quad 1\leqslant i\leqslant 5, \qquad |E|\,d=\sum_{i=1}^5a_i.
\end{equation}
Given $q\in H^1(E)$, integrating by parts $\Delta\phi$ times $q$ gives
\[
\int_E\nabla\phi\cdot\nabla q\,d\bx = 
-\int_Edq\,d\bx + 
\int_{\partial E}gq\,dS.
\]
Let $\overline{q}$ denote the mean of $q$ over $E$. By the last expression,
as $d$ is constant, we have
\begin{IEEEeqnarray*}{rCcCl}
\|\bv\|_{L^2(E)} & = & \|\nabla\phi\|_{L^2(E)}  & = &  
\sup_{\stackrel{q\in H^1(E),\|\nabla q\|_{0,E}=1\,}{\overline{q}=0}}
\int_{E}\nabla\phi\cdot\nabla q\,d\bx \\[5pt]
\yesnumber\label{equi0}
&&&=&\sup_{\stackrel{q\in H^1(E),\|\nabla q\|_{0,E}=1}{\overline{q}=0}}
\int_{\partial E}gq\,dS.\quad
\end{IEEEeqnarray*}
As a consequence of the definition of $d$ and $g$ in~(\ref{ai}), we have 
obtained an
expression of the norm $\|\bv\|_{L^2(E)}$ in terms of the coefficients of $\bv$
with respecto to basis $B_E$.
Now let $\hat E$ be a fixed reference pyramid, and let $F:\hat E\to E$ be an
affine mapping from $\hat E$ onto $E$ (remember that only pyramids with
parallelogram basis are considered), which can be written as
\[
{\bx}=F(\hat{\bx})=M_E\,\hat{\bx}+{\bf c}.
\] 
Given $q\in H^1(E)$ let $\hat q := q \circ F$. There 
exists constants $c_0$ and $c_1$ depending only on the
aspect ratio of $E$ such that
\begin{equation}\label{a1}
\frac{c_0}{h_E}\|\nabla q\|_{L^2(E)}^2\leqslant \|\nabla\hat q\|_{L^2(\hat E)}^2\leqslant
\frac{c_1}{h_E}\|\nabla q\|_{L^2(E)}^2,
\end{equation}
and, on the other hand, it holds
\[
\int_{\hat E}\hat q\,d\hat\bx = 0 \quad \Longleftrightarrow \quad \int_Eq\,d\bx=0.
\]
Now setting $\left|J(\hat{\bx})\right|_{f_i}|= |f_i|/|\hat f_i|\sim h_E^2$
we have
\[
\int_{\partial E}gq\,dS = \|\nabla \hat q\|_{0,E}
\left(\int_{\partial \hat E}\frac{\hat g\,\hat q\,|J|}{\|\nabla\hat q\|_{0,\hat E}}\,d\hat S
\right).
\]
It follows that
\begin{multline*}
\|\bv\|_{L^2(E)}=\sup\Bigg\{\|\nabla \hat q\|_{L^2(\hat E)}
\left(\int_{\partial \hat E}\frac{g|J|\hat q}{\|\nabla\hat q\|_{L^2(\hat E)}} 
\right):\\ \quad q\in H^1(E),\,\|\nabla q\|_{L^2(E)}=1,\,\overline{q}=0\Bigg\},
\end{multline*}
and taking~\eqref{a1} into account we obtain
% \begin{multline*}
% \|\bv\|_{L^2(E)}\leqslant \sup\Bigg\{\|\hat \nabla q\|_{L^2(\hat E)}\left(-\int_{\hat E}d|\det M_E|\frac{\hat q}{\|\nabla\hat q\|_{L^2(\hat E)}} + \int_{\partial \hat E}g|J|\frac{\hat q}{\|\nabla \hat q\|_{L^2(\hat E)}} \right):\\ \quad \hat q\in H^1(\hat E),\frac{c_0}{h_E^\frac12}\le\|\nabla \hat q\|_{L^2(\hat E)}\le\frac{c_1}{h_E^\frac12},\int_{\hat E}\hat q=0\quad\Bigg\}
% \end{multline*}
\begin{IEEEeqnarray*}{rCCl}
\|\bv\|_{L^2(E)} & \leqslant & c_1^{\nicefrac12}h_E^{-\nicefrac12}\sup&\Bigg\{
\int_{\partial \hat E}g|J|\hat q\,d\hat S:\\ 
\yesnumber\label{mult1}
&&&\quad\quad\hat q\in H^1(\hat E),\|\nabla \hat q\|_{L^2(\hat E)}=1,\overline{q}=0\Bigg\}\quad
\end{IEEEeqnarray*}
and 
\begin{IEEEeqnarray*}{rCCl}
\|\bv\|_{L^2(E)} & \geqslant & c_0^{\nicefrac12}h_E^{-\nicefrac12} \sup&\Bigg\{
\int_{\partial \hat E}g|J|\hat q \,d\hat S:\\
\yesnumber\label{mult2}
&&&\,\hat q\in H^1(\hat E),\|\nabla \hat q\|_{L^2(\hat E)}=1,\int_{\hat E} \hat q\,d\hat\bx=0\Bigg\}.\quad
\end{IEEEeqnarray*}
% and so
% \begin{multline*}
% \|\bv\|_{L^2(E)}\leqslant \frac{c_1}{h_E^\frac12}\sup\Bigg\{-\int_{\hat E}d|\det M_E|\hat q + \int_{\partial \hat E}g|J|\hat q:\\ \quad \hat q\in H^1(\hat E),\|\nabla \hat q\|_{L^2(\hat E)}=1,\int_{\hat E}\hat q=0\quad\Bigg\}.
% \end{multline*}
We remark that
\[
\int_{\hat E}d|\det M_E|\,d\hat\bx=\int_{\partial\hat E}g|J|\,d\hat S.
\]
Now, let $\{\hat\bv_i\}$ be the basis of $V_h(\hat E)$ which is dual to the degrees of
freedom over the faces of $\hat{E}$, and let
\begin{equation}\label{hatai}
\hat a_i=g|_{f_i}\,|J|_{f_i}|\,|\hat f_i|, \qquad 1\leqslant i\leqslant 5.
\end{equation}
Take $\hat\bv = \sum_{i=1}^5 \hat a_i\hat\bv_i$.
From~\eqref{equiv0} applied to the pyramid $\hat E$ we know that
\begin{equation}\label{equiv1}
c_{\hat E}^2\|\hat\bv\|_{L^2(\hat E)}^2\leqslant \frac1{h_{\hat E}}\sum_{i=1}^5
\hat a_i^2\leqslant C_{\hat E}^2\|\hat\bv\|_{L^2(\hat E)}^2\mbox{,}
\end{equation}
and using~\eqref{equi0} for $\hat E$ instead of $E$ we have %(with constants in the equivalence depending only on $\hat E$) %from our previous reasoning %(with $h_{\hat E}\sim 1$)
\begin{IEEEeqnarray*}{rCl}
\|\hat\bv\|_{L^2(\hat E)} & = & 
\sup_{\stackrel{\hat q\in H^1(\hat E),\|\nabla \hat q\|_{L^2(\hat E)}=1}{\int_{\hat E}\hat q\,=\,0}}
\int_{\partial \hat E}g|J|\hat q\,d\hat S.
\end{IEEEeqnarray*}
It follows from~\eqref{equiv1} that
\begin{IEEEeqnarray}{rCl}
  \nonumber
  \left(\frac1{h_{\hat E}}\sum_{i=1}^5 \hat a_i^2\right)^\frac12 & \sim &
  \sup_{\stackrel{\hat q\in H^1(\hat E),\|\nabla \hat q\|_{L^2(\hat E)}=1}
  {\int_{\hat E}\hat q\,=\,0}} 
  \int_{\partial \hat E}g|J|\hat q\,d\hat S
  \mbox{,}
  \\
  &&
\label{auxlabel3}
\end{IEEEeqnarray}
where the constants in this equivalence depend only on the aspect ratio of $E$
and so, since $\hat E$ can be considered with $h_{\hat E}\sim 1$, equation~\eqref{auxlabel3} together
with~\eqref{mult1} and~\eqref{mult2} give
\[
\frac1{h_E}\sum_{i=1}^5\hat a_i^2\sim \|\bv\|_{L^2(E)}^2.
\]
Using~\eqref{ai} and~\eqref{hatai},
\[
\hat a_i = a_i\frac{|J|_{f_i}|}{|f_i|}|\hat f_i|\sim a_i, \qquad 1\leqslant i\leqslant 5,
\]
and then
\[
\frac1{h_E}\sum_{i=1}^5a_i^2\sim \|\bv\|_{L^2(E)}^2
\]
as we wanted, since the constants in this equivalence also depend only on the aspect ratio of $E$.
\end{proof}
Proposition~(\ref{stabilizing_term}) yields the next corollary.
\begin{corollary}\label{equivalence} For all $\bv\in V_h(E)$ and all pyramidal
$E\in\Th$ it holds
\begin{IEEEeqnarray*}{rCcCl}
  c_E\,a^E(\bv,\bv) & \leqslant & h_E^{-1}\langle(\bv)_B,(\bv)_B\rangle & \leqslant
  & C_E\,a^E(\bv,\bv)
\end{IEEEeqnarray*}
where $c_E$ and $C_E$ depend only on the shape regularity of $E$.
\end{corollary}
\section{Discrete Problem}
The discrete problem in the Finite--Virtual Element scheme is stated as follows.
\begin{problem}\label{mixedDiscrete}
To find $\bu_{\Th}\in V_{\Th}$ and $p_{\Th}\in Q_{\Th}$ such that
$\forall \bv\in V_{\Th} \forall q\in Q_{\Th}$
  \begin{IEEEeqnarray*}{rCCCl}
    a_{\scriptscriptstyle\Th}(\bu_{\scriptscriptstyle\Th},\bv) & + &
      b_{\scriptscriptstyle\Th}(\bv,p_{\scriptscriptstyle\Th}) & = & 0 \\[5pt]
                    & - & b_{\scriptscriptstyle\Th}(\bu_{\scriptscriptstyle\Th},q) & = & (f,q).
  \end{IEEEeqnarray*}
\end{problem}
In Section~\ref{sec:well_posedness} we will be concerned in proving existence
and uniqueness for Problem~\ref{mixedDiscrete}. It will be done
showing coercivity for one of the bilinear forms and a
discrete version of the $\inf$--$\sup$ condition for the other.
\begin{lemma}\label{lemma_for_coercivity} For all $E\in\Th$, all $\bw\in W(E)$ and all $\bv\in V_h(E)$
\begin{IEEEeqnarray}{rCcCl} 
a_h^E(\bw,\bv) & = & a^E(\bw,\bv)       & &\label{L1}\\
c_E\,a^E(\bv,\bv)      & \leqslant & a_h^E(\bv,\bv) & \leqslant & C_E\,a^E(\bv,\bv)\label{L2}
\end{IEEEeqnarray}
with $c_E = C_E = 1$ when $E$ is tetrahedral or prismatic and $c_E,C_E$ depending
only on the shape regularity of $E$ when $E$ is pyramidal.
\end{lemma}
\begin{proof} The relation $\bw = \pi^E\bw$ for all $\bw\in W(E)$,
condition~\eqref{projection}
and the symmetry of $a^E$ imply
\[
  a_h^E(\bw,\bv)=a^E(\pi^E\bw,\pi^E\bv)=a^E(\pi^E\bw,\bv)=a^E(\bw,\bv)
\]
which proves~\eqref{L1}.
To prove~(\ref{L2}), if $E$ is not a Pyramid this property is already a consequence
of Remark~\ref{ah_equal_a}. In the case of a Pyramid, firstly a simple computation yields
\begin{IEEEeqnarray}{rCcCl}
  \label{comput}
  a^E(\bv,\bv)&=&a^E(\bv-\pi^E\bv,\bv-\pi^E\bv)&+&a^E(\pi^E\bv,\pi^E\bv).
\end{IEEEeqnarray}
Corollary~\ref{equivalence} and~(\ref{comput}) give
\begin{IEEEeqnarray*}{rCl}
a_h^E(\bv,\bv) &\leqslant& a^E(\pi^E\bv,\pi^E\bv) + C_E\,a^E(\bv-\pi^E\bv,\bv-\pi^E\bv) \\[5pt]
               &\leqslant& \max\{1,C_E\}\,a^E(\bv,\bv).
\end{IEEEeqnarray*}
In a similar manner, it holds
\[
  a_h^E(\bv,\bv)\geqslant \min\{1,c_E\}\,a^E(\bv,\bv).
\]
\end{proof}
\section{A Projection Space $W(E)$}
Since it holds clearly that $W(E)=V_h(E)$ when $E$ is a tetrahedron or a prism,
the purpose of this Section is to characterize $W(E)$ only when $E$ is a pyramid.
The Section finishes with some computational insights. 
\begin{lemma}\label{L3}
Let $\hat E$ be the reference pyramid 
of Definition~\ref{defi_of_ref_pyr} and recall the notation for the faces
in Table~\ref{pyramidNotationTableFaces}.
If $\hat\bv\in P_1(\hat E)^3$ verifies $\hat\bv \cdot \hat\bn =0$ on
$\hat f_1$, $\hat f_2$, $\hat f_3$ and
$\hat f_5$, then $\hat\bv(\hat\bx)=(0,c\hat x_2,0)$ for a constant $c$.
\end{lemma}
\begin{proof}
The conditions of $\hat\bv\cdot\hat\bn$ over the faces $\hat f_1$, $\hat f_2$ and
$\hat f_5$ yield
\[
\hat\bv(\hat\bx) = (b_1\hat x_1, c_2\hat x_2,d_3\hat x_3)'.
\]
Using that $\hat\bv\cdot\hat\bn=0$ on
$\hat f_3$  we have $b_1=d_3=0$. Then $\hat \bv=(0,c_2\hat x_2,0)'$ as we wanted to prove.
\end{proof}
\begin{lemma}\label{L4}
Let $P$ be any pyramid on a physical mesh. Then $\mbox{dim\,}W(P)\leqslant 4$.
\end{lemma}
\begin{proof}
As $W(P)\subseteq V_{\textit{h}}(P)$, we have $\mbox{dim\,}W(P)\leqslant 5$. 
We will prove that $W(P)\ne V_{\textit{h}}(P)$ by showing that there exists no field
$\bv=\nabla q_2$ for $q_2\in P_2(P)$ if we impose the restriccion that the normal
component of $\bv$ vanishes on four different faces of $P$ while is different from zero on
the remaining face.

Let $\hat E$ be the reference pyramid of Definition~\ref{defi_of_ref_pyr}  and
let us use the same notation for the faces. Let $F(\hat {\bx}) = M_P \hat{\bx} + \bx_P$
be an affine map from $\hat E$ onto $P$ and let
$f_i=F(\hat f_i)$ denote the faces of $\hat E$ and $P$. Suppose
that $\bv=\nabla q_2$ for a $q_2 \in P_2(P)$ and that
$\bv\cdot\bn=0$ on $f_1$, $f_2$, $f_3$ and $f_5$, while
$\bv\cdot\bn=1$ on $f_4$. Now we transform it with~(\ref{transfDiv}) to get
\begin{equation}\label{piola}
\bv({\bx}) = \frac1{|M_P|}M_P\hat\bv(\hat{\bx}), \qquad {\bx}=F(\hat {\bx}),
\end{equation}
where $\hat\bv$ is in $P_1(\hat E)^3$. If $\phi\in P_1(f_i)$, put $\hat\phi = \phi\circ F$.
Using the properties of the transformed
degrees of freedom we explained in Section~\ref{auxlabel110} we have, for $i=1,2,3,5$,
\[
\iint_{\hat f_i}\hat\bv\cdot {\hat\bn}\,\hat\phi\,d\hat S = \iint_{f_i}\bv\cdot {\bn}\,\phi\,dS =0
\]
for all $\phi\in P_1(f_i)$.
Since $\hat\bv|_{\hat f_i}\cdot {\hat\bn}$ is itself a $P_1(\hat f_i)$ polynomial, 
this implies $\hat\bv|_{\hat f_i}\cdot {\hat\bn}$ vanishes identically over $\hat f_i$.
By Lemma~\ref{L3} we have $\hat\bv(\hat{\bx})=(0,c\hat x_2,0)'$. Then
\[
  \bv({\bx})=\frac1{|M_P|}M_P(0,c\hat x_2,0)' = \frac {c\,\hat x_2}{|M_P|}\boldsymbol{m}_2 
\]
where we write $M_P=[\boldsymbol{m}_1\,\,\boldsymbol{m}_2\,\,\boldsymbol{m}_3]$ 
columnwise. So on $f_4$ we have 
\[
	\bv({\bx})\cdot {\bn}=\frac {c\,\hat x_2}{|M_P|}\boldsymbol{m}_2\cdot {\bn} 
\]
but $\hat x_2$ is not a constant on $f_4$ since it ranges from $0$ to $1$. Moreover,
$\boldsymbol{m}_2\cdot {\bn}$ is not zero because $\boldsymbol{m}_2$ is a 
vector {\color{Orange}\#\#\#\#\#\#\#\# quiere decir no perpend? transversal} to $f_4$.
Then $\bv({\bx})\cdot {\bn}$ is not a constant on $f_4$, which contradicts our 
definition of $\bv$.  
\end{proof}
Now we explicit the pyramidal projection space.
\begin{proposition}
Let $P$ be a pyramid in the mesh. Then $W(P)=P_0^3(P)+{\bx} P_0(P)$.
\end{proposition}
\begin{proof}
First $P_0^3(P)+{\bx} P_0(P)\subseteq W(P).$
Since by Lemma~\ref{L4} both spaces have the same dimension the assertion concludes.
\end{proof}
If $P$ is a physical pyramid, given a field $\bv \in V_h(P)$ we can construct $\pi^P\bv$
as follows. We choose a basis $\{\bw_i\}$ of $W(P)$, for
example,
\[
\left\{ (1,0,0)', (0,1,0)', (0,0,1)', (x,y,z)' \right\} =: \{
\bw_i: 1\leqslant i\leqslant 4\}
\]
with $\bw_i=\nabla q_i$, $i=1,2,3,4$. Then $a^P(\bv,\bw_i)$ is 
calculable from the degrees of freedom of $\bv$ by
\[
a^P(\bv,\bw_i) = \int_P\bv\cdot\nabla q_i\,d\bx =
-\int_P\mbox{div\,}\bv\,q_i\,d\bx + \iint_{\partial P}\bv\cdot\bn\,q_i\,dS.
\]
Then, if $\pi^P\bv=\sum_{j=1}^4\alpha_j\bw_j$ we can
compute the coefficients $\alpha_j$ by solving the linear system
\[
\sum_{j=1}^4\alpha_j a^P(\bw_j,\bw_i) =
a^P(\bv,\bw_i), \qquad 1\leqslant i\leqslant 4.
\]
In order to compute the stabilization part of the discrete
bilinear form $a_\textit{h}^P$ we need to write $\pi^P\bv$ in terms
of the basis $\{\bv_i\}$ of $V_h(E)$ associated with
the degrees of freedom. That is (always in the pyramidal case)
\[
\iint_{f_i}\bv_j\cdot\bn\,dS =\delta_{ij}, \qquad 1\leqslant i,j\leqslant 5.
\]
In this case, we have $\pi^P\bv = \sum_{i=1}^5\beta_i\bv_i$,
with
\[
\beta_i=\sum_{j=1}^4\alpha_j\iint_{f_i}\bw_j\cdot\bn\,dS.
\]
\end{chapter}
