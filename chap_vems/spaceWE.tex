\section{The Space $W(E)$}
We know that when $E$ is a tetrahedron or a prism $W(E)=V_h(E)$. The purpose of this Section is to characterize $W(E)$ when $E$ is a pyramid. The Section finishes with some computational insights. 
\begin{lemma}\label{L3}
Let $\hat P$ be the reference pyramid 
of Definition~\ref{defi_of_ref_pyr}.
Then if $\bv \in\mathcal P_1(\hat P)^3$ verifies $\bv \cdot \bn =0$ on $\hat f_1$, $\hat f_2$, $\hat f_3$ and
$\hat f_5$, then $\bv({\bf x})=(0,cx_2,0)$ with $c$ constant.
\end{lemma}
\begin{proof}
Using that $\bv\cdot\bn$ is constant on $\hat f_1$, $\hat f_2$ and
$\hat f_5$ we obtain that $\bv$ is of the form
\[
\bv({\bf x}) = (a_1+b_1x_1, a_2+c_2x_2,a_3+d_3x_3),
\]
and since $\bv\cdot\bn=0$ on those faces it result
$a_1=a_2=a_3=0$. Now, using that $\bv\cdot\bn=0$ on
$\hat f_3$  we have that $b_1x_1+d_3x_3|_{\hat f_3}=d_3+(b_1-d_3)x_1|_{\hat f_3}=0$,
so $b_1=d_3=0$. Then $\bv=(0,c_2x_2,0)$ as we wanted to prove.
\end{proof}

\begin{lemma}\label{L4}
Let $P$ be a pyramid. Then $\mbox{dim\,}W(P)\le4$.
\end{lemma}
\begin{proof}
We have $W(P)\subseteq V(P)$ and $\mbox{dim\,}V(P)=5$. In order to prove that $W(P)\ne
V(P)$ we will show that there exists no field
$\bv=\nabla q_2$ with $q_2\in \mathcal P_2(P)$ with normal
component vanishing on four faces of $P$ and being constant different from $0$ on
the another face.

Let $\hat P$ be the reference pyramid of Lemma \ref{L3} and
use the same notation for the faces. Let $F(\hat {\bf x})=B\hat
{\bf x} + {\bf b}$ be an affine map from $\hat P$ onto $P$ and we denote
$f_i=F(\hat f_i)$. Suppose
that $\bv=\nabla q_2\in \mathcal P_2(P)$ is such that
$\bv\cdot\bn=0$ on $f_1$, $f_2$, $f_3$ and $f_5$, while
$\bv\cdot\bn=1$ on $f_4$. Now we consider $\hat\bv$
obtained via the Piola trasform from $\bv$, that is
\begin{equation}\label{piola}
\bv({\bf x}) = \frac1{|B|}B\hat\bv(\hat{\bf x}), \qquad {\bf x}=F(\hat {\bf x}),
\end{equation}
which is in $\mathcal P_1(\hat P)^3$. Using properties of the Piola transform
\cite[pages 12--14]{ricardoMixed} we have for $i=1,2,3,5$,
\[
\int_{\hat f_i}\hat \bv\cdot {\bf n}\hat\phi = \int_{f_i}\bv\cdot {\bf n}\phi =0\qquad \forall \phi\in\mathcal P_1(f_i).
\]
with $\hat \phi = \phi\circ F$. Since $\hat\bv|_{\hat f_i}\cdot {\bf n}\in\mathcal P_1(\hat f_i)$, this implies $\hat\bv|_{\hat f_i}\cdot {\bf n}=0$ for $i=1,2,3,5$. From the previous Lemma we obtain that
\[
\hat\bv(\hat{\bf x})=(0,c\hat x_2,0).
\]
Then
\[
\bv({\bf x})=\frac1{|B|}B(0,c\hat x_2,0)^t = \frac c{|B|}{\bf b}_2 \hat x_2
\]
if $B=[{\bf b}_1\,\,{\bf b}_2\,\,{\bf b}_3]$ (that s, ${\bf b}_i$, $i=1,2,3$,  are the columns of $B$). Then, on $f_4$ we have 
\[
\bv({\bf x})\cdot {\bf n}=\frac c{|B|}\hat x_2{\bf b}_2\cdot {\bf n} 
\]
and we note that $\hat x_2$ is not constant on $f_4$, it varies from $0$ to $1$, and ${\bf b}_2\cdot {\bf n}\ne 0$, since ${\bf b}_2$ is a transversal vector to the face $f_4$. Then, $\bv({\bf x})\cdot {\bf n}$ is not constant on $f_4$, which is a contradicts our definition of $\bf v$.  
\end{proof}

\begin{proposition}
Let $P$ be a pyramid. Then $W(P)=\mathcal
P_0^3(P)+{\bf x}\mathcal P_0(P)$.
\end{proposition}
\begin{proof}
We have
\[
\mathcal P_0^3(P)+{\bf x}\mathcal P_0(P)\subseteq \hat
V(P).
\]
Since
\[
\mbox{dim\,}\left(\mathcal P_0^3(P)+{\bf x}\mathcal
P_0(P)\right)=4
\]
and that from Lemma \ref{L4}, $\mbox{dim\,}\hat
V(P)\le4$, we conclude the assertion.
\end{proof}

Given a field $\bv \in V_h(E)$, we can construct $\Pi_w^E\bv$
as follows. We choose a basis $\{\bw_i\}$ of $W(E)$, for
example,
\[
\left\{ (1,0,0), (0,1,0), (0,0,1), (x,y,z) \right\} =: \{
\bw_i: 1\leqslant i\leqslant 4\}
\]
with $\bw_i=\nabla q_i$, $i=1,2,3,4$. Then $a^E(\bv,\bw_i)$ is calculable from $\bv$'s degrees of freedom
\[
a^E(\bv,\bw_i) = \int_E\bv\cdot\nabla q_i =
-\int_E\mbox{div\,}\bv\,q_i + \int_{\partial E}\bv\cdot\bn\,q_i.
\]
Then if $\Pi_w^E\bv=\sum_{j=1}^4\alpha_j\bw_j$ we can
compute the coefficients $\alpha_j$ by solving the linear system
\[
\sum_{j=1}^4\alpha_j a^E(\bw_j,\bw_i) =
a^E(\bv,\bw_i), \qquad i=1,2,3,4.
\]
In order to compute the stabilization part of the discrete
bilinear form we need to write $\Pi_w^E\bv$ as a linear
combination of the basis $\{\bv_i\}$ of $V_h(E)$ associated with
the degrees of freedom, this is (always in the pyramidal case)
\[
\int_{f_i}\bv_j\cdot\bn =\delta_{ij}, \qquad 1\leqslant i,j\leqslant 5.
\]
In this case, we have $\Pi_w^E\bv = \sum_{i=1}^5\beta_i\bv_i$,
with
\[
\beta_i=\sum_{j=1}^4\alpha_j\int_{f_i}\bw_j\cdot\bn.
\]