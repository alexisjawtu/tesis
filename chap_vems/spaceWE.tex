\section{A Projection Space $W(E)$}
Since it holds clearly that $W(E)=V_h(E)$ when $E$ is a tetrahedron or a prism,
the purpose of this Section is to characterize $W(E)$ only when $E$ is a pyramid.
The Section finishes with some computational insights. 
\begin{lemma}\label{L3}
Let $\hat E$ be the reference pyramid 
of Definition~\ref{defi_of_ref_pyr} and recall the notation for the faces
in Table~\ref{pyramidNotationTableFaces}.
If $\hat\bv\in P_1(\hat E)^3$ verifies $\hat\bv \cdot \hat\bn =0$ on
$\hat f_1$, $\hat f_2$, $\hat f_3$ and
$\hat f_5$, then $\hat\bv(\hat\bx)=(0,c\hat x_2,0)$ for a constant $c$.
\end{lemma}
\begin{proof}
The conditions of $\hat\bv\cdot\hat\bn$ over the faces $\hat f_1$, $\hat f_2$ and
$\hat f_5$ yield
\[
\hat\bv(\hat\bx) = (b_1\hat x_1, c_2\hat x_2,d_3\hat x_3)'.
\]
Using that $\hat\bv\cdot\hat\bn=0$ on
$\hat f_3$  we have $b_1=d_3=0$. Then $\hat \bv=(0,c_2\hat x_2,0)'$ as we wanted to prove.
\end{proof}
\begin{lemma}\label{L4}
Let $P$ be any pyramid on a physical mesh. Then $\mbox{dim\,}W(P)\leqslant 4$.
\end{lemma}
\begin{proof}
As $W(P)\subseteq V_{\textit{h}}(P)$, we have $\mbox{dim\,}W(P)\leqslant 5$. 
We will prove that $W(P)\ne V_{\textit{h}}(P)$ by showing that there exists no field
$\bv=\nabla q_2$ for $q_2\in P_2(P)$ if we impose the restriccion that the normal
component of $\bv$ vanishes on four different faces of $P$ while is different from zero on
the remaining face.

Let $\hat E$ be the reference pyramid of Definition~\ref{defi_of_ref_pyr}  and
let us use the same notation for the faces. Let $F(\hat {\bx}) = M_P \hat{\bx} + \bx_P$
be an affine map from $\hat E$ onto $P$ and let
$f_i=F(\hat f_i)$ denote the faces of $\hat E$ and $P$. Suppose
that $\bv=\nabla q_2$ for a $q_2 \in P_2(P)$ and that
$\bv\cdot\bn=0$ on $f_1$, $f_2$, $f_3$ and $f_5$, while
$\bv\cdot\bn=1$ on $f_4$. Now we transform it with~(\ref{transfDiv}) to get
\begin{equation}\label{piola}
\bv({\bx}) = \frac1{|M_P|}M_P\hat\bv(\hat{\bx}), \qquad {\bx}=F(\hat {\bx}),
\end{equation}
where $\hat\bv$ is in $P_1(\hat E)^3$. If $\phi\in P_1(f_i)$, put $\hat\phi = \phi\circ F$.
Using the properties of the transformed
degrees of freedom we explained in Section~\ref{auxlabel110} we have, for $i=1,2,3,5$,
\[
\iint_{\hat f_i}\hat\bv\cdot {\hat\bn}\,\hat\phi\,d\hat S = \iint_{f_i}\bv\cdot {\bn}\,\phi\,dS =0
\]
for all $\phi\in P_1(f_i)$.
Since $\hat\bv|_{\hat f_i}\cdot {\hat\bn}$ is itself a $P_1(\hat f_i)$ polynomial, 
this implies $\hat\bv|_{\hat f_i}\cdot {\hat\bn}$ vanishes identically over $\hat f_i$.
By Lemma~\ref{L3} we have $\hat\bv(\hat{\bx})=(0,c\hat x_2,0)'$. Then
\[
  \bv({\bx})=\frac1{|M_P|}M_P(0,c\hat x_2,0)' = \frac {c\,\hat x_2}{|M_P|}\boldsymbol{m}_2 
\]
where we write $M_P=[\boldsymbol{m}_1\,\,\boldsymbol{m}_2\,\,\boldsymbol{m}_3]$ 
columnwise. So on $f_4$ we have 
\[
	\bv({\bx})\cdot {\bn}=\frac {c\,\hat x_2}{|M_P|}\boldsymbol{m}_2\cdot {\bn} 
\]
but $\hat x_2$ is not a constant on $f_4$ since it ranges from $0$ to $1$. Moreover,
$\boldsymbol{m}_2\cdot {\bn}$ is not zero because $\boldsymbol{m}_2$ is a 
vector whose direction is not contained in $f_4$.
Then $\bv({\bx})\cdot {\bn}$ is not a constant on $f_4$, which contradicts our 
definition of $\bv$.  
\end{proof}
Now we explicit the pyramidal projection space.
\begin{proposition}
Let $P$ be a pyramid in the mesh. Then $W(P)=P_0^3(P)+{\bx} P_0(P)$.
\end{proposition}
\begin{proof}
First $P_0^3(P)+{\bx} P_0(P)\subseteq W(P).$
Since by Lemma~\ref{L4} both spaces have the same dimension the assertion concludes.
\end{proof}
If $P$ is a physical pyramid, given a field $\bv \in V_h(P)$ we can construct $\pi^P\bv$
as follows. We choose a basis $\{\bw_i\}$ of $W(P)$, for
example,
\[
\left\{ (1,0,0)', (0,1,0)', (0,0,1)', (x,y,z)' \right\} =: \{
\bw_i: 1\leqslant i\leqslant 4\}
\]
with $\bw_i=\nabla q_i$, $i=1,2,3,4$. Then $a^P(\bv,\bw_i)$ is 
calculable from the degrees of freedom of $\bv$ by
\[
a^P(\bv,\bw_i) = \int_P\bv\cdot\nabla q_i\,d\bx =
-\int_P\mbox{div\,}\bv\,q_i\,d\bx + \iint_{\partial P}\bv\cdot\bn\,q_i\,dS.
\]
Then, if $\pi^P\bv=\sum_{j=1}^4\alpha_j\bw_j$ we can
compute the coefficients $\alpha_j$ by solving the linear system
\[
\sum_{j=1}^4\alpha_j a^P(\bw_j,\bw_i) =
a^P(\bv,\bw_i), \qquad 1\leqslant i\leqslant 4.
\]
In order to compute the stabilization part of the discrete
bilinear form $a_\textit{h}^P$ we need to write $\pi^P\bv$ in terms
of the basis $\{\bv_i\}$ of $V_h(E)$ associated with
the degrees of freedom. That is (always in the pyramidal case)
\[
\iint_{f_i}\bv_j\cdot\bn\,dS =\delta_{ij}, \qquad 1\leqslant i,j\leqslant 5.
\]
In this case, we have $\pi^P\bv = \sum_{i=1}^5\beta_i\bv_i$,
with
\[
\beta_i=\sum_{j=1}^4\alpha_j\iint_{f_i}\bw_j\cdot\bn\,dS.
\]