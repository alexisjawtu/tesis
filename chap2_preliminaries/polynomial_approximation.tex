\section{Polynomial Approximation} % (fold)
\label{sec:polynomial_approximation}

\begin{defi}
  Given $f\in\mathcal{C}^{k}(\Omega)$ the $k$--th degree
  Taylor polynomial of $f$ centered at ${\by}\in\Omega$, $T_{\by}^kf$, is defined by
  \begin{IEEEeqnarray}{rCl}\label{taylor}
      (T_{\by}^kf)({\bx})&=&\sum_{|\alpha|\leqslant k} \frac{f^{(\alpha)}({\by})}{\alpha!}(\bx-{\by})^{\alpha}.
  \end{IEEEeqnarray}
\end{defi}

\begin{defi} $\Omega$ is star--shaped with respect to $B$ if, for all
$x\in\Omega$, the closed convex hull of $\{x\}\cup\Omega$ is 
a subset of $\Omega$.
\end{defi}

\begin{defi}
Assume that $\Omega$ is star--shaped with respect to a set $B\subseteq\Omega$
of positive measure. Given an integer $k\geqslant 0$ and 
$f\in W^{k+1,p}(\Omega)$ we introduce the averaged Taylor polynomial
of $f$, $\Qb_{k,B}f\in P_k$ defined by
\begin{IEEEeqnarray}{rCl}\label{averagedTaylor}
  (\Qb_{k,B}f)({\bx}) & = & \frac{1}{|B|} \int_B (T_{\by}^kf)({\bx})\,dy 
\end{IEEEeqnarray}
with $T_{\by}^k$ as in~(\ref{taylor}) but with distributional
derivatives.

Given a field $\bu = (u_1, u_2, u_3)' \in W^{m,p} (\Omega)^3$,
the vectorial averaged Taylor polynomial of $\bu$ is defined
componentwise as
\begin{IEEEeqnarray*}{rCl}
  \Qbb_{m,B}\,\bu  & = &  
  (\Qb_{m,B}\,u_1 , \Qb_{m,B}\,u_2 , \Qb_{m,B}\,u_3 ).
\end{IEEEeqnarray*}
\end{defi}
\begin{lemma}\label{avg_taylor_commutes}
Let $\beta$ be a multi--index such that  $|\beta| \leqslant m$,
then 
\begin{IEEEeqnarray}{rCl}
  \partial^\beta \Qb_{m,B} f & = & \Qb_{m-|\beta|,B} \partial^\beta f.
\end{IEEEeqnarray}
\end{lemma}
\begin{lemma} \label{aux_label40}
  Let $\Omega\subseteq\mathbb{R}^n$ be an open connected set
  with diameter $d$ which is star--shaped with respect to a 
  set $B\subseteq\Omega$ of positive measure. Given $p\geqslant 1$
  and an integer
  $k\geqslant 0$ and $f\in W^{k+1,p}(\Omega)$ there exists a 
  positive $C=C(k,n)$ such that, for $|\beta|\geqslant k+1$,
  \begin{IEEEeqnarray*}{rCl}
      \|\partial^{\beta}(f-\Qb_{k,B}f)\|_{L^p(\Omega)}
        &\leqslant&C\frac{|\Omega|^{\nicefrac1p}}{|B|^{\nicefrac1p}}
          d^{k-|\beta|+1}|f|_{k+1,p,\Omega}.
  \end{IEEEeqnarray*}
  In particular, if $\Omega$ is convex,
  \begin{IEEEeqnarray*}{rCl}
    \|\partial^{\beta}(f-\Qb_{k,\Omega}f)\|_{L^p(\Omega)}
        &\leqslant&Cd^{k-|\beta|+1}|f|_{k+1,p,\Omega}.
  \end{IEEEeqnarray*}
\end{lemma} 
\begin{proof} In view of Lemma~\ref{avg_taylor_commutes} we may only prove
the estimate for the case $|\beta| = 0$ and, for $|\beta|>0$, apply it to
$\partial^{\beta}f-\Qb_{k-|\beta|,B}\partial^{\beta}f$.

Consider $|\beta| = 0$ and take $q$ as the H\"older conjugate of $p\geqslant 1$.
By density we may assume $f\in \mathcal{C}^\infty(\Omega)$.
Firt use Taylor's Theorem
\begin{IEEEeqnarray*}{rCl}
  f(\bx)-(T_{\by}^kf)(\bx) & = & {k+1}
    \sum_{|\alpha|=k+1} \frac{(\bx-\by)^\alpha}{\alpha!}
    \int_0^1 \partial^\alpha f(t\by+(1-t)\bx)\,t^k\,dt
\end{IEEEeqnarray*}
which implies
\begin{IEEEeqnarray*}{rCl}
  f(\bx)-\Qb_{k,B}f(\bx) & = & \frac{k+1}{B}
    \sum_{|\alpha|=k+1} \int_B\int_0^1 \frac{(\bx-\by)^\alpha}{\alpha!}
      \partial^\alpha f(t\by+(1-t)\bx)\,t^k\,dtd\by.
\end{IEEEeqnarray*}
By H\"older's inequality twice (once in a finite dimensional version) we have
\begin{IEEEeqnarray*}{rCl}
  \int_{\Omega}|f(\bx)-\Qb_{k,B}f(\bx)|^p\,d\bx&\leqslant& C\frac{d^{p(k+1)}}{|B|^p}
    \sum_{|\alpha|=k+1}\int_{\Omega}
      (\int_B\int_0^1|\partial^\alpha f(t\by+(1-t)\bx)|^p\,dtd\by)
      (\frac{|B|}{qk+1})^{\nicefrac{p}{q}}
    \\[5pt]
&=& \frac{C}{(qk+1)^{\nicefrac{p}{q}}}\frac{d^{p(k+1)}}{|B|}
    \sum_{|\alpha|=k+1} \int_{\Omega}\int_B\int_0^1
      |\partial^\alpha f(t\by+(1-t)\bx)|^p\,dtd\by d\bx.
\end{IEEEeqnarray*}
For any multi--index $\alpha$ we split
\begin{IEEEeqnarray}{rCl}
\label{first_auxiliary}
  \int_{\Omega}\int_B\int_0^1
      |\partial^\alpha f(t\by+(1-t)\bx)|^p\,dtd\by d\bx
   & = & 
 \int_{\Omega}\int_B\int_0^{\nicefrac12}
      |\partial^\alpha f(t\by+(1-t)\bx)|^p\,dtd\by d\bx\\
\label{second_auxiliary}
   &  &\, +  
\int_{\Omega}\int_B\int_{\nicefrac12}^1
      |\partial^\alpha f(t\by+(1-t)\bx)|^p\,dtd\by d\bx.
\end{IEEEeqnarray}
Let $\phi_\alpha$ be the extension by zero of $\partial^\alpha f$ to $\mathbb{R}^n$.
By Fubini's Theorem and change of variables,
\begin{IEEEeqnarray*}{rCl}
  (\ref{first_auxiliary}) &\leqslant& \int_{B}\int_0^{\nicefrac12}\int_{\mathbb{R}^n}
      |\phi_\alpha(\bz)|^p(1-t)^{-n}\,d\bz dt d\by\\
      &\leqslant& 2^{n-1} |B|\|\partial^\alpha f\|^p_{p,\Omega}.
\end{IEEEeqnarray*}
\begin{IEEEeqnarray*}{rCl}
  (\ref{second_auxiliary}) &\leqslant& \int_{\Omega}\int_{\nicefrac12}^1\int_{\mathbb{R}^n}
      |\phi_\alpha(t\by)|^p\,d\by dt d\bx\\
  &=& \int_{\Omega}\int_{\nicefrac12}^1\int_{\mathbb{R}^n}
      |\phi_\alpha(\bz)|^pt^{-n}\,d\bz dt d\bx\\
      &\leqslant& 2^{n-1} |\Omega|\|\partial^\alpha f\|^p_{p,\Omega}.
\end{IEEEeqnarray*}
Summing theese up for any $\alpha$ of order $k+1$ we obtain
\begin{IEEEeqnarray*}{rCl}
  \|f-\Qb_{k,B}f\|_p^p & \leqslant & 
  \frac{C}{(qk+1)^{\nicefrac{p}{q}}}d^{p(k+1)}\frac{|\Omega|}{|B|} |f|^p_{p,k+1,\Omega}.
\end{IEEEeqnarray*}
\end{proof}
For the following paragraphs we refer to the exposition in
Theorem 3.2 of~\cite{dupontScott}.
\begin{theorem}
  \label{aux_label21}
let $m\geqslant 0$ and $p$, $\bar{p}\in [1,\infty]$. Suppose
\begin{IEEEeqnarray*}{rCl}
  \frac{1}{\bar{p}} - \frac{1}{p} + \frac{m+1}{3} & \geqslant & 0
\end{IEEEeqnarray*}
and that there exists $\sigma$ with 
\begin{IEEEeqnarray*}{rCcCl}
  0 & < & \sigma & \leqslant & 
  \max\left\{
    \left\lfloor \frac{m+1}{3} \right\rfloor,
    \frac{1}{\bar{p}} - \frac{1}{p} + \frac{m+1}{3},
    \min\left\{1-\frac{1}{\bar{p}},\frac{1}{\bar{p}}\right\}
  \right\}\mbox{,}
\end{IEEEeqnarray*}
then there is a positive $C$ depending only on $m$,$\sigma$ and $\Omega$ such
that, for all $g\in W^{m+1,p}(\Omega)$
\begin{IEEEeqnarray}{rCl} \label{aux_label19}
  \|\partial^{\beta}(g-\Qb_m g)\|_{L^{\bar{p}}(\Omega)} & \leqslant & C|g|_{W^{m+1,p}(\Omega)}
\end{IEEEeqnarray}
whenever $0 \geqslant |\beta| \geqslant m+1$.
\end{theorem}
% section polynomial_approximation (end)
