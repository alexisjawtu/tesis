\chapter{Preliminaries}
\label{chap_prelim}
\section{Preliminaries} % (fold)
\label{sec:preliminaries}
\subsection{Definitions, notations} % (fold)
\label{sub:definitions_notations}
\begin{defi}
A Finite Element in $\mathbb{R}^n$ is defined by the following triple $(E, P_E, \Sigma) $:
\vspace{5pt}
\noindent
$E$ is an open non empty domain of $\mathbb{R}^n$ with Lipschitz--continuous 
boundary.\\[5pt]
\noindent
$P_E$ is a finite--dimensional space of real--valued functions defined over $E$.\\[5pt]
\noindent
$\Sigma$ is a finite set of linearly independent linear functionals acting 
on a functional space containing $P_E$. The elements of $\Sigma$ are often
referred to as  \emph{degrees of freedom} or \emph{moments} of the Finite
Element.
\end{defi}
\noindent As we are going to deal with vectorial Finite Elements and
Virtual Elements, we are going to need to use the following notational convention.
\begin{defi} If\hspace{7pt}$V$ denotes a functional space of vectorial
fields in $\mathbb{R}^n$, whether finite dimensional
or not, $(V)_i$ denotes the space of scalar functions in $i-$th
coordinate projection
of $V$, for $1\leqslant i\leqslant n$.
\end{defi}
\begin{defi} 
Given  a Finite Elements $E$ set
\begin{IEEEeqnarray*}{rCl}
  h_E & = & \mbox{ diameter of the element } E.\\
  \rho_E & = & \mbox{ supremum of the diameters of the spheres inscribed in } E.
\end{IEEEeqnarray*}
Consider a family of meshes $\{\mathcal{T}_n\}_{n\in\mathbb{N}}$ such that 
there is a sequence of elements $E_{n_k}\in\mathcal{T}_{n_k}$ for which
	$h_{E_{n_k}}$ tends to zero as $k$ tends to infinity.
\begin{itemize}
	\item [i)] The family of meshes is isotropic if 
	there is a positive constant $\sigma$ such that
	for each $n$ and each $E\in\mathcal{T}_n$ 
	\[
		h_E \leqslant \sigma\rho_E.
	\]
	\item [ii)] The family of meshes is anisotropic if it is not
	isotropic.
\end{itemize}
\end{defi}

\subsection{Polynomials} % (fold)
\label{sub:polynomials}
The finite elements involved in this Thesis are built with piecewise
polynomial functions or rational functions (that is, quotients of polynomials).
For that reason we state some handy notations here.
\begin{defi}
\begin{IEEEeqnarray*}{rCl}
          P_k & = & \{\,\mbox{polynomials of degree less than or equal to k}\,\}\\[5pt]
  \tilde{P}_k & = & \{\,\mbox{homogeneous polynomials of degree k}\,\}\\[5pt]
\end{IEEEeqnarray*}
In the case of a line segment \emph{e} or a subdomain \emph{f} of a plane (typically
an edge or a face of a polyhedron) we will write
\begin{IEEEeqnarray*}{rCl}
	P_k (e) & = & \{\,\mbox{polynomials of maximum degree k in arc length on e}\,\}\\[5pt]
	P_k (f) & = & \{\,\mbox{polynomials of maximum degree k in $\xi_1$,$\xi_2$ on f}\,\}
\end{IEEEeqnarray*}
where we have used an orthogonal coordinate system $(\xi_1,\xi_2)$ in the plane
containing $f$.\\[5pt]
Moreover, we will make the identifications
\begin{IEEEeqnarray*}{rClCrCl}
	P_k(e)  & = & \{p|_e\,:\,p\in P_k\} &\quad\mbox{ and }\quad&P_k(f)
			& = & \{p|_f\,:\,p\in P_k\}.
\end{IEEEeqnarray*}
\end{defi}
\noindent The tensor product of polynomials. % (fold)
\begin{defi} \label{tensor_product} Given two polynomials 
\begin{IEEEeqnarray*}{rClcrCl}
	p(x)&=&\sum a_ix^i&\quad\mbox{ and }\quad&q(y)&=&\sum b_jy^j
\end{IEEEeqnarray*}
the tensor product of $p$ and $q$ is the following
polynomial of \emph{separate variables}
\begin{IEEEeqnarray}{rCl}
	p\otimes q (x,y)&=&\sum_{i,j}a_ib_jx^iy^j.
\end{IEEEeqnarray}
If $\mathcal{P}$ and $\mathcal{Q}$ are two polynomial spaces
\begin{IEEEeqnarray}{rCl}
	\mathcal{P}\otimes \mathcal{Q} & = &\{p\otimes q\,|\,p\in\mathcal{P}, q\in\mathcal{Q}\}.
\end{IEEEeqnarray}
\end{defi}
\noindent In particular we will use the notation.
\begin{notation}
  $Q_{k,k,k} = P_k(\hat{x}_1)\otimes P_k(\hat{x}_2)\otimes P_k(\hat{x}_3)$.
\end{notation}
\begin{remark}\label{tensor_prod_dim} an observation that we will use several times is that,
from the definition, $\dim \mathcal{P}\otimes \mathcal{Q} = \dim \mathcal{P} \dim \mathcal{Q}$.
\end{remark}
% subsubsection tensor_product (end)
% subsection polynomials (end)
\subsection{Functional Spaces. Trace Spaces.} % (fold)
\label{sub:functional_spaces_trace_spaces}
\begin{defi}
\begin{IEEEeqnarray*}{rCl}
	W^p(\curl, E) & = & \{\bu\in W^{1,p}(E)^3\,:\,
	\curl\bu\in W^{1,1}(E)^3\}\\
	\label{normaWpcurl}\yesnumber \|\bu\|_{_{W^p(\curl, E)}} & = & 
	\|\bu\|_{_{W^{1,p}(E)}} +
	\| \curl\bu \|_{_{W^{1,1}(E)}}. 
\end{IEEEeqnarray*}
\end{defi}
To talk  about the regularity of a solution in a \emph{non--convex} domain
we will make use of the following weighted Sobolev space.
\begin{defi} Suppose $\Omega$ non--convex, $\Lambda \subseteq \Omega$ such that 
$\Lambda closure$ contains one singular vertex $\bv$ and one singular edge $\be$. Given two
positive parameters $\beta$ and $\delta$, the symbol $V_{\beta,\delta}^{1,2}(\Lambda)$
denotes the space
\begin{IEEEeqnarray}{rCl}\label{weighted_sobolev}
	V_{\beta,\delta}^{1,2}(\Lambda) & = &
	  \left\{v\in \mathcal D'(\Lambda):
	    R^{\beta-1+|\alpha|}\theta^{\delta-1+|\alpha|}D^\alpha v\in L^2(\Lambda),
	    \alpha\in \mathbb N_0^3, |\alpha|\leqslant 1
	  \right\}
\end{IEEEeqnarray}
where $R({\bf x})$ is the distance of ${\bf x}$ to the vertices
of $\Lambda$,
$r({\bf x})$ is the distance from ${\bf x}$ to the edges
of $\Lambda$ and
finally $\theta({\bf x})$ is the angular distance
$\theta({\bf x})=\frac{r({\bf x})}{R({\bf x})}$.
\end{defi}
\begin{lemma} Let two disjoint Lipschitz domains $\Omega_1$ and $\Omega_2$
be given  in $\mathbb{R}^n$, such that $\overline{\Omega_1}\cap\overline{\Omega_2}$ is an
$(n-1)$--dimensional surface $f$ of positive measure. Take
$\Omega = \Omega_1\cup \Omega_2\cup f$. Let $\bu_1 \in H(\text{div},\Omega_1)$ 
and $\bu_2 \in H(\text{div},\Omega_2)$. Consider 
\begin{IEEEeqnarray*}{rCl}
	\bu & = &
	  \begin{cases}
	  	\bu_1, &\text{ on }\Omega_1\\
	  	\bu_2, &\text{ on }\Omega_2	  	
	  \end{cases}.
\end{IEEEeqnarray*}
Then $\bu$ is in $H(\text{div},\Omega)$ if and only if
the normal traces of $\boldsymbol{u}_1$ and $\boldsymbol{u}_2$ coincide on $f$.
\end{lemma}
\begin{proof}
	{\color{red} TODO}
\end{proof}
% subsection functional_spaces_trace_spaces (end)
\subsection{Transformations} % (fold)

\{*** traido de seccion $\tilde{E}$
y usamos todo lo que estudiamos en la Sección~\ref{sub:transformaciones_entre_prismas}
para transformar $\hat{E} $ en $\tilde{E} $ v\'ia


y para transformar campos, rotores y el interpolador. Por ejemplo, usaremos que 
\begin{IEEEeqnarray*}{rCl}
    \hat{\pi}_i & = & h_i\tilde{\pi}_i \\
    (\textbf{curl}\,\hat{\textbf{u}})_3 & = & h_1h_2(\textbf{curl}\,\tilde{\textbf{u}})_3.
\end{IEEEeqnarray*}


\}

\label{sub:transformations}
\begin{equation}\label{push-forward}
	{\color{red} \ldots \mbox{push-forward} \ldots y que los pull--backs conmutan con los interpoladores}
\end{equation}
% subsection transformations (end)
\begin{figure}[!h]
	\centering
	\subfloat[Reference Prism]
	{
		\referencePrismTikz
	}
	\caption{Prisms}
\end{figure}

\begin{defi}\label{defi_of_ref_pyr}
  The reference pyramid $\hat P$ is the one with vertices at $(0,0,0)$,
$(1,0,0)$, $(0,1,0)$, $(1,1,0)$ and $(0,0,1)$. We denote by $\hat
f_1$ the face with vertices at $(0,0,0)$, $(0,1,0)$, $(0,0,1)$, by
$\hat f_2$ the one with vertices at $(0,0,0)$, $(1,0,0)$,
$(0,0,1)$, by $\hat f_3$ the one with vertices at $(1,0,0)$,
$(1,1,0)$, $(0,0,1)$, by $\hat f_4$ the one with vertices at
$(1,1,0)$, $(0,1,0)$, $(0,0,1)$ and by $\hat f_5$ the square
basis (cfr. Figure~\ref{refpyr}).
\end{defi}

\begin{figure}
	\centering
	\subfloat[Reference Pyramid]
	{
		\label{refpyr}
		\referencePyramidTikz
	}
	\caption{Pyramids}
\end{figure}
\begin{defi} We say that the unisolvent Finite Element $(E, P_E, \Sigma)$ is
$H(\text{div})$--conforming if every time we take two
\emph{push-forward} elements $(E_1, P_{E_1}, \Sigma_1)$
and $(E_2, P_{E_2}, \Sigma_2)$ according
to~(\ref{sub:transformations}), and denote $\pi_1$, $\pi_2$
the interpolation operators determined by the degrees
of freedom $\Sigma_1$ and $\Sigma_2$, the field defined as
\begin{IEEEeqnarray*}{rCl}
	\bw & = &
	  \begin{cases}
	  	\pi_1(\bu|_{\Omega_1}), &\text{ on }\Omega_1\\
	  	\pi_2(\bu|_{\Omega_2}), &\text{ on }\Omega_2	  	
	  \end{cases}
\end{IEEEeqnarray*}
results in $H(\text{div},\Omega_1\cup\Omega_2)$.
\end{defi}
% section preliminares (end)
	
\section{Regularity of the solution for a model Elliptic Problem}

\macroTetraRegularity
\macroPrismRegularity
\noindent We are concerned in solving 
the following problem.
\begin{problem}\label{mixedContinuous}
Suppose we have a simply connected Lipschitz polyhedron
$\Omega\subseteq\mathbb{R}^3$ which is not convex. Given $f\in L^2(\Omega)$
we look for a $\bu\in H(\dv,\Omega)$ such that 
\begin{IEEEeqnarray*}{rCl}
  \boldsymbol{u}              & = & \nabla p \\
  -\text{div\,}\boldsymbol{u} & = & f \\
   p|_{\partial\Omega}
  & = & 0.
\end{IEEEeqnarray*}
\mbox{la cond de borde VA?\quad COMO LO PONGO? \quad (ver apunte de FEM2012 talvez)}
\end{problem}
Due to the polyhedron not being convex there will be ingoing
vertices and edges whose interior angles are obtuse. In the following
we are going to recall how to formalize the singularities in a polyhedron
and how to classify the regularity of the solution of problem~(\ref{mixedContinuous})\\[5pt]

In~(\cite{put_ref_here}) it is stated that the solution
  \begin{itemize}
    \item
  $\Omega=\cup_{\ell=1}^N \Lambda_\ell$ macroelements
      \begin{itemize}
       \item ({\color{blue}tetrahedral or prismatic}).
    \end{itemize}
  \item One singular edge  per $\Lambda_\ell$.
  \item One singular vertex per $\Lambda_\ell$.\\[5pt]
  \item
$
\|v\|^{1,2}_{\beta,\delta} := \left\{\sum_{|\alpha|\leqslant 1}
\|R^{\beta-1+|\alpha|}\theta^{\delta-1+|\alpha|}D^\alpha v\|_{L^2(\Lambda)}^2\right\}^{1/2}
$
\item $R({\bf x}) \,=\, d({\bf x},\mbox{ vertices of }\Lambda$),
\item $r({\bf x}) \,=\, d({\bf x},\mbox{ edges of }\Lambda$),
$\theta({\bf x})\,=\,\frac{r({\bf x})}{R({\bf x})}$
\item
$
V^{1,2}_{\beta,\delta} = \left\{v\in \mathcal D'(\Lambda): \|v\|^{1,2}_{\beta,\delta} < \infty\right\}
$
  \end{itemize}
\begin{itemize}
\item $\{\xi_i\}:$ directions of three concurrent edges of $\Lambda_\ell$
\item $\xi_3$ = direction of singular edge
\item $\beta > \frac12-\lambda_{\bv}^{(\ell)} > 0$
\item $\delta > 1-\lambda_{\be}^{(\ell)} = 1-\pi/\omega_{\be}$
\end{itemize}

 \begin{theorem}
Solutions $\bu$ and $p$ of~(\ref{P}) satisfy
\begin{itemize}
  \item $p\in H^1(\Omega)$
  \item for each $\Lambda_{\ell}$
  \begin{itemize}
    \item $\bu = \bu_r + \bu_s$\\[5pt]
    \item $\bu_r\in H^1(\Omega)^3$ \\[5pt]
    \item $\bu_s\cdot \xi_i\in V^{1,2}_{\beta,\delta}(\Lambda_\ell), \quad i=1,2$ \\[5pt]
    \item $\bu_s\cdot\xi_3\in V^{1,2}_{\beta,0}(\Lambda_\ell)$
  \end{itemize}
\end{itemize}
\end{theorem}
