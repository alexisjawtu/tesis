\chapter{Preliminaries}\label{auxlabel207}
\section*{Introducci\'on al cap\'itulo}
En este cap\'itulo recolectamos definiciones y propiedades
dentro de la teor\'ia que desarrollamos y fijamos nuestra 
elecci\'on de notaci\'on, est\'andar en la mayor\'ia de los casos 
en la literatura.
\section*{Introduction to the chapter}
In this chapter we gather definitions and properties
within the theory we develop and
fix our chosen notation, which is standard in most cases in
the literature.
\label{chap_prelim}
\section{Preliminaries} % (fold)

\label{sec:preliminaries}
\subsection{Definitions and Notations} % (fold)
\label{sub:definitions_notations}
For the first definition we refer the reader to~\cite{ciarlet}. 
\begin{defi}
A Finite Element in $\mathbb{R}^n$ is defined by the following triple $(E, P_E, \Sigma) $:
\begin{enumerate}
  \item 
$E$ is an open non empty polytope of $\mathbb{R}^n$ with Lipschitz--continuous 
boundary.
  \item
$P_E$ is a finite--dimensional space of real--valued scalar or vectorial 
functions defined over $E$.
  \item
$\Sigma$ is a finite set of linearly independent linear functionals acting 
on a functional space containing $P_E$. The elements of $\Sigma$ are often
referred to as  \emph{degrees of freedom} or \emph{moments} of the Finite
Element.
\end{enumerate}
\end{defi}
\begin{defi} 
Given  a Finite Element $E$ set
\begin{IEEEeqnarray*}{rCl}
  h_E & = & \mbox{ diameter of the element } E.\\
  \rho_E & = & \mbox{ supremum of the diameters of the spheres contained in } E.
\end{IEEEeqnarray*}
Consider a family of meshes $\{\pazocal{T}_n\}_{n\in\mathbb{N}}$ such that 
$\max_{E\in\pazocal{T}_{n}} h_E$
tends to zero as $n$ tends to infinity.
\begin{itemize}
	\item [i)] The family of meshes is isotropic if 
	there is a positive constant $\sigma$ such that
	for each $n$ and each $E\in\pazocal{T}_n$ 
	\[
		h_E \leqslant \sigma\rho_E.
	\]
	\item [ii)] The family of meshes is anisotropic if it is not
	isotropic (cfr. Figure~\ref{auxlabel12}).
\end{itemize}
\end{defi}
\begin{figure}[!h]\centering
  \subfloat
  {
    \begin{tikzpicture}[scale=2]
      \prismaticMacroelement{8}{2}{.63}{4}{black}{black}
    \end{tikzpicture}\hspace{1cm}
    \begin{tikzpicture}[scale=2]
      \prismaticMacroelement{8}{3}{.63}{4}{black}{black}
    \end{tikzpicture}\hspace{1cm}
    \begin{tikzpicture}[scale=2]
      \prismaticMacroelement{8}{4}{.63}{4}{black}{black}
    \end{tikzpicture}\hspace{1cm}
    \begin{tikzpicture}[scale=2]
      \prismaticMacroelement{8}{5}{.63}{4}{black}{black}
    \end{tikzpicture}
  }
  \caption{Elements of an anisotropic family of meshes.}
  \label{auxlabel12}
\end{figure}
\begin{defi}
  If $V(E)$ denotes a functional space over an element $E$ from a 
  mesh $\pazocal{T}_n$ for a domain $\Omega$, the symbol $V(\pazocal{T}_n)$
  denotes the space of functions in $\Omega$ whose restrictions to each 
  $E\in\pazocal{T}_n$ belongs to $V(E)$.
\end{defi}
\subsection{Polynomials} % (fold)
\label{sub:polynomials}
The finite elements involved in this Thesis are built with piecewise
polynomial functions or rational functions (that is, quotients of polynomials).
As we will see, the virtual elements also involve functions of these
two kinds amongst others.
For that reason we state some handy notations here.
\begin{defi}
\begin{IEEEeqnarray*}{rCl}
          P_k & = & \{\,\mbox{polynomials of degree less than or equal to k}\,\}.\\[5pt]
  \tilde{P}_k & = & \{\,\mbox{homogeneous polynomials of degree k}\,\}.\\[5pt]
\end{IEEEeqnarray*}
In the case of a line segment $\be$ or a subdomain $f$ of a plane (typically
an edge or a face of a polyhedron) we will write interchangeably
\begin{IEEEeqnarray*}{rCcCl}
	P_k (\be) & = & P_k(t) & = & \{\,\mbox{polynomials of maximum degree k in arc length on $\be$}\,\}\mbox{,}\\[5pt]
	P_k (f) & = & P_k(t_1,t_2) & = & \{\,\mbox{polynomials of maximum degree k in ${\xi_1}$, ${\xi_2}$ on f}\,\}\mbox{,}
\end{IEEEeqnarray*}
where we have used an orthogonal coordinate system $(\xi_1,\xi_2)$ in the plane
containing $f$ and the variable $\xi$ along the edge $\be$.
Moreover, we will make the identifications
\begin{IEEEeqnarray*}{rClCrCl}
	P_k(\be)  & = & \{p|_{\be}\,:\,p\in P_k\} &\quad\mbox{ and }\quad&P_k(f)
			& = & \{p|_f\,:\,p\in P_k\}.
\end{IEEEeqnarray*}
\end{defi}
\noindent The tensor product of polynomials. % (fold)
\begin{defi} \label{tensor_product} Given two polynomials 
\begin{IEEEeqnarray*}{rClcrCl}
	p(x)&=&\sum_i a_ix^i&\quad\mbox{ and }\quad&q(y)&=&\sum_j b_jy^j
\end{IEEEeqnarray*}
the tensor product of $p$ and $q$ is the following
polynomial of \emph{separate variables}
\begin{IEEEeqnarray}{rCl}
	p\otimes q (x,y)&=&\sum_{i,j}a_ib_jx^iy^j.
\end{IEEEeqnarray}
If $\mathbb{Q}_1$ and $\mathbb{Q}_2$ are two polynomial spaces
\begin{IEEEeqnarray}{rCl}
	\mathbb{Q}_1\otimes \mathbb{Q}_2 & = &\{p\otimes q\,|\,p\in\mathbb{Q}_1, q\in\mathbb{Q}_2\}.
\end{IEEEeqnarray}
\end{defi}
\begin{remark}\label{tensor_prod_dim} An observation that we will use several times is that,
from the definition, $\dim \pazocal{P}\otimes \pazocal{Q} = \dim \pazocal{P} \dim \pazocal{Q}$.
\end{remark}
% subsubsection tensor_product (end)
% subsection polynomials (end)
\subsection{Functional Spaces and Trace Spaces} % (fold)
\label{sub:functional_spaces_trace_spaces}
Let $\bx$, $\by$, \dots\, denote points in $\mathbb{R}^n$
and let $d\bx$, $d\by$, \dots denote Lebesgue measure. If $\Omega$
is a measurable set, $1\leqslant p\leqslant\infty$ and $f$ is a 
real or complex valued measurable function, we say $f\in L^p(\Omega)$
if
\begin{IEEEeqnarray*}{rCcCl}
    \|f\|_{L^p(\Omega)}\,=\,\|f\|_{0,p,\Omega}& := &\left\{\int_\Omega |f(\bx)|^p\,d\bx\right\}^{\nicefrac1p} 
    & < & \infty
\end{IEEEeqnarray*}
with the usual modification when $p=\infty$.

Let $\mathbb{Z}_{\geqslant 0}$ denote the set of nonnegative integers. A
\emph{multi--index} $\balpha$ is an $n$--tuple of nonnegative integers:
$\balpha = (\alpha_1,\ldots,\alpha_n)$, $\alpha_i\in\mathbb{Z}_{\geqslant 0}$,
${\s1\leqslant i\leqslant n}$. With multi--indices we will establish the following
notations:
\begin{IEEEeqnarray*}{rCl}
  |\balpha|&=&\sum_i\alpha_i\mbox{,}\\[5pt]
  \balpha&\leqslant&\boldsymbol{\beta}\mbox{\quad iff\quad} \alpha_i\leqslant\beta_i
  \mbox{,\quad}
  {\s 1\leqslant i\leqslant n}\mbox{,}\\[5pt]
  \balpha+\boldsymbol{\beta}&=&(\alpha_1+\beta_1,\ldots,\alpha_n+\beta_n)\mbox{,}\\[5pt]
  \balpha-\boldsymbol{\beta}&=&(\max\{\alpha_1-\beta_1,0\},\ldots,\max\{\alpha_n-\beta_n,0\})\mbox{,}\\[5pt]
  \balpha!&=&\Pi_{i=1}^n \alpha_i!\mbox{,}\\[5pt]
  \bx^{\balpha}&=&\Pi_{i=1}^n x_i^{\alpha_i}\mbox{\quad and}\\[5pt]
  {\s\partial}^{\balpha}\,=\,\left(\frac{{\s\partial}}{{\s\partial}\bx}\right)^{\balpha}&=&\frac{{\s\partial}^{|\balpha|}}{{\s\partial} x_1^{\alpha_1}
  \ldots{\s\partial} x_n^{\alpha_n}}\\[5pt]
  &=&\Pi_{i=1}^n \left(\frac{{\s\partial}}{{\s\partial} x_i}\right)^{\alpha_i}.
\end{IEEEeqnarray*}
If $|\alpha| = 0$, $\partial^{\alpha}f=f$.\\

Let $\Omega$ be an open set with Lipschitz boundary, let $\pazocal{C}^\infty(\Omega)$ denote
the space of infinitely differentiable functions in $\Omega$.                            %For $f\in\pazocal{C}^\infty(\Omega)$
Let $\mathcal{D}(\Omega)$ denote the vectorial subspace of $\pazocal{C}^\infty(\Omega)$
functions that have compact support in $\Omega$, called the \emph{space of test functions}
with the usual topology (cfr.~\cite{rudin}, page 136) 
for which we say a sequence $\phi_n\to 0$ in the topology of $\mathcal{D}(\Omega)$
if and only if there is a compact $K\subseteq\Omega$ which contains the support
of every $\phi_n$, and $\partial^{\alpha}\phi_n\to 0$ uniformly on $K$, as $n\to\infty$,
for every multi--index $\balpha$. The dual $\mathcal{D}'(\Omega)$ of $\mathcal{D}(\Omega)$
is called the space of \emph{distributions} on $\Omega$. If $\phi\in\mathcal{D}'(\Omega)$
and $\alpha$ is a multi--index, $\partial^{\alpha}\phi$ is called a distributional
or weak derivative of $\phi$, where $\partial^{\alpha}\phi$ is defined by
\begin{IEEEeqnarray*}{rCl}
  (\partial^{\alpha}\phi)(f) & = & (-1)^{|\alpha|}\phi(\partial^{\alpha}f)\mbox{,\qquad}
    f\in\mathcal{D}(\Omega).
\end{IEEEeqnarray*}
A distribution $\phi\in\mathcal{D}'(\Omega)$ will be identified with a function
$\psi$ defined on $\Omega$ if for each $f\in \mathcal{D}(\Omega)$, $\psi f\in L^1(\Omega)$
and $\phi(f) = \int_\Omega \psi f\,d\bx$. In this case we shall let
$\phi$ denote the identified function, $\psi$, as well. 

If $m\in\mathbb{Z}_{\geqslant 0}$ and if for each multi--index $\alpha$
with $|\alpha|\leqslant m$, $\partial^{\alpha}\phi$ is  given by a function such that
\begin{IEEEeqnarray*}{rCcccCl}
  \|\phi\|_{W^{m,p}(\Omega)} & = & 
  \|\phi\|_{m,p,\Omega} & := & 
  \left\{\sum_{|\alpha|\leqslant m}\|{\s\partial}^\alpha\phi\|^p_{L^{p}(\Omega)}\right\}^{\nicefrac1p} 
  & < & \infty\mbox{,}
\end{IEEEeqnarray*}
then we will say $\phi\in W^{m,p}(\Omega)$. For $\phi\in W^{m,p}(\Omega)$ let
\begin{IEEEeqnarray*}{rCcCl}
  |\phi|_{W^{m,p}(\Omega)} & = & |\phi|_{m,p,\Omega} 
    & := & \left\{\sum_{|\alpha|=m}\|\partial\phi\|_{L^{p}(\Omega)}\right\}^{\nicefrac1p}.
\end{IEEEeqnarray*}
In the previous notation we will skip the number $2$ for the special case $p=2$ and 
write $\|\phi\|_{W^{m,2}(\Omega)}=\|\phi\|_{m,\Omega}$,
$|\phi|_{W^{m,2}(\Omega)}=|\phi|_{m,\Omega}$.
Sometimes we will write, in the vectorial case, $|\bu|^p = |u_1|^p + |u_2|^p + |u_3|^p$.\\

In the definitions of the following two functional spaces, the differential
operators must be understood in the distributional sense.
\begin{defi} As in page 26 of~\cite{giraultRaviart} we make the present definition. Let 
$\Omega\subseteq\mathbb{R}^3$ be a lipschitz domain.
  \begin{IEEEeqnarray*}{rCl}
    H(\Div, \Omega) & := & \{\bu\in L^2(\Omega)^3\,:\,\dv \bu \in L^2(\Omega)\}\mbox{,}\\[5pt]
    \|\bu\|_{H(\Div, \Omega)} & := & \left(\|\bu\|^2_{L^2(\Omega)^3}+
      \|\dv \bu\|^2_{L^2(\Omega)}\right)^{\nicefrac12}.
  \end{IEEEeqnarray*}
\end{defi}
\begin{defi} As in page 32 of~\cite{giraultRaviart} we make the present definition. Let 
$\Omega\subseteq\mathbb{R}^3$ be a lipschitz domain.
  \begin{IEEEeqnarray*}{rCl}
    H(\bcurl,\Omega) & := & \{\bu\in L^2(\Omega)^3\,:\,\curl \bu \in L^2(\Omega)^3\}\mbox{,} \\[5pt]
    \|\bu\|_{H(\bcurl,\Omega)} & := & \left(\|\bu\|^2_{L^2(\Omega)^3}+
      \|\curl\bu\|^2_{L^2(\Omega)^3}\right)^{\nicefrac12}.
  \end{IEEEeqnarray*}
\end{defi}
\begin{defi} As in page 4 of~\cite{ariel} we make the present definition.
\begin{IEEEeqnarray*}{rCl}
  W^p(\bcurl, \Omega) & = & \{\bu\in W^{1,p}(\Omega)^3\,:\,
  \curl\bu\in W^{1,1}(\Omega)^3\}\mbox{,}\\
  \label{normaWpcurl}\yesnumber \|\bu\|_{_{W^p(\bcurl, \Omega)}} & = & 
  \|\bu\|_{_{W^{1,p}(\Omega)}} +
  \| \curl\bu \|_{_{W^{1,1}(\Omega)}}. 
\end{IEEEeqnarray*}
\end{defi}
The concept of a component that is normal to the boundary can be extended
to $H(\Div,\Omega)$ using a density argument. This is what we call \textsl{the
normal trace}. Let $\bnu$ be, in the following, the unit outward normal
to $\partial\Omega$. For a function $\bv\in\pazocal{C}^\infty(\overline{\Omega})^3$
the normal trace $\gamma_{\boldsymbol{\nu}}$ is defined simply by
\begin{IEEEeqnarray}{rCl}\label{normal_trace}
  \gamma_{\boldsymbol{\nu}}(\bv) & := & \bv|_{\partial\Omega}\cdot\boldsymbol{\nu}.
\end{IEEEeqnarray}
The proof of next Theorem can be found in~\cite{monk}, page $53$ and uses merely
the density of $\pazocal{C}^\infty(\overline{\Omega})^3$ in $H(\Div,\Omega)$.
\begin{theorem} The operator $\gamma_{\boldsymbol{\nu}}$ defined in~(\ref{normal_trace})
can be extended by continuity to a continuous linear map $\gamma_{\boldsymbol{\nu}}$ from
$H(\Div,\Omega)$ onto $(H^{\nicefrac12}(\partial\Omega))'.$
\end{theorem}
With regard to trace properties in $H(\bcurl)$, we must verify that
functions in this space have a well--defined tangential trace (cfr.
page 34 of~\cite{giraultRaviart} and page 59 of~\cite{monk}). The next Definition
and proof of the Theorem after it can be found in~\cite{chenDuZou} and gives the
form that the surface degrees of freedom in $H(\bcurl)$--Conforming
Elements will take.

For a smooth vector function $\bv\in\pazocal{C}^\infty(\overline{\Omega})^3$ we define
the two traces
\begin{IEEEeqnarray}{rCl}
\label{aux_label80}\gamma_t(\bv)&=&\bnu\times\bv|_{\partial\Omega}\\
\gamma_T(\bv)&=&(\bnu\times\bv|_{\partial\Omega})\times\bnu
\end{IEEEeqnarray}
Theorem $3.29$ in~\cite{monk} states that~(\ref{aux_label80}) can be extended
by continuity to a continuous linear map from $H(\bcurl,\Omega)$ into 
$(H^{\nicefrac12}(\partial\Omega)')^3$
\begin{defi}
  \begin{IEEEeqnarray}{rCl}
  \nonumber
    Y(\partial\Omega) &=& \left\{ \boldsymbol{f}\in 
    (H^{\nicefrac12}(\partial\Omega)')^3\,:\,\mbox{ there exists }\bu\in 
    H(\bcurl,\Omega) \right.\\
    &&\quad\left.\mbox{with } \gamma_t(\bu) = \boldsymbol{f}\right\}\mbox{,}
  \end{IEEEeqnarray}
  \begin{IEEEeqnarray*}{rCl}
    \|\boldsymbol{f}\|_{Y(\partial\Omega)} &=& 
    \inf_{\bu\in H(\bcurl,\Omega), \gamma_t(\bu) = \boldsymbol{f}}
    \|\bu\|_{H(\bcurl,\Omega)}.
  \end{IEEEeqnarray*}
\end{defi}
So we can consider the epimorphism $\gamma_t\,:\,H(\bcurl,\Omega)\rightarrow
Y(\partial\Omega)$ with which we have the following Theorem (cfr. Theorem 3.31
in~\cite{monk}).
\begin{theorem} The map 
$\gamma_T\,:\,H(\bcurl,\Omega)\rightarrow
Y(\partial\Omega)'$ is well--defined. For any $\bv$, $\bphi$ in $H(\bcurl,\Omega)$
\begin{IEEEeqnarray}{rCl}\label{aux_label5}
   \int_\Omega \curl\bv\cdot\bphi\,d\bx 
    - \int_\Omega \bv\cdot\curl\bphi\,d\bx & = & 
    \langle\gamma_t(\bv), \gamma_T(\bphi)\rangle_{\partial\Omega}
 \end{IEEEeqnarray}
($\langle\cdot,\cdot\rangle$ denotes a duality pairing).
\end{theorem}
%%=============================================================================
% {\color{blue}\#\#\#\#\#\#\#\# es el de pag. 51 Adams?.}
% {\color{blue}\#\#\#\#\#\#\#\# tal vez mejor el Teorema 3.26 de monk, pag.55?.}
% {\color{blue}\#\#\#\#\#\#\#\# poner un esbozo de la prueba, mirar chap 3 adams
%  si queda tiempo.}
%%=============================================================================
\begin{defi}
  We shall say that a domain $\Omega$ has the \textsl{segment property}
  if for every $x$ in the boundary of $\Omega$ there exists an open set
  $U_x$ and a nonzero vector $y_x$ such that $x\in U_x$ and if 
  $z\in\overline{\Omega}\cap U_x$, then $z+ty_x \in \Omega$ for $0<t<1$.
\end{defi}
A domain having this property must have $(n-1)$--dimensional boundary
and cannot simultaneously lie on both sides of any given part of its
boundary.  Since the polyhedra we will consider to build finite elements are
  convex and obviously satisfy the segment property,
  with techniques that can be learned from Chapter $3$
  in the book~\cite{adams} we can prove the following Proposition for our 
  elements.
\begin{proposition}\label{density_wpcurl}
  The space $\pazocal{C}^\infty(\bar{E})^3$ is dense in
  $W^p(\bcurl,{E})$ with its norm defined in~(\ref{normaWpcurl}).
\end{proposition}

With elementary applications of integration by parts for distributional
derivatives we get the following two Lemata.
\begin{lemma} Let two disjoint Lipschitz domains $\Omega_1$ and $\Omega_2$
be given  in $\mathbb{R}^3$, such that $\overline{\Omega_1}\cap\overline{\Omega_2}$ is an
$2$--dimensional surface $f$ of positive measure. Take
$\Omega = \Omega_1\cup \Omega_2\cup f$. Let $\bu_1 \in H(\Div,\Omega_1)$ 
and $\bu_2 \in H(\Div,\Omega_2)$. Consider 
\begin{IEEEeqnarray*}{rCl}
  \bu & = &
    \begin{cases}
      \bu_1, &\text{ on }\Omega_1\\
      \bu_2, &\text{ on }\Omega_2     
    \end{cases}.
\end{IEEEeqnarray*}
Then $\bu$ is in $H(\Div,\Omega)$ if and only if
the normal traces of $\boldsymbol{u}_1$ and $\boldsymbol{u}_2$ coincide on $f$.
\end{lemma}
\begin{lemma} Let two disjoint Lipschitz domains $\Omega_1$ and $\Omega_2$
be given  in $\mathbb{R}^3$, such that $\overline{\Omega_1}\cap\overline{\Omega_2}$ is an
$2$--dimensional surface $f$ of positive measure. Take
$\Omega = \Omega_1\cup \Omega_2\cup f$. Let $\bu_1 \in H(\bcurl,\Omega_1)$ 
and $\bu_2 \in H(\bcurl,\Omega_2)$. Consider 
\begin{IEEEeqnarray*}{rCl}
	\bu & = &
	  \begin{cases}
	  	\bu_1, &\text{ on }\Omega_1\\
	  	\bu_2, &\text{ on }\Omega_2	  	
	  \end{cases}.
\end{IEEEeqnarray*}
Then $\bu$ is in $H(\bcurl,\Omega)$ if and only if
the tangential traces $\gamma_T(\cdot)$
of $\bu_1$ and $\bu_2$, as defined in~(\ref{aux_label5}), coincide on $f$.
\end{lemma}
%=================
%\begin{proof}
%	{\color{red} TODO}
%\end{proof}
%=================
% subsection functional_spaces_trace_spaces (end)

\begin{defi}\label{defi_of_ref_prism}
  Let $\hat T$ be the triangle $\{ 0 < x + y < 1 \}$. 
  The reference prism is the interior of 
  $\hat T\times\{ 0 < z < 1 \}$ (cfr. Figure~\ref{reference_prism}).
\end{defi}
\begin{defi}\label{defi_of_ref_pyr}
The reference pyramid $\hat E$ is the polyhedron 
$\{\bx\in\mathbb{R}^3\,:\,0< x_3< 1,
0<  x_1<  1-x_3, 0<  x_2<  1-x_3\}$
with vertices at $(0,0,0)'$,
$(1,0,0)'$, $(0,1,0)'$, $(1,1,0)'$ and $(0,0,1)'$ (cfr. Figure~\ref{refpyr}).
\end{defi}
\begin{defi}\label{def_of_ref_elems}
The reference tetrahedron is (cfr. Figure~\ref{reftetr})
$\{\bx\in\mathbb{R}^3\,:\,x_1 > 0, x_2 > 0, x_3 > 0, x_1+x_2+x_3 < 1\}$.
\end{defi}

\begin{figure}
	\centering
  \subfloat[Pyramid]
  {
    \label{refpyr}
    \referencePyramidTikz{1.5}
  }
  \hspace{1cm}
  \subfloat[Prism]
  {
    \label{reference_prism}
    \referencePrismTikz{1.5}
  }\\
  \subfloat[Tetrahedron]
  {
    \label{reftetr}
    \referenceTetrahedronTikz{1.5}
  }
  \hspace{1cm}
  \subfloat[Triangle]
  {
    \label{reference_triangle}
    \referenceTriangleTikz{1.3}
  }
	\caption{Reference Elements}
\end{figure}

\section{Regularity of the solution for a model Elliptic Problem}
\label{sec:regularity}
\macroRegularity
%\macroPrismRegularity
\noindent We are concerned in solving 
the following problem.
\begin{problem}\label{mixedContinuous}
Suppose we have a simply connected Lipschitz polyhedron
$\Omega\subseteq\mathbb{R}^3$ which is not convex. Given $f\in L^2(\Omega)$
we look for a $\bu\in H(\dv,\Omega)$ such that 
\begin{IEEEeqnarray*}{rCl}
  \boldsymbol{u}              & = & \nabla p \\
  -\text{div\,}\boldsymbol{u} & = & f \\
   p|_{\partial\Omega}
  & = & 0.
\end{IEEEeqnarray*}
\mbox{\color{Orange}la cond de borde VA?\quad COMO LO PONGO? \quad (ver apunte de FEM2012 talvez)}
\end{problem}
\begin{problem}[Weak formulation.]\label{weakMixedContinuous}
Suppose we have a simply connected Lipschitz polyhedron
$\Omega\subseteq\mathbb{R}^3$ which is not convex. Given $f\in L^2(\Omega)$,
find       $(\bu, p)  \in  H(\dv, \Omega) \times L^2(\Omega)$ 
    such that for all   $(\bv, q)  \in  H(\dv, \Omega) \times L^2(\Omega)$
  \begin{IEEEeqnarray*}{rCl}
    \int_{\Omega} \bu\cdot\bv\,d\bx + 
    \int_{\Omega} p\,\dv\bv\,d\bx                     & = & 0\\
     -\int_{\Omega} q\dv\bu\,d\bx     & = & 
    \int_{\Omega} f\,q\,d\bx.    
  \end{IEEEeqnarray*}
\end{problem}
Due to the polyhedron not being convex there will be ingoing
vertices and edges whose interior angles are obtuse. In the following
we are going to recall how to formalize the singularities in a polyhedron
and how to classify the regularity of the solution of problem~(\ref{mixedContinuous}).\\

Recall the description in page 522 of~\cite{apelNicaise} for the following 
definition. 
\begin{defi}
Suppose that a vertex
$\bv\in\partial\Omega$ is the origin of our Cartesian system of coordinates. Let $C_{\bv}$ be the infinite
polyhedral cone of $\mathbb{R}^3$ which coincides with $\Omega$ in a neighbourhood of 
$\bv$. Set $G_{\bv} = C_{\bv}\cap S^2(\bv)$, the intersection of $C_{\bv}$ with
the unit sphere centered at $\bv$. The vertex singular exponent related to $\bv$
is defined by $\lambda_{\bv} := -\nicefrac12 + \sqrt{\lambda_{\bv,1} + \nicefrac14}$
where 
$\lambda_{\bv,n} > 0$, $n\in\mathbb{N}$, are the eigenvalues, in increasing order, of a positive
Laplace--Beltrami operator $\Delta'$ on $G_{\bv}$ with Dirichlet boundary
conditions. For any edge of $\partial\Omega$ the edge
singular exponent is 
$\pi/\omega_{\be}$ where $\omega_{\be}$ is the interior angle between
the faces sharing  $\be$.
We say that
  \begin{enumerate}
    \item $\be$ is singular if $\lambda_{\be} < 1.$ 
    \item $\bv$ is singular if $\lambda_{\bv} < \dfrac12$
  \end{enumerate}
\end{defi}
The motivation of calling \textsl{singular} an edge or a vertex is that
the solutions of elliptic problems in non--convex domains have singularities
at those vertices or approaching those edges.

To talk about the regularity of a solution in a \emph{non--convex} domain
we will make use of the following weighted Sobolev spaces.
\begin{defi} Suppose $\Omega$ is a non--convex polihedron where $\Lambda \subseteq \Omega$ is
a subdomain such that 
$\bar{\Lambda}$ contains at most one singular vertex $\bv$ or at most one singular
edge $\be$ of $\Omega$.
In case it contained both, the edge is incident to the vertex. Fix a local system
of coordinates in $\Lambda$ with origin at $\bv$ (in case we had a singular
edge and no singular vertex, the origin is at one end of the edge).
$R(\bx)$ will be the distance from $\bx$ to $\bv$,
$r(\bx)$ will be the distance from $\bx$ to $\be$ and $\theta(\bx)$ will be
the \textsl{angular distance} $\theta(\bx)=\frac{r(\bx)}{R(\bx)}$. With these weights,
given two 
positive parameters $\beta$ and $\delta$ we define the norm
\begin{IEEEeqnarray}{rCl}\label{weighted_norm}
  \|v\|^{1,2}_{\beta,\delta} & := & \left\{\sum_{|\alpha|\leqslant 1}
  \|R^{\beta-1+|\alpha|}\theta^{\delta-1+|\alpha|}\,\partial^\alpha v\|_{L^2(\Lambda)}^2\right\}^{1/2}.
\end{IEEEeqnarray}
%\left\{v\in \mathcal D'(\Lambda):
%      R^{\beta-1+|\alpha|}\theta^{\delta-1+|\alpha|}\partial^\alpha v\in L^2(\Lambda),
%      \alpha\in \mathbb N_0^3, |\alpha|\leqslant 1
%\right\}
And the symbol $V_{\beta,\delta}^{1,2}(\Lambda)$ will
denote the space
\begin{IEEEeqnarray}{rCl}\label{weighted_sobolev}
  V_{\beta,\delta}^{1,2}(\Lambda) & = &
  \left\{ v \in \mathcal D'(\Lambda): \|v\|^{1,2}_{\beta,\delta} < \infty\right\}.  
\end{IEEEeqnarray}
\end{defi}
\begin{remark}
If in the last definition we had $\beta = \delta$, which will hold true in the
case of just an edge singularity, then we would write 
$V_{\delta, \delta}^{1,2}(\Lambda)  = V_{\delta}^{1,2}(\Lambda)$
and, consistently, 
\begin{IEEEeqnarray*}{rCl}
\|v\|_{\delta}^{1,2} & := & \left\{\sum_{|\alpha|\leqslant 1}
\|r^{\delta-1+|\alpha|}\partial^\alpha v\|_{L^2(\Lambda)}^2\right\}^{1/2}.
\end{IEEEeqnarray*}
\end{remark}
%========
%First we introduce the space $V^{1,2}_{\beta,\delta}(\Lambda)$ for a macroelement $\Lambda$ as
%where $R(\bx)$ is the distance of $\bx$ to the vertices of $\Lambda$, $r(\bx)$ is the distance from $\bx$ to the edges of $\Lambda$ and finally $\theta(\bx)$ is the angular distance $\theta(\bx)=\frac{r(\bx)}{R(\bx)}$.
%========
Let $\Omega=\cup_{\ell=1}^N \Lambda_\ell$ be 
a decomposition of $\Omega$ in
{\bf prismatic} or {\bf tetrahedral} macro--elements having, each one of them,
at most a singular edge $\lambda_{\be}^{(\ell)}$ and a singular vertex
$\lambda_{\bv}^{(\ell)}$. This corresponds to the fact
that the sequence of meshes we will introduce 
at the end of this thesis has a first coarse term consisting only of prisms
and tetrahedra.
We have the following regularity result, for which we refer again
to~\cite{apelNicaise}.
\begin{theorem}\label{thm_regularity}
The solutions $\bu$ and $p$ of problem \eqref{weakMixedContinuous} satisfy
\[
  p\in H^1(\Omega)
\] 
and for each $\ell$
\[
  \bu=\bu_r + \bu_s
\]
with $\bu_r\in H^1(\Omega)$ and
\[
  \bu_s\cdot \xi_i\in V^{1,2}_{\beta_{\ell},\delta_{\ell}}(\Lambda_\ell), \quad i=1,2, \qquad
  \bu_s\cdot\xi_3\in V^{1,2}_{\beta_{\ell},0}(\Lambda_\ell)
\]
where $\xi_i$, $i=1,2,3$, are the directions of three edges of $\Lambda_\ell$ 
concurrent to $\bv$ with $\xi_3$ being the direction of the
singular edge, and $\beta_{\ell},\delta_{\ell}\geqslant 0$
satisfying $\beta_{\ell}>\frac12-\lambda_{\bv}^{(\ell)}$ and
$\delta_{\ell}>1-\lambda_{\be}^{(\ell)}$.
Furthermore, the following estimates hold:
%provided
%\begin{IEEEeqnarray*}{rCl}
%  {\color{violet} \delta_{\ell}} & > & 1 - \frac{\pi}{\omega_{\textbf{e}}}\text{,}\\
%  {\color{violet} \beta_{\ell}} & > & \frac{1}{2} - \lambda_{\textbf{v}}.
%\end{IEEEeqnarray*}
\begin{IEEEeqnarray}{rCl}
  \label{aux_label11}
  \| \bu_r \|_{H^1(\Omega)} & \leqslant & c\,\|f\|_{L^2(\Omega)}\\[5pt]
  \| \bu_s\cdot\xi_i \|_{V_{\beta_{\ell},\delta_{\ell}}^{1,2}(\Lambda_\ell)} & \leqslant & c\,\| f \|_{L^2(\Omega)}\\[5pt]
  \| \bu_s\cdot\xi_3 \|_{V_{\beta_{\ell},0}^{1,2}(\Lambda_\ell)}      & \leqslant & c\,\| f \|_{L^2(\Omega)}
\end{IEEEeqnarray}
\end{theorem}
\begin{remark}\label{sobreBetaYDelta}
Note that it is always possible to take $0<\beta_{\ell}=\delta_{\ell}<1$ in the previous Theorem
and note the dependence of $\beta$ and $\delta$ on the macro--element, which
we made explicit by putting the subindex $\ell$, meaning the result is ahout
local regularity.
\end{remark}
\section{Polynomial Approximation} % (fold)
\label{sec:polynomial_approximation}

\begin{defi}
  Given $f\in\pazocal{C}^{k}(\Omega)$ the $k$--th degree
  Taylor polynomial of $f$ centered at ${\by}\in\Omega$, $T_{\by}^kf$, is defined by
  \begin{IEEEeqnarray}{rCl}\label{taylor}
      (T_{\by}^kf)({\bx})&=&\sum_{|{\balpha}|\leqslant k} 
      \frac{f^{({\balpha})}({\by})}{{\balpha}!}(\bx-{\by})^{{\balpha}}.
  \end{IEEEeqnarray}
\end{defi}

\begin{defi} $\Omega$ is star--shaped with respect to $B$ if, for all
$x\in\Omega$, the closed convex hull of $\{x\}\cup B$ is 
a subset of $\Omega$.
\end{defi}

\begin{defi}
Assume that $\Omega$ is star--shaped with respect to a set $B\subseteq\Omega$
of positive measure. Given an integer $k\geqslant 0$ and 
$f\in W^{k+1,p}(\Omega)$ we introduce the averaged Taylor polynomial
of $f$, $\Qb_{k,B}f\in P_k$ defined by
\begin{IEEEeqnarray}{rCl}\label{averagedTaylor}
  (\Qb_{k,B}f)({\bx}) & = & \frac{1}{|B|} \int_B (T_{\by}^kf)({\bx})\,dy 
\end{IEEEeqnarray}
with $T_{\by}^k$ as in~(\ref{taylor}) but with distributional
derivatives.

Given a field $\bu = (u_1, u_2, u_3)' \in W^{m,p} (\Omega)^3$,
the vectorial averaged Taylor polynomial of $\bu$ is defined
componentwise as
\begin{IEEEeqnarray*}{rCl}
  \Qbb_{m,B}\,\bu  & = &  
  (\Qb_{m,B}\,u_1 , \Qb_{m,B}\,u_2 , \Qb_{m,B}\,u_3 ).
\end{IEEEeqnarray*}
\end{defi}
\begin{lemma}\label{avg_taylor_commutes}
Let ${\bbeta}$ be a multi--index such that  $|{\bbeta}| \leqslant m$,
then 
\begin{IEEEeqnarray}{rCl}
  \partial^{\bbeta} \Qb_{m,B} f & = & \Qb_{m-|{\bbeta}|,B} \partial^{\bbeta} f.
\end{IEEEeqnarray}
\end{lemma}
\begin{lemma} \label{aux_label40}
  Let $\Omega\subseteq\mathbb{R}^n$ be an open connected set
  with diameter $d$ which is star--shaped with respect to a 
  set $B\subseteq\Omega$ of positive measure. Given $p\geqslant 1$
  and an integer
  $k\geqslant 0$ and $f\in W^{k+1,p}(\Omega)$ there exists a 
  positive $C=C(k,n)$ such that, for $|{\bbeta}|\geqslant k+1$,
  \begin{IEEEeqnarray*}{rCl}
      \|\partial^{{\bbeta}}(f-\Qb_{k,B}f)\|_{L^p(\Omega)}
        &\leqslant&C\frac{|\Omega|^{\nicefrac1p}}{|B|^{\nicefrac1p}}
          d^{k-|{\bbeta}|+1}|f|_{k+1,p,\Omega}.
  \end{IEEEeqnarray*}
  In particular, if $\Omega$ is convex,
  \begin{IEEEeqnarray*}{rCl}
    \|\partial^{{\bbeta}}(f-\Qb_{k,\Omega}f)\|_{L^p(\Omega)}
        &\leqslant&Cd^{k-|{\bbeta}|+1}|f|_{k+1,p,\Omega}.
  \end{IEEEeqnarray*}
\end{lemma} 
\begin{proof} In view of Lemma~\ref{avg_taylor_commutes} we may only prove
the estimate for the case $|{\bbeta}| = 0$ and, for $|{\bbeta}|>0$, apply it to
$\partial^{{\bbeta}}f-\Qb_{k-|{\bbeta}|,B}\partial^{{\bbeta}}f$.

Consider $|{\bbeta}| = 0$ and take $q$ as the H\"older conjugate of $p\geqslant 1$.
By density we may assume $f\in \pazocal{C}^\infty(\Omega)$.
Firt use Taylor's Theorem
\begin{IEEEeqnarray*}{rCl}
  f(\bx)-(T_{\by}^kf)(\bx) & = & {(k+1)}
    \sum_{|{\balpha}|=k+1} \frac{(\bx-\by)^{\balpha}}{{\balpha}!}
    \int_0^1 \partial^{\balpha} f(t\by+(1-t)\bx)\,t^k\,dt
\end{IEEEeqnarray*}
which implies
\begin{IEEEeqnarray*}{rCl}
  f(\bx)-\Qb_{k,B}f(\bx) & = & \frac{k+1}{B}
    \sum_{|{\balpha}|=k+1} \int_B\int_0^1 \frac{(\bx-\by)^{\balpha}}{{\balpha}!}
      \partial^{\balpha} f(t\by+(1-t)\bx)\,t^k\,dt\,d\by.
\end{IEEEeqnarray*}
By H\"older's inequality twice (once in a finite dimensional version) we have
\begin{IEEEeqnarray*}{rCl}
  \IEEEeqnarraymulticol{3}{l}{\int_{\Omega}|f(\bx)-\Qb_{k,B}f(\bx)|^p\,d\bx\,\leqslant\,} \\
  \IEEEeqnarraymulticol{3}{r}{
\begin{IEEEeqnarraybox}{rCl}
  &\leqslant&
    C\dfrac{d^{\,p(k+1)}}{|B|^p}
      \sum_{|{\balpha}|=k+1}\int_{\Omega}
        \left(\int_B\int_0^1|\partial^{\balpha} f(t\by+(1-t)\bx)|^p\,dt\,d\by\right)
        \left(\frac{|B|}{qk+1}\right)^{\nicefrac{p}{q}}
        \\[5pt]
  &=&\frac{C}{(qk+1)^{\nicefrac{p}{q}}}\frac{d^{\,p(k+1)}}{|B|}
    \sum_{|{\balpha}|=k+1} \int_{\Omega}\int_B\int_0^1
      |\partial^{\balpha} f(t\by+(1-t)\bx)|^p\,dt\,d\by d\bx.
\end{IEEEeqnarraybox}}
\end{IEEEeqnarray*}
For any multi--index ${\balpha}$ we split
\begin{IEEEeqnarray}{rCl}
\IEEEeqnarraymulticol{3}{l}{\nonumber
\int_{\Omega}\int_B\int_0^1
  |\partial^{\balpha} f(t\by+(1-t)\bx)|^p\,dt\,d\by d\bx\, = \,}\\[5pt]
\IEEEeqnarraymulticol{3}{r}{
\begin{IEEEeqnarraybox}{rCl}
\hspace{2.7cm}& = & \int_{\Omega}\int_B\int_0^{\nicefrac12}
      |\partial^{\balpha} f(t\by+(1-t)\bx)|^p\,dt\,d\by d\bx\\[5pt]
\label{first_auxiliary}
   &  &\, +  
\int_{\Omega}\int_B\int_{\nicefrac12}^1
      |\partial^{\balpha} f(t\by+(1-t)\bx)|^p\,dt\,d\by d\bx.\hspace{.3cm}
\end{IEEEeqnarraybox}} 
\end{IEEEeqnarray}
Let $\phi_{\balpha}$ be the extension by zero of $\partial^{\balpha} f$ to $\mathbb{R}^n$.
By Fubini's Theorem and change of variables, the first term in~(\ref{first_auxiliary})
is less than or equal to
\begin{IEEEeqnarray*}{rCl}
  \int_{B}\int_0^{\nicefrac12}\int_{\mathbb{R}^n}
      |\phi_{\balpha}(\bz)|^p(1-t)^{-n}\,d\bz\,dt\, d\by
      &\leqslant& 2^{n-1} |B|\|\partial^{\balpha} f\|^p_{p,\Omega}.
\end{IEEEeqnarray*}
The second term in~(\ref{first_auxiliary}) is less than or equal to
\begin{IEEEeqnarray*}{rCl}
  \int_{\Omega}\int_{\nicefrac12}^1\int_{\mathbb{R}^n}
      |\phi_{\balpha}(t\by)|^p\,d\by\,dt\, d\bx
  &=& \int_{\Omega}\int_{\nicefrac12}^1\int_{\mathbb{R}^n}
      |\phi_{\balpha}(\bz)|^pt^{-n}\,d\bz\,dt\, d\bx\\
      &\leqslant& 2^{n-1} |\Omega|\|\partial^{\balpha} f\|^p_{p,\Omega}.
\end{IEEEeqnarray*}
Summing theese up for any ${\balpha}$ of order $k+1$ we obtain
\begin{IEEEeqnarray*}{rCl}
  \|f-\Qb_{k,B}f\|_p^p & \leqslant & 
  \frac{C}{(qk+1)^{\nicefrac{p}{q}}}d^{\,p(k+1)}\frac{|\Omega|}{|B|} |f|^p_{p,k+1,\Omega}.
\end{IEEEeqnarray*}
\end{proof}
For the following paragraphs we refer to the exposition in
Theorem 3.2 of~\cite{dupontScott}.
\begin{theorem}
  \label{aux_label21}
Let $m\geqslant 0$ and $p$, $\bar{p}\in [1,\infty]$. Suppose
\begin{IEEEeqnarray*}{rCl}
  \frac{1}{\bar{p}} - \frac{1}{p} + \frac{m+1}{3} & \geqslant & 0
\end{IEEEeqnarray*}
and that there exists $\sigma$ with 
\begin{IEEEeqnarray*}{rCcCl}
  0 & < & \sigma & \leqslant & 
  \max\left\{
    \left\lfloor \frac{m+1}{3} \right\rfloor,
    \frac{1}{\bar{p}} - \frac{1}{p} + \frac{m+1}{3},
    \min\left\{1-\frac{1}{\bar{p}},\frac{1}{\bar{p}}\right\}
  \right\}\mbox{,}
\end{IEEEeqnarray*}
then there is a positive $C$ depending only on $m$,$\sigma$ and $\Omega$ such
that, for all $g\in W^{m+1,p}(\Omega)$
\begin{IEEEeqnarray}{rCl} \label{aux_label19}
  \|\partial^{{\bbeta}}(g-\Qb_m g)\|_{L^{\bar{p}}(\Omega)} & \leqslant & C|g|_{W^{m+1,p}(\Omega)}
\end{IEEEeqnarray}
whenever $0 \geqslant |{\bbeta}| \geqslant m+1$.
\end{theorem}
% section polynomial_approximation (end)


%==========================================================================
%\begin{defi} We say that the unisolvent Finite Element $(E, P_E, \Sigma)$ is
%$H(\bcurl)$--conforming if every time we take two
%\emph{push-forward} elements $(E_1, P_{E_1}, \Sigma_1)$
%and $(E_2, P_{E_2}, \Sigma_2)$ according
%to~(\ref{sub:transformations}), and denote $\pi_1$, $\pi_2$
%the interpolation operators determined by the degrees
%of freedom $\Sigma_1$ and $\Sigma_2$, the field defined as
%\begin{IEEEeqnarray*}{rCl}
% \bw & = &
%   \begin{cases}
%     \pi_1(\bu|_{\Omega_1}), &\text{ on }\Omega_1\\
%     \pi_2(\bu|_{\Omega_2}), &\text{ on }\Omega_2      
%   \end{cases}
%\end{IEEEeqnarray*}
%results in $H(\bcurl,\Omega_1\cup\Omega_2)$.
%\end{defi}
% section preliminares (end)
%==========================================================================