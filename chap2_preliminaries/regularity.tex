\section{Regularity of the solution for a model Elliptic Problem}
\label{sec:regularity}
\macroRegularity
%\macroPrismRegularity
\noindent We are concerned in solving 
the following problem.
\begin{problem}\label{mixedContinuous}
Suppose we have a simply connected Lipschitz polyhedron
$\Omega\subseteq\mathbb{R}^3$ which is not convex. Given $f\in L^2(\Omega)$
we look for a $\bu\in H(\dv,\Omega)$ such that 
\begin{IEEEeqnarray*}{rCl}
  \boldsymbol{u}              & = & \nabla p \\
  -\text{div\,}\boldsymbol{u} & = & f \\
   p|_{\partial\Omega}
  & = & 0.
\end{IEEEeqnarray*}
\mbox{\color{Orange}la cond de borde VA?\quad COMO LO PONGO? \quad (ver apunte de FEM2012 talvez)}
\end{problem}
\begin{problem}[Weak formulation.]\label{weakMixedContinuous}
Suppose we have a simply connected Lipschitz polyhedron
$\Omega\subseteq\mathbb{R}^3$ which is not convex. Given $f\in L^2(\Omega)$,
find       $(\bu, p)  \in  H(\dv, \Omega) \times L^2(\Omega)$ 
    such that for all   $(\bv, q)  \in  H(\dv, \Omega) \times L^2(\Omega)$
  \begin{IEEEeqnarray*}{rCl}
    \int_{\Omega} \bu\cdot\bv\,d\bx + 
    \int_{\Omega} p\,\dv\bv\,d\bx                     & = & 0\\
     -\int_{\Omega} q\dv\bu\,d\bx     & = & 
    \int_{\Omega} f\,q\,d\bx.    
  \end{IEEEeqnarray*}
\end{problem}
Due to the polyhedron not being convex there will be ingoing
vertices and edges whose interior angles are obtuse. In the following
we are going to recall how to formalize the singularities in a polyhedron
and how to classify the regularity of the solution of problem~(\ref{mixedContinuous}).\\

{\color{Orange}\#\#\#\#\#\#\#\# continue here.}

Suppose that
S is the origin of our Cartesian system of co-ordinates. Let C be the infinite
S
polyhedral cone of R3 which coincides with ) in a neighbourhood of S; we set
G "C WS2(S), the intersection of C with the unit sphere centered at S. Then the
S
S
S
vertex singular exponent related to S is given by


\begin{defi}
  Suppose $\be$ is an edge and $bv$ is a vertex both in a polihedron
  $\Omega\subseteq \mathbb{R}^3$. We say that
  \begin{enumerate}
    \item $\be$ is singular if the interior angle between the faces sharing
    $\be$ is greater than $\pi$.
    \item $\bv$ is singular if 
  $\lambda_{\bv} := -\dfrac12 + \sqrt{\lambda_{\bv,1} + \dfrac14} < \dfrac12$
  where 
  $\lambda_{\bv,n} > 0$,$n\in\mathbb{N}$, 
  are the eigenvalues (in increasing order) of a (positive)
Laplace--Beltrami operator on a manifold 
  \end{enumerate}
  $\lambda_e$ 
  is ... and 
   is --- 
  $\delta > 1-\lambda_{\be}^{(\ell)} = 1-\pi/\omega_{\be}$
\end{defi}
{\color{blue} poner que beta y delta dependen de l; o sea, regularidad
local, R = dist a EL vertice de  $\Lambda_l$.}

The motivation of calling \textsl{singular} an edge or a vertex is that
the solutions of elliptic problems in non--convex domains have singularities
at those vertices or approaching those edges.

To talk  about the regularity of a solution in a \emph{non--convex} domain
we will make use of the following weighted Sobolev spaces.
\begin{defi} Suppose $\Omega$ is a non--convex polihedron where $\Lambda \subseteq \Omega$ such that 
$\bar{\Lambda}$ contains one singular vertex $\bv$ and one singular edge $\be$. Given two
positive parameters $\beta$ and $\delta$, the symbol $V_{\beta,\delta}^{1,2}(\Lambda)$
denotes the space
\begin{IEEEeqnarray}{rCl}\label{weighted_sobolev}
  V_{\beta,\delta}^{1,2}(\Lambda) & = &
    \left\{v\in \mathcal D'(\Lambda):
      R^{\beta-1+|\alpha|}\theta^{\delta-1+|\alpha|}D^\alpha v\in L^2(\Lambda),
      \alpha\in \mathbb N_0^3, |\alpha|\leqslant 1
    \right\}
\end{IEEEeqnarray}
where $R({\bf x})$ is the distance of ${\bf x}$ to the vertices
of $\Lambda$,
$r({\bf x})$ is the distance from ${\bf x}$ to the edges
of $\Lambda$ and
finally $\theta({\bf x})$ is the angular distance
$\theta({\bf x})=\frac{r({\bf x})}{R({\bf x})}$.
\end{defi}



\noindent Assuming a decomposition of $\Omega=\cup_{\ell=1}^N \Lambda_\ell$ in tetrahedral macroelements having at most a singular edge and a singular vertex, we have the following regularity result.
First we introduce the space $V^{1,2}_{\beta,\delta}(\Lambda)$ for a macroelement $\Lambda$ as
\[
V^{1,2}_{\beta,\delta} = \left\{v\in \mathcal D'(\Lambda): R^{\beta-1+|\alpha|}\theta^{\delta-1+|\alpha|}D^\alpha v\in L^2(\Lambda), \alpha\in \mathbb N^3, |\alpha|\le1\right\}
\]
where $R({\bf x})$ is the distance of ${\bf x}$ to the vertices of $\Lambda$, $r({\bf x})$ is the distance from ${\bf x}$ to the edges of $\Lambda$ and finally $\theta({\bf x})$ is the angular distance $\theta({\bf x})=\frac{r({\bf x})}{R({\bf x})}$.
\[
\|v\|^{1,2}_{\beta,\delta} := \left\{\sum_{|\alpha|\leqslant 1}
\|R^{\beta-1+|\alpha|}\theta^{\delta-1+|\alpha|}D^\alpha v\|_{L^2(\Lambda)}^2\right\}^{1/2}
\]
\[
V^{1,2}_{\beta,\delta} = \left\{v\in \mathcal D'(\Lambda): \|v\|^{1,2}_{\beta,\delta} < \infty\right\}.
\]
If $\beta = \delta$ we will write $V_{\delta,\delta,1,2}(\Lambda_\ell)  = V_{\delta,1,2}(\Lambda)$
and, consistently, 
\[
\|v\|_{\delta,1,2} := \left\{\sum_{|\alpha|\leqslant 1}
\|r^{\delta-1+|\alpha|}\partial^\alpha v\|_{L^2(\Lambda)}^2\right\}^{1/2}.
\]
\begin{theorem}\label{thm_regularity}
The solutions $\bu$ and $p$ of problem \eqref{weakMixedContinuous} satisfy
\[
p\in H^1(\Omega)
\] 
and for each $\ell$
\[
\bu=\bu_r + \bu_s
\]
with $\bu_r\in H^1(\Omega)$ and
\[
\bu_s\cdot \xi_i\in V^{1,2}_{\beta,\delta}(\Lambda_\ell), \quad i=1,2, \qquad
\bu_s\cdot\xi_3\in V^{1,2}_{\beta,0}(\Lambda_\ell)
\]
where $\xi_i$, $i=1,2,3$, are the directions of three concurrent
edges of $\Lambda_\ell$ with $\xi_3$ being the direction of the
singular edge if it exists in $\Omega_\ell$, and $\beta,\delta\ge0$
satisfying $\beta>\frac12-\lambda_{\bv}^{(\ell)}$ and
$\delta>1-\lambda_{\be}^{(\ell)}$, $v$ and $e$ being the singular
vertex and edge, respectively, if they exist.
\end{theorem}
\begin{IEEEeqnarray}{rCl}
  \label{aux_label11}
  \| \bu_r \|_{H^1(\Omega)} & \leqslant & c\,\|f\|_{L^2(\Omega)}\\[5pt]
  \| \bu_s\cdot\xi_i \|_{V_{\beta,\delta}^{1,2}(\Lambda_\ell)} & \leqslant & c\,\| f \|_{L^2(\Omega)}\\[5pt]
  \| \bu_s\cdot\xi_3 \|_{V_{\beta,0}^{1,2}(\Lambda_\ell)}      & \leqslant & c\,\| f \|_{L^2(\Omega)}
\end{IEEEeqnarray}
  provided
\begin{IEEEeqnarray*}{rCl}
  {\color{violet} \delta} & > & 1 - \frac{\pi}{\omega_{\textbf{e}}}\text{,}\\
  {\color{violet} \beta} & > & \frac{1}{2} - \lambda_{\textbf{v}}.
\end{IEEEeqnarray*}
\begin{remark}\label{sobreBetaYDelta}
\textcolor{red}{Note that it is always possible in the previous Theorem to take $0<\beta=\delta<1$.} 
\end{remark}
